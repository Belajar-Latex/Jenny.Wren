% REV00 Tue 04 May 2021 13:55:16 WIB
% START Tue 04 May 2021 13:55:16 WIB

\chapter{XXX}

\includegraphics[scale=2.3]{01-01-01}

BOOK THE SECOND -- BIRDS OF A FEATHER



Chapter 2

STILL EDUCATIONAL


The person of the house, doll’s dressmaker and manufacturer of
ornamental pincushions and pen-wipers, sat in her quaint little low
arm-chair, singing in the dark, until Lizzie came back. The person
of the house had attained that dignity while yet of very tender years
indeed, through being the only trustworthy person IN the house.

‘Well Lizzie-Mizzie-Wizzie,’ said she, breaking off in her song, ‘what’s
the news out of doors?’

‘What’s the news in doors?’ returned Lizzie, playfully smoothing the
bright long fair hair which grew very luxuriant and beautiful on the
head of the doll’s dressmaker.

‘Let me see, said the blind man. Why the last news is, that I don’t mean
to marry your brother.’

‘No?’

‘No-o,’ shaking her head and her chin. ‘Don’t like the boy.’

‘What do you say to his master?’

‘I say that I think he’s bespoke.’

Lizzie finished putting the hair carefully back over the misshapen
shoulders, and then lighted a candle. It showed the little parlour to
be dingy, but orderly and clean. She stood it on the mantelshelf, remote
from the dressmaker’s eyes, and then put the room door open, and the
house door open, and turned the little low chair and its occupant
towards the outer air. It was a sultry night, and this was a
fine-weather arrangement when the day’s work was done. To complete
it, she seated herself in a chair by the side of the little chair, and
protectingly drew under her arm the spare hand that crept up to her.

‘This is what your loving Jenny Wren calls the best time in the day and
night,’ said the person of the house. Her real name was Fanny Cleaver;
but she had long ago chosen to bestow upon herself the appellation of
Miss Jenny Wren.

‘I have been thinking,’ Jenny went on, ‘as I sat at work to-day, what
a thing it would be, if I should be able to have your company till I am
married, or at least courted. Because when I am courted, I shall make
Him do some of the things that you do for me. He couldn’t brush my hair
like you do, or help me up and down stairs like you do, and he couldn’t
do anything like you do; but he could take my work home, and he could
call for orders in his clumsy way. And he shall too. I’LL trot him
about, I can tell him!’

Jenny Wren had her personal vanities--happily for her--and no intentions
were stronger in her breast than the various trials and torments that
were, in the fulness of time, to be inflicted upon ‘him.’

‘Wherever he may happen to be just at present, or whoever he may happen
to be,’ said Miss Wren, ‘I know his tricks and his manners, and I give
him warning to look out.’

‘Don’t you think you are rather hard upon him?’ asked her friend,
smiling, and smoothing her hair.

‘Not a bit,’ replied the sage Miss Wren, with an air of vast experience.
‘My dear, they don’t care for you, those fellows, if you’re NOT hard
upon ‘em. But I was saying If I should be able to have your company. Ah!
What a large If! Ain’t it?’

‘I have no intention of parting company, Jenny.’

‘Don’t say that, or you’ll go directly.’

‘Am I so little to be relied upon?’

‘You’re more to be relied upon than silver and gold.’ As she said it,
Miss Wren suddenly broke off, screwed up her eyes and her chin, and
looked prodigiously knowing. ‘Aha!

     Who comes here?
     A Grenadier.
     What does he want?
     A pot of beer.

And nothing else in the world, my dear!’

A man’s figure paused on the pavement at the outer door. ‘Mr Eugene
Wrayburn, ain’t it?’ said Miss Wren.

‘So I am told,’ was the answer.

‘You may come in, if you’re good.’

‘I am not good,’ said Eugene, ‘but I’ll come in.’

He gave his hand to Jenny Wren, and he gave his hand to Lizzie, and he
stood leaning by the door at Lizzie’s side. He had been strolling with
his cigar, he said, (it was smoked out and gone by this time,) and he
had strolled round to return in that direction that he might look in as
he passed. Had she not seen her brother to-night?

‘Yes,’ said Lizzie, whose manner was a little troubled.

Gracious condescension on our brother’s part! Mr Eugene Wrayburn thought
he had passed my young gentleman on the bridge yonder. Who was his
friend with him?

‘The schoolmaster.’

‘To be sure. Looked like it.’

Lizzie sat so still, that one could not have said wherein the fact of
her manner being troubled was expressed; and yet one could not have
doubted it. Eugene was as easy as ever; but perhaps, as she sat with
her eyes cast down, it might have been rather more perceptible that
his attention was concentrated upon her for certain moments, than its
concentration upon any subject for any short time ever was, elsewhere.

‘I have nothing to report, Lizzie,’ said Eugene. ‘But, having promised
you that an eye should be always kept on Mr Riderhood through my friend
Lightwood, I like occasionally to renew my assurance that I keep my
promise, and keep my friend up to the mark.’

‘I should not have doubted it, sir.’

‘Generally, I confess myself a man to be doubted,’ returned Eugene,
coolly, ‘for all that.’

‘Why are you?’ asked the sharp Miss Wren.

‘Because, my dear,’ said the airy Eugene, ‘I am a bad idle dog.’

‘Then why don’t you reform and be a good dog?’ inquired Miss Wren.

‘Because, my dear,’ returned Eugene, ‘there’s nobody who makes it worth
my while. Have you considered my suggestion, Lizzie?’ This in a lower
voice, but only as if it were a graver matter; not at all to the
exclusion of the person of the house.

‘I have thought of it, Mr Wrayburn, but I have not been able to make up
my mind to accept it.’

‘False pride!’ said Eugene.

‘I think not, Mr Wrayburn. I hope not.’

‘False pride!’ repeated Eugene. ‘Why, what else is it? The thing is
worth nothing in itself. The thing is worth nothing to me. What can it
be worth to me? You know the most I make of it. I propose to be of some
use to somebody--which I never was in this world, and never shall be on
any other occasion--by paying some qualified person of your own sex and
age, so many (or rather so few) contemptible shillings, to come here,
certain nights in the week, and give you certain instruction which you
wouldn’t want if you hadn’t been a self-denying daughter and sister.
You know that it’s good to have it, or you would never have so devoted
yourself to your brother’s having it. Then why not have it: especially
when our friend Miss Jenny here would profit by it too? If I proposed to
be the teacher, or to attend the lessons--obviously incongruous!--but
as to that, I might as well be on the other side of the globe, or not
on the globe at all. False pride, Lizzie. Because true pride wouldn’t
shame, or be shamed by, your thankless brother. True pride wouldn’t have
schoolmasters brought here, like doctors, to look at a bad case. True
pride would go to work and do it. You know that, well enough, for you
know that your own true pride would do it to-morrow, if you had the ways
and means which false pride won’t let me supply. Very well. I add no
more than this. Your false pride does wrong to yourself and does wrong
to your dead father.’

‘How to my father, Mr Wrayburn?’ she asked, with an anxious face.

‘How to your father? Can you ask! By perpetuating the consequences of
his ignorant and blind obstinacy. By resolving not to set right the
wrong he did you. By determining that the deprivation to which he
condemned you, and which he forced upon you, shall always rest upon his
head.’

It chanced to be a subtle string to sound, in her who had so spoken to
her brother within the hour. It sounded far more forcibly, because of
the change in the speaker for the moment; the passing appearance of
earnestness, complete conviction, injured resentment of suspicion,
generous and unselfish interest. All these qualities, in him usually so
light and careless, she felt to be inseparable from some touch of their
opposites in her own breast. She thought, had she, so far below him
and so different, rejected this disinterestedness, because of some vain
misgiving that he sought her out, or heeded any personal attractions
that he might descry in her? The poor girl, pure of heart and purpose,
could not bear to think it. Sinking before her own eyes, as she
suspected herself of it, she drooped her head as though she had done him
some wicked and grievous injury, and broke into silent tears.

‘Don’t be distressed,’ said Eugene, very, very kindly. ‘I hope it is not
I who have distressed you. I meant no more than to put the matter in its
true light before you; though I acknowledge I did it selfishly enough,
for I am disappointed.’

Disappointed of doing her a service. How else COULD he be disappointed?

‘It won’t break my heart,’ laughed Eugene; ‘it won’t stay by me
eight-and-forty hours; but I am genuinely disappointed. I had set my
fancy on doing this little thing for you and for our friend Miss Jenny.
The novelty of my doing anything in the least useful, had its charms. I
see, now, that I might have managed it better. I might have affected to
do it wholly for our friend Miss J. I might have got myself up, morally,
as Sir Eugene Bountiful. But upon my soul I can’t make flourishes, and I
would rather be disappointed than try.’

If he meant to follow home what was in Lizzie’s thoughts, it was
skilfully done. If he followed it by mere fortuitous coincidence, it was
done by an evil chance.

‘It opened out so naturally before me,’ said Eugene. ‘The ball seemed so
thrown into my hands by accident! I happen to be originally brought into
contact with you, Lizzie, on those two occasions that you know of. I
happen to be able to promise you that a watch shall be kept upon that
false accuser, Riderhood. I happen to be able to give you some little
consolation in the darkest hour of your distress, by assuring you that I
don’t believe him. On the same occasion I tell you that I am the idlest
and least of lawyers, but that I am better than none, in a case I have
noted down with my own hand, and that you may be always sure of my best
help, and incidentally of Lightwood’s too, in your efforts to clear
your father. So, it gradually takes my fancy that I may help you--so
easily!--to clear your father of that other blame which I mentioned
a few minutes ago, and which is a just and real one. I hope I have
explained myself; for I am heartily sorry to have distressed you. I hate
to claim to mean well, but I really did mean honestly and simply well,
and I want you to know it.’

‘I have never doubted that, Mr Wrayburn,’ said Lizzie; the more
repentant, the less he claimed.

‘I am very glad to hear it. Though if you had quite understood my whole
meaning at first, I think you would not have refused. Do you think you
would?’

‘I--don’t know that I should, Mr Wrayburn.’

‘Well! Then why refuse now you do understand it?’

‘It’s not easy for me to talk to you,’ returned Lizzie, in some
confusion, ‘for you see all the consequences of what I say, as soon as I
say it.’

‘Take all the consequences,’ laughed Eugene, ‘and take away my
disappointment. Lizzie Hexam, as I truly respect you, and as I am your
friend and a poor devil of a gentleman, I protest I don’t even now
understand why you hesitate.’

There was an appearance of openness, trustfulness, unsuspecting
generosity, in his words and manner, that won the poor girl over; and
not only won her over, but again caused her to feel as though she had
been influenced by the opposite qualities, with vanity at their head.

‘I will not hesitate any longer, Mr Wrayburn. I hope you will not
think the worse of me for having hesitated at all. For myself and for
Jenny--you let me answer for you, Jenny dear?’

The little creature had been leaning back, attentive, with her elbows
resting on the elbows of her chair, and her chin upon her hands. Without
changing her attitude, she answered, ‘Yes!’ so suddenly that it rather
seemed as if she had chopped the monosyllable than spoken it.

‘For myself and for Jenny, I thankfully accept your kind offer.’

‘Agreed! Dismissed!’ said Eugene, giving Lizzie his hand before lightly
waving it, as if he waved the whole subject away. ‘I hope it may not be
often that so much is made of so little!’

Then he fell to talking playfully with Jenny Wren. ‘I think of setting
up a doll, Miss Jenny,’ he said.

‘You had better not,’ replied the dressmaker.

‘Why not?’

‘You are sure to break it. All you children do.’

‘But that makes good for trade, you know, Miss Wren,’ returned Eugene.
‘Much as people’s breaking promises and contracts and bargains of all
sorts, makes good for MY trade.’

‘I don’t know about that,’ Miss Wren retorted; ‘but you had better by
half set up a pen-wiper, and turn industrious, and use it.’

‘Why, if we were all as industrious as you, little Busy-Body, we should
begin to work as soon as we could crawl, and there would be a bad
thing!’

‘Do you mean,’ returned the little creature, with a flush suffusing her
face, ‘bad for your backs and your legs?’

‘No, no, no,’ said Eugene; shocked--to do him justice--at the thought of
trifling with her infirmity. ‘Bad for business, bad for business. If we
all set to work as soon as we could use our hands, it would be all over
with the dolls’ dressmakers.’

‘There’s something in that,’ replied Miss Wren; ‘you have a sort of an
idea in your noddle sometimes.’ Then, in a changed tone; ‘Talking of
ideas, my Lizzie,’ they were sitting side by side as they had sat at
first, ‘I wonder how it happens that when I am work, work, working here,
all alone in the summer-time, I smell flowers.’

‘As a commonplace individual, I should say,’ Eugene suggested
languidly--for he was growing weary of the person of the house--‘that
you smell flowers because you DO smell flowers.’

‘No I don’t,’ said the little creature, resting one arm upon the elbow
of her chair, resting her chin upon that hand, and looking vacantly
before her; ‘this is not a flowery neighbourhood. It’s anything but
that. And yet as I sit at work, I smell miles of flowers. I smell roses,
till I think I see the rose-leaves lying in heaps, bushels, on the
floor. I smell fallen leaves, till I put down my hand--so--and expect to
make them rustle. I smell the white and the pink May in the hedges, and
all sorts of flowers that I never was among. For I have seen very few
flowers indeed, in my life.’

‘Pleasant fancies to have, Jenny dear!’ said her friend: with a glance
towards Eugene as if she would have asked him whether they were given
the child in compensation for her losses.

‘So I think, Lizzie, when they come to me. And the birds I hear! Oh!’
cried the little creature, holding out her hand and looking upward, ‘how
they sing!’

There was something in the face and action for the moment, quite
inspired and beautiful. Then the chin dropped musingly upon the hand
again.

‘I dare say my birds sing better than other birds, and my flowers smell
better than other flowers. For when I was a little child,’ in a tone as
though it were ages ago, ‘the children that I used to see early in the
morning were very different from any others that I ever saw. They were
not like me; they were not chilled, anxious, ragged, or beaten; they
were never in pain. They were not like the children of the neighbours;
they never made me tremble all over, by setting up shrill noises, and
they never mocked me. Such numbers of them too! All in white dresses,
and with something shining on the borders, and on their heads, that I
have never been able to imitate with my work, though I know it so
well. They used to come down in long bright slanting rows, and say all
together, “Who is this in pain! Who is this in pain!” When I told them
who it was, they answered, “Come and play with us!” When I said “I never
play! I can’t play!” they swept about me and took me up, and made me
light. Then it was all delicious ease and rest till they laid me
down, and said, all together, “Have patience, and we will come again.”
 Whenever they came back, I used to know they were coming before I saw
the long bright rows, by hearing them ask, all together a long way off,
“Who is this in pain! Who is this in pain!” And I used to cry out, “O my
blessed children, it’s poor me. Have pity on me. Take me up and make me
light!”’

By degrees, as she progressed in this remembrance, the hand was raised,
the late ecstatic look returned, and she became quite beautiful. Having
so paused for a moment, silent, with a listening smile upon her face,
she looked round and recalled herself.

‘What poor fun you think me; don’t you, Mr Wrayburn? You may well look
tired of me. But it’s Saturday night, and I won’t detain you.’

‘That is to say, Miss Wren,’ observed Eugene, quite ready to profit by
the hint, ‘you wish me to go?’

‘Well, it’s Saturday night,’ she returned, ‘and my child’s coming
home. And my child is a troublesome bad child, and costs me a world of
scolding. I would rather you didn’t see my child.’

‘A doll?’ said Eugene, not understanding, and looking for an
explanation.

But Lizzie, with her lips only, shaping the two words, ‘Her father,’ he
delayed no longer. He took his leave immediately. At the corner of the
street he stopped to light another cigar, and possibly to ask himself
what he was doing otherwise. If so, the answer was indefinite and vague.
Who knows what he is doing, who is careless what he does!

A man stumbled against him as he turned away, who mumbled some maudlin
apology. Looking after this man, Eugene saw him go in at the door by
which he himself had just come out.

On the man’s stumbling into the room, Lizzie rose to leave it.

‘Don’t go away, Miss Hexam,’ he said in a submissive manner, speaking
thickly and with difficulty. ‘Don’t fly from unfortunate man in
shattered state of health. Give poor invalid honour of your company. It
ain’t--ain’t catching.’

Lizzie murmured that she had something to do in her own room, and went
away upstairs.

‘How’s my Jenny?’ said the man, timidly. ‘How’s my Jenny Wren, best of
children, object dearest affections broken-hearted invalid?’

To which the person of the house, stretching out her arm in an attitude
of command, replied with irresponsive asperity: ‘Go along with you! Go
along into your corner! Get into your corner directly!’

The wretched spectacle made as if he would have offered some
remonstrance; but not venturing to resist the person of the house,
thought better of it, and went and sat down on a particular chair of
disgrace.

‘Oh-h-h!’ cried the person of the house, pointing her little finger,
‘You bad old boy! Oh-h-h you naughty, wicked creature! WHAT do you mean
by it?’

The shaking figure, unnerved and disjointed from head to foot, put
out its two hands a little way, as making overtures of peace and
reconciliation. Abject tears stood in its eyes, and stained the blotched
red of its cheeks. The swollen lead-coloured under lip trembled with a
shameful whine. The whole indecorous threadbare ruin, from the broken
shoes to the prematurely-grey scanty hair, grovelled. Not with any sense
worthy to be called a sense, of this dire reversal of the places of
parent and child, but in a pitiful expostulation to be let off from a
scolding.

‘I know your tricks and your manners,’ cried Miss Wren. ‘I know where
you’ve been to!’ (which indeed it did not require discernment to
discover). ‘Oh, you disgraceful old chap!’

The very breathing of the figure was contemptible, as it laboured and
rattled in that operation, like a blundering clock.

‘Slave, slave, slave, from morning to night,’ pursued the person of the
house, ‘and all for this! WHAT do you mean by it?’

There was something in that emphasized ‘What,’ which absurdly frightened
the figure. As often as the person of the house worked her way round to
it--even as soon as he saw that it was coming--he collapsed in an extra
degree.

‘I wish you had been taken up, and locked up,’ said the person of the
house. ‘I wish you had been poked into cells and black holes, and run
over by rats and spiders and beetles. I know their tricks and their
manners, and they’d have tickled you nicely. Ain’t you ashamed of
yourself?’

‘Yes, my dear,’ stammered the father.

‘Then,’ said the person of the house, terrifying him by a grand muster
of her spirits and forces before recurring to the emphatic word, ‘WHAT
do you mean by it?’

‘Circumstances over which had no control,’ was the miserable creature’s
plea in extenuation.

‘I’LL circumstance you and control you too,’ retorted the person of the
house, speaking with vehement sharpness, ‘if you talk in that way. I’ll
give you in charge to the police, and have you fined five shillings when
you can’t pay, and then I won’t pay the money for you, and you’ll be
transported for life. How should you like to be transported for life?’

‘Shouldn’t like it. Poor shattered invalid. Trouble nobody long,’ cried
the wretched figure.

‘Come, come!’ said the person of the house, tapping the table near her
in a business-like manner, and shaking her head and her chin; ‘you know
what you’ve got to do. Put down your money this instant.’

The obedient figure began to rummage in its pockets.

‘Spent a fortune out of your wages, I’ll be bound!’ said the person of
the house. ‘Put it here! All you’ve got left! Every farthing!’

Such a business as he made of collecting it from his dogs’-eared
pockets; of expecting it in this pocket, and not finding it; of not
expecting it in that pocket, and passing it over; of finding no pocket
where that other pocket ought to be!

‘Is this all?’ demanded the person of the house, when a confused heap of
pence and shillings lay on the table.

‘Got no more,’ was the rueful answer, with an accordant shake of the
head.

‘Let me make sure. You know what you’ve got to do. Turn all your pockets
inside out, and leave ‘em so!’ cried the person of the house.

He obeyed. And if anything could have made him look more abject or more
dismally ridiculous than before, it would have been his so displaying
himself.

‘Here’s but seven and eightpence halfpenny!’ exclaimed Miss Wren, after
reducing the heap to order. ‘Oh, you prodigal old son! Now you shall be
starved.’

‘No, don’t starve me,’ he urged, whimpering.

‘If you were treated as you ought to be,’ said Miss Wren, ‘you’d be fed
upon the skewers of cats’ meat;--only the skewers, after the cats had
had the meat. As it is, go to bed.’

When he stumbled out of the corner to comply, he again put out both his
hands, and pleaded: ‘Circumstances over which no control--’

‘Get along with you to bed!’ cried Miss Wren, snapping him up. ‘Don’t
speak to me. I’m not going to forgive you. Go to bed this moment!’

Seeing another emphatic ‘What’ upon its way, he evaded it by complying
and was heard to shuffle heavily up stairs, and shut his door, and throw
himself on his bed. Within a little while afterwards, Lizzie came down.

‘Shall we have our supper, Jenny dear?’

‘Ah! bless us and save us, we need have something to keep us going,’
returned Miss Jenny, shrugging her shoulders.

Lizzie laid a cloth upon the little bench (more handy for the person of
the house than an ordinary table), and put upon it such plain fare as
they were accustomed to have, and drew up a stool for herself.

‘Now for supper! What are you thinking of, Jenny darling?’

‘I was thinking,’ she returned, coming out of a deep study, ‘what I
would do to Him, if he should turn out a drunkard.’

‘Oh, but he won’t,’ said Lizzie. ‘You’ll take care of that, beforehand.’

‘I shall try to take care of it beforehand, but he might deceive me.
Oh, my dear, all those fellows with their tricks and their manners do
deceive!’ With the little fist in full action. ‘And if so, I tell you
what I think I’d do. When he was asleep, I’d make a spoon red hot, and
I’d have some boiling liquor bubbling in a saucepan, and I’d take it
out hissing, and I’d open his mouth with the other hand--or perhaps he’d
sleep with his mouth ready open--and I’d pour it down his throat, and
blister it and choke him.’

‘I am sure you would do no such horrible thing,’ said Lizzie.

‘Shouldn’t I? Well; perhaps I shouldn’t. But I should like to!’

‘I am equally sure you would not.’

‘Not even like to? Well, you generally know best. Only you haven’t
always lived among it as I have lived--and your back isn’t bad and your
legs are not queer.’

As they went on with their supper, Lizzie tried to bring her round to
that prettier and better state. But, the charm was broken. The person
of the house was the person of a house full of sordid shames and cares,
with an upper room in which that abased figure was infecting even
innocent sleep with sensual brutality and degradation. The doll’s
dressmaker had become a little quaint shrew; of the world, worldly; of
the earth, earthy.

Poor doll’s dressmaker! How often so dragged down by hands that should
have raised her up; how often so misdirected when losing her way on the
eternal road, and asking guidance! Poor, poor little doll’s dressmaker!



Chapter 3

A PIECE OF WORK


Britannia, sitting meditating one fine day (perhaps in the attitude in
which she is presented on the copper coinage), discovers all of a sudden
that she wants Veneering in Parliament. It occurs to her that Veneering
is ‘a representative man’--which cannot in these times be doubted--and
that Her Majesty’s faithful Commons are incomplete without him. So,
Britannia mentions to a legal gentleman of her acquaintance that if
Veneering will ‘put down’ five thousand pounds, he may write a couple
of initial letters after his name at the extremely cheap rate of two
thousand five hundred per letter. It is clearly understood between
Britannia and the legal gentleman that nobody is to take up the five
thousand pounds, but that being put down they will disappear by magical
conjuration and enchantment.

The legal gentleman in Britannia’s confidence going straight from that
lady to Veneering, thus commissioned, Veneering declares himself highly
flattered, but requires breathing time to ascertain ‘whether his friends
will rally round him.’ Above all things, he says, it behoves him to be
clear, at a crisis of this importance, ‘whether his friends will rally
round him.’ The legal gentleman, in the interests of his client cannot
allow much time for this purpose, as the lady rather thinks she knows
somebody prepared to put down six thousand pounds; but he says he will
give Veneering four hours.

Veneering then says to Mrs Veneering, ‘We must work,’ and throws himself
into a Hansom cab. Mrs Veneering in the same moment relinquishes baby
to Nurse; presses her aquiline hands upon her brow, to arrange the
throbbing intellect within; orders out the carriage; and repeats in
a distracted and devoted manner, compounded of Ophelia and any
self-immolating female of antiquity you may prefer, ‘We must work.’

Veneering having instructed his driver to charge at the Public in the
streets, like the Life-Guards at Waterloo, is driven furiously to Duke
Street, Saint James’s. There, he finds Twemlow in his lodgings, fresh
from the hands of a secret artist who has been doing something to his
hair with yolks of eggs. The process requiring that Twemlow shall, for
two hours after the application, allow his hair to stick upright and dry
gradually, he is in an appropriate state for the receipt of startling
intelligence; looking equally like the Monument on Fish Street Hill, and
King Priam on a certain incendiary occasion not wholly unknown as a neat
point from the classics.

‘My dear Twemlow,’ says Veneering, grasping both his hands, ‘as the
dearest and oldest of my friends--’

[‘Then there can be no more doubt about it in future,’ thinks Twemlow,
‘and I AM!’)

‘--Are you of opinion that your cousin, Lord Snigsworth, would give his
name as a Member of my Committee? I don’t go so far as to ask for his
lordship; I only ask for his name. Do you think he would give me his
name?’

In sudden low spirits, Twemlow replies, ‘I don’t think he would.’

‘My political opinions,’ says Veneering, not previously aware of having
any, ‘are identical with those of Lord Snigsworth, and perhaps as a
matter of public feeling and public principle, Lord Snigsworth would
give me his name.’

‘It might be so,’ says Twemlow; ‘but--’ And perplexedly scratching his
head, forgetful of the yolks of eggs, is the more discomfited by being
reminded how stickey he is.

‘Between such old and intimate friends as ourselves,’ pursues Veneering,
‘there should in such a case be no reserve. Promise me that if I ask you
to do anything for me which you don’t like to do, or feel the slightest
difficulty in doing, you will freely tell me so.’

This, Twemlow is so kind as to promise, with every appearance of most
heartily intending to keep his word.

‘Would you have any objection to write down to Snigsworthy Park, and ask
this favour of Lord Snigsworth? Of course if it were granted I should
know that I owed it solely to you; while at the same time you would put
it to Lord Snigsworth entirely upon public grounds. Would you have any
objection?’

Says Twemlow, with his hand to his forehead, ‘You have exacted a promise
from me.’

‘I have, my dear Twemlow.’

‘And you expect me to keep it honourably.’

‘I do, my dear Twemlow.’

‘ON the whole, then;--observe me,’ urges Twemlow with great nicety, as
if; in the case of its having been off the whole, he would have done it
directly--‘ON the whole, I must beg you to excuse me from addressing any
communication to Lord Snigsworth.’

‘Bless you, bless you!’ says Veneering; horribly disappointed, but
grasping him by both hands again, in a particularly fervent manner.

It is not to be wondered at that poor Twemlow should decline to inflict
a letter on his noble cousin (who has gout in the temper), inasmuch
as his noble cousin, who allows him a small annuity on which he lives,
takes it out of him, as the phrase goes, in extreme severity; putting
him, when he visits at Snigsworthy Park, under a kind of martial law;
ordaining that he shall hang his hat on a particular peg, sit on a
particular chair, talk on particular subjects to particular people, and
perform particular exercises: such as sounding the praises of the Family
Varnish (not to say Pictures), and abstaining from the choicest of the
Family Wines unless expressly invited to partake.

‘One thing, however, I CAN do for you,’ says Twemlow; ‘and that is, work
for you.’

Veneering blesses him again.

‘I’ll go,’ says Twemlow, in a rising hurry of spirits, ‘to the
club;--let us see now; what o’clock is it?’

‘Twenty minutes to eleven.’

‘I’ll be,’ says Twemlow, ‘at the club by ten minutes to twelve, and I’ll
never leave it all day.’

Veneering feels that his friends are rallying round him, and says,
‘Thank you, thank you. I knew I could rely upon you. I said to Anastatia
before leaving home just now to come to you--of course the first friend
I have seen on a subject so momentous to me, my dear Twemlow--I said to
Anastatia, “We must work.”’

‘You were right, you were right,’ replies Twemlow. ‘Tell me. Is SHE
working?’

‘She is,’ says Veneering.

‘Good!’ cries Twemlow, polite little gentleman that he is. ‘A woman’s
tact is invaluable. To have the dear sex with us, is to have everything
with us.’

‘But you have not imparted to me,’ remarks Veneering, ‘what you think of
my entering the House of Commons?’

‘I think,’ rejoins Twemlow, feelingly, ‘that it is the best club in
London.’

Veneering again blesses him, plunges down stairs, rushes into his
Hansom, and directs the driver to be up and at the British Public, and
to charge into the City.

Meanwhile Twemlow, in an increasing hurry of spirits, gets his hair down
as well as he can--which is not very well; for, after these glutinous
applications it is restive, and has a surface on it somewhat in the
nature of pastry--and gets to the club by the appointed time. At the
club he promptly secures a large window, writing materials, and all
the newspapers, and establishes himself; immoveable, to be respectfully
contemplated by Pall Mall. Sometimes, when a man enters who nods to
him, Twemlow says, ‘Do you know Veneering?’ Man says, ‘No; member of
the club?’ Twemlow says, ‘Yes. Coming in for Pocket-Breaches.’ Man says,
‘Ah! Hope he may find it worth the money!’ yawns, and saunters out.
Towards six o’clock of the afternoon, Twemlow begins to persuade
himself that he is positively jaded with work, and thinks it much to be
regretted that he was not brought up as a Parliamentary agent.

From Twemlow’s, Veneering dashes at Podsnap’s place of business. Finds
Podsnap reading the paper, standing, and inclined to be oratorical
over the astonishing discovery he has made, that Italy is not England.
Respectfully entreats Podsnap’s pardon for stopping the flow of his
words of wisdom, and informs him what is in the wind. Tells Podsnap that
their political opinions are identical. Gives Podsnap to understand that
he, Veneering, formed his political opinions while sitting at the feet
of him, Podsnap. Seeks earnestly to know whether Podsnap ‘will rally
round him?’

Says Podsnap, something sternly, ‘Now, first of all, Veneering, do you
ask my advice?’

Veneering falters that as so old and so dear a friend--

‘Yes, yes, that’s all very well,’ says Podsnap; ‘but have you made up
your mind to take this borough of Pocket-Breaches on its own terms, or
do you ask my opinion whether you shall take it or leave it alone?’

Veneering repeats that his heart’s desire and his soul’s thirst are,
that Podsnap shall rally round him.

‘Now, I’ll be plain with you, Veneering,’ says Podsnap, knitting his
brows. ‘You will infer that I don’t care about Parliament, from the fact
of my not being there?’

Why, of course Veneering knows that! Of course Veneering knows that if
Podsnap chose to go there, he would be there, in a space of time that
might be stated by the light and thoughtless as a jiffy.

‘It is not worth my while,’ pursues Podsnap, becoming handsomely
mollified, ‘and it is the reverse of important to my position. But it
is not my wish to set myself up as law for another man, differently
situated. You think it IS worth YOUR while, and IS important to YOUR
position. Is that so?’

Always with the proviso that Podsnap will rally round him, Veneering
thinks it is so.

‘Then you don’t ask my advice,’ says Podsnap. ‘Good. Then I won’t give
it you. But you do ask my help. Good. Then I’ll work for you.’

Veneering instantly blesses him, and apprises him that Twemlow is
already working. Podsnap does not quite approve that anybody should
be already working--regarding it rather in the light of a liberty--but
tolerates Twemlow, and says he is a well-connected old female who will
do no harm.

‘I have nothing very particular to do to-day,’ adds Podsnap, ‘and I’ll
mix with some influential people. I had engaged myself to dinner, but
I’ll send Mrs Podsnap and get off going myself; and I’ll dine with you
at eight. It’s important we should report progress and compare notes.
Now, let me see. You ought to have a couple of active energetic fellows,
of gentlemanly manners, to go about.’

Veneering, after cogitation, thinks of Boots and Brewer.

‘Whom I have met at your house,’ says Podsnap. ‘Yes. They’ll do very
well. Let them each have a cab, and go about.’

Veneering immediately mentions what a blessing he feels it, to possess
a friend capable of such grand administrative suggestions, and really
is elated at this going about of Boots and Brewer, as an idea wearing
an electioneering aspect and looking desperately like business. Leaving
Podsnap, at a hand-gallop, he descends upon Boots and Brewer, who
enthusiastically rally round him by at once bolting off in cabs, taking
opposite directions. Then Veneering repairs to the legal gentleman in
Britannia’s confidence, and with him transacts some delicate affairs
of business, and issues an address to the independent electors of
Pocket-Breaches, announcing that he is coming among them for their
suffrages, as the mariner returns to the home of his early childhood: a
phrase which is none the worse for his never having been near the place
in his life, and not even now distinctly knowing where it is.

Mrs Veneering, during the same eventful hours, is not idle. No sooner
does the carriage turn out, all complete, than she turns into it, all
complete, and gives the word ‘To Lady Tippins’s.’ That charmer dwells
over a staymaker’s in the Belgravian Borders, with a life-size model
in the window on the ground floor of a distinguished beauty in a blue
petticoat, stay-lace in hand, looking over her shoulder at the town in
innocent surprise. As well she may, to find herself dressing under the
circumstances.

Lady Tippins at home? Lady Tippins at home, with the room darkened,
and her back (like the lady’s at the ground-floor window, though for a
different reason) cunningly turned towards the light. Lady Tippins is
so surprised by seeing her dear Mrs Veneering so early--in the middle of
the night, the pretty creature calls it--that her eyelids almost go up,
under the influence of that emotion.

To whom Mrs Veneering incoherently communicates, how that Veneering
has been offered Pocket-Breaches; how that it is the time for rallying
round; how that Veneering has said ‘We must work’; how that she is here,
as a wife and mother, to entreat Lady Tippins to work; how that the
carriage is at Lady Tippins’s disposal for purposes of work; how that
she, proprietress of said bran new elegant equipage, will return home on
foot--on bleeding feet if need be--to work (not specifying how), until
she drops by the side of baby’s crib.

‘My love,’ says Lady Tippins, ‘compose yourself; we’ll bring him in.’
And Lady Tippins really does work, and work the Veneering horses too;
for she clatters about town all day, calling upon everybody she knows,
and showing her entertaining powers and green fan to immense advantage,
by rattling on with, My dear soul, what do you think? What do
you suppose me to be? You’ll never guess. I’m pretending to be an
electioneering agent. And for what place of all places? Pocket-Breaches.
And why? Because the dearest friend I have in the world has bought it.
And who is the dearest friend I have in the world? A man of the name of
Veneering. Not omitting his wife, who is the other dearest friend I have
in the world; and I positively declare I forgot their baby, who is the
other. And we are carrying on this little farce to keep up appearances,
and isn’t it refreshing! Then, my precious child, the fun of it is that
nobody knows who these Veneerings are, and that they know nobody, and
that they have a house out of the Tales of the Genii, and give dinners
out of the Arabian Nights. Curious to see ‘em, my dear? Say you’ll know
‘em. Come and dine with ‘em. They shan’t bore you. Say who shall meet
you. We’ll make up a party of our own, and I’ll engage that they shall
not interfere with you for one single moment. You really ought to see
their gold and silver camels. I call their dinner-table, the Caravan.
Do come and dine with my Veneerings, my own Veneerings, my exclusive
property, the dearest friends I have in the world! And above all, my
dear, be sure you promise me your vote and interest and all sorts of
plumpers for Pocket-Breaches; for we couldn’t think of spending sixpence
on it, my love, and can only consent to be brought in by the spontaneous
thingummies of the incorruptible whatdoyoucallums.

Now, the point of view seized by the bewitching Tippins, that this same
working and rallying round is to keep up appearances, may have something
in it, but not all the truth. More is done, or considered to be
done--which does as well--by taking cabs, and ‘going about,’ than the
fair Tippins knew of. Many vast vague reputations have been made,
solely by taking cabs and going about. This particularly obtains in all
Parliamentary affairs. Whether the business in hand be to get a man in,
or get a man out, or get a man over, or promote a railway, or jockey
a railway, or what else, nothing is understood to be so effectual as
scouring nowhere in a violent hurry--in short, as taking cabs and going
about.

Probably because this reason is in the air, Twemlow, far from being
singular in his persuasion that he works like a Trojan, is capped by
Podsnap, who in his turn is capped by Boots and Brewer. At eight o’clock
when all these hard workers assemble to dine at Veneering’s, it is
understood that the cabs of Boots and Brewer mustn’t leave the door, but
that pails of water must be brought from the nearest baiting-place,
and cast over the horses’ legs on the very spot, lest Boots and Brewer
should have instant occasion to mount and away. Those fleet messengers
require the Analytical to see that their hats are deposited where they
can be laid hold of at an instant’s notice; and they dine (remarkably
well though) with the air of firemen in charge of an engine, expecting
intelligence of some tremendous conflagration.

Mrs Veneering faintly remarks, as dinner opens, that many such days
would be too much for her.

‘Many such days would be too much for all of us,’ says Podsnap; ‘but
we’ll bring him in!’

‘We’ll bring him in,’ says Lady Tippins, sportively waving her green
fan. ‘Veneering for ever!’

‘We’ll bring him in!’ says Twemlow.

‘We’ll bring him in!’ say Boots and Brewer.

Strictly speaking, it would be hard to show cause why they should not
bring him in, Pocket-Breaches having closed its little bargain, and
there being no opposition. However, it is agreed that they must ‘work’
to the last, and that if they did not work, something indefinite would
happen. It is likewise agreed that they are all so exhausted with the
work behind them, and need to be so fortified for the work before them,
as to require peculiar strengthening from Veneering’s cellar. Therefore,
the Analytical has orders to produce the cream of the cream of his
binns, and therefore it falls out that rallying becomes rather a trying
word for the occasion; Lady Tippins being observed gamely to inculcate
the necessity of rearing round their dear Veneering; Podsnap advocating
roaring round him; Boots and Brewer declaring their intention of reeling
round him; and Veneering thanking his devoted friends one and all, with
great emotion, for rarullarulling round him.

In these inspiring moments, Brewer strikes out an idea which is the
great hit of the day. He consults his watch, and says (like Guy Fawkes),
he’ll now go down to the House of Commons and see how things look.

‘I’ll keep about the lobby for an hour or so,’ says Brewer, with a
deeply mysterious countenance, ‘and if things look well, I won’t come
back, but will order my cab for nine in the morning.’

‘You couldn’t do better,’ says Podsnap.

Veneering expresses his inability ever to acknowledge this last service.
Tears stand in Mrs Veneering’s affectionate eyes. Boots shows envy,
loses ground, and is regarded as possessing a second-rate mind. They all
crowd to the door, to see Brewer off. Brewer says to his driver, ‘Now,
is your horse pretty fresh?’ eyeing the animal with critical scrutiny.
Driver says he’s as fresh as butter. ‘Put him along then,’ says Brewer;
‘House of Commons.’ Driver darts up, Brewer leaps in, they cheer him as
he departs, and Mr Podsnap says, ‘Mark my words, sir. That’s a man of
resource; that’s a man to make his way in life.’

When the time comes for Veneering to deliver a neat and appropriate
stammer to the men of Pocket-Breaches, only Podsnap and Twemlow
accompany him by railway to that sequestered spot. The legal gentleman
is at the Pocket-Breaches Branch Station, with an open carriage with a
printed bill ‘Veneering for ever’ stuck upon it, as if it were a wall;
and they gloriously proceed, amidst the grins of the populace, to a
feeble little town hall on crutches, with some onions and bootlaces
under it, which the legal gentleman says are a Market; and from the
front window of that edifice Veneering speaks to the listening earth.
In the moment of his taking his hat off, Podsnap, as per agreement made
with Mrs Veneering, telegraphs to that wife and mother, ‘He’s up.’

Veneering loses his way in the usual No Thoroughfares of speech, and
Podsnap and Twemlow say Hear hear! and sometimes, when he can’t by any
means back himself out of some very unlucky No Thoroughfare, ‘He-a-a-r
He-a-a-r!’ with an air of facetious conviction, as if the ingenuity of
the thing gave them a sensation of exquisite pleasure. But Veneering
makes two remarkably good points; so good, that they are supposed
to have been suggested to him by the legal gentleman in Britannia’s
confidence, while briefly conferring on the stairs.

Point the first is this. Veneering institutes an original comparison
between the country, and a ship; pointedly calling the ship, the Vessel
of the State, and the Minister the Man at the Helm. Veneering’s object
is to let Pocket-Breaches know that his friend on his right (Podsnap) is
a man of wealth. Consequently says he, ‘And, gentlemen, when the timbers
of the Vessel of the State are unsound and the Man at the Helm is
unskilful, would those great Marine Insurers, who rank among our
world-famed merchant-princes--would they insure her, gentlemen? Would
they underwrite her? Would they incur a risk in her? Would they have
confidence in her? Why, gentlemen, if I appealed to my honourable friend
upon my right, himself among the greatest and most respected of that
great and much respected class, he would answer No!’

Point the second is this. The telling fact that Twemlow is related to
Lord Snigsworth, must be let off. Veneering supposes a state of public
affairs that probably never could by any possibility exist (though this
is not quite certain, in consequence of his picture being unintelligible
to himself and everybody else), and thus proceeds. ‘Why, gentlemen, if
I were to indicate such a programme to any class of society, I say it
would be received with derision, would be pointed at by the finger of
scorn. If I indicated such a programme to any worthy and intelligent
tradesman of your town--nay, I will here be personal, and say Our
town--what would he reply? He would reply, “Away with it!” That’s what
HE would reply, gentlemen. In his honest indignation he would reply,
“Away with it!” But suppose I mounted higher in the social scale.
Suppose I drew my arm through the arm of my respected friend upon my
left, and, walking with him through the ancestral woods of his family,
and under the spreading beeches of Snigsworthy Park, approached the
noble hall, crossed the courtyard, entered by the door, went up the
staircase, and, passing from room to room, found myself at last in
the august presence of my friend’s near kinsman, Lord Snigsworth. And
suppose I said to that venerable earl, “My Lord, I am here before your
lordship, presented by your lordship’s near kinsman, my friend upon my
left, to indicate that programme;” what would his lordship answer? Why,
he would answer, “Away with it!” That’s what he would answer, gentlemen.
“Away with it!” Unconsciously using, in his exalted sphere, the exact
language of the worthy and intelligent tradesman of our town, the near
and dear kinsman of my friend upon my left would answer in his wrath,
“Away with it!”’

Veneering finishes with this last success, and Mr Podsnap telegraphs to
Mrs Veneering, ‘He’s down.’

Then, dinner is had at the Hotel with the legal gentleman, and then
there are in due succession, nomination, and declaration. Finally Mr
Podsnap telegraphs to Mrs Veneering, ‘We have brought him in.’

Another gorgeous dinner awaits them on their return to the Veneering
halls, and Lady Tippins awaits them, and Boots and Brewer await
them. There is a modest assertion on everybody’s part that everybody
single-handed ‘brought him in’; but in the main it is conceded by all,
that that stroke of business on Brewer’s part, in going down to the
house that night to see how things looked, was the master-stroke.

A touching little incident is related by Mrs Veneering, in the course of
the evening. Mrs Veneering is habitually disposed to be tearful, and
has an extra disposition that way after her late excitement. Previous
to withdrawing from the dinner-table with Lady Tippins, she says, in a
pathetic and physically weak manner:

‘You will all think it foolish of me, I know, but I must mention it. As
I sat by Baby’s crib, on the night before the election, Baby was very
uneasy in her sleep.’

The Analytical chemist, who is gloomily looking on, has diabolical
impulses to suggest ‘Wind’ and throw up his situation; but represses
them.

‘After an interval almost convulsive, Baby curled her little hands in
one another and smiled.’

Mrs Veneering stopping here, Mr Podsnap deems it incumbent on him to
say: ‘I wonder why!’

‘Could it be, I asked myself,’ says Mrs Veneering, looking about her for
her pocket-handkerchief, ‘that the Fairies were telling Baby that her
papa would shortly be an M. P.?’

So overcome by the sentiment is Mrs Veneering, that they all get up
to make a clear stage for Veneering, who goes round the table to the
rescue, and bears her out backward, with her feet impressively scraping
the carpet: after remarking that her work has been too much for her
strength. Whether the fairies made any mention of the five thousand
pounds, and it disagreed with Baby, is not speculated upon.

Poor little Twemlow, quite done up, is touched, and still continues
touched after he is safely housed over the livery-stable yard in
Duke Street, Saint James’s. But there, upon his sofa, a tremendous
consideration breaks in upon the mild gentleman, putting all softer
considerations to the rout.

‘Gracious heavens! Now I have time to think of it, he never saw one of
his constituents in all his days, until we saw them together!’

After having paced the room in distress of mind, with his hand to his
forehead, the innocent Twemlow returns to his sofa and moans:

‘I shall either go distracted, or die, of this man. He comes upon me too
late in life. I am not strong enough to bear him!’



Chapter 4

CUPID PROMPTED


To use the cold language of the world, Mrs Alfred Lammle rapidly
improved the acquaintance of Miss Podsnap. To use the warm language of
Mrs Lammle, she and her sweet Georgiana soon became one: in heart, in
mind, in sentiment, in soul.

Whenever Georgiana could escape from the thraldom of Podsnappery; could
throw off the bedclothes of the custard-coloured phaeton, and get up;
could shrink out of the range of her mother’s rocking, and (so to speak)
rescue her poor little frosty toes from being rocked over; she repaired
to her friend, Mrs Alfred Lammle. Mrs Podsnap by no means objected. As
a consciously ‘splendid woman,’ accustomed to overhear herself so
denominated by elderly osteologists pursuing their studies in dinner
society, Mrs Podsnap could dispense with her daughter. Mr Podsnap, for
his part, on being informed where Georgiana was, swelled with patronage
of the Lammles. That they, when unable to lay hold of him, should
respectfully grasp at the hem of his mantle; that they, when they could
not bask in the glory of him the sun, should take up with the pale
reflected light of the watery young moon his daughter; appeared quite
natural, becoming, and proper. It gave him a better opinion of the
discretion of the Lammles than he had heretofore held, as showing that
they appreciated the value of the connexion. So, Georgiana repairing
to her friend, Mr Podsnap went out to dinner, and to dinner, and yet to
dinner, arm in arm with Mrs Podsnap: settling his obstinate head in his
cravat and shirt-collar, much as if he were performing on the Pandean
pipes, in his own honour, the triumphal march, See the conquering
Podsnap comes, Sound the trumpets, beat the drums!

It was a trait in Mr Podsnap’s character (and in one form or other
it will be generally seen to pervade the depths and shallows of
Podsnappery), that he could not endure a hint of disparagement of any
friend or acquaintance of his. ‘How dare you?’ he would seem to say, in
such a case. ‘What do you mean? I have licensed this person. This person
has taken out MY certificate. Through this person you strike at me,
Podsnap the Great. And it is not that I particularly care for the
person’s dignity, but that I do most particularly care for Podsnap’s.’
Hence, if any one in his presence had presumed to doubt the
responsibility of the Lammles, he would have been mightily huffed. Not
that any one did, for Veneering, M.P., was always the authority for
their being very rich, and perhaps believed it. As indeed he might, if
he chose, for anything he knew of the matter.

Mr and Mrs Lammle’s house in Sackville Street, Piccadilly, was but
a temporary residence. It has done well enough, they informed their
friends, for Mr Lammle when a bachelor, but it would not do now. So,
they were always looking at palatial residences in the best situations,
and always very nearly taking or buying one, but never quite concluding
the bargain. Hereby they made for themselves a shining little reputation
apart. People said, on seeing a vacant palatial residence, ‘The very
thing for the Lammles!’ and wrote to the Lammles about it, and the
Lammles always went to look at it, but unfortunately it never exactly
answered. In short, they suffered so many disappointments, that they
began to think it would be necessary to build a palatial residence.
And hereby they made another shining reputation; many persons of their
acquaintance becoming by anticipation dissatisfied with their own
houses, and envious of the non-existent Lammle structure.

The handsome fittings and furnishings of the house in Sackville Street
were piled thick and high over the skeleton up-stairs, and if it ever
whispered from under its load of upholstery, ‘Here I am in the closet!’
it was to very few ears, and certainly never to Miss Podsnap’s. What
Miss Podsnap was particularly charmed with, next to the graces of
her friend, was the happiness of her friend’s married life. This was
frequently their theme of conversation.

‘I am sure,’ said Miss Podsnap, ‘Mr Lammle is like a lover. At least
I--I should think he was.’

‘Georgiana, darling!’ said Mrs Lammle, holding up a forefinger, ‘Take
care!’

‘Oh my goodness me!’ exclaimed Miss Podsnap, reddening. ‘What have I
said now?’

‘Alfred, you know,’ hinted Mrs Lammle, playfully shaking her head. ‘You
were never to say Mr Lammle any more, Georgiana.’

‘Oh! Alfred, then. I am glad it’s no worse. I was afraid I had said
something shocking. I am always saying something wrong to ma.’

‘To me, Georgiana dearest?’

‘No, not to you; you are not ma. I wish you were.’

Mrs Lammle bestowed a sweet and loving smile upon her friend, which Miss
Podsnap returned as she best could. They sat at lunch in Mrs Lammle’s
own boudoir.

‘And so, dearest Georgiana, Alfred is like your notion of a lover?’

‘I don’t say that, Sophronia,’ Georgiana replied, beginning to conceal
her elbows. ‘I haven’t any notion of a lover. The dreadful wretches that
ma brings up at places to torment me, are not lovers. I only mean that
Mr--’

‘Again, dearest Georgiana?’

‘That Alfred--’

‘Sounds much better, darling.’

‘--Loves you so. He always treats you with such delicate gallantry and
attention. Now, don’t he?’

‘Truly, my dear,’ said Mrs Lammle, with a rather singular expression
crossing her face. ‘I believe that he loves me, fully as much as I love
him.’

‘Oh, what happiness!’ exclaimed Miss Podsnap.

‘But do you know, my Georgiana,’ Mrs Lammle resumed presently, ‘that
there is something suspicious in your enthusiastic sympathy with
Alfred’s tenderness?’

‘Good gracious no, I hope not!’

‘Doesn’t it rather suggest,’ said Mrs Lammle archly, ‘that my
Georgiana’s little heart is--’

‘Oh don’t!’ Miss Podsnap blushingly besought her. ‘Please don’t! I
assure you, Sophronia, that I only praise Alfred, because he is your
husband and so fond of you.’

Sophronia’s glance was as if a rather new light broke in upon her. It
shaded off into a cool smile, as she said, with her eyes upon her lunch,
and her eyebrows raised:

‘You are quite wrong, my love, in your guess at my meaning. What I
insinuated was, that my Georgiana’s little heart was growing conscious
of a vacancy.’

‘No, no, no,’ said Georgiana. ‘I wouldn’t have anybody say anything to
me in that way for I don’t know how many thousand pounds.’

‘In what way, my Georgiana?’ inquired Mrs Lammle, still smiling coolly
with her eyes upon her lunch, and her eyebrows raised.

‘YOU know,’ returned poor little Miss Podsnap. ‘I think I should go out
of my mind, Sophronia, with vexation and shyness and detestation, if
anybody did. It’s enough for me to see how loving you and your husband
are. That’s a different thing. I couldn’t bear to have anything of that
sort going on with myself. I should beg and pray to--to have the person
taken away and trampled upon.’

Ah! here was Alfred. Having stolen in unobserved, he playfully leaned on
the back of Sophronia’s chair, and, as Miss Podsnap saw him, put one
of Sophronia’s wandering locks to his lips, and waved a kiss from it
towards Miss Podsnap.

‘What is this about husbands and detestations?’ inquired the captivating
Alfred.

‘Why, they say,’ returned his wife, ‘that listeners never hear any good
of themselves; though you--but pray how long have you been here, sir?’

‘This instant arrived, my own.’

‘Then I may go on--though if you had been here but a moment or two
sooner, you would have heard your praises sounded by Georgiana.’

‘Only, if they were to be called praises at all which I really don’t
think they were,’ explained Miss Podsnap in a flutter, ‘for being so
devoted to Sophronia.’

‘Sophronia!’ murmured Alfred. ‘My life!’ and kissed her hand. In return
for which she kissed his watch-chain.

‘But it was not I who was to be taken away and trampled upon, I hope?’
said Alfred, drawing a seat between them.

‘Ask Georgiana, my soul,’ replied his wife.

Alfred touchingly appealed to Georgiana.

‘Oh, it was nobody,’ replied Miss Podsnap. ‘It was nonsense.’

‘But if you are determined to know, Mr Inquisitive Pet, as I suppose you
are,’ said the happy and fond Sophronia, smiling, ‘it was any one who
should venture to aspire to Georgiana.’

‘Sophronia, my love,’ remonstrated Mr Lammle, becoming graver, ‘you are
not serious?’

‘Alfred, my love,’ returned his wife, ‘I dare say Georgiana was not, but
I am.’

‘Now this,’ said Mr Lammle, ‘shows the accidental combinations that
there are in things! Could you believe, my Ownest, that I came in here
with the name of an aspirant to our Georgiana on my lips?’

‘Of course I could believe, Alfred,’ said Mrs Lammle, ‘anything that YOU
told me.’

‘You dear one! And I anything that YOU told me.’

How delightful those interchanges, and the looks accompanying them! Now,
if the skeleton up-stairs had taken that opportunity, for instance, of
calling out ‘Here I am, suffocating in the closet!’

‘I give you my honour, my dear Sophronia--’

‘And I know what that is, love,’ said she.

‘You do, my darling--that I came into the room all but uttering young
Fledgeby’s name. Tell Georgiana, dearest, about young Fledgeby.’

‘Oh no, don’t! Please don’t!’ cried Miss Podsnap, putting her fingers in
her ears. ‘I’d rather not.’

Mrs Lammle laughed in her gayest manner, and, removing her Georgiana’s
unresisting hands, and playfully holding them in her own at arms’
length, sometimes near together and sometimes wide apart, went on:

‘You must know, you dearly beloved little goose, that once upon a
time there was a certain person called young Fledgeby. And this young
Fledgeby, who was of an excellent family and rich, was known to two
other certain persons, dearly attached to one another and called Mr and
Mrs Alfred Lammle. So this young Fledgeby, being one night at the play,
there sees with Mr and Mrs Alfred Lammle, a certain heroine called--’

‘No, don’t say Georgiana Podsnap!’ pleaded that young lady almost in
tears. ‘Please don’t. Oh do do do say somebody else! Not Georgiana
Podsnap. Oh don’t, don’t, don’t!’

‘No other,’ said Mrs Lammle, laughing airily, and, full of affectionate
blandishments, opening and closing Georgiana’s arms like a pair of
compasses, ‘than my little Georgiana Podsnap. So this young Fledgeby goes
to that Alfred Lammle and says--’

‘Oh ple-e-e-ease don’t!’ Georgiana, as if the supplication were being
squeezed out of her by powerful compression. ‘I so hate him for saying
it!’

‘For saying what, my dear?’ laughed Mrs Lammle.

‘Oh, I don’t know what he said,’ cried Georgiana wildly, ‘but I hate him
all the same for saying it.’

‘My dear,’ said Mrs Lammle, always laughing in her most captivating way,
‘the poor young fellow only says that he is stricken all of a heap.’

‘Oh, what shall I ever do!’ interposed Georgiana. ‘Oh my goodness what a
Fool he must be!’

‘--And implores to be asked to dinner, and to make a fourth at the play
another time. And so he dines to-morrow and goes to the Opera with
us. That’s all. Except, my dear Georgiana--and what will you think of
this!--that he is infinitely shyer than you, and far more afraid of you
than you ever were of any one in all your days!’

In perturbation of mind Miss Podsnap still fumed and plucked at her
hands a little, but could not help laughing at the notion of anybody’s
being afraid of her. With that advantage, Sophronia flattered her and
rallied her more successfully, and then the insinuating Alfred flattered
her and rallied her, and promised that at any moment when she might
require that service at his hands, he would take young Fledgeby out and
trample on him. Thus it remained amicably understood that young Fledgeby
was to come to admire, and that Georgiana was to come to be admired; and
Georgiana with the entirely new sensation in her breast of having that
prospect before her, and with many kisses from her dear Sophronia in
present possession, preceded six feet one of discontented footman (an
amount of the article that always came for her when she walked home) to
her father’s dwelling.

The happy pair being left together, Mrs Lammle said to her husband:

‘If I understand this girl, sir, your dangerous fascinations have
produced some effect upon her. I mention the conquest in good time
because I apprehend your scheme to be more important to you than your
vanity.’

There was a mirror on the wall before them, and her eyes just caught
him smirking in it. She gave the reflected image a look of the deepest
disdain, and the image received it in the glass. Next moment they
quietly eyed each other, as if they, the principals, had had no part in
that expressive transaction.

It may have been that Mrs Lammle tried in some manner to excuse her
conduct to herself by depreciating the poor little victim of whom she
spoke with acrimonious contempt. It may have been too that in this she
did not quite succeed, for it is very difficult to resist confidence,
and she knew she had Georgiana’s.

Nothing more was said between the happy pair. Perhaps conspirators
who have once established an understanding, may not be over-fond of
repeating the terms and objects of their conspiracy. Next day came; came
Georgiana; and came Fledgeby.

Georgiana had by this time seen a good deal of the house and its
frequenters. As there was a certain handsome room with a billiard table
in it--on the ground floor, eating out a backyard--which might have
been Mr Lammle’s office, or library, but was called by neither name, but
simply Mr Lammle’s room, so it would have been hard for stronger female
heads than Georgiana’s to determine whether its frequenters were men
of pleasure or men of business. Between the room and the men there were
strong points of general resemblance. Both were too gaudy, too slangey,
too odorous of cigars, and too much given to horseflesh; the latter
characteristic being exemplified in the room by its decorations, and in
the men by their conversation. High-stepping horses seemed necessary to
all Mr Lammle’s friends--as necessary as their transaction of business
together in a gipsy way at untimely hours of the morning and evening,
and in rushes and snatches. There were friends who seemed to be always
coming and going across the Channel, on errands about the Bourse, and
Greek and Spanish and India and Mexican and par and premium and discount
and three quarters and seven eighths. There were other friends who
seemed to be always lolling and lounging in and out of the City, on
questions of the Bourse, and Greek and Spanish and India and Mexican and
par and premium and discount and three quarters and seven eighths. They
were all feverish, boastful, and indefinably loose; and they all ate and
drank a great deal; and made bets in eating and drinking. They all spoke
of sums of money, and only mentioned the sums and left the money to
be understood; as ‘five and forty thousand Tom,’ or ‘Two hundred and
twenty-two on every individual share in the lot Joe.’ They seemed to
divide the world into two classes of people; people who were making
enormous fortunes, and people who were being enormously ruined. They
were always in a hurry, and yet seemed to have nothing tangible to do;
except a few of them (these, mostly asthmatic and thick-lipped) who were
for ever demonstrating to the rest, with gold pencil-cases which they
could hardly hold because of the big rings on their forefingers, how
money was to be made. Lastly, they all swore at their grooms, and the
grooms were not quite as respectful or complete as other men’s grooms;
seeming somehow to fall short of the groom point as their masters fell
short of the gentleman point.

Young Fledgeby was none of these. Young Fledgeby had a peachy cheek,
or a cheek compounded of the peach and the red red red wall on which
it grows, and was an awkward, sandy-haired, small-eyed youth, exceeding
slim (his enemies would have said lanky), and prone to self-examination
in the articles of whisker and moustache. While feeling for the whisker
that he anxiously expected, Fledgeby underwent remarkable fluctuations
of spirits, ranging along the whole scale from confidence to despair.
There were times when he started, as exclaiming ‘By Jupiter here it is
at last!’ There were other times when, being equally depressed, he would
be seen to shake his head, and give up hope. To see him at those periods
leaning on a chimneypiece, like as on an urn containing the ashes of his
ambition, with the cheek that would not sprout, upon the hand on which
that cheek had forced conviction, was a distressing sight.

Not so was Fledgeby seen on this occasion. Arrayed in superb raiment,
with his opera hat under his arm, he concluded his self-examination
hopefully, awaited the arrival of Miss Podsnap, and talked small-talk
with Mrs Lammle. In facetious homage to the smallness of his talk, and
the jerky nature of his manners, Fledgeby’s familiars had agreed to
confer upon him (behind his back) the honorary title of Fascination
Fledgeby.

‘Warm weather, Mrs Lammle,’ said Fascination Fledgeby. Mrs Lammle
thought it scarcely as warm as it had been yesterday. ‘Perhaps not,’
said Fascination Fledgeby, with great quickness of repartee; ‘but I
expect it will be devilish warm to-morrow.’

He threw off another little scintillation. ‘Been out to-day, Mrs
Lammle?’

Mrs Lammle answered, for a short drive.

‘Some people,’ said Fascination Fledgeby, ‘are accustomed to take long
drives; but it generally appears to me that if they make ‘em too long,
they overdo it.’

Being in such feather, he might have surpassed himself in his next
sally, had not Miss Podsnap been announced. Mrs Lammle flew to embrace
her darling little Georgy, and when the first transports were over,
presented Mr Fledgeby. Mr Lammle came on the scene last, for he was
always late, and so were the frequenters always late; all hands being
bound to be made late, by private information about the Bourse, and
Greek and Spanish and India and Mexican and par and premium and discount
and three quarters and seven eighths.

A handsome little dinner was served immediately, and Mr Lammle sat
sparkling at his end of the table, with his servant behind his chair,
and HIS ever-lingering doubts upon the subject of his wages behind
himself. Mr Lammle’s utmost powers of sparkling were in requisition
to-day, for Fascination Fledgeby and Georgiana not only struck each
other speechless, but struck each other into astonishing attitudes;
Georgiana, as she sat facing Fledgeby, making such efforts to conceal
her elbows as were totally incompatible with the use of a knife and
fork; and Fledgeby, as he sat facing Georgiana, avoiding her countenance
by every possible device, and betraying the discomposure of his mind in
feeling for his whiskers with his spoon, his wine glass, and his bread.

So, Mr and Mrs Alfred Lammle had to prompt, and this is how they
prompted.

‘Georgiana,’ said Mr Lammle, low and smiling, and sparkling all over,
like a harlequin; ‘you are not in your usual spirits. Why are you not in
your usual spirits, Georgiana?’

Georgiana faltered that she was much the same as she was in general; she
was not aware of being different.

‘Not aware of being different!’ retorted Mr Alfred Lammle. ‘You, my dear
Georgiana! Who are always so natural and unconstrained with us! Who are
such a relief from the crowd that are all alike! Who are the embodiment
of gentleness, simplicity, and reality!’

Miss Podsnap looked at the door, as if she entertained confused thoughts
of taking refuge from these compliments in flight.

‘Now, I will be judged,’ said Mr Lammle, raising his voice a little, ‘by
my friend Fledgeby.’

‘Oh DON’T!’ Miss Podsnap faintly ejaculated: when Mrs Lammle took the
prompt-book.

‘I beg your pardon, Alfred, my dear, but I cannot part with Mr Fledgeby
quite yet; you must wait for him a moment. Mr Fledgeby and I are engaged
in a personal discussion.’

Fledgeby must have conducted it on his side with immense art, for no
appearance of uttering one syllable had escaped him.

‘A personal discussion, Sophronia, my love? What discussion? Fledgeby, I
am jealous. What discussion, Fledgeby?’

‘Shall I tell him, Mr Fledgeby?’ asked Mrs Lammle.

Trying to look as if he knew anything about it, Fascination replied,
‘Yes, tell him.’

‘We were discussing then,’ said Mrs Lammle, ‘if you MUST know, Alfred,
whether Mr Fledgeby was in his usual flow of spirits.’

‘Why, that is the very point, Sophronia, that Georgiana and I were
discussing as to herself! What did Fledgeby say?’

‘Oh, a likely thing, sir, that I am going to tell you everything, and be
told nothing! What did Georgiana say?’

‘Georgiana said she was doing her usual justice to herself to-day, and I
said she was not.’

‘Precisely,’ exclaimed Mrs Lammle, ‘what I said to Mr Fledgeby.’ Still,
it wouldn’t do. They would not look at one another. No, not even
when the sparkling host proposed that the quartette should take an
appropriately sparkling glass of wine. Georgiana looked from her wine
glass at Mr Lammle and at Mrs Lammle; but mightn’t, couldn’t, shouldn’t,
wouldn’t, look at Mr Fledgeby. Fascination looked from his wine glass
at Mrs Lammle and at Mr Lammle; but mightn’t, couldn’t, shouldn’t,
wouldn’t, look at Georgiana.

More prompting was necessary. Cupid must be brought up to the mark. The
manager had put him down in the bill for the part, and he must play it.

‘Sophronia, my dear,’ said Mr Lammle, ‘I don’t like the colour of your
dress.’

‘I appeal,’ said Mrs Lammle, ‘to Mr Fledgeby.’

‘And I,’ said Mr Lammle, ‘to Georgiana.’

‘Georgy, my love,’ remarked Mrs Lammle aside to her dear girl, ‘I rely
upon you not to go over to the opposition. Now, Mr Fledgeby.’

Fascination wished to know if the colour were not called rose-colour?
Yes, said Mr Lammle; actually he knew everything; it was really
rose-colour. Fascination took rose-colour to mean the colour of roses.
(In this he was very warmly supported by Mr and Mrs Lammle.) Fascination
had heard the term Queen of Flowers applied to the Rose. Similarly, it
might be said that the dress was the Queen of Dresses. [‘Very happy,
Fledgeby!’ from Mr Lammle.) Notwithstanding, Fascination’s opinion
was that we all had our eyes--or at least a large majority of us--and
that--and--and his farther opinion was several ands, with nothing beyond
them.

‘Oh, Mr Fledgeby,’ said Mrs Lammle, ‘to desert me in that way! Oh, Mr
Fledgeby, to abandon my poor dear injured rose and declare for blue!’

‘Victory, victory!’ cried Mr Lammle; ‘your dress is condemned, my dear.’

‘But what,’ said Mrs Lammle, stealing her affectionate hand towards her
dear girl’s, ‘what does Georgy say?’

‘She says,’ replied Mr Lammle, interpreting for her, ‘that in her eyes
you look well in any colour, Sophronia, and that if she had expected to
be embarrassed by so pretty a compliment as she has received, she would
have worn another colour herself. Though I tell her, in reply, that it
would not have saved her, for whatever colour she had worn would have
been Fledgeby’s colour. But what does Fledgeby say?’

‘He says,’ replied Mrs Lammle, interpreting for him, and patting the
back of her dear girl’s hand, as if it were Fledgeby who was patting it,
‘that it was no compliment, but a little natural act of homage that
he couldn’t resist. And,’ expressing more feeling as if it were more
feeling on the part of Fledgeby, ‘he is right, he is right!’

Still, no not even now, would they look at one another. Seeming to gnash
his sparkling teeth, studs, eyes, and buttons, all at once, Mr Lammle
secretly bent a dark frown on the two, expressive of an intense desire
to bring them together by knocking their heads together.

‘Have you heard this opera of to-night, Fledgeby?’ he asked, stopping
very short, to prevent himself from running on into ‘confound you.’

‘Why no, not exactly,’ said Fledgeby. ‘In fact I don’t know a note of
it.’

‘Neither do you know it, Georgy?’ said Mrs Lammle. ‘N-no,’ replied
Georgiana, faintly, under the sympathetic coincidence.

‘Why, then,’ said Mrs Lammle, charmed by the discovery which flowed from
the premises, ‘you neither of you know it! How charming!’

Even the craven Fledgeby felt that the time was now come when he must
strike a blow. He struck it by saying, partly to Mrs Lammle and partly
to the circumambient air, ‘I consider myself very fortunate in being
reserved by--’

As he stopped dead, Mr Lammle, making that gingerous bush of his
whiskers to look out of, offered him the word ‘Destiny.’

‘No, I wasn’t going to say that,’ said Fledgeby. ‘I was going to say
Fate. I consider it very fortunate that Fate has written in the book
of--in the book which is its own property--that I should go to that
opera for the first time under the memorable circumstances of going with
Miss Podsnap.’

To which Georgiana replied, hooking her two little fingers in one
another, and addressing the tablecloth, ‘Thank you, but I generally go
with no one but you, Sophronia, and I like that very much.’

Content perforce with this success for the time, Mr Lammle let Miss
Podsnap out of the room, as if he were opening her cage door, and Mrs
Lammle followed. Coffee being presently served up stairs, he kept a
watch on Fledgeby until Miss Podsnap’s cup was empty, and then directed
him with his finger (as if that young gentleman were a slow Retriever)
to go and fetch it. This feat he performed, not only without failure,
but even with the original embellishment of informing Miss Podsnap that
green tea was considered bad for the nerves. Though there Miss Podsnap
unintentionally threw him out by faltering, ‘Oh, is it indeed? How does
it act?’ Which he was not prepared to elucidate.

The carriage announced, Mrs Lammle said; ‘Don’t mind me, Mr Fledgeby, my
skirts and cloak occupy both my hands, take Miss Podsnap.’ And he
took her, and Mrs Lammle went next, and Mr Lammle went last, savagely
following his little flock, like a drover.

But he was all sparkle and glitter in the box at the Opera, and there he
and his dear wife made a conversation between Fledgeby and Georgiana in
the following ingenious and skilful manner. They sat in this order:
Mrs Lammle, Fascination Fledgeby, Georgiana, Mr Lammle. Mrs Lammle made
leading remarks to Fledgeby, only requiring monosyllabic replies. Mr
Lammle did the like with Georgiana. At times Mrs Lammle would lean
forward to address Mr Lammle to this purpose.

‘Alfred, my dear, Mr Fledgeby very justly says, apropos of the last
scene, that true constancy would not require any such stimulant as the
stage deems necessary.’ To which Mr Lammle would reply, ‘Ay, Sophronia,
my love, but as Georgiana has observed to me, the lady had no sufficient
reason to know the state of the gentleman’s affections.’ To which Mrs
Lammle would rejoin, ‘Very true, Alfred; but Mr Fledgeby points
out,’ this. To which Alfred would demur: ‘Undoubtedly, Sophronia, but
Georgiana acutely remarks,’ that. Through this device the two young
people conversed at great length and committed themselves to a variety
of delicate sentiments, without having once opened their lips, save to
say yes or no, and even that not to one another.

Fledgeby took his leave of Miss Podsnap at the carriage door, and the
Lammles dropped her at her own home, and on the way Mrs Lammle archly
rallied her, in her fond and protecting manner, by saying at intervals,
‘Oh little Georgiana, little Georgiana!’ Which was not much; but the
tone added, ‘You have enslaved your Fledgeby.’

And thus the Lammles got home at last, and the lady sat down moody and
weary, looking at her dark lord engaged in a deed of violence with a
bottle of soda-water as though he were wringing the neck of some unlucky
creature and pouring its blood down his throat. As he wiped his dripping
whiskers in an ogreish way, he met her eyes, and pausing, said, with no
very gentle voice:

‘Well?’

‘Was such an absolute Booby necessary to the purpose?’

‘I know what I am doing. He is no such dolt as you suppose.’

‘A genius, perhaps?’

‘You sneer, perhaps; and you take a lofty air upon yourself perhaps!
But I tell you this:--when that young fellow’s interest is concerned,
he holds as tight as a horse-leech. When money is in question with that
young fellow, he is a match for the Devil.’

‘Is he a match for you?’

‘He is. Almost as good a one as you thought me for you. He has no
quality of youth in him, but such as you have seen to-day. Touch him
upon money, and you touch no booby then. He really is a dolt, I suppose,
in other things; but it answers his one purpose very well.’

‘Has she money in her own right in any case?’

‘Ay! she has money in her own right in any case. You have done so well
to-day, Sophronia, that I answer the question, though you know I object
to any such questions. You have done so well to-day, Sophronia, that you
must be tired. Get to bed.’



Chapter 5

MERCURY PROMPTING


Fledgeby deserved Mr Alfred Lammle’s eulogium. He was the meanest
cur existing, with a single pair of legs. And instinct (a word we all
clearly understand) going largely on four legs, and reason always on
two, meanness on four legs never attains the perfection of meanness on
two.

The father of this young gentleman had been a money-lender, who
had transacted professional business with the mother of this
young gentleman, when he, the latter, was waiting in the vast dark
ante-chambers of the present world to be born. The lady, a widow, being
unable to pay the money-lender, married him; and in due course, Fledgeby
was summoned out of the vast dark ante-chambers to come and be presented
to the Registrar-General. Rather a curious speculation how Fledgeby
would otherwise have disposed of his leisure until Doomsday.

Fledgeby’s mother offended her family by marrying Fledgeby’s father. It
is one of the easiest achievements in life to offend your family when
your family want to get rid of you. Fledgeby’s mother’s family had
been very much offended with her for being poor, and broke with her
for becoming comparatively rich. Fledgeby’s mother’s family was the
Snigsworth family. She had even the high honour to be cousin to Lord
Snigsworth--so many times removed that the noble Earl would have had no
compunction in removing her one time more and dropping her clean outside
the cousinly pale; but cousin for all that.

Among her pre-matrimonial transactions with Fledgeby’s father,
Fledgeby’s mother had raised money of him at a great disadvantage on a
certain reversionary interest. The reversion falling in soon after they
were married, Fledgeby’s father laid hold of the cash for his separate
use and benefit. This led to subjective differences of opinion, not to
say objective interchanges of boot-jacks, backgammon boards, and other
such domestic missiles, between Fledgeby’s father and Fledgeby’s mother,
and those led to Fledgeby’s mother spending as much money as she
could, and to Fledgeby’s father doing all he couldn’t to restrain her.
Fledgeby’s childhood had been, in consequence, a stormy one; but the
winds and the waves had gone down in the grave, and Fledgeby flourished
alone.

He lived in chambers in the Albany, did Fledgeby, and maintained a
spruce appearance. But his youthful fire was all composed of sparks from
the grindstone; and as the sparks flew off, went out, and never warmed
anything, be sure that Fledgeby had his tools at the grindstone, and
turned it with a wary eye.

Mr Alfred Lammle came round to the Albany to breakfast with Fledgeby.
Present on the table, one scanty pot of tea, one scanty loaf, two scanty
pats of butter, two scanty rashers of bacon, two pitiful eggs, and an
abundance of handsome china bought a secondhand bargain.

‘What did you think of Georgiana?’ asked Mr Lammle.

‘Why, I’ll tell you,’ said Fledgeby, very deliberately.

‘Do, my boy.’

‘You misunderstand me,’ said Fledgeby. ‘I don’t mean I’ll tell you that.
I mean I’ll tell you something else.’

‘Tell me anything, old fellow!’

‘Ah, but there you misunderstand me again,’ said Fledgeby. ‘I mean I’ll
tell you nothing.’

Mr Lammle sparkled at him, but frowned at him too.

‘Look here,’ said Fledgeby. ‘You’re deep and you’re ready. Whether I am
deep or not, never mind. I am not ready. But I can do one thing, Lammle,
I can hold my tongue. And I intend always doing it.’

‘You are a long-headed fellow, Fledgeby.’

‘May be, or may not be. If I am a short-tongued fellow, it may amount to
the same thing. Now, Lammle, I am never going to answer questions.’

‘My dear fellow, it was the simplest question in the world.’

‘Never mind. It seemed so, but things are not always what they seem. I
saw a man examined as a witness in Westminster Hall. Questions put to
him seemed the simplest in the world, but turned out to be anything
rather than that, after he had answered ‘em. Very well. Then he should
have held his tongue. If he had held his tongue he would have kept out
of scrapes that he got into.’

‘If I had held my tongue, you would never have seen the subject of my
question,’ remarked Lammle, darkening.

‘Now, Lammle,’ said Fascination Fledgeby, calmly feeling for his
whisker, ‘it won’t do. I won’t be led on into a discussion. I can’t
manage a discussion. But I can manage to hold my tongue.’

‘Can?’ Mr Lammle fell back upon propitiation. ‘I should think you could!
Why, when these fellows of our acquaintance drink and you drink with
them, the more talkative they get, the more silent you get. The more
they let out, the more you keep in.’

‘I don’t object, Lammle,’ returned Fledgeby, with an internal chuckle,
‘to being understood, though I object to being questioned. That
certainly IS the way I do it.’

‘And when all the rest of us are discussing our ventures, none of us
ever know what a single venture of yours is!’

‘And none of you ever will from me, Lammle,’ replied Fledgeby, with
another internal chuckle; ‘that certainly IS the way I do it.’

‘Why of course it is, I know!’ rejoined Lammle, with a flourish of
frankness, and a laugh, and stretching out his hands as if to show
the universe a remarkable man in Fledgeby. ‘If I hadn’t known it of my
Fledgeby, should I have proposed our little compact of advantage, to my
Fledgeby?’

‘Ah!’ remarked Fascination, shaking his head slyly. ‘But I am not to
be got at in that way. I am not vain. That sort of vanity don’t pay,
Lammle. No, no, no. Compliments only make me hold my tongue the more.’

Alfred Lammle pushed his plate away (no great sacrifice under the
circumstances of there being so little in it), thrust his hands in his
pockets, leaned back in his chair, and contemplated Fledgeby in silence.
Then he slowly released his left hand from its pocket, and made that
bush of his whiskers, still contemplating him in silence. Then he slowly
broke silence, and slowly said: ‘What--the--Dev-il is this fellow about
this morning?’

‘Now, look here, Lammle,’ said Fascination Fledgeby, with the meanest
of twinkles in his meanest of eyes: which were too near together, by
the way: ‘look here, Lammle; I am very well aware that I didn’t show to
advantage last night, and that you and your wife--who, I consider, is
a very clever woman and an agreeable woman--did. I am not calculated to
show to advantage under that sort of circumstances. I know very well you
two did show to advantage, and managed capitally. But don’t you on that
account come talking to me as if I was your doll and puppet, because I
am not.

‘And all this,’ cried Alfred, after studying with a look the meanness
that was fain to have the meanest help, and yet was so mean as to turn
upon it: ‘all this because of one simple natural question!’

‘You should have waited till I thought proper to say something about it
of myself. I don’t like your coming over me with your Georgianas, as if
you was her proprietor and mine too.’

‘Well, when you are in the gracious mind to say anything about it of
yourself,’ retorted Lammle, ‘pray do.’

‘I have done it. I have said you managed capitally. You and your wife
both. If you’ll go on managing capitally, I’ll go on doing my part. Only
don’t crow.’

‘I crow!’ exclaimed Lammle, shrugging his shoulders.

‘Or,’ pursued the other--‘or take it in your head that people are your
puppets because they don’t come out to advantage at the particular
moments when you do, with the assistance of a very clever and agreeable
wife. All the rest keep on doing, and let Mrs Lammle keep on doing. Now,
I have held my tongue when I thought proper, and I have spoken when I
thought proper, and there’s an end of that. And now the question is,’
proceeded Fledgeby, with the greatest reluctance, ‘will you have another
egg?’

‘No, I won’t,’ said Lammle, shortly.

‘Perhaps you’re right and will find yourself better without it,’ replied
Fascination, in greatly improved spirits. ‘To ask you if you’ll have
another rasher would be unmeaning flattery, for it would make you
thirsty all day. Will you have some more bread and butter?’

‘No, I won’t,’ repeated Lammle.

‘Then I will,’ said Fascination. And it was not a mere retort for the
sound’s sake, but was a cheerful cogent consequence of the refusal; for
if Lammle had applied himself again to the loaf, it would have been so
heavily visited, in Fledgeby’s opinion, as to demand abstinence from
bread, on his part, for the remainder of that meal at least, if not for
the whole of the next.

Whether this young gentleman (for he was but three-and-twenty) combined
with the miserly vice of an old man, any of the open-handed vices of
a young one, was a moot point; so very honourably did he keep his own
counsel. He was sensible of the value of appearances as an investment,
and liked to dress well; but he drove a bargain for every moveable about
him, from the coat on his back to the china on his breakfast-table;
and every bargain by representing somebody’s ruin or somebody’s loss,
acquired a peculiar charm for him. It was a part of his avarice to take,
within narrow bounds, long odds at races; if he won, he drove harder
bargains; if he lost, he half starved himself until next time. Why money
should be so precious to an Ass too dull and mean to exchange it for any
other satisfaction, is strange; but there is no animal so sure to get
laden with it, as the Ass who sees nothing written on the face of the
earth and sky but the three letters L. S. D.--not Luxury, Sensuality,
Dissoluteness, which they often stand for, but the three dry letters.
Your concentrated Fox is seldom comparable to your concentrated Ass in
money-breeding.

Fascination Fledgeby feigned to be a young gentleman living on his
means, but was known secretly to be a kind of outlaw in the bill-broking
line, and to put money out at high interest in various ways. His circle
of familiar acquaintance, from Mr Lammle round, all had a touch of the
outlaw, as to their rovings in the merry greenwood of Jobbery Forest,
lying on the outskirts of the Share-Market and the Stock Exchange.

‘I suppose you, Lammle,’ said Fledgeby, eating his bread and butter,
‘always did go in for female society?’

‘Always,’ replied Lammle, glooming considerably under his late
treatment.

‘Came natural to you, eh?’ said Fledgeby.

‘The sex were pleased to like me, sir,’ said Lammle sulkily, but with
the air of a man who had not been able to help himself.

‘Made a pretty good thing of marrying, didn’t you?’ asked Fledgeby.

The other smiled (an ugly smile), and tapped one tap upon his nose.

‘My late governor made a mess of it,’ said Fledgeby. ‘But Geor--is the
right name Georgina or Georgiana?’

‘Georgiana.’

‘I was thinking yesterday, I didn’t know there was such a name. I
thought it must end in ina.’

‘Why?’

‘Why, you play--if you can--the Concertina, you know,’ replied
Fledgeby, meditating very slowly. ‘And you have--when you catch it--the
Scarlatina. And you can come down from a balloon in a parach--no you
can’t though. Well, say Georgeute--I mean Georgiana.’

‘You were going to remark of Georgiana--?’ Lammle moodily hinted, after
waiting in vain.

‘I was going to remark of Georgiana, sir,’ said Fledgeby, not at all
pleased to be reminded of his having forgotten it, ‘that she don’t seem
to be violent. Don’t seem to be of the pitching-in order.’

‘She has the gentleness of the dove, Mr Fledgeby.’

‘Of course you’ll say so,’ replied Fledgeby, sharpening, the moment his
interest was touched by another. ‘But you know, the real look-out is
this:--what I say, not what you say. I say having my late governor
and my late mother in my eye--that Georgiana don’t seem to be of the
pitching-in order.’

The respected Mr Lammle was a bully, by nature and by usual practice.
Perceiving, as Fledgeby’s affronts cumulated, that conciliation by no
means answered the purpose here, he now directed a scowling look
into Fledgeby’s small eyes for the effect of the opposite treatment.
Satisfied by what he saw there, he burst into a violent passion and
struck his hand upon the table, making the china ring and dance.

‘You are a very offensive fellow, sir,’ cried Mr Lammle, rising. ‘You
are a highly offensive scoundrel. What do you mean by this behaviour?’

‘I say!’ remonstrated Fledgeby. ‘Don’t break out.’

‘You are a very offensive fellow sir,’ repeated Mr Lammle. ‘You are a
highly offensive scoundrel!’

‘I SAY, you know!’ urged Fledgeby, quailing.

‘Why, you coarse and vulgar vagabond!’ said Mr Lammle, looking fiercely
about him, ‘if your servant was here to give me sixpence of your
money to get my boots cleaned afterwards--for you are not worth the
expenditure--I’d kick you.’

‘No you wouldn’t,’ pleaded Fledgeby. ‘I am sure you’d think better of
it.’

‘I tell you what, Mr Fledgeby,’ said Lammle advancing on him. ‘Since
you presume to contradict me, I’ll assert myself a little. Give me your
nose!’

Fledgeby covered it with his hand instead, and said, retreating, ‘I beg
you won’t!’

‘Give me your nose, sir,’ repeated Lammle.

Still covering that feature and backing, Mr Fledgeby reiterated
(apparently with a severe cold in his head), ‘I beg, I beg, you won’t.’

‘And this fellow,’ exclaimed Lammle, stopping and making the most of his
chest--‘This fellow presumes on my having selected him out of all the
young fellows I know, for an advantageous opportunity! This fellow
presumes on my having in my desk round the corner, his dirty note of
hand for a wretched sum payable on the occurrence of a certain event,
which event can only be of my and my wife’s bringing about! This fellow,
Fledgeby, presumes to be impertinent to me, Lammle. Give me your nose
sir!’

‘No! Stop! I beg your pardon,’ said Fledgeby, with humility.

‘What do you say, sir?’ demanded Mr Lammle, seeming too furious to
understand.

‘I beg your pardon,’ repeated Fledgeby.

‘Repeat your words louder, sir. The just indignation of a gentleman has
sent the blood boiling to my head. I don’t hear you.’

‘I say,’ repeated Fledgeby, with laborious explanatory politeness, ‘I
beg your pardon.’

Mr Lammle paused. ‘As a man of honour,’ said he, throwing himself into a
chair, ‘I am disarmed.’

Mr Fledgeby also took a chair, though less demonstratively, and by
slow approaches removed his hand from his nose. Some natural diffidence
assailed him as to blowing it, so shortly after its having assumed a
personal and delicate, not to say public, character; but he overcame
his scruples by degrees, and modestly took that liberty under an implied
protest.

‘Lammle,’ he said sneakingly, when that was done, ‘I hope we are friends
again?’

‘Mr Fledgeby,’ returned Lammle, ‘say no more.’

‘I must have gone too far in making myself disagreeable,’ said Fledgeby,
‘but I never intended it.’

‘Say no more, say no more!’ Mr Lammle repeated in a magnificent tone.
‘Give me your’--Fledgeby started--‘hand.’

They shook hands, and on Mr Lammle’s part, in particular, there ensued
great geniality. For, he was quite as much of a dastard as the other,
and had been in equal danger of falling into the second place for good,
when he took heart just in time, to act upon the information conveyed to
him by Fledgeby’s eye.

The breakfast ended in a perfect understanding. Incessant machinations
were to be kept at work by Mr and Mrs Lammle; love was to be made for
Fledgeby, and conquest was to be insured to him; he on his part
very humbly admitting his defects as to the softer social arts, and
entreating to be backed to the utmost by his two able coadjutors.

Little recked Mr Podsnap of the traps and toils besetting his Young
Person. He regarded her as safe within the Temple of Podsnappery, hiding
the fulness of time when she, Georgiana, should take him, Fitz-Podsnap,
who with all his worldly goods should her endow. It would call a blush
into the cheek of his standard Young Person to have anything to do with
such matters save to take as directed, and with worldly goods as per
settlement to be endowed. Who giveth this woman to be married to this
man? I, Podsnap. Perish the daring thought that any smaller creation
should come between!

It was a public holiday, and Fledgeby did not recover his spirits or his
usual temperature of nose until the afternoon. Walking into the City in
the holiday afternoon, he walked against a living stream setting out of
it; and thus, when he turned into the precincts of St Mary Axe, he found
a prevalent repose and quiet there. A yellow overhanging plaster-fronted
house at which he stopped was quiet too. The blinds were all drawn down,
and the inscription Pubsey and Co. seemed to doze in the counting-house
window on the ground-floor giving on the sleepy street.

Fledgeby knocked and rang, and Fledgeby rang and knocked, but no
one came. Fledgeby crossed the narrow street and looked up at the
house-windows, but nobody looked down at Fledgeby. He got out of temper,
crossed the narrow street again, and pulled the housebell as if it were
the house’s nose, and he were taking a hint from his late experience.
His ear at the keyhole seemed then, at last, to give him assurance that
something stirred within. His eye at the keyhole seemed to confirm his
ear, for he angrily pulled the house’s nose again, and pulled and pulled
and continued to pull, until a human nose appeared in the dark doorway.

‘Now you sir!’ cried Fledgeby. ‘These are nice games!’

He addressed an old Jewish man in an ancient coat, long of skirt, and
wide of pocket. A venerable man, bald and shining at the top of his
head, and with long grey hair flowing down at its sides and mingling
with his beard. A man who with a graceful Eastern action of homage bent
his head, and stretched out his hands with the palms downward, as if to
deprecate the wrath of a superior.

‘What have you been up to?’ said Fledgeby, storming at him.

‘Generous Christian master,’ urged the Jewish man, ‘it being holiday, I
looked for no one.’

‘Holiday he blowed!’ said Fledgeby, entering. ‘What have YOU got to do
with holidays? Shut the door.’

With his former action the old man obeyed. In the entry hung his rusty
large-brimmed low-crowned hat, as long out of date as his coat; in the
corner near it stood his staff--no walking-stick but a veritable staff.
Fledgeby turned into the counting-house, perched himself on a business
stool, and cocked his hat. There were light boxes on shelves in the
counting-house, and strings of mock beads hanging up. There were samples
of cheap clocks, and samples of cheap vases of flowers. Foreign toys,
all.

Perched on the stool with his hat cocked on his head and one of his legs
dangling, the youth of Fledgeby hardly contrasted to advantage with the
age of the Jewish man as he stood with his bare head bowed, and his eyes
(which he only raised in speaking) on the ground. His clothing was worn
down to the rusty hue of the hat in the entry, but though he looked
shabby he did not look mean. Now, Fledgeby, though not shabby, did look
mean.

‘You have not told me what you were up to, you sir,’ said Fledgeby,
scratching his head with the brim of his hat.

‘Sir, I was breathing the air.’

‘In the cellar, that you didn’t hear?’

‘On the house-top.’

‘Upon my soul! That’s a way of doing business.’

‘Sir,’ the old man represented with a grave and patient air, ‘there must
be two parties to the transaction of business, and the holiday has left
me alone.’

‘Ah! Can’t be buyer and seller too. That’s what the Jews say; ain’t it?’

‘At least we say truly, if we say so,’ answered the old man with a
smile.

‘Your people need speak the truth sometimes, for they lie enough,’
remarked Fascination Fledgeby.

‘Sir, there is,’ returned the old man with quiet emphasis, ‘too much
untruth among all denominations of men.’

Rather dashed, Fascination Fledgeby took another scratch at his
intellectual head with his hat, to gain time for rallying.

‘For instance,’ he resumed, as though it were he who had spoken last,
‘who but you and I ever heard of a poor Jew?’

‘The Jews,’ said the old man, raising his eyes from the ground with his
former smile. ‘They hear of poor Jews often, and are very good to them.’

‘Bother that!’ returned Fledgeby. ‘You know what I mean. You’d persuade
me if you could, that you are a poor Jew. I wish you’d confess how much
you really did make out of my late governor. I should have a better
opinion of you.’

The old man only bent his head, and stretched out his hands as before.

‘Don’t go on posturing like a Deaf and Dumb School,’ said the ingenious
Fledgeby, ‘but express yourself like a Christian--or as nearly as you
can.’

‘I had had sickness and misfortunes, and was so poor,’ said the old
man, ‘as hopelessly to owe the father, principal and interest. The son
inheriting, was so merciful as to forgive me both, and place me here.’

He made a little gesture as though he kissed the hem of an imaginary
garment worn by the noble youth before him. It was humbly done, but
picturesquely, and was not abasing to the doer.

‘You won’t say more, I see,’ said Fledgeby, looking at him as if he
would like to try the effect of extracting a double-tooth or two, ‘and
so it’s of no use my putting it to you. But confess this, Riah; who
believes you to be poor now?’

‘No one,’ said the old man.

‘There you’re right,’ assented Fledgeby.

‘No one,’ repeated the old man with a grave slow wave of his head. ‘All
scout it as a fable. Were I to say “This little fancy business is not
mine”;’ with a lithe sweep of his easily-turning hand around him,
to comprehend the various objects on the shelves; ‘“it is the little
business of a Christian young gentleman who places me, his servant, in
trust and charge here, and to whom I am accountable for every single
bead,” they would laugh. When, in the larger money-business, I tell the
borrowers--’

‘I say, old chap!’ interposed Fledgeby, ‘I hope you mind what you DO
tell ‘em?’

‘Sir, I tell them no more than I am about to repeat. When I tell them,
“I cannot promise this, I cannot answer for the other, I must see my
principal, I have not the money, I am a poor man and it does not rest
with me,” they are so unbelieving and so impatient, that they sometimes
curse me in Jehovah’s name.’

‘That’s deuced good, that is!’ said Fascination Fledgeby.

‘And at other times they say, “Can it never be done without these
tricks, Mr Riah? Come, come, Mr Riah, we know the arts of your
people”--my people!--“If the money is to be lent, fetch it, fetch it; if
it is not to be lent, keep it and say so.” They never believe me.’

‘THAT’S all right,’ said Fascination Fledgeby.

‘They say, “We know, Mr Riah, we know. We have but to look at you, and
we know.”’

‘Oh, a good ‘un are you for the post,’ thought Fledgeby, ‘and a good ‘un
was I to mark you out for it! I may be slow, but I am precious sure.’

Not a syllable of this reflection shaped itself in any scrap of Mr
Fledgeby’s breath, lest it should tend to put his servant’s price up.
But looking at the old man as he stood quiet with his head bowed and his
eyes cast down, he felt that to relinquish an inch of his baldness,
an inch of his grey hair, an inch of his coat-skirt, an inch of his
hat-brim, an inch of his walking-staff, would be to relinquish hundreds
of pounds.

‘Look here, Riah,’ said Fledgeby, mollified by these self-approving
considerations. ‘I want to go a little more into buying-up queer bills.
Look out in that direction.’

‘Sir, it shall be done.’

‘Casting my eye over the accounts, I find that branch of business pays
pretty fairly, and I am game for extending it. I like to know people’s
affairs likewise. So look out.’

‘Sir, I will, promptly.’

‘Put it about in the right quarters, that you’ll buy queer bills by the
lump--by the pound weight if that’s all--supposing you see your way to a
fair chance on looking over the parcel. And there’s one thing more. Come
to me with the books for periodical inspection as usual, at eight on
Monday morning.’

Riah drew some folding tablets from his breast and noted it down.

‘That’s all I wanted to say at the present time,’ continued Fledgeby in
a grudging vein, as he got off the stool, ‘except that I wish you’d take
the air where you can hear the bell, or the knocker, either one of the
two or both. By-the-by how DO you take the air at the top of the house?
Do you stick your head out of a chimney-pot?’

‘Sir, there are leads there, and I have made a little garden there.’

‘To bury your money in, you old dodger?’

‘A thumbnail’s space of garden would hold the treasure I bury, master,’
said Riah. ‘Twelve shillings a week, even when they are an old man’s
wages, bury themselves.’

‘I should like to know what you really are worth,’ returned Fledgeby,
with whom his growing rich on that stipend and gratitude was a very
convenient fiction. ‘But come! Let’s have a look at your garden on the
tiles, before I go!’

The old man took a step back, and hesitated.

‘Truly, sir, I have company there.’

‘Have you, by George!’ said Fledgeby; ‘I suppose you happen to know
whose premises these are?’

‘Sir, they are yours, and I am your servant in them.’

‘Oh! I thought you might have overlooked that,’ retorted Fledgeby, with
his eyes on Riah’s beard as he felt for his own; ‘having company on my
premises, you know!’

‘Come up and see the guests, sir. I hope for your admission that they
can do no harm.’

Passing him with a courteous reverence, specially unlike any action that
Mr Fledgeby could for his life have imparted to his own head and hands,
the old man began to ascend the stairs. As he toiled on before, with his
palm upon the stair-rail, and his long black skirt, a very gaberdine,
overhanging each successive step, he might have been the leader in some
pilgrimage of devotional ascent to a prophet’s tomb. Not troubled by any
such weak imagining, Fascination Fledgeby merely speculated on the time
of life at which his beard had begun, and thought once more what a good
‘un he was for the part.

Some final wooden steps conducted them, stooping under a low penthouse
roof, to the house-top. Riah stood still, and, turning to his master,
pointed out his guests.

Lizzie Hexam and Jenny Wren. For whom, perhaps with some old instinct of
his race, the gentle Jew had spread a carpet. Seated on it, against
no more romantic object than a blackened chimney-stack over which some
bumble creeper had been trained, they both pored over one book; both
with attentive faces; Jenny with the sharper; Lizzie with the more
perplexed. Another little book or two were lying near, and a common
basket of common fruit, and another basket full of strings of beads and
tinsel scraps. A few boxes of humble flowers and evergreens completed
the garden; and the encompassing wilderness of dowager old chimneys
twirled their cowls and fluttered their smoke, rather as if they were
bridling, and fanning themselves, and looking on in a state of airy
surprise.

Taking her eyes off the book, to test her memory of something in it,
Lizzie was the first to see herself observed. As she rose, Miss Wren
likewise became conscious, and said, irreverently addressing the great
chief of the premises: ‘Whoever you are, I can’t get up, because my
back’s bad and my legs are queer.’

‘This is my master,’ said Riah, stepping forward.

[‘Don’t look like anybody’s master,’ observed Miss Wren to herself, with
a hitch of her chin and eyes.)

‘This, sir,’ pursued the old man, ‘is a little dressmaker for little
people. Explain to the master, Jenny.’

‘Dolls; that’s all,’ said Jenny, shortly. ‘Very difficult to fit too,
because their figures are so uncertain. You never know where to expect
their waists.’

‘Her friend,’ resumed the old man, motioning towards Lizzie; ‘and as
industrious as virtuous. But that they both are. They are busy early and
late, sir, early and late; and in bye-times, as on this holiday, they go
to book-learning.’

‘Not much good to be got out of that,’ remarked Fledgeby.

‘Depends upon the person!’ quoth Miss Wren, snapping him up.

‘I made acquaintance with my guests, sir,’ pursued the Jew, with an
evident purpose of drawing out the dressmaker, ‘through their coming
here to buy of our damage and waste for Miss Jenny’s millinery. Our
waste goes into the best of company, sir, on her rosy-cheeked little
customers. They wear it in their hair, and on their ball-dresses, and
even (so she tells me) are presented at Court with it.’

‘Ah!’ said Fledgeby, on whose intelligence this doll-fancy made rather
strong demands; ‘she’s been buying that basketful to-day, I suppose?’

‘I suppose she has,’ Miss Jenny interposed; ‘and paying for it too, most
likely!’

‘Let’s have a look at it,’ said the suspicious chief. Riah handed it to
him. ‘How much for this now?’

‘Two precious silver shillings,’ said Miss Wren.

Riah confirmed her with two nods, as Fledgeby looked to him. A nod for
each shilling.

‘Well,’ said Fledgeby, poking into the contents of the basket with his
forefinger, ‘the price is not so bad. You have got good measure, Miss
What-is-it.’

‘Try Jenny,’ suggested that young lady with great calmness.

‘You have got good measure, Miss Jenny; but the price is not so
bad.--And you,’ said Fledgeby, turning to the other visitor, ‘do you buy
anything here, miss?’

‘No, sir.’

‘Nor sell anything neither, miss?’

‘No, sir.’

Looking askew at the questioner, Jenny stole her hand up to her
friend’s, and drew her friend down, so that she bent beside her on her
knee.

‘We are thankful to come here for rest, sir,’ said Jenny. ‘You see, you
don’t know what the rest of this place is to us; does he, Lizzie? It’s
the quiet, and the air.’

‘The quiet!’ repeated Fledgeby, with a contemptuous turn of his head
towards the City’s roar. ‘And the air!’ with a ‘Poof!’ at the smoke.

‘Ah!’ said Jenny. ‘But it’s so high. And you see the clouds rushing
on above the narrow streets, not minding them, and you see the golden
arrows pointing at the mountains in the sky from which the wind comes,
and you feel as if you were dead.’

The little creature looked above her, holding up her slight transparent
hand.

‘How do you feel when you are dead?’ asked Fledgeby, much perplexed.

‘Oh, so tranquil!’ cried the little creature, smiling. ‘Oh, so peaceful
and so thankful! And you hear the people who are alive, crying, and
working, and calling to one another down in the close dark streets, and
you seem to pity them so! And such a chain has fallen from you, and such
a strange good sorrowful happiness comes upon you!’

Her eyes fell on the old man, who, with his hands folded, quietly looked
on.

‘Why it was only just now,’ said the little creature, pointing at him,
‘that I fancied I saw him come out of his grave! He toiled out at
that low door so bent and worn, and then he took his breath and stood
upright, and looked all round him at the sky, and the wind blew upon
him, and his life down in the dark was over!--Till he was called back
to life,’ she added, looking round at Fledgeby with that lower look of
sharpness. ‘Why did you call him back?’

‘He was long enough coming, anyhow,’ grumbled Fledgeby.

‘But you are not dead, you know,’ said Jenny Wren. ‘Get down to life!’

Mr Fledgeby seemed to think it rather a good suggestion, and with a nod
turned round. As Riah followed to attend him down the stairs, the little
creature called out to the Jew in a silvery tone, ‘Don’t be long gone.
Come back, and be dead!’ And still as they went down they heard the
little sweet voice, more and more faintly, half calling and half
singing, ‘Come back and be dead, Come back and be dead!’

When they got down into the entry, Fledgeby, pausing under the shadow of
the broad old hat, and mechanically poising the staff, said to the old
man:

‘That’s a handsome girl, that one in her senses.’

‘And as good as handsome,’ answered Riah.

‘At all events,’ observed Fledgeby, with a dry whistle, ‘I hope she
ain’t bad enough to put any chap up to the fastenings, and get the
premises broken open. You look out. Keep your weather eye awake and
don’t make any more acquaintances, however handsome. Of course you
always keep my name to yourself?’

‘Sir, assuredly I do.’

‘If they ask it, say it’s Pubsey, or say it’s Co, or say it’s anything
you like, but what it is.’

His grateful servant--in whose race gratitude is deep, strong, and
enduring--bowed his head, and actually did now put the hem of his coat
to his lips: though so lightly that the wearer knew nothing of it.

Thus, Fascination Fledgeby went his way, exulting in the artful
cleverness with which he had turned his thumb down on a Jew, and the old
man went his different way up-stairs. As he mounted, the call or song
began to sound in his ears again, and, looking above, he saw the face
of the little creature looking down out of a Glory of her long bright
radiant hair, and musically repeating to him, like a vision:

‘Come up and be dead! Come up and be dead!’



Chapter 6

A RIDDLE WITHOUT AN ANSWER


Again Mr Mortimer Lightwood and Mr Eugene Wrayburn sat together in the
Temple. This evening, however, they were not together in the place of
business of the eminent solicitor, but in another dismal set of
chambers facing it on the same second-floor; on whose dungeon-like black
outer-door appeared the legend:

		PRIVATE

		MR EUGENE WRAYBURN

		MR MORTIMER LIGHTWOOD

		(Mr Lightwood’s Offices opposite.)

Appearances indicated that this establishment was a very recent
institution. The white letters of the inscription were extremely white
and extremely strong to the sense of smell, the complexion of the
tables and chairs was (like Lady Tippins’s) a little too blooming to
be believed in, and the carpets and floorcloth seemed to rush at the
beholder’s face in the unusual prominency of their patterns. But the
Temple, accustomed to tone down both the still life and the human life
that has much to do with it, would soon get the better of all that.

‘Well!’ said Eugene, on one side of the fire, ‘I feel tolerably
comfortable. I hope the upholsterer may do the same.’

‘Why shouldn’t he?’ asked Lightwood, from the other side of the fire.

‘To be sure,’ pursued Eugene, reflecting, ‘he is not in the secret of
our pecuniary affairs, so perhaps he may be in an easy frame of mind.’

‘We shall pay him,’ said Mortimer.

‘Shall we, really?’ returned Eugene, indolently surprised. ‘You don’t
say so!’

‘I mean to pay him, Eugene, for my part,’ said Mortimer, in a slightly
injured tone.

‘Ah! I mean to pay him too,’ retorted Eugene. ‘But then I mean so much
that I--that I don’t mean.’

‘Don’t mean?’

‘So much that I only mean and shall always only mean and nothing more,
my dear Mortimer. It’s the same thing.’

His friend, lying back in his easy chair, watched him lying back in his
easy chair, as he stretched out his legs on the hearth-rug, and said,
with the amused look that Eugene Wrayburn could always awaken in him
without seeming to try or care:

‘Anyhow, your vagaries have increased the bill.’

‘Calls the domestic virtues vagaries!’ exclaimed Eugene, raising his
eyes to the ceiling.

‘This very complete little kitchen of ours,’ said Mortimer, ‘in which
nothing will ever be cooked--’

‘My dear, dear Mortimer,’ returned his friend, lazily lifting his head
a little to look at him, ‘how often have I pointed out to you that its
moral influence is the important thing?’

‘Its moral influence on this fellow!’ exclaimed Lightwood, laughing.

‘Do me the favour,’ said Eugene, getting out of his chair with much
gravity, ‘to come and inspect that feature of our establishment which
you rashly disparage.’ With that, taking up a candle, he conducted
his chum into the fourth room of the set of chambers--a little narrow
room--which was very completely and neatly fitted as a kitchen. ‘See!’
said Eugene, ‘miniature flour-barrel, rolling-pin, spice-box, shelf of
brown jars, chopping-board, coffee-mill, dresser elegantly furnished
with crockery, saucepans and pans, roasting jack, a charming kettle, an
armoury of dish-covers. The moral influence of these objects, in forming
the domestic virtues, may have an immense influence upon me; not upon
you, for you are a hopeless case, but upon me. In fact, I have an idea
that I feel the domestic virtues already forming. Do me the favour to
step into my bedroom. Secretaire, you see, and abstruse set of solid
mahogany pigeon-holes, one for every letter of the alphabet. To what use
do I devote them? I receive a bill--say from Jones. I docket it neatly
at the secretaire, JONES, and I put it into pigeonhole J. It’s the next
thing to a receipt and is quite as satisfactory to ME. And I very much
wish, Mortimer,’ sitting on his bed, with the air of a philosopher
lecturing a disciple, ‘that my example might induce YOU to cultivate
habits of punctuality and method; and, by means of the moral influences
with which I have surrounded you, to encourage the formation of the
domestic virtues.’

Mortimer laughed again, with his usual commentaries of ‘How CAN you be
so ridiculous, Eugene!’ and ‘What an absurd fellow you are!’ but when
his laugh was out, there was something serious, if not anxious, in his
face. Despite that pernicious assumption of lassitude and indifference,
which had become his second nature, he was strongly attached to his
friend. He had founded himself upon Eugene when they were yet boys at
school; and at this hour imitated him no less, admired him no less,
loved him no less, than in those departed days.

‘Eugene,’ said he, ‘if I could find you in earnest for a minute, I would
try to say an earnest word to you.’

‘An earnest word?’ repeated Eugene. ‘The moral influences are beginning
to work. Say on.’

‘Well, I will,’ returned the other, ‘though you are not earnest yet.’

‘In this desire for earnestness,’ murmured Eugene, with the air of one
who was meditating deeply, ‘I trace the happy influences of the little
flour-barrel and the coffee-mill. Gratifying.’

‘Eugene,’ resumed Mortimer, disregarding the light interruption, and
laying a hand upon Eugene’s shoulder, as he, Mortimer, stood before him
seated on his bed, ‘you are withholding something from me.’

Eugene looked at him, but said nothing.

‘All this past summer, you have been withholding something from me.
Before we entered on our boating vacation, you were as bent upon it as I
have seen you upon anything since we first rowed together. But you cared
very little for it when it came, often found it a tie and a drag upon
you, and were constantly away. Now it was well enough half-a-dozen
times, a dozen times, twenty times, to say to me in your own odd manner,
which I know so well and like so much, that your disappearances were
precautions against our boring one another; but of course after a short
while I began to know that they covered something. I don’t ask what it
is, as you have not told me; but the fact is so. Say, is it not?’

‘I give you my word of honour, Mortimer,’ returned Eugene, after a
serious pause of a few moments, ‘that I don’t know.’

‘Don’t know, Eugene?’

‘Upon my soul, don’t know. I know less about myself than about most
people in the world, and I don’t know.’

‘You have some design in your mind?’

‘Have I? I don’t think I have.’

‘At any rate, you have some subject of interest there which used not to
be there?’

‘I really can’t say,’ replied Eugene, shaking his head blankly, after
pausing again to reconsider. ‘At times I have thought yes; at other
times I have thought no. Now, I have been inclined to pursue such a
subject; now I have felt that it was absurd, and that it tired and
embarrassed me. Absolutely, I can’t say. Frankly and faithfully, I would
if I could.’

So replying, he clapped a hand, in his turn, on his friend’s shoulder,
as he rose from his seat upon the bed, and said:

‘You must take your friend as he is. You know what I am, my dear
Mortimer. You know how dreadfully susceptible I am to boredom. You know
that when I became enough of a man to find myself an embodied conundrum,
I bored myself to the last degree by trying to find out what I meant.
You know that at length I gave it up, and declined to guess any more.
Then how can I possibly give you the answer that I have not discovered?
The old nursery form runs, “Riddle-me-riddle-me-ree, p’raps you can’t
tell me what this may be?” My reply runs, “No. Upon my life, I can’t.”’

So much of what was fantastically true to his own knowledge of this
utterly careless Eugene, mingled with the answer, that Mortimer could
not receive it as a mere evasion. Besides, it was given with an engaging
air of openness, and of special exemption of the one friend he valued,
from his reckless indifference.

‘Come, dear boy!’ said Eugene. ‘Let us try the effect of smoking. If it
enlightens me at all on this question, I will impart unreservedly.’

They returned to the room they had come from, and, finding it heated,
opened a window. Having lighted their cigars, they leaned out of this
window, smoking, and looking down at the moonlight, as it shone into the
court below.

‘No enlightenment,’ resumed Eugene, after certain minutes of silence. ‘I
feel sincerely apologetic, my dear Mortimer, but nothing comes.’

‘If nothing comes,’ returned Mortimer, ‘nothing can come from it. So
I shall hope that this may hold good throughout, and that there may be
nothing on foot. Nothing injurious to you, Eugene, or--’

Eugene stayed him for a moment with his hand on his arm, while he took a
piece of earth from an old flowerpot on the window-sill and dexterously
shot it at a little point of light opposite; having done which to his
satisfaction, he said, ‘Or?’

‘Or injurious to any one else.’

‘How,’ said Eugene, taking another little piece of earth, and shooting
it with great precision at the former mark, ‘how injurious to any one
else?’

‘I don’t know.’

‘And,’ said Eugene, taking, as he said the word, another shot, ‘to whom
else?’

‘I don’t know.’

Checking himself with another piece of earth in his hand, Eugene looked
at his friend inquiringly and a little suspiciously. There was no
concealed or half-expressed meaning in his face.

‘Two belated wanderers in the mazes of the law,’ said Eugene, attracted
by the sound of footsteps, and glancing down as he spoke, ‘stray into
the court. They examine the door-posts of number one, seeking the name
they want. Not finding it at number one, they come to number two. On the
hat of wanderer number two, the shorter one, I drop this pellet. Hitting
him on the hat, I smoke serenely, and become absorbed in contemplation
of the sky.’

Both the wanderers looked up towards the window; but, after
interchanging a mutter or two, soon applied themselves to the door-posts
below. There they seemed to discover what they wanted, for they
disappeared from view by entering at the doorway. ‘When they emerge,’
said Eugene, ‘you shall see me bring them both down’; and so prepared
two pellets for the purpose.

He had not reckoned on their seeking his name, or Lightwood’s. But
either the one or the other would seem to be in question, for now there
came a knock at the door. ‘I am on duty to-night,’ said Mortimer, ‘stay
you where you are, Eugene.’ Requiring no persuasion, he stayed there,
smoking quietly, and not at all curious to know who knocked, until
Mortimer spoke to him from within the room, and touched him. Then,
drawing in his head, he found the visitors to be young Charley Hexam
and the schoolmaster; both standing facing him, and both recognized at a
glance.

‘You recollect this young fellow, Eugene?’ said Mortimer.

‘Let me look at him,’ returned Wrayburn, coolly. ‘Oh, yes, yes. I
recollect him!’

He had not been about to repeat that former action of taking him by the
chin, but the boy had suspected him of it, and had thrown up his arm
with an angry start. Laughingly, Wrayburn looked to Lightwood for an
explanation of this odd visit.

‘He says he has something to say.’

‘Surely it must be to you, Mortimer.’

‘So I thought, but he says no. He says it is to you.’

‘Yes, I do say so,’ interposed the boy. ‘And I mean to say what I want
to say, too, Mr Eugene Wrayburn!’

Passing him with his eyes as if there were nothing where he stood,
Eugene looked on to Bradley Headstone. With consummate indolence, he
turned to Mortimer, inquiring: ‘And who may this other person be?’

‘I am Charles Hexam’s friend,’ said Bradley; ‘I am Charles Hexam’s
schoolmaster.’

‘My good sir, you should teach your pupils better manners,’ returned
Eugene.

Composedly smoking, he leaned an elbow on the chimneypiece, at the side
of the fire, and looked at the schoolmaster. It was a cruel look, in its
cold disdain of him, as a creature of no worth. The schoolmaster looked
at him, and that, too, was a cruel look, though of the different kind,
that it had a raging jealousy and fiery wrath in it.

Very remarkably, neither Eugene Wrayburn nor Bradley Headstone looked at
all at the boy. Through the ensuing dialogue, those two, no matter
who spoke, or whom was addressed, looked at each other. There was some
secret, sure perception between them, which set them against one another
in all ways.

‘In some high respects, Mr Eugene Wrayburn,’ said Bradley, answering
him with pale and quivering lips, ‘the natural feelings of my pupils are
stronger than my teaching.’

‘In most respects, I dare say,’ replied Eugene, enjoying his cigar,
‘though whether high or low is of no importance. You have my name very
correctly. Pray what is yours?’

‘It cannot concern you much to know, but--’

‘True,’ interposed Eugene, striking sharply and cutting him short at his
mistake, ‘it does not concern me at all to know. I can say Schoolmaster,
which is a most respectable title. You are right, Schoolmaster.’

It was not the dullest part of this goad in its galling of Bradley
Headstone, that he had made it himself in a moment of incautious anger.
He tried to set his lips so as to prevent their quivering, but they
quivered fast.

‘Mr Eugene Wrayburn,’ said the boy, ‘I want a word with you. I have
wanted it so much, that we have looked out your address in the book, and
we have been to your office, and we have come from your office here.’

‘You have given yourself much trouble, Schoolmaster,’ observed
Eugene, blowing the feathery ash from his cigar. ‘I hope it may prove
remunerative.’

‘And I am glad to speak,’ pursued the boy, ‘in presence of Mr Lightwood,
because it was through Mr Lightwood that you ever saw my sister.’

For a mere moment, Wrayburn turned his eyes aside from the schoolmaster
to note the effect of the last word on Mortimer, who, standing on the
opposite side of the fire, as soon as the word was spoken, turned his
face towards the fire and looked down into it.

‘Similarly, it was through Mr Lightwood that you ever saw her again, for
you were with him on the night when my father was found, and so I found
you with her on the next day. Since then, you have seen my sister often.
You have seen my sister oftener and oftener. And I want to know why?’

‘Was this worth while, Schoolmaster?’ murmured Eugene, with the air of
a disinterested adviser. ‘So much trouble for nothing? You should know
best, but I think not.’

‘I don’t know, Mr Wrayburn,’ answered Bradley, with his passion rising,
‘why you address me--’

‘Don’t you? said Eugene. ‘Then I won’t.’

He said it so tauntingly in his perfect placidity, that the respectable
right-hand clutching the respectable hair-guard of the respectable watch
could have wound it round his throat and strangled him with it. Not
another word did Eugene deem it worth while to utter, but stood leaning
his head upon his hand, smoking, and looking imperturbably at the
chafing Bradley Headstone with his clutching right-hand, until Bradley
was wellnigh mad.

‘Mr Wrayburn,’ proceeded the boy, ‘we not only know this that I have
charged upon you, but we know more. It has not yet come to my sister’s
knowledge that we have found it out, but we have. We had a plan, Mr
Headstone and I, for my sister’s education, and for its being advised
and overlooked by Mr Headstone, who is a much more competent authority,
whatever you may pretend to think, as you smoke, than you could produce,
if you tried. Then, what do we find? What do we find, Mr Lightwood? Why,
we find that my sister is already being taught, without our knowing
it. We find that while my sister gives an unwilling and cold ear to our
schemes for her advantage--I, her brother, and Mr Headstone, the most
competent authority, as his certificates would easily prove, that could
be produced--she is wilfully and willingly profiting by other schemes.
Ay, and taking pains, too, for I know what such pains are. And so does
Mr Headstone! Well! Somebody pays for this, is a thought that naturally
occurs to us; who pays? We apply ourselves to find out, Mr Lightwood,
and we find that your friend, this Mr Eugene Wrayburn, here, pays. Then
I ask him what right has he to do it, and what does he mean by it, and
how comes he to be taking such a liberty without my consent, when I
am raising myself in the scale of society by my own exertions and Mr
Headstone’s aid, and have no right to have any darkness cast upon my
prospects, or any imputation upon my respectability, through my sister?’

The boyish weakness of this speech, combined with its great selfishness,
made it a poor one indeed. And yet Bradley Headstone, used to the little
audience of a school, and unused to the larger ways of men, showed a
kind of exultation in it.

‘Now I tell Mr Eugene Wrayburn,’ pursued the boy, forced into the use
of the third person by the hopelessness of addressing him in the first,
‘that I object to his having any acquaintance at all with my sister, and
that I request him to drop it altogether. He is not to take it into his
head that I am afraid of my sister’s caring for HIM--’

(As the boy sneered, the Master sneered, and Eugene blew off the
feathery ash again.)

--‘But I object to it, and that’s enough. I am more important to my
sister than he thinks. As I raise myself, I intend to raise her;
she knows that, and she has to look to me for her prospects. Now I
understand all this very well, and so does Mr Headstone. My sister is an
excellent girl, but she has some romantic notions; not about such things
as your Mr Eugene Wrayburns, but about the death of my father and other
matters of that sort. Mr Wrayburn encourages those notions to make
himself of importance, and so she thinks she ought to be grateful to
him, and perhaps even likes to be. Now I don’t choose her to be grateful
to him, or to be grateful to anybody but me, except Mr Headstone. And
I tell Mr Wrayburn that if he don’t take heed of what I say, it will be
worse for her. Let him turn that over in his memory, and make sure of
it. Worse for her!’

A pause ensued, in which the schoolmaster looked very awkward.

‘May I suggest, Schoolmaster,’ said Eugene, removing his fast-waning
cigar from his lips to glance at it, ‘that you can now take your pupil
away.’

‘And Mr Lightwood,’ added the boy, with a burning face, under the
flaming aggravation of getting no sort of answer or attention, ‘I hope
you’ll take notice of what I have said to your friend, and of what
your friend has heard me say, word by word, whatever he pretends to the
contrary. You are bound to take notice of it, Mr Lightwood, for, as I
have already mentioned, you first brought your friend into my sister’s
company, and but for you we never should have seen him. Lord knows none
of us ever wanted him, any more than any of us will ever miss him. Now
Mr Headstone, as Mr Eugene Wrayburn has been obliged to hear what I had
to say, and couldn’t help himself, and as I have said it out to the last
word, we have done all we wanted to do, and may go.’

‘Go down-stairs, and leave me a moment, Hexam,’ he returned. The boy
complying with an indignant look and as much noise as he could make,
swung out of the room; and Lightwood went to the window, and leaned
there, looking out.

‘You think me of no more value than the dirt under your feet,’ said
Bradley to Eugene, speaking in a carefully weighed and measured tone, or
he could not have spoken at all.

‘I assure you, Schoolmaster,’ replied Eugene, ‘I don’t think about you.’

‘That’s not true,’ returned the other; ‘you know better.’

‘That’s coarse,’ Eugene retorted; ‘but you DON’T know better.’

‘Mr Wrayburn, at least I know very well that it would be idle to set
myself against you in insolent words or overbearing manners. That lad
who has just gone out could put you to shame in half-a-dozen branches of
knowledge in half an hour, but you can throw him aside like an inferior.
You can do as much by me, I have no doubt, beforehand.’

‘Possibly,’ remarked Eugene.

‘But I am more than a lad,’ said Bradley, with his clutching hand, ‘and
I WILL be heard, sir.’

‘As a schoolmaster,’ said Eugene, ‘you are always being heard. That
ought to content you.’

‘But it does not content me,’ replied the other, white with passion. ‘Do
you suppose that a man, in forming himself for the duties I discharge,
and in watching and repressing himself daily to discharge them well,
dismisses a man’s nature?’

‘I suppose you,’ said Eugene, ‘judging from what I see as I look at you,
to be rather too passionate for a good schoolmaster.’ As he spoke, he
tossed away the end of his cigar.

‘Passionate with you, sir, I admit I am. Passionate with you, sir, I
respect myself for being. But I have not Devils for my pupils.’

‘For your Teachers, I should rather say,’ replied Eugene.

‘Mr Wrayburn.’

‘Schoolmaster.’

‘Sir, my name is Bradley Headstone.’

‘As you justly said, my good sir, your name cannot concern me. Now, what
more?’

‘This more. Oh, what a misfortune is mine,’ cried Bradley, breaking off
to wipe the starting perspiration from his face as he shook from head to
foot, ‘that I cannot so control myself as to appear a stronger creature
than this, when a man who has not felt in all his life what I have felt
in a day can so command himself!’ He said it in a very agony, and even
followed it with an errant motion of his hands as if he could have torn
himself.

Eugene Wrayburn looked on at him, as if he found him beginning to be
rather an entertaining study.

‘Mr Wrayburn, I desire to say something to you on my own part.’

‘Come, come, Schoolmaster,’ returned Eugene, with a languid approach to
impatience as the other again struggled with himself; ‘say what you have
to say. And let me remind you that the door is standing open, and your
young friend waiting for you on the stairs.’

‘When I accompanied that youth here, sir, I did so with the purpose of
adding, as a man whom you should not be permitted to put aside, in case
you put him aside as a boy, that his instinct is correct and right.’
Thus Bradley Headstone, with great effort and difficulty.

‘Is that all?’ asked Eugene.

‘No, sir,’ said the other, flushed and fierce. ‘I strongly support him
in his disapproval of your visits to his sister, and in his objection to
your officiousness--and worse--in what you have taken upon yourself to
do for her.’

‘Is THAT all?’ asked Eugene.

‘No, sir. I determined to tell you that you are not justified in these
proceedings, and that they are injurious to his sister.’

‘Are you her schoolmaster as well as her brother’s?--Or perhaps you
would like to be?’ said Eugene.

It was a stab that the blood followed, in its rush to Bradley
Headstone’s face, as swiftly as if it had been dealt with a dagger.
‘What do you mean by that?’ was as much as he could utter.

‘A natural ambition enough,’ said Eugene, coolly. ‘Far be it from me
to say otherwise. The sister who is something too much upon your lips,
perhaps--is so very different from all the associations to which she had
been used, and from all the low obscure people about her, that it is a
very natural ambition.’

‘Do you throw my obscurity in my teeth, Mr Wrayburn?’

‘That can hardly be, for I know nothing concerning it, Schoolmaster, and
seek to know nothing.’

‘You reproach me with my origin,’ said Bradley Headstone; ‘you cast
insinuations at my bringing-up. But I tell you, sir, I have worked my
way onward, out of both and in spite of both, and have a right to be
considered a better man than you, with better reasons for being proud.’

‘How I can reproach you with what is not within my knowledge, or how
I can cast stones that were never in my hand, is a problem for the
ingenuity of a schoolmaster to prove,’ returned Eugene. ‘Is THAT all?’

‘No, sir. If you suppose that boy--’

‘Who really will be tired of waiting,’ said Eugene, politely.

‘If you suppose that boy to be friendless, Mr Wrayburn, you deceive
yourself. I am his friend, and you shall find me so.’

‘And you will find HIM on the stairs,’ remarked Eugene.

‘You may have promised yourself, sir, that you could do what you
chose here, because you had to deal with a mere boy, inexperienced,
friendless, and unassisted. But I give you warning that this mean
calculation is wrong. You have to do with a man also. You have to do
with me. I will support him, and, if need be, require reparation for
him. My hand and heart are in this cause, and are open to him.’

‘And--quite a coincidence--the door is open,’ remarked Eugene.

‘I scorn your shifty evasions, and I scorn you,’ said the schoolmaster.
‘In the meanness of your nature you revile me with the meanness of my
birth. I hold you in contempt for it. But if you don’t profit by this
visit, and act accordingly, you will find me as bitterly in earnest
against you as I could be if I deemed you worth a second thought on my
own account.’

With a consciously bad grace and stiff manner, as Wrayburn looked so
easily and calmly on, he went out with these words, and the heavy door
closed like a furnace-door upon his red and white heats of rage.

‘A curious monomaniac,’ said Eugene. ‘The man seems to believe that
everybody was acquainted with his mother!’

Mortimer Lightwood being still at the window, to which he had in
delicacy withdrawn, Eugene called to him, and he fell to slowly pacing
the room.

‘My dear fellow,’ said Eugene, as he lighted another cigar, ‘I fear my
unexpected visitors have been troublesome. If as a set-off (excuse the
legal phrase from a barrister-at-law) you would like to ask Tippins to
tea, I pledge myself to make love to her.’

‘Eugene, Eugene, Eugene,’ replied Mortimer, still pacing the room, ‘I am
sorry for this. And to think that I have been so blind!’

‘How blind, dear boy?’ inquired his unmoved friend.

‘What were your words that night at the river-side public-house?’ said
Lightwood, stopping. ‘What was it that you asked me? Did I feel like a
dark combination of traitor and pickpocket when I thought of that girl?’

‘I seem to remember the expression,’ said Eugene.

‘How do YOU feel when you think of her just now?’

His friend made no direct reply, but observed, after a few whiffs of his
cigar, ‘Don’t mistake the situation. There is no better girl in all this
London than Lizzie Hexam. There is no better among my people at home; no
better among your people.’

‘Granted. What follows?’

‘There,’ said Eugene, looking after him dubiously as he paced away to
the other end of the room, ‘you put me again upon guessing the riddle
that I have given up.’

‘Eugene, do you design to capture and desert this girl?’

‘My dear fellow, no.’

‘Do you design to marry her?’

‘My dear fellow, no.’

‘Do you design to pursue her?’

‘My dear fellow, I don’t design anything. I have no design whatever.
I am incapable of designs. If I conceived a design, I should speedily
abandon it, exhausted by the operation.’

‘Oh Eugene, Eugene!’

‘My dear Mortimer, not that tone of melancholy reproach, I entreat. What
can I do more than tell you all I know, and acknowledge my ignorance
of all I don’t know! How does that little old song go, which, under
pretence of being cheerful, is by far the most lugubrious I ever heard
in my life?

     “Away with melancholy,
     Nor doleful changes ring
     On life and human folly,
     But merrily merrily sing
                              Fal la!”

Don’t let us sing Fal la, my dear Mortimer (which is comparatively
unmeaning), but let us sing that we give up guessing the riddle
altogether.’

‘Are you in communication with this girl, Eugene, and is what these
people say true?’

‘I concede both admissions to my honourable and learned friend.’

‘Then what is to come of it? What are you doing? Where are you going?’

‘My dear Mortimer, one would think the schoolmaster had left behind him
a catechizing infection. You are ruffled by the want of another cigar.
Take one of these, I entreat. Light it at mine, which is in perfect
order. So! Now do me the justice to observe that I am doing all I can
towards self-improvement, and that you have a light thrown on those
household implements which, when you only saw them as in a glass darkly,
you were hastily--I must say hastily--inclined to depreciate. Sensible
of my deficiencies, I have surrounded myself with moral influences
expressly meant to promote the formation of the domestic virtues.
To those influences, and to the improving society of my friend from
boyhood, commend me with your best wishes.’

‘Ah, Eugene!’ said Lightwood, affectionately, now standing near him,
so that they both stood in one little cloud of smoke; ‘I would that you
answered my three questions! What is to come of it? What are you doing?
Where are you going?’

‘And my dear Mortimer,’ returned Eugene, lightly fanning away the smoke
with his hand for the better exposition of his frankness of face and
manner, ‘believe me, I would answer them instantly if I could. But
to enable me to do so, I must first have found out the troublesome
conundrum long abandoned. Here it is. Eugene Wrayburn.’ Tapping his
forehead and breast. ‘Riddle-me, riddle-me-ree, perhaps you can’t tell
me what this may be?--No, upon my life I can’t. I give it up!’



Chapter 7

IN WHICH A FRIENDLY MOVE IS ORIGINATED


The arrangement between Mr Boffin and his literary man, Mr Silas Wegg,
so far altered with the altered habits of Mr Boffin’s life, as that
the Roman Empire usually declined in the morning and in the eminently
aristocratic family mansion, rather than in the evening, as of yore,
and in Boffin’s Bower. There were occasions, however, when Mr Boffin,
seeking a brief refuge from the blandishments of fashion, would present
himself at the Bower after dark, to anticipate the next sallying
forth of Wegg, and would there, on the old settle, pursue the downward
fortunes of those enervated and corrupted masters of the world who were
by this time on their last legs. If Wegg had been worse paid for his
office, or better qualified to discharge it, he would have considered
these visits complimentary and agreeable; but, holding the position of
a handsomely-remunerated humbug, he resented them. This was quite
according to rule, for the incompetent servant, by whomsoever employed,
is always against his employer. Even those born governors, noble and
right honourable creatures, who have been the most imbecile in high
places, have uniformly shown themselves the most opposed (sometimes in
belying distrust, sometimes in vapid insolence) to THEIR employer. What
is in such wise true of the public master and servant, is equally true
of the private master and servant all the world over.

When Mr Silas Wegg did at last obtain free access to ‘Our House’, as he
had been wont to call the mansion outside which he had sat shelterless
so long, and when he did at last find it in all particulars as different
from his mental plans of it as according to the nature of things it
well could be, that far-seeing and far-reaching character, by way of
asserting himself and making out a case for compensation, affected to
fall into a melancholy strain of musing over the mournful past; as if
the house and he had had a fall in life together.

‘And this, sir,’ Silas would say to his patron, sadly nodding his head
and musing, ‘was once Our House! This, sir, is the building from which I
have so often seen those great creatures, Miss Elizabeth, Master
George, Aunt Jane, and Uncle Parker’--whose very names were of his own
inventing--‘pass and repass! And has it come to this, indeed! Ah dear
me, dear me!’

So tender were his lamentations, that the kindly Mr Boffin was quite
sorry for him, and almost felt mistrustful that in buying the house he
had done him an irreparable injury.

Two or three diplomatic interviews, the result of great subtlety on Mr
Wegg’s part, but assuming the mask of careless yielding to a fortuitous
combination of circumstances impelling him towards Clerkenwell, had
enabled him to complete his bargain with Mr Venus.

‘Bring me round to the Bower,’ said Silas, when the bargain was closed,
‘next Saturday evening, and if a sociable glass of old Jamaikey warm
should meet your views, I am not the man to begrudge it.’

‘You are aware of my being poor company, sir,’ replied Mr Venus, ‘but be
it so.’

It being so, here is Saturday evening come, and here is Mr Venus come,
and ringing at the Bower-gate.

Mr Wegg opens the gate, descries a sort of brown paper truncheon under
Mr Venus’s arm, and remarks, in a dry tone: ‘Oh! I thought perhaps you
might have come in a cab.’

‘No, Mr Wegg,’ replies Venus. ‘I am not above a parcel.’

‘Above a parcel! No!’ says Wegg, with some dissatisfaction. But does not
openly growl, ‘a certain sort of parcel might be above you.’

‘Here is your purchase, Mr Wegg,’ says Venus, politely handing it over,
‘and I am glad to restore it to the source from whence it--flowed.’

‘Thankee,’ says Wegg. ‘Now this affair is concluded, I may mention to
you in a friendly way that I’ve my doubts whether, if I had consulted a
lawyer, you could have kept this article back from me. I only throw it
out as a legal point.’

‘Do you think so, Mr Wegg? I bought you in open contract.’

‘You can’t buy human flesh and blood in this country, sir; not alive,
you can’t,’ says Wegg, shaking his head. ‘Then query, bone?’

‘As a legal point?’ asks Venus.

‘As a legal point.’

‘I am not competent to speak upon that, Mr Wegg,’ says Venus, reddening
and growing something louder; ‘but upon a point of fact I think myself
competent to speak; and as a point of fact I would have seen you--will
you allow me to say, further?’

‘I wouldn’t say more than further, if I was you,’ Mr Wegg suggests,
pacifically.

--‘Before I’d have given that packet into your hand without being paid
my price for it. I don’t pretend to know how the point of law may stand,
but I’m thoroughly confident upon the point of fact.’

As Mr Venus is irritable (no doubt owing to his disappointment in love),
and as it is not the cue of Mr Wegg to have him out of temper, the
latter gentleman soothingly remarks, ‘I only put it as a little case; I
only put it ha’porthetically.’

‘Then I’d rather, Mr Wegg, you put it another time, penn’orth-etically,’
is Mr Venus’s retort, ‘for I tell you candidly I don’t like your little
cases.’

Arrived by this time in Mr Wegg’s sitting-room, made bright on the
chilly evening by gaslight and fire, Mr Venus softens and compliments
him on his abode; profiting by the occasion to remind Wegg that he
(Venus) told him he had got into a good thing.

‘Tolerable,’ Wegg rejoins. ‘But bear in mind, Mr Venus, that there’s
no gold without its alloy. Mix for yourself and take a seat in the
chimbley-corner. Will you perform upon a pipe, sir?’

‘I am but an indifferent performer, sir,’ returns the other; ‘but I’ll
accompany you with a whiff or two at intervals.’

So, Mr Venus mixes, and Wegg mixes; and Mr Venus lights and puffs, and
Wegg lights and puffs.

‘And there’s alloy even in this metal of yours, Mr Wegg, you was
remarking?’

‘Mystery,’ returns Wegg. ‘I don’t like it, Mr Venus. I don’t like to
have the life knocked out of former inhabitants of this house, in the
gloomy dark, and not know who did it.’

‘Might you have any suspicions, Mr Wegg?’

‘No,’ returns that gentleman. ‘I know who profits by it. But I’ve no
suspicions.’

Having said which, Mr Wegg smokes and looks at the fire with a most
determined expression of Charity; as if he had caught that cardinal
virtue by the skirts as she felt it her painful duty to depart from him,
and held her by main force.

‘Similarly,’ resumes Wegg, ‘I have observations as I can offer upon
certain points and parties; but I make no objections, Mr Venus. Here
is an immense fortune drops from the clouds upon a person that shall be
nameless. Here is a weekly allowance, with a certain weight of coals,
drops from the clouds upon me. Which of us is the better man? Not the
person that shall be nameless. That’s an observation of mine, but I
don’t make it an objection. I take my allowance and my certain weight of
coals. He takes his fortune. That’s the way it works.’

‘It would be a good thing for me, if I could see things in the calm
light you do, Mr Wegg.’

‘Again look here,’ pursues Silas, with an oratorical flourish of his
pipe and his wooden leg: the latter having an undignified tendency
to tilt him back in his chair; ‘here’s another observation, Mr Venus,
unaccompanied with an objection. Him that shall be nameless is liable to
be talked over. He gets talked over. Him that shall be nameless, having
me at his right hand, naturally looking to be promoted higher, and you
may perhaps say meriting to be promoted higher--’

(Mr Venus murmurs that he does say so.)

‘--Him that shall be nameless, under such circumstances passes me by,
and puts a talking-over stranger above my head. Which of us two is the
better man? Which of us two can repeat most poetry? Which of us two has,
in the service of him that shall be nameless, tackled the Romans, both
civil and military, till he has got as husky as if he’d been weaned and
ever since brought up on sawdust? Not the talking-over stranger. Yet the
house is as free to him as if it was his, and he has his room, and is
put upon a footing, and draws about a thousand a year. I am banished to
the Bower, to be found in it like a piece of furniture whenever wanted.
Merit, therefore, don’t win. That’s the way it works. I observe it,
because I can’t help observing it, being accustomed to take a powerful
sight of notice; but I don’t object. Ever here before, Mr Venus?’

‘Not inside the gate, Mr Wegg.’

‘You’ve been as far as the gate then, Mr Venus?’

‘Yes, Mr Wegg, and peeped in from curiosity.’

‘Did you see anything?’

‘Nothing but the dust-yard.’

Mr Wegg rolls his eyes all round the room, in that ever unsatisfied
quest of his, and then rolls his eyes all round Mr Venus; as if
suspicious of his having something about him to be found out.

‘And yet, sir,’ he pursues, ‘being acquainted with old Mr Harmon, one
would have thought it might have been polite in you, too, to give him a
call. And you’re naturally of a polite disposition, you are.’ This last
clause as a softening compliment to Mr Venus.

‘It is true, sir,’ replies Venus, winking his weak eyes, and running
his fingers through his dusty shock of hair, ‘that I was so, before a
certain observation soured me. You understand to what I allude, Mr Wegg?
To a certain written statement respecting not wishing to be regarded in
a certain light. Since that, all is fled, save gall.’

‘Not all,’ says Mr Wegg, in a tone of sentimental condolence.

‘Yes, sir,’ returns Venus, ‘all! The world may deem it harsh, but I’d
quite as soon pitch into my best friend as not. Indeed, I’d sooner!’

Involuntarily making a pass with his wooden leg to guard himself as Mr
Venus springs up in the emphasis of this unsociable declaration, Mr Wegg
tilts over on his back, chair and all, and is rescued by that harmless
misanthrope, in a disjointed state and ruefully rubbing his head.

‘Why, you lost your balance, Mr Wegg,’ says Venus, handing him his pipe.

‘And about time to do it,’ grumbles Silas, ‘when a man’s visitors,
without a word of notice, conduct themselves with the sudden wiciousness
of Jacks-in-boxes! Don’t come flying out of your chair like that, Mr
Venus!’

‘I ask your pardon, Mr Wegg. I am so soured.’

‘Yes, but hang it,’ says Wegg argumentatively, ‘a well-governed mind can
be soured sitting! And as to being regarded in lights, there’s bumpey
lights as well as bony. IN which,’ again rubbing his head, ‘I object to
regard myself.’

‘I’ll bear it in memory, sir.’

‘If you’ll be so good.’ Mr Wegg slowly subdues his ironical tone and his
lingering irritation, and resumes his pipe. ‘We were talking of old Mr
Harmon being a friend of yours.’

‘Not a friend, Mr Wegg. Only known to speak to, and to have a little
deal with now and then. A very inquisitive character, Mr Wegg, regarding
what was found in the dust. As inquisitive as secret.’

‘Ah! You found him secret?’ returns Wegg, with a greedy relish.

‘He had always the look of it, and the manner of it.’

‘Ah!’ with another roll of his eyes. ‘As to what was found in the dust
now. Did you ever hear him mention how he found it, my dear friend?
Living on the mysterious premises, one would like to know. For instance,
where he found things? Or, for instance, how he set about it? Whether
he began at the top of the mounds, or whether he began at the bottom.
Whether he prodded’; Mr Wegg’s pantomime is skilful and expressive here;
‘or whether he scooped? Should you say scooped, my dear Mr Venus; or
should you as a man--say prodded?’

‘I should say neither, Mr Wegg.’

‘As a fellow-man, Mr Venus--mix again--why neither?’

‘Because I suppose, sir, that what was found, was found in the sorting
and sifting. All the mounds are sorted and sifted?’

‘You shall see ‘em and pass your opinion. Mix again.’

On each occasion of his saying ‘mix again’, Mr Wegg, with a hop on
his wooden leg, hitches his chair a little nearer; more as if he were
proposing that himself and Mr Venus should mix again, than that they
should replenish their glasses.

‘Living (as I said before) on the mysterious premises,’ says Wegg when
the other has acted on his hospitable entreaty, ‘one likes to know.
Would you be inclined to say now--as a brother--that he ever hid things
in the dust, as well as found ‘em?’

‘Mr Wegg, on the whole I should say he might.’

Mr Wegg claps on his spectacles, and admiringly surveys Mr Venus from
head to foot.

‘As a mortal equally with myself, whose hand I take in mine for the
first time this day, having unaccountably overlooked that act so full of
boundless confidence binding a fellow-creetur TO a fellow creetur,’ says
Wegg, holding Mr Venus’s palm out, flat and ready for smiting, and now
smiting it; ‘as such--and no other--for I scorn all lowlier ties betwixt
myself and the man walking with his face erect that alone I call my
Twin--regarded and regarding in this trustful bond--what do you think he
might have hid?’

‘It is but a supposition, Mr Wegg.’

‘As a Being with his hand upon his heart,’ cries Wegg; and the
apostrophe is not the less impressive for the Being’s hand being
actually upon his rum and water; ‘put your supposition into language,
and bring it out, Mr Venus!’

‘He was the species of old gentleman, sir,’ slowly returns that
practical anatomist, after drinking, ‘that I should judge likely to
take such opportunities as this place offered, of stowing away money,
valuables, maybe papers.’

‘As one that was ever an ornament to human life,’ says Mr Wegg, again
holding out Mr Venus’s palm as if he were going to tell his fortune by
chiromancy, and holding his own up ready for smiting it when the time
should come; ‘as one that the poet might have had his eye on, in writing
the national naval words:

     Helm a-weather, now lay her close,
            Yard arm and yard arm she lies;
     Again, cried I, Mr Venus, give her t’other dose,
            Man shrouds and grapple, sir, or she flies!

--that is to say, regarded in the light of true British Oak, for such
you are explain, Mr Venus, the expression “papers”!’

‘Seeing that the old gentleman was generally cutting off some near
relation, or blocking out some natural affection,’ Mr Venus rejoins, ‘he
most likely made a good many wills and codicils.’

The palm of Silas Wegg descends with a sounding smack upon the palm
of Venus, and Wegg lavishly exclaims, ‘Twin in opinion equally with
feeling! Mix a little more!’

Having now hitched his wooden leg and his chair close in front of Mr
Venus, Mr Wegg rapidly mixes for both, gives his visitor his glass,
touches its rim with the rim of his own, puts his own to his lips, puts
it down, and spreading his hands on his visitor’s knees thus addresses
him:

‘Mr Venus. It ain’t that I object to being passed over for a stranger,
though I regard the stranger as a more than doubtful customer. It ain’t
for the sake of making money, though money is ever welcome. It ain’t for
myself, though I am not so haughty as to be above doing myself a good
turn. It’s for the cause of the right.’

Mr Venus, passively winking his weak eyes both at once, demands: ‘What
is, Mr Wegg?’

‘The friendly move, sir, that I now propose. You see the move, sir?’

‘Till you have pointed it out, Mr Wegg, I can’t say whether I do or
not.’

‘If there IS anything to be found on these premises, let us find it
together. Let us make the friendly move of agreeing to look for it
together. Let us make the friendly move of agreeing to share the
profits of it equally betwixt us. In the cause of the right.’ Thus Silas
assuming a noble air.

‘Then,’ says Mr Venus, looking up, after meditating with his hair held
in his hands, as if he could only fix his attention by fixing his head;
‘if anything was to be unburied from under the dust, it would be kept a
secret by you and me? Would that be it, Mr Wegg?’

‘That would depend upon what it was, Mr Venus. Say it was money, or
plate, or jewellery, it would be as much ours as anybody else’s.’

Mr Venus rubs an eyebrow, interrogatively.

‘In the cause of the right it would. Because it would be unknowingly
sold with the mounds else, and the buyer would get what he was never
meant to have, and never bought. And what would that be, Mr Venus, but
the cause of the wrong?’

‘Say it was papers,’ Mr Venus propounds.

‘According to what they contained we should offer to dispose of ‘em to
the parties most interested,’ replies Wegg, promptly.

‘In the cause of the right, Mr Wegg?’

‘Always so, Mr Venus. If the parties should use them in the cause of the
wrong, that would be their act and deed. Mr Venus. I have an opinion of
you, sir, to which it is not easy to give mouth. Since I called upon you
that evening when you were, as I may say, floating your powerful mind in
tea, I have felt that you required to be roused with an object. In this
friendly move, sir, you will have a glorious object to rouse you.’

Mr Wegg then goes on to enlarge upon what throughout has been uppermost
in his crafty mind:--the qualifications of Mr Venus for such a search.
He expatiates on Mr Venus’s patient habits and delicate manipulation; on
his skill in piecing little things together; on his knowledge of various
tissues and textures; on the likelihood of small indications leading him
on to the discovery of great concealments. ‘While as to myself,’ says
Wegg, ‘I am not good at it. Whether I gave myself up to prodding,
or whether I gave myself up to scooping, I couldn’t do it with that
delicate touch so as not to show that I was disturbing the mounds.
Quite different with YOU, going to work (as YOU would) in the light of
a fellow-man, holily pledged in a friendly move to his brother man.’ Mr
Wegg next modestly remarks on the want of adaptation in a wooden leg
to ladders and such like airy perches, and also hints at an inherent
tendency in that timber fiction, when called into action for the
purposes of a promenade on an ashey slope, to stick itself into the
yielding foothold, and peg its owner to one spot. Then, leaving this
part of the subject, he remarks on the special phenomenon that before
his installation in the Bower, it was from Mr Venus that he first heard
of the legend of hidden wealth in the Mounds: ‘which’, he observes with
a vaguely pious air, ‘was surely never meant for nothing.’ Lastly,
he returns to the cause of the right, gloomily foreshadowing the
possibility of something being unearthed to criminate Mr Boffin (of whom
he once more candidly admits it cannot be denied that he profits by a
murder), and anticipating his denunciation by the friendly movers to
avenging justice. And this, Mr Wegg expressly points out, not at all for
the sake of the reward--though it would be a want of principle not to
take it.

To all this, Mr Venus, with his shock of dusty hair cocked after the
manner of a terrier’s ears, attends profoundly. When Mr Wegg, having
finished, opens his arms wide, as if to show Mr Venus how bare his
breast is, and then folds them pending a reply, Mr Venus winks at him
with both eyes some little time before speaking.

‘I see you have tried it by yourself, Mr Wegg,’ he says when he does
speak. ‘You have found out the difficulties by experience.’

‘No, it can hardly be said that I have tried it,’ replies Wegg, a little
dashed by the hint. ‘I have just skimmed it. Skimmed it.’

‘And found nothing besides the difficulties?’

Wegg shakes his head.

‘I scarcely know what to say to this, Mr Wegg,’ observes Venus, after
ruminating for a while.

‘Say yes,’ Wegg naturally urges.

‘If I wasn’t soured, my answer would be no. But being soured, Mr Wegg,
and driven to reckless madness and desperation, I suppose it’s Yes.’

Wegg joyfully reproduces the two glasses, repeats the ceremony of
clinking their rims, and inwardly drinks with great heartiness to the
health and success in life of the young lady who has reduced Mr Venus to
his present convenient state of mind.

The articles of the friendly move are then severally recited and agreed
upon. They are but secrecy, fidelity, and perseverance. The Bower to
be always free of access to Mr Venus for his researches, and every
precaution to be taken against their attracting observation in the
neighbourhood.

‘There’s a footstep!’ exclaims Venus.

‘Where?’ cries Wegg, starting.

‘Outside. St!’

They are in the act of ratifying the treaty of friendly move, by shaking
hands upon it. They softly break off, light their pipes which have gone
out, and lean back in their chairs. No doubt, a footstep. It approaches
the window, and a hand taps at the glass. ‘Come in!’ calls Wegg; meaning
come round by the door. But the heavy old-fashioned sash is slowly
raised, and a head slowly looks in out of the dark background of night.

‘Pray is Mr Silas Wegg here? Oh! I see him!’

The friendly movers might not have been quite at their ease, even
though the visitor had entered in the usual manner. But, leaning on the
breast-high window, and staring in out of the darkness, they find the
visitor extremely embarrassing. Especially Mr Venus: who removes his
pipe, draws back his head, and stares at the starer, as if it were his
own Hindoo baby come to fetch him home.

‘Good evening, Mr Wegg. The yard gate-lock should be looked to, if you
please; it don’t catch.’

‘Is it Mr Rokesmith?’ falters Wegg.

‘It is Mr Rokesmith. Don’t let me disturb you. I am not coming in. I
have only a message for you, which I undertook to deliver on my way home
to my lodgings. I was in two minds about coming beyond the gate without
ringing: not knowing but you might have a dog about.’

‘I wish I had,’ mutters Wegg, with his back turned as he rose from his
chair. ‘St! Hush! The talking-over stranger, Mr Venus.’

‘Is that any one I know?’ inquires the staring Secretary.

‘No, Mr Rokesmith. Friend of mine. Passing the evening with me.’

‘Oh! I beg his pardon. Mr Boffin wishes you to know that he does not
expect you to stay at home any evening, on the chance of his coming. It
has occurred to him that he may, without intending it, have been a tie
upon you. In future, if he should come without notice, he will take his
chance of finding you, and it will be all the same to him if he does
not. I undertook to tell you on my way. That’s all.’

With that, and ‘Good night,’ the Secretary lowers the window, and
disappears. They listen, and hear his footsteps go back to the gate, and
hear the gate close after him.

‘And for that individual, Mr Venus,’ remarks Wegg, when he is fully
gone, ‘I have been passed over! Let me ask you what you think of him?’

Apparently, Mr Venus does not know what to think of him, for he makes
sundry efforts to reply, without delivering himself of any other
articulate utterance than that he has ‘a singular look’.

‘A double look, you mean, sir,’ rejoins Wegg, playing bitterly upon the
word. ‘That’s HIS look. Any amount of singular look for me, but not a
double look! That’s an under-handed mind, sir.’

‘Do you say there’s something against him?’ Venus asks.

‘Something against him?’ repeats Wegg. ‘Something? What would the relief
be to my feelings--as a fellow-man--if I wasn’t the slave of truth, and
didn’t feel myself compelled to answer, Everything!’

See into what wonderful maudlin refuges, featherless ostriches plunge
their heads! It is such unspeakable moral compensation to Wegg, to be
overcome by the consideration that Mr Rokesmith has an underhanded mind!

‘On this starlight night, Mr Venus,’ he remarks, when he is showing that
friendly mover out across the yard, and both are something the worse
for mixing again and again: ‘on this starlight night to think that
talking-over strangers, and underhanded minds, can go walking home under
the sky, as if they was all square!’

‘The spectacle of those orbs,’ says Mr Venus, gazing upward with his hat
tumbling off; ‘brings heavy on me her crushing words that she did not
wish to regard herself nor yet to be regarded in that--’

‘I know! I know! You needn’t repeat ‘em,’ says Wegg, pressing his hand.
‘But think how those stars steady me in the cause of the right against
some that shall be nameless. It isn’t that I bear malice. But see how
they glisten with old remembrances! Old remembrances of what, sir?’

Mr Venus begins drearily replying, ‘Of her words, in her own
handwriting, that she does not wish to regard herself, nor yet--’ when
Silas cuts him short with dignity.

‘No, sir! Remembrances of Our House, of Master George, of Aunt Jane, of
Uncle Parker, all laid waste! All offered up sacrifices to the minion of
fortune and the worm of the hour!’



Chapter 8

IN WHICH AN INNOCENT ELOPEMENT OCCURS


The minion of fortune and the worm of the hour, or in less cutting
language, Nicodemus Boffin, Esquire, the Golden Dustman, had become
as much at home in his eminently aristocratic family mansion as he
was likely ever to be. He could not but feel that, like an eminently
aristocratic family cheese, it was much too large for his wants, and
bred an infinite amount of parasites; but he was content to regard this
drawback on his property as a sort of perpetual Legacy Duty. He felt the
more resigned to it, forasmuch as Mrs Boffin enjoyed herself completely,
and Miss Bella was delighted.

That young lady was, no doubt, an acquisition to the Boffins. She
was far too pretty to be unattractive anywhere, and far too quick of
perception to be below the tone of her new career. Whether it improved
her heart might be a matter of taste that was open to question; but as
touching another matter of taste, its improvement of her appearance and
manner, there could be no question whatever.

And thus it soon came about that Miss Bella began to set Mrs Boffin
right; and even further, that Miss Bella began to feel ill at ease, and
as it were responsible, when she saw Mrs Boffin going wrong. Not that so
sweet a disposition and so sound a nature could ever go very wrong even
among the great visiting authorities who agreed that the Boffins were
‘charmingly vulgar’ (which for certain was not their own case in saying
so), but that when she made a slip on the social ice on which all the
children of Podsnappery, with genteel souls to be saved, are required to
skate in circles, or to slide in long rows, she inevitably tripped Miss
Bella up (so that young lady felt), and caused her to experience great
confusion under the glances of the more skilful performers engaged in
those ice-exercises.

At Miss Bella’s time of life it was not to be expected that she should
examine herself very closely on the congruity or stability of her
position in Mr Boffin’s house. And as she had never been sparing of
complaints of her old home when she had no other to compare it with,
so there was no novelty of ingratitude or disdain in her very much
preferring her new one.

‘An invaluable man is Rokesmith,’ said Mr Boffin, after some two or
three months. ‘But I can’t quite make him out.’

Neither could Bella, so she found the subject rather interesting.

‘He takes more care of my affairs, morning, noon, and night,’ said Mr
Boffin, ‘than fifty other men put together either could or would; and
yet he has ways of his own that are like tying a scaffolding-pole right
across the road, and bringing me up short when I am almost a-walking arm
in arm with him.’

‘May I ask how so, sir?’ inquired Bella.

‘Well, my dear,’ said Mr Boffin, ‘he won’t meet any company here, but
you. When we have visitors, I should wish him to have his regular place
at the table like ourselves; but no, he won’t take it.’

‘If he considers himself above it,’ said Miss Bella, with an airy toss
of her head, ‘I should leave him alone.’

‘It ain’t that, my dear,’ replied Mr Boffin, thinking it over. ‘He don’t
consider himself above it.’

‘Perhaps he considers himself beneath it,’ suggested Bella. ‘If so, he
ought to know best.’

‘No, my dear; nor it ain’t that, neither. No,’ repeated Mr Boffin, with
a shake of his head, after again thinking it over; ‘Rokesmith’s a modest
man, but he don’t consider himself beneath it.’

‘Then what does he consider, sir?’ asked Bella.

‘Dashed if I know!’ said Mr Boffin. ‘It seemed at first as if it
was only Lightwood that he objected to meet. And now it seems to be
everybody, except you.’

Oho! thought Miss Bella. ‘In--deed! That’s it, is it!’ For Mr Mortimer
Lightwood had dined there two or three times, and she had met him
elsewhere, and he had shown her some attention. ‘Rather cool in a
Secretary--and Pa’s lodger--to make me the subject of his jealousy!’

That Pa’s daughter should be so contemptuous of Pa’s lodger was odd;
but there were odder anomalies than that in the mind of the spoilt girl:
spoilt first by poverty, and then by wealth. Be it this history’s part,
however, to leave them to unravel themselves.

‘A little too much, I think,’ Miss Bella reflected scornfully, ‘to
have Pa’s lodger laying claim to me, and keeping eligible people off!
A little too much, indeed, to have the opportunities opened to me by Mr
and Mrs Boffin, appropriated by a mere Secretary and Pa’s lodger!’

Yet it was not so very long ago that Bella had been fluttered by the
discovery that this same Secretary and lodger seem to like her. Ah! but
the eminently aristocratic mansion and Mrs Boffin’s dressmaker had not
come into play then.

In spite of his seemingly retiring manners a very intrusive person, this
Secretary and lodger, in Miss Bella’s opinion. Always a light in his
office-room when we came home from the play or Opera, and he always at
the carriage-door to hand us out. Always a provoking radiance too on
Mrs Boffin’s face, and an abominably cheerful reception of him, as if it
were possible seriously to approve what the man had in his mind!

‘You never charge me, Miss Wilfer,’ said the Secretary, encountering her
by chance alone in the great drawing-room, ‘with commissions for home.
I shall always be happy to execute any commands you may have in that
direction.’

‘Pray what may you mean, Mr Rokesmith?’ inquired Miss Bella, with
languidly drooping eyelids.

‘By home? I mean your father’s house at Holloway.’

She coloured under the retort--so skilfully thrust, that the words
seemed to be merely a plain answer, given in plain good faith--and said,
rather more emphatically and sharply:

‘What commissions and commands are you speaking of?’

‘Only little words of remembrance as I assume you sent somehow or
other,’ replied the Secretary with his former air. ‘It would be a
pleasure to me if you would make me the bearer of them. As you know, I
come and go between the two houses every day.’

‘You needn’t remind me of that, sir.’

She was too quick in this petulant sally against ‘Pa’s lodger’; and she
felt that she had been so when she met his quiet look.

‘They don’t send many--what was your expression?--words of remembrance
to me,’ said Bella, making haste to take refuge in ill-usage.

‘They frequently ask me about you, and I give them such slight
intelligence as I can.’

‘I hope it’s truly given,’ exclaimed Bella.

‘I hope you cannot doubt it, for it would be very much against you, if
you could.’

‘No, I do not doubt it. I deserve the reproach, which is very just
indeed. I beg your pardon, Mr Rokesmith.’

‘I should beg you not to do so, but that it shows you to such admirable
advantage,’ he replied with earnestness. ‘Forgive me; I could not help
saying that. To return to what I have digressed from, let me add that
perhaps they think I report them to you, deliver little messages, and
the like. But I forbear to trouble you, as you never ask me.’

‘I am going, sir,’ said Bella, looking at him as if he had reproved her,
‘to see them tomorrow.’

‘Is that,’ he asked, hesitating, ‘said to me, or to them?’

‘To which you please.’

‘To both? Shall I make it a message?’

‘You can if you like, Mr Rokesmith. Message or no message, I am going to
see them tomorrow.’

‘Then I will tell them so.’

He lingered a moment, as though to give her the opportunity of
prolonging the conversation if she wished. As she remained silent, he
left her. Two incidents of the little interview were felt by Miss Bella
herself, when alone again, to be very curious. The first was, that he
unquestionably left her with a penitent air upon her, and a penitent
feeling in her heart. The second was, that she had not an intention or
a thought of going home, until she had announced it to him as a settled
design.

‘What can I mean by it, or what can he mean by it?’ was her mental
inquiry: ‘He has no right to any power over me, and how do I come to
mind him when I don’t care for him?’

Mrs Boffin, insisting that Bella should make tomorrow’s expedition
in the chariot, she went home in great grandeur. Mrs Wilfer and Miss
Lavinia had speculated much on the probabilities and improbabilities of
her coming in this gorgeous state, and, on beholding the chariot from
the window at which they were secreted to look out for it, agreed
that it must be detained at the door as long as possible, for the
mortification and confusion of the neighbours. Then they repaired to
the usual family room, to receive Miss Bella with a becoming show of
indifference.

The family room looked very small and very mean, and the downward
staircase by which it was attained looked very narrow and very crooked.
The little house and all its arrangements were a poor contrast to the
eminently aristocratic dwelling. ‘I can hardly believe,’ thought Bella,
‘that I ever did endure life in this place!’

Gloomy majesty on the part of Mrs Wilfer, and native pertness on the
part of Lavvy, did not mend the matter. Bella really stood in natural
need of a little help, and she got none.

‘This,’ said Mrs Wilfer, presenting a cheek to be kissed, as sympathetic
and responsive as the back of the bowl of a spoon, ‘is quite an honour!
You will probably find your sister Lavvy grown, Bella.’

‘Ma,’ Miss Lavinia interposed, ‘there can be no objection to your being
aggravating, because Bella richly deserves it; but I really must request
that you will not drag in such ridiculous nonsense as my having grown
when I am past the growing age.’

‘I grew, myself,’ Mrs Wilfer sternly proclaimed, ‘after I was married.’

‘Very well, Ma,’ returned Lavvy, ‘then I think you had much better have
left it alone.’

The lofty glare with which the majestic woman received this answer,
might have embarrassed a less pert opponent, but it had no effect upon
Lavinia: who, leaving her parent to the enjoyment of any amount of
glaring at she might deem desirable under the circumstances, accosted
her sister, undismayed.

‘I suppose you won’t consider yourself quite disgraced, Bella, if I give
you a kiss? Well! And how do you do, Bella? And how are your Boffins?’

‘Peace!’ exclaimed Mrs Wilfer. ‘Hold! I will not suffer this tone of
levity.’

‘My goodness me! How are your Spoffins, then?’ said Lavvy, ‘since Ma so
very much objects to your Boffins.’

‘Impertinent girl! Minx!’ said Mrs Wilfer, with dread severity.

‘I don’t care whether I am a Minx, or a Sphinx,’ returned Lavinia,
coolly, tossing her head; ‘it’s exactly the same thing to me, and I’d
every bit as soon be one as the other; but I know this--I’ll not grow
after I’m married!’

‘You will not? YOU will not?’ repeated Mrs Wilfer, solemnly.

‘No, Ma, I will not. Nothing shall induce me.’

Mrs Wilfer, having waved her gloves, became loftily pathetic.

‘But it was to be expected;’ thus she spake. ‘A child of mine deserts me
for the proud and prosperous, and another child of mine despises me. It
is quite fitting.’

‘Ma,’ Bella struck in, ‘Mr and Mrs Boffin are prosperous, no doubt; but
you have no right to say they are proud. You must know very well that
they are not.’

‘In short, Ma,’ said Lavvy, bouncing over to the enemy without a word
of notice, ‘you must know very well--or if you don’t, more shame for
you!--that Mr and Mrs Boffin are just absolute perfection.’

‘Truly,’ returned Mrs Wilfer, courteously receiving the deserter, ‘it
would seem that we are required to think so. And this, Lavinia, is
my reason for objecting to a tone of levity. Mrs Boffin (of whose
physiognomy I can never speak with the composure I would desire to
preserve), and your mother, are not on terms of intimacy. It is not
for a moment to be supposed that she and her husband dare to presume to
speak of this family as the Wilfers. I cannot therefore condescend to
speak of them as the Boffins. No; for such a tone--call it familiarity,
levity, equality, or what you will--would imply those social
interchanges which do not exist. Do I render myself intelligible?’

Without taking the least notice of this inquiry, albeit delivered in an
imposing and forensic manner, Lavinia reminded her sister, ‘After all,
you know, Bella, you haven’t told us how your Whatshisnames are.’

‘I don’t want to speak of them here,’ replied Bella, suppressing
indignation, and tapping her foot on the floor. ‘They are much too kind
and too good to be drawn into these discussions.’

‘Why put it so?’ demanded Mrs Wilfer, with biting sarcasm. ‘Why adopt a
circuitous form of speech? It is polite and it is obliging; but why do
it? Why not openly say that they are much too kind and too good for US?
We understand the allusion. Why disguise the phrase?’

‘Ma,’ said Bella, with one beat of her foot, ‘you are enough to drive a
saint mad, and so is Lavvy.’

‘Unfortunate Lavvy!’ cried Mrs Wilfer, in a tone of commiseration. ‘She
always comes for it. My poor child!’ But Lavvy, with the suddenness of
her former desertion, now bounced over to the other enemy: very sharply
remarking, ‘Don’t patronize ME, Ma, because I can take care of myself.’

‘I only wonder,’ resumed Mrs Wilfer, directing her observations to her
elder daughter, as safer on the whole than her utterly unmanageable
younger, ‘that you found time and inclination to tear yourself from
Mr and Mrs Boffin, and come to see us at all. I only wonder that our
claims, contending against the superior claims of Mr and Mrs Boffin,
had any weight. I feel I ought to be thankful for gaining so much, in
competition with Mr and Mrs Boffin.’ (The good lady bitterly emphasized
the first letter of the word Boffin, as if it represented her chief
objection to the owners of that name, and as if she could have born
Doffin, Moffin, or Poffin much better.)

‘Ma,’ said Bella, angrily, ‘you force me to say that I am truly sorry I
did come home, and that I never will come home again, except when poor
dear Pa is here. For, Pa is too magnanimous to feel envy and spite
towards my generous friends, and Pa is delicate enough and gentle enough
to remember the sort of little claim they thought I had upon them and
the unusually trying position in which, through no act of my own, I had
been placed. And I always did love poor dear Pa better than all the rest
of you put together, and I always do and I always shall!’

Here Bella, deriving no comfort from her charming bonnet and her elegant
dress, burst into tears.

‘I think, R.W.,’ cried Mrs Wilfer, lifting up her eyes and
apostrophising the air, ‘that if you were present, it would be a
trial to your feelings to hear your wife and the mother of your family
depreciated in your name. But Fate has spared you this, R.W., whatever
it may have thought proper to inflict upon her!’

Here Mrs Wilfer burst into tears.

‘I hate the Boffins!’ protested Miss Lavinia. ‘I don’t care who objects
to their being called the Boffins. I WILL call ‘em the Boffins. The
Boffins, the Boffins, the Boffins! And I say they are mischief-making
Boffins, and I say the Boffins have set Bella against me, and I tell the
Boffins to their faces:’ which was not strictly the fact, but the
young lady was excited: ‘that they are detestable Boffins, disreputable
Boffins, odious Boffins, beastly Boffins. There!’

Here Miss Lavinia burst into tears.

The front garden-gate clanked, and the Secretary was seen coming at a
brisk pace up the steps. ‘Leave Me to open the door to him,’ said Mrs
Wilfer, rising with stately resignation as she shook her head and dried
her eyes; ‘we have at present no stipendiary girl to do so. We have
nothing to conceal. If he sees these traces of emotion on our cheeks,
let him construe them as he may.’

With those words she stalked out. In a few moments she stalked in again,
proclaiming in her heraldic manner, ‘Mr Rokesmith is the bearer of a
packet for Miss Bella Wilfer.’

Mr Rokesmith followed close upon his name, and of course saw what was
amiss. But he discreetly affected to see nothing, and addressed Miss
Bella.

‘Mr Boffin intended to have placed this in the carriage for you
this morning. He wished you to have it, as a little keepsake he had
prepared--it is only a purse, Miss Wilfer--but as he was disappointed in
his fancy, I volunteered to come after you with it.’

Bella took it in her hand, and thanked him.

‘We have been quarrelling here a little, Mr Rokesmith, but not more than
we used; you know our agreeable ways among ourselves. You find me just
going. Good-bye, mamma. Good-bye, Lavvy!’ and with a kiss for each Miss
Bella turned to the door. The Secretary would have attended her, but
Mrs Wilfer advancing and saying with dignity, ‘Pardon me! Permit me to
assert my natural right to escort my child to the equipage which is
in waiting for her,’ he begged pardon and gave place. It was a very
magnificent spectacle indeed, to see Mrs Wilfer throw open the
house-door, and loudly demand with extended gloves, ‘The male domestic
of Mrs Boffin!’ To whom presenting himself, she delivered the brief but
majestic charge, ‘Miss Wilfer. Coming out!’ and so delivered her over,
like a female Lieutenant of the Tower relinquishing a State Prisoner.
The effect of this ceremonial was for some quarter of an hour afterwards
perfectly paralyzing on the neighbours, and was much enhanced by the
worthy lady airing herself for that term in a kind of splendidly serene
trance on the top step.

When Bella was seated in the carriage, she opened the little packet in
her hand. It contained a pretty purse, and the purse contained a bank
note for fifty pounds. ‘This shall be a joyful surprise for poor dear
Pa,’ said Bella, ‘and I’ll take it myself into the City!’

As she was uninformed respecting the exact locality of the place of
business of Chicksey Veneering and Stobbles, but knew it to be near
Mincing Lane, she directed herself to be driven to the corner of that
darksome spot. Thence she despatched ‘the male domestic of Mrs Boffin,’
in search of the counting-house of Chicksey Veneering and Stobbles, with
a message importing that if R. Wilfer could come out, there was a lady
waiting who would be glad to speak with him. The delivery of these
mysterious words from the mouth of a footman caused so great an
excitement in the counting-house, that a youthful scout was instantly
appointed to follow Rumty, observe the lady, and come in with his
report. Nor was the agitation by any means diminished, when the scout
rushed back with the intelligence that the lady was ‘a slap-up gal in a
bang-up chariot.’

Rumty himself, with his pen behind his ear under his rusty hat, arrived
at the carriage-door in a breathless condition, and had been fairly
lugged into the vehicle by his cravat and embraced almost unto choking,
before he recognized his daughter. ‘My dear child!’ he then panted,
incoherently. ‘Good gracious me! What a lovely woman you are! I thought
you had been unkind and forgotten your mother and sister.’

‘I have just been to see them, Pa dear.’

‘Oh! and how--how did you find your mother?’ asked R. W., dubiously.

‘Very disagreeable, Pa, and so was Lavvy.’

‘They are sometimes a little liable to it,’ observed the patient cherub;
‘but I hope you made allowances, Bella, my dear?’

‘No. I was disagreeable too, Pa; we were all of us disagreeable
together. But I want you to come and dine with me somewhere, Pa.’

‘Why, my dear, I have already partaken of a--if one might mention such
an article in this superb chariot--of a--Saveloy,’ replied R. Wilfer,
modestly dropping his voice on the word, as he eyed the canary-coloured
fittings.

‘Oh! That’s nothing, Pa!’

‘Truly, it ain’t as much as one could sometimes wish it to be, my
dear,’ he admitted, drawing his hand across his mouth. ‘Still, when
circumstances over which you have no control, interpose obstacles
between yourself and Small Germans, you can’t do better than bring a
contented mind to hear on’--again dropping his voice in deference to the
chariot--‘Saveloys!’

‘You poor good Pa! Pa, do, I beg and pray, get leave for the rest of the
day, and come and pass it with me!’

‘Well, my dear, I’ll cut back and ask for leave.’

‘But before you cut back,’ said Bella, who had already taken him by the
chin, pulled his hat off, and begun to stick up his hair in her old way,
‘do say that you are sure I am giddy and inconsiderate, but have never
really slighted you, Pa.’

‘My dear, I say it with all my heart. And might I likewise observe,’ her
father delicately hinted, with a glance out at window, ‘that perhaps
it might be calculated to attract attention, having one’s hair publicly
done by a lovely woman in an elegant turn-out in Fenchurch Street?’

Bella laughed and put on his hat again. But when his boyish figure
bobbed away, its shabbiness and cheerful patience smote the tears out
of her eyes. ‘I hate that Secretary for thinking it of me,’ she said to
herself, ‘and yet it seems half true!’

Back came her father, more like a boy than ever, in his release from
school. ‘All right, my dear. Leave given at once. Really very handsomely
done!’

‘Now where can we find some quiet place, Pa, in which I can wait for you
while you go on an errand for me, if I send the carriage away?’

It demanded cogitation. ‘You see, my dear,’ he explained, ‘you really
have become such a very lovely woman, that it ought to be a very quiet
place.’ At length he suggested, ‘Near the garden up by the Trinity House
on Tower Hill.’ So, they were driven there, and Bella dismissed the
chariot; sending a pencilled note by it to Mrs Boffin, that she was with
her father.

‘Now, Pa, attend to what I am going to say, and promise and vow to be
obedient.’

‘I promise and vow, my dear.’

‘You ask no questions. You take this purse; you go to the nearest place
where they keep everything of the very very best, ready made; you buy
and put on, the most beautiful suit of clothes, the most beautiful hat,
and the most beautiful pair of bright boots (patent leather, Pa, mind!)
that are to be got for money; and you come back to me.’

‘But, my dear Bella--’

‘Take care, Pa!’ pointing her forefinger at him, merrily. ‘You have
promised and vowed. It’s perjury, you know.’

There was water in the foolish little fellow’s eyes, but she kissed them
dry (though her own were wet), and he bobbed away again. After half an
hour, he came back, so brilliantly transformed, that Bella was obliged
to walk round him in ecstatic admiration twenty times, before she could
draw her arm through his, and delightedly squeeze it.

‘Now, Pa,’ said Bella, hugging him close, ‘take this lovely woman out to
dinner.’

‘Where shall we go, my dear?’

‘Greenwich!’ said Bella, valiantly. ‘And be sure you treat this lovely
woman with everything of the best.’

While they were going along to take boat, ‘Don’t you wish, my dear,’
said R. W., timidly, ‘that your mother was here?’

‘No, I don’t, Pa, for I like to have you all to myself to-day. I was
always your little favourite at home, and you were always mine. We have
run away together often, before now; haven’t we, Pa?’

‘Ah, to be sure we have! Many a Sunday when your mother was--was a
little liable to it,’ repeating his former delicate expression after
pausing to cough.

‘Yes, and I am afraid I was seldom or never as good as I ought to have
been, Pa. I made you carry me, over and over again, when you should
have made me walk; and I often drove you in harness, when you would much
rather have sat down and read your news-paper: didn’t I?’

‘Sometimes, sometimes. But Lor, what a child you were! What a companion
you were!’

‘Companion? That’s just what I want to be to-day, Pa.’

‘You are safe to succeed, my love. Your brothers and sisters have all
in their turns been companions to me, to a certain extent, but only to a
certain extent. Your mother has, throughout life, been a companion that
any man might--might look up to--and--and commit the sayings of, to
memory--and--form himself upon--if he--’

‘If he liked the model?’ suggested Bella.

‘We-ell, ye-es,’ he returned, thinking about it, not quite satisfied
with the phrase: ‘or perhaps I might say, if it was in him. Supposing,
for instance, that a man wanted to be always marching, he would find
your mother an inestimable companion. But if he had any taste for
walking, or should wish at any time to break into a trot, he might
sometimes find it a little difficult to keep step with your mother.
Or take it this way, Bella,’ he added, after a moment’s reflection;
‘Supposing that a man had to go through life, we won’t say with a
companion, but we’ll say to a tune. Very good. Supposing that the tune
allotted to him was the Dead March in Saul. Well. It would be a very
suitable tune for particular occasions--none better--but it would
be difficult to keep time with in the ordinary run of domestic
transactions. For instance, if he took his supper after a hard day, to
the Dead March in Saul, his food might be likely to sit heavy on him.
Or, if he was at any time inclined to relieve his mind by singing a
comic song or dancing a hornpipe, and was obliged to do it to the Dead
March in Saul, he might find himself put out in the execution of his
lively intentions.’

‘Poor Pa!’ thought Bella, as she hung upon his arm.

‘Now, what I will say for you, my dear,’ the cherub pursued mildly and
without a notion of complaining, ‘is, that you are so adaptable. So
adaptable.’

‘Indeed I am afraid I have shown a wretched temper, Pa. I am afraid
I have been very complaining, and very capricious. I seldom or never
thought of it before. But when I sat in the carriage just now and saw
you coming along the pavement, I reproached myself.’

‘Not at all, my dear. Don’t speak of such a thing.’

A happy and a chatty man was Pa in his new clothes that day. Take it
for all in all, it was perhaps the happiest day he had ever known in his
life; not even excepting that on which his heroic partner had approached
the nuptial altar to the tune of the Dead March in Saul.

The little expedition down the river was delightful, and the little
room overlooking the river into which they were shown for dinner was
delightful. Everything was delightful. The park was delightful, the
punch was delightful, the dishes of fish were delightful, the wine
was delightful. Bella was more delightful than any other item in the
festival; drawing Pa out in the gayest manner; making a point of always
mentioning herself as the lovely woman; stimulating Pa to order things,
by declaring that the lovely woman insisted on being treated with them;
and in short causing Pa to be quite enraptured with the consideration
that he WAS the Pa of such a charming daughter.

And then, as they sat looking at the ships and steamboats making their
way to the sea with the tide that was running down, the lovely woman
imagined all sorts of voyages for herself and Pa. Now, Pa, in the
character of owner of a lumbering square-sailed collier, was tacking
away to Newcastle, to fetch black diamonds to make his fortune with;
now, Pa was going to China in that handsome threemasted ship, to bring
home opium, with which he would for ever cut out Chicksey Veneering
and Stobbles, and to bring home silks and shawls without end for the
decoration of his charming daughter. Now, John Harmon’s disastrous fate
was all a dream, and he had come home and found the lovely woman just
the article for him, and the lovely woman had found him just the article
for her, and they were going away on a trip, in their gallant bark,
to look after their vines, with streamers flying at all points, a band
playing on deck and Pa established in the great cabin. Now, John Harmon
was consigned to his grave again, and a merchant of immense wealth
(name unknown) had courted and married the lovely woman, and he was
so enormously rich that everything you saw upon the river sailing or
steaming belonged to him, and he kept a perfect fleet of yachts for
pleasure, and that little impudent yacht which you saw over there, with
the great white sail, was called The Bella, in honour of his wife, and
she held her state aboard when it pleased her, like a modern Cleopatra.
Anon, there would embark in that troop-ship when she got to Gravesend, a
mighty general, of large property (name also unknown), who wouldn’t
hear of going to victory without his wife, and whose wife was the lovely
woman, and she was destined to become the idol of all the red coats and
blue jackets alow and aloft. And then again: you saw that ship being
towed out by a steam-tug? Well! where did you suppose she was going to?
She was going among the coral reefs and cocoa-nuts and all that sort of
thing, and she was chartered for a fortunate individual of the name
of Pa (himself on board, and much respected by all hands), and she
was going, for his sole profit and advantage, to fetch a cargo of
sweet-smelling woods, the most beautiful that ever were seen, and the
most profitable that ever were heard of; and her cargo would be a great
fortune, as indeed it ought to be: the lovely woman who had purchased
her and fitted her expressly for this voyage, being married to an Indian
Prince, who was a Something-or-Other, and who wore Cashmere shawls all
over himself and diamonds and emeralds blazing in his turban, and was
beautifully coffee-coloured and excessively devoted, though a little too
jealous. Thus Bella ran on merrily, in a manner perfectly enchanting to
Pa, who was as willing to put his head into the Sultan’s tub of water as
the beggar-boys below the window were to put THEIR heads in the mud.

‘I suppose, my dear,’ said Pa after dinner, ‘we may come to the
conclusion at home, that we have lost you for good?’

Bella shook her head. Didn’t know. Couldn’t say. All she was able to
report was, that she was most handsomely supplied with everything she
could possibly want, and that whenever she hinted at leaving Mr and Mrs
Boffin, they wouldn’t hear of it.

‘And now, Pa,’ pursued Bella, ‘I’ll make a confession to you. I am the
most mercenary little wretch that ever lived in the world.’

‘I should hardly have thought it of you, my dear,’ returned her father,
first glancing at himself; and then at the dessert.

‘I understand what you mean, Pa, but it’s not that. It’s not that I care
for money to keep as money, but I do care so much for what it will buy!’

‘Really I think most of us do,’ returned R. W.

‘But not to the dreadful extent that I do, Pa. O-o!’ cried Bella,
screwing the exclamation out of herself with a twist of her dimpled
chin. ‘I AM so mercenary!’

With a wistful glance R. W. said, in default of having anything better
to say: ‘About when did you begin to feel it coming on, my dear?’

‘That’s it, Pa. That’s the terrible part of it. When I was at home, and
only knew what it was to be poor, I grumbled but didn’t so much mind.
When I was at home expecting to be rich, I thought vaguely of all the
great things I would do. But when I had been disappointed of my splendid
fortune, and came to see it from day to day in other hands, and to have
before my eyes what it could really do, then I became the mercenary
little wretch I am.’

‘It’s your fancy, my dear.’

‘I can assure you it’s nothing of the sort, Pa!’ said Bella, nodding at
him, with her very pretty eyebrows raised as high as they would go, and
looking comically frightened. ‘It’s a fact. I am always avariciously
scheming.’

‘Lor! But how?’

‘I’ll tell you, Pa. I don’t mind telling YOU, because we have always
been favourites of each other’s, and because you are not like a Pa, but
more like a sort of a younger brother with a dear venerable chubbiness
on him. And besides,’ added Bella, laughing as she pointed a rallying
finger at his face, ‘because I have got you in my power. This is a
secret expedition. If ever you tell of me, I’ll tell of you. I’ll tell
Ma that you dined at Greenwich.’

‘Well; seriously, my dear,’ observed R. W., with some trepidation of
manner, ‘it might be as well not to mention it.’

‘Aha!’ laughed Bella. ‘I knew you wouldn’t like it, sir! So you keep my
confidence, and I’ll keep yours. But betray the lovely woman, and you
shall find her a serpent. Now, you may give me a kiss, Pa, and I should
like to give your hair a turn, because it has been dreadfully neglected
in my absence.’

R. W. submitted his head to the operator, and the operator went on
talking; at the same time putting separate locks of his hair through
a curious process of being smartly rolled over her two revolving
forefingers, which were then suddenly pulled out of it in opposite
lateral directions. On each of these occasions the patient winced and
winked.

‘I have made up my mind that I must have money, Pa. I feel that I can’t
beg it, borrow it, or steal it; and so I have resolved that I must marry
it.’

R. W. cast up his eyes towards her, as well as he could under the
operating circumstances, and said in a tone of remonstrance, ‘My de-ar
Bella!’

‘Have resolved, I say, Pa, that to get money I must marry money. In
consequence of which, I am always looking out for money to captivate.’

‘My de-a-r Bella!’

‘Yes, Pa, that is the state of the case. If ever there was a mercenary
plotter whose thoughts and designs were always in her mean occupation, I
am the amiable creature. But I don’t care. I hate and detest being
poor, and I won’t be poor if I can marry money. Now you are deliciously
fluffy, Pa, and in a state to astonish the waiter and pay the bill.’

‘But, my dear Bella, this is quite alarming at your age.’

‘I told you so, Pa, but you wouldn’t believe it,’ returned Bella, with a
pleasant childish gravity. ‘Isn’t it shocking?’

‘It would be quite so, if you fully knew what you said, my dear, or
meant it.’

‘Well, Pa, I can only tell you that I mean nothing else. Talk to me of
love!’ said Bella, contemptuously: though her face and figure certainly
rendered the subject no incongruous one. ‘Talk to me of fiery dragons!
But talk to me of poverty and wealth, and there indeed we touch upon
realities.’

‘My De-ar, this is becoming Awful--’ her father was emphatically
beginning: when she stopped him.

‘Pa, tell me. Did you marry money?’

‘You know I didn’t, my dear.’

Bella hummed the Dead March in Saul, and said, after all it signified
very little! But seeing him look grave and downcast, she took him round
the neck and kissed him back to cheerfulness again.

‘I didn’t mean that last touch, Pa; it was only said in joke. Now mind!
You are not to tell of me, and I’ll not tell of you. And more than that;
I promise to have no secrets from you, Pa, and you may make certain
that, whatever mercenary things go on, I shall always tell you all about
them in strict confidence.’

Fain to be satisfied with this concession from the lovely woman, R. W.
rang the bell, and paid the bill. ‘Now, all the rest of this, Pa,’ said
Bella, rolling up the purse when they were alone again, hammering it
small with her little fist on the table, and cramming it into one of the
pockets of his new waistcoat, ‘is for you, to buy presents with for them
at home, and to pay bills with, and to divide as you like, and spend
exactly as you think proper. Last of all take notice, Pa, that it’s
not the fruit of any avaricious scheme. Perhaps if it was, your little
mercenary wretch of a daughter wouldn’t make so free with it!’

After which, she tugged at his coat with both hands, and pulled him all
askew in buttoning that garment over the precious waistcoat pocket, and
then tied her dimples into her bonnet-strings in a very knowing way, and
took him back to London. Arrived at Mr Boffin’s door, she set him with
his back against it, tenderly took him by the ears as convenient handles
for her purpose, and kissed him until he knocked muffled double knocks
at the door with the back of his head. That done, she once more reminded
him of their compact and gaily parted from him.

Not so gaily, however, but that tears filled her eyes as he went away
down the dark street. Not so gaily, but that she several times said,
‘Ah, poor little Pa! Ah, poor dear struggling shabby little Pa!’
before she took heart to knock at the door. Not so gaily, but that the
brilliant furniture seemed to stare her out of countenance as if it
insisted on being compared with the dingy furniture at home. Not so
gaily, but that she fell into very low spirits sitting late in her own
room, and very heartily wept, as she wished, now that the deceased old
John Harmon had never made a will about her, now that the deceased young
John Harmon had lived to marry her. ‘Contradictory things to wish,’ said
Bella, ‘but my life and fortunes are so contradictory altogether that
what can I expect myself to be!’



Chapter 9

IN WHICH THE ORPHAN MAKES HIS WILL


The Secretary, working in the Dismal Swamp betimes next morning, was
informed that a youth waited in the hall who gave the name of Sloppy.
The footman who communicated this intelligence made a decent pause
before uttering the name, to express that it was forced on his
reluctance by the youth in question, and that if the youth had had
the good sense and good taste to inherit some other name it would have
spared the feelings of him the bearer.

‘Mrs Boffin will be very well pleased,’ said the Secretary in a
perfectly composed way. ‘Show him in.’

Mr Sloppy being introduced, remained close to the door: revealing
in various parts of his form many surprising, confounding, and
incomprehensible buttons.

‘I am glad to see you,’ said John Rokesmith, in a cheerful tone of
welcome. ‘I have been expecting you.’

Sloppy explained that he had meant to come before, but that the Orphan
(of whom he made mention as Our Johnny) had been ailing, and he had
waited to report him well.

‘Then he is well now?’ said the Secretary.

‘No he ain’t,’ said Sloppy.

Mr Sloppy having shaken his head to a considerable extent, proceeded
to remark that he thought Johnny ‘must have took ‘em from the Minders.’
Being asked what he meant, he answered, them that come out upon him and
partickler his chest. Being requested to explain himself, he stated that
there was some of ‘em wot you couldn’t kiver with a sixpence. Pressed to
fall back upon a nominative case, he opined that they wos about as
red as ever red could be. ‘But as long as they strikes out’ards, sir,’
continued Sloppy, ‘they ain’t so much. It’s their striking in’ards
that’s to be kep off.’

John Rokesmith hoped the child had had medical attendance? Oh yes, said
Sloppy, he had been took to the doctor’s shop once. And what did the
doctor call it? Rokesmith asked him. After some perplexed reflection,
Sloppy answered, brightening, ‘He called it something as wos wery
long for spots.’ Rokesmith suggested measles. ‘No,’ said Sloppy with
confidence, ‘ever so much longer than THEM, sir!’ (Mr Sloppy was
elevated by this fact, and seemed to consider that it reflected credit
on the poor little patient.)

‘Mrs Boffin will be sorry to hear this,’ said Rokesmith.

‘Mrs Higden said so, sir, when she kep it from her, hoping as Our Johnny
would work round.’

‘But I hope he will?’ said Rokesmith, with a quick turn upon the
messenger.

‘I hope so,’ answered Sloppy. ‘It all depends on their striking
in’ards.’ He then went on to say that whether Johnny had ‘took ‘em’
from the Minders, or whether the Minders had ‘took ’em from Johnny,
the Minders had been sent home and had ‘got ’em.’ Furthermore, that Mrs
Higden’s days and nights being devoted to Our Johnny, who was never out
of her lap, the whole of the mangling arrangements had devolved upon
himself, and he had had ‘rayther a tight time’. The ungainly piece of
honesty beamed and blushed as he said it, quite enraptured with the
remembrance of having been serviceable.

‘Last night,’ said Sloppy, ‘when I was a-turning at the wheel pretty
late, the mangle seemed to go like Our Johnny’s breathing. It begun
beautiful, then as it went out it shook a little and got unsteady, then
as it took the turn to come home it had a rattle-like and lumbered a
bit, then it come smooth, and so it went on till I scarce know’d which
was mangle and which was Our Johnny. Nor Our Johnny, he scarce know’d
either, for sometimes when the mangle lumbers he says, “Me choking,
Granny!” and Mrs Higden holds him up in her lap and says to me “Bide a
bit, Sloppy,” and we all stops together. And when Our Johnny gets his
breathing again, I turns again, and we all goes on together.’

Sloppy had gradually expanded with his description into a stare and a
vacant grin. He now contracted, being silent, into a half-repressed gush
of tears, and, under pretence of being heated, drew the under part of
his sleeve across his eyes with a singularly awkward, laborious, and
roundabout smear.

‘This is unfortunate,’ said Rokesmith. ‘I must go and break it to Mrs
Boffin. Stay you here, Sloppy.’

Sloppy stayed there, staring at the pattern of the paper on the wall,
until the Secretary and Mrs Boffin came back together. And with Mrs
Boffin was a young lady (Miss Bella Wilfer by name) who was better worth
staring at, it occurred to Sloppy, than the best of wall-papering.

‘Ah, my poor dear pretty little John Harmon!’ exclaimed Mrs Boffin.

‘Yes mum,’ said the sympathetic Sloppy.

‘You don’t think he is in a very, very bad way, do you?’ asked the
pleasant creature with her wholesome cordiality.

Put upon his good faith, and finding it in collision with his
inclinations, Sloppy threw back his head and uttered a mellifluous howl,
rounded off with a sniff.

‘So bad as that!’ cried Mrs Boffin. ‘And Betty Higden not to tell me of
it sooner!’

‘I think she might have been mistrustful, mum,’ answered Sloppy,
hesitating.

‘Of what, for Heaven’s sake?’

‘I think she might have been mistrustful, mum,’ returned Sloppy with
submission, ‘of standing in Our Johnny’s light. There’s so much trouble
in illness, and so much expense, and she’s seen such a lot of its being
objected to.’

‘But she never can have thought,’ said Mrs Boffin, ‘that I would grudge
the dear child anything?’

‘No mum, but she might have thought (as a habit-like) of its standing
in Johnny’s light, and might have tried to bring him through it
unbeknownst.’

Sloppy knew his ground well. To conceal herself in sickness, like a
lower animal; to creep out of sight and coil herself away and die; had
become this woman’s instinct. To catch up in her arms the sick child who
was dear to her, and hide it as if it were a criminal, and keep off all
ministration but such as her own ignorant tenderness and patience could
supply, had become this woman’s idea of maternal love, fidelity, and
duty. The shameful accounts we read, every week in the Christian year,
my lords and gentlemen and honourable boards, the infamous records of
small official inhumanity, do not pass by the people as they pass by
us. And hence these irrational, blind, and obstinate prejudices, so
astonishing to our magnificence, and having no more reason in them--God
save the Queen and Confound their politics--no, than smoke has in coming
from fire!

‘It’s not a right place for the poor child to stay in,’ said Mrs Boffin.
‘Tell us, dear Mr Rokesmith, what to do for the best.’

He had already thought what to do, and the consultation was very short.
He could pave the way, he said, in half an hour, and then they would go
down to Brentford. ‘Pray take me,’ said Bella. Therefore a carriage was
ordered, of capacity to take them all, and in the meantime Sloppy
was regaled, feasting alone in the Secretary’s room, with a complete
realization of that fairy vision--meat, beer, vegetables, and pudding.
In consequence of which his buttons became more importunate of public
notice than before, with the exception of two or three about the region
of the waistband, which modestly withdrew into a creasy retirement.

Punctual to the time, appeared the carriage and the Secretary. He sat
on the box, and Mr Sloppy graced the rumble. So, to the Three Magpies as
before: where Mrs Boffin and Miss Bella were handed out, and whence they
all went on foot to Mrs Betty Higden’s.

But, on the way down, they had stopped at a toy-shop, and had bought
that noble charger, a description of whose points and trappings had on
the last occasion conciliated the then worldly-minded orphan, and also a
Noah’s ark, and also a yellow bird with an artificial voice in him,
and also a military doll so well dressed that if he had only been of
life-size his brother-officers in the Guards might never have found him
out. Bearing these gifts, they raised the latch of Betty Higden’s door,
and saw her sitting in the dimmest and furthest corner with poor Johnny
in her lap.

‘And how’s my boy, Betty?’ asked Mrs Boffin, sitting down beside her.

‘He’s bad! He’s bad!’ said Betty. ‘I begin to be afeerd he’ll not be
yours any more than mine. All others belonging to him have gone to
the Power and the Glory, and I have a mind that they’re drawing him to
them--leading him away.’

‘No, no, no,’ said Mrs Boffin.

‘I don’t know why else he clenches his little hand as if it had hold of
a finger that I can’t see. Look at it,’ said Betty, opening the wrappers
in which the flushed child lay, and showing his small right hand lying
closed upon his breast. ‘It’s always so. It don’t mind me.’

‘Is he asleep?’

‘No, I think not. You’re not asleep, my Johnny?’

‘No,’ said Johnny, with a quiet air of pity for himself; and without
opening his eyes.

‘Here’s the lady, Johnny. And the horse.’

Johnny could bear the lady, with complete indifference, but not the
horse. Opening his heavy eyes, he slowly broke into a smile on beholding
that splendid phenomenon, and wanted to take it in his arms. As it was
much too big, it was put upon a chair where he could hold it by the mane
and contemplate it. Which he soon forgot to do.

But, Johnny murmuring something with his eyes closed, and Mrs Boffin
not knowing what, old Betty bent her ear to listen and took pains to
understand. Being asked by her to repeat what he had said, he did so two
or three times, and then it came out that he must have seen more than
they supposed when he looked up to see the horse, for the murmur was,
‘Who is the boofer lady?’ Now, the boofer, or beautiful, lady was Bella;
and whereas this notice from the poor baby would have touched her of
itself; it was rendered more pathetic by the late melting of her heart
to her poor little father, and their joke about the lovely woman. So,
Bella’s behaviour was very tender and very natural when she kneeled on
the brick floor to clasp the child, and when the child, with a child’s
admiration of what is young and pretty, fondled the boofer lady.

‘Now, my good dear Betty,’ said Mrs Boffin, hoping that she saw her
opportunity, and laying her hand persuasively on her arm; ‘we have come
to remove Johnny from this cottage to where he can be taken better care
of.’

Instantly, and before another word could be spoken, the old woman
started up with blazing eyes, and rushed at the door with the sick
child.

‘Stand away from me every one of ye!’ she cried out wildly. ‘I see what
ye mean now. Let me go my way, all of ye. I’d sooner kill the Pretty,
and kill myself!’

‘Stay, stay!’ said Rokesmith, soothing her. ‘You don’t understand.’

‘I understand too well. I know too much about it, sir. I’ve run from
it too many a year. No! Never for me, nor for the child, while there’s
water enough in England to cover us!’

The terror, the shame, the passion of horror and repugnance, firing the
worn face and perfectly maddening it, would have been a quite terrible
sight, if embodied in one old fellow-creature alone. Yet it ‘crops
up’--as our slang goes--my lords and gentlemen and honourable boards, in
other fellow-creatures, rather frequently!

‘It’s been chasing me all my life, but it shall never take me nor mine
alive!’ cried old Betty. ‘I’ve done with ye. I’d have fastened door and
window and starved out, afore I’d ever have let ye in, if I had known
what ye came for!’

But, catching sight of Mrs Boffin’s wholesome face, she relented, and
crouching down by the door and bending over her burden to hush it, said
humbly: ‘Maybe my fears has put me wrong. If they have so, tell me, and
the good Lord forgive me! I’m quick to take this fright, I know, and my
head is summ’at light with wearying and watching.’

‘There, there, there!’ returned Mrs Boffin. ‘Come, come! Say no more of
it, Betty. It was a mistake, a mistake. Any one of us might have made it
in your place, and felt just as you do.’

‘The Lord bless ye!’ said the old woman, stretching out her hand.

‘Now, see, Betty,’ pursued the sweet compassionate soul, holding the
hand kindly, ‘what I really did mean, and what I should have begun by
saying out, if I had only been a little wiser and handier. We want to
move Johnny to a place where there are none but children; a place set
up on purpose for sick children; where the good doctors and nurses pass
their lives with children, talk to none but children, touch none but
children, comfort and cure none but children.’

‘Is there really such a place?’ asked the old woman, with a gaze of
wonder.

‘Yes, Betty, on my word, and you shall see it. If my home was a better
place for the dear boy, I’d take him to it; but indeed indeed it’s not.’

‘You shall take him,’ returned Betty, fervently kissing the comforting
hand, ‘where you will, my deary. I am not so hard, but that I believe
your face and voice, and I will, as long as I can see and hear.’

This victory gained, Rokesmith made haste to profit by it, for he saw
how woefully time had been lost. He despatched Sloppy to bring the
carriage to the door; caused the child to be carefully wrapped up; bade
old Betty get her bonnet on; collected the toys, enabling the little
fellow to comprehend that his treasures were to be transported with
him; and had all things prepared so easily that they were ready for
the carriage as soon as it appeared, and in a minute afterwards were
on their way. Sloppy they left behind, relieving his overcharged breast
with a paroxysm of mangling.

At the Children’s Hospital, the gallant steed, the Noah’s ark, yellow
bird, and the officer in the Guards, were made as welcome as their
child-owner. But the doctor said aside to Rokesmith, ‘This should have
been days ago. Too late!’

However, they were all carried up into a fresh airy room, and there
Johnny came to himself, out of a sleep or a swoon or whatever it was,
to find himself lying in a little quiet bed, with a little platform over
his breast, on which were already arranged, to give him heart and urge
him to cheer up, the Noah’s ark, the noble steed, and the yellow bird;
with the officer in the Guards doing duty over the whole, quite as much
to the satisfaction of his country as if he had been upon Parade. And at
the bed’s head was a coloured picture beautiful to see, representing as
it were another Johnny seated on the knee of some Angel surely who loved
little children. And, marvellous fact, to lie and stare at: Johnny had
become one of a little family, all in little quiet beds (except two
playing dominoes in little arm-chairs at a little table on the hearth):
and on all the little beds were little platforms whereon were to be
seen dolls’ houses, woolly dogs with mechanical barks in them not very
dissimilar from the artificial voice pervading the bowels of the yellow
bird, tin armies, Moorish tumblers, wooden tea things, and the riches of
the earth.

As Johnny murmured something in his placid admiration, the ministering
women at his bed’s head asked him what he said. It seemed that he wanted
to know whether all these were brothers and sisters of his? So they told
him yes. It seemed then, that he wanted to know whether God had brought
them all together there? So they told him yes again. They made out then,
that he wanted to know whether they would all get out of pain? So they
answered yes to that question likewise, and made him understand that the
reply included himself.

Johnny’s powers of sustaining conversation were as yet so very
imperfectly developed, even in a state of health, that in sickness they
were little more than monosyllabic. But, he had to be washed and tended,
and remedies were applied, and though those offices were far, far more
skilfully and lightly done than ever anything had been done for him in
his little life, so rough and short, they would have hurt and tired him
but for an amazing circumstance which laid hold of his attention. This
was no less than the appearance on his own little platform in pairs,
of All Creation, on its way into his own particular ark: the elephant
leading, and the fly, with a diffident sense of his size, politely
bringing up the rear. A very little brother lying in the next bed with a
broken leg, was so enchanted by this spectacle that his delight exalted
its enthralling interest; and so came rest and sleep.

‘I see you are not afraid to leave the dear child here, Betty,’
whispered Mrs Boffin.

‘No, ma’am. Most willingly, most thankfully, with all my heart and
soul.’

So, they kissed him, and left him there, and old Betty was to come back
early in the morning, and nobody but Rokesmith knew for certain how that
the doctor had said, ‘This should have been days ago. Too late!’

But, Rokesmith knowing it, and knowing that his bearing it in mind would
be acceptable thereafter to that good woman who had been the only light
in the childhood of desolate John Harmon dead and gone, resolved that
late at night he would go back to the bedside of John Harmon’s namesake,
and see how it fared with him.

The family whom God had brought together were not all asleep, but were
all quiet. From bed to bed, a light womanly tread and a pleasant fresh
face passed in the silence of the night. A little head would lift itself
up into the softened light here and there, to be kissed as the face went
by--for these little patients are very loving--and would then submit
itself to be composed to rest again. The mite with the broken leg was
restless, and moaned; but after a while turned his face towards Johnny’s
bed, to fortify himself with a view of the ark, and fell asleep. Over
most of the beds, the toys were yet grouped as the children had left
them when they last laid themselves down, and, in their innocent
grotesqueness and incongruity, they might have stood for the children’s
dreams.

The doctor came in too, to see how it fared with Johnny. And he and
Rokesmith stood together, looking down with compassion on him.

‘What is it, Johnny?’ Rokesmith was the questioner, and put an arm round
the poor baby as he made a struggle.

‘Him!’ said the little fellow. ‘Those!’

The doctor was quick to understand children, and, taking the horse,
the ark, the yellow bird, and the man in the Guards, from Johnny’s bed,
softly placed them on that of his next neighbour, the mite with the
broken leg.

With a weary and yet a pleased smile, and with an action as if he
stretched his little figure out to rest, the child heaved his body on
the sustaining arm, and seeking Rokesmith’s face with his lips, said:

‘A kiss for the boofer lady.’

Having now bequeathed all he had to dispose of, and arranged his affairs
in this world, Johnny, thus speaking, left it.



Chapter 10

A SUCCESSOR


Some of the Reverend Frank Milvey’s brethren had found themselves
exceedingly uncomfortable in their minds, because they were required to
bury the dead too hopefully. But, the Reverend Frank, inclining to the
belief that they were required to do one or two other things (say out of
nine-and-thirty) calculated to trouble their consciences rather more if
they would think as much about them, held his peace.

Indeed, the Reverend Frank Milvey was a forbearing man, who noticed many
sad warps and blights in the vineyard wherein he worked, and did not
profess that they made him savagely wise. He only learned that the more
he himself knew, in his little limited human way, the better he could
distantly imagine what Omniscience might know.

Wherefore, if the Reverend Frank had had to read the words that troubled
some of his brethren, and profitably touched innumerable hearts, in
a worse case than Johnny’s, he would have done so out of the pity and
humility of his soul. Reading them over Johnny, he thought of his own
six children, but not of his poverty, and read them with dimmed eyes.
And very seriously did he and his bright little wife, who had been
listening, look down into the small grave and walk home arm-in-arm.

There was grief in the aristocratic house, and there was joy in the
Bower. Mr Wegg argued, if an orphan were wanted, was he not an orphan
himself; and could a better be desired? And why go beating about
Brentford bushes, seeking orphans forsooth who had established no claims
upon you and made no sacrifices for you, when here was an orphan ready
to your hand who had given up in your cause, Miss Elizabeth, Master
George, Aunt Jane, and Uncle Parker?

Mr Wegg chuckled, consequently, when he heard the tidings. Nay, it was
afterwards affirmed by a witness who shall at present be nameless,
that in the seclusion of the Bower he poked out his wooden leg, in the
stage-ballet manner, and executed a taunting or triumphant pirouette on
the genuine leg remaining to him.

John Rokesmith’s manner towards Mrs Boffin at this time, was more the
manner of a young man towards a mother, than that of a Secretary towards
his employer’s wife. It had always been marked by a subdued affectionate
deference that seemed to have sprung up on the very day of his
engagement; whatever was odd in her dress or her ways had seemed to have
no oddity for him; he had sometimes borne a quietly-amused face in her
company, but still it had seemed as if the pleasure her genial temper
and radiant nature yielded him, could have been quite as naturally
expressed in a tear as in a smile. The completeness of his sympathy with
her fancy for having a little John Harmon to protect and rear, he
had shown in every act and word, and now that the kind fancy was
disappointed, he treated it with a manly tenderness and respect for
which she could hardly thank him enough.

‘But I do thank you, Mr Rokesmith,’ said Mrs Boffin, ‘and I thank you
most kindly. You love children.’

‘I hope everybody does.’

‘They ought,’ said Mrs Boffin; ‘but we don’t all of us do what we ought,
do us?’

John Rokesmith replied, ‘Some among us supply the short-comings of the
rest. You have loved children well, Mr Boffin has told me.’

‘Not a bit better than he has, but that’s his way; he puts all the good
upon me. You speak rather sadly, Mr Rokesmith.’

‘Do I?’

‘It sounds to me so. Were you one of many children?’ He shook his head.

‘An only child?’

‘No there was another. Dead long ago.’

‘Father or mother alive?’

‘Dead.’--

‘And the rest of your relations?’

‘Dead--if I ever had any living. I never heard of any.’

At this point of the dialogue Bella came in with a light step. She
paused at the door a moment, hesitating whether to remain or retire;
perplexed by finding that she was not observed.

‘Now, don’t mind an old lady’s talk,’ said Mrs Boffin, ‘but tell me. Are
you quite sure, Mr Rokesmith, that you have never had a disappointment
in love?’

‘Quite sure. Why do you ask me?’

‘Why, for this reason. Sometimes you have a kind of kept-down manner
with you, which is not like your age. You can’t be thirty?’

‘I am not yet thirty.’

Deeming it high time to make her presence known, Bella coughed here to
attract attention, begged pardon, and said she would go, fearing that
she interrupted some matter of business.

‘No, don’t go,’ rejoined Mrs Boffin, ‘because we are coming to business,
instead of having begun it, and you belong to it as much now, my dear
Bella, as I do. But I want my Noddy to consult with us. Would somebody
be so good as find my Noddy for me?’

Rokesmith departed on that errand, and presently returned accompanied by
Mr Boffin at his jog-trot. Bella felt a little vague trepidation as to
the subject-matter of this same consultation, until Mrs Boffin announced
it.

‘Now, you come and sit by me, my dear,’ said that worthy soul, taking
her comfortable place on a large ottoman in the centre of the room,
and drawing her arm through Bella’s; ‘and Noddy, you sit here, and Mr
Rokesmith you sit there. Now, you see, what I want to talk about, is
this. Mr and Mrs Milvey have sent me the kindest note possible (which
Mr Rokesmith just now read to me out aloud, for I ain’t good at
handwritings), offering to find me another little child to name and
educate and bring up. Well. This has set me thinking.’

[‘And she is a steam-ingein at it,’ murmured Mr Boffin, in an admiring
parenthesis, ‘when she once begins. It mayn’t be so easy to start her;
but once started, she’s a ingein.’)

‘--This has set me thinking, I say,’ repeated Mrs Boffin, cordially
beaming under the influence of her husband’s compliment, ‘and I have
thought two things. First of all, that I have grown timid of reviving
John Harmon’s name. It’s an unfortunate name, and I fancy I should
reproach myself if I gave it to another dear child, and it proved again
unlucky.’

‘Now, whether,’ said Mr Boffin, gravely propounding a case for his
Secretary’s opinion; ‘whether one might call that a superstition?’

‘It is a matter of feeling with Mrs Boffin,’ said Rokesmith, gently.
‘The name has always been unfortunate. It has now this new unfortunate
association connected with it. The name has died out. Why revive it?
Might I ask Miss Wilfer what she thinks?’

‘It has not been a fortunate name for me,’ said Bella, colouring--‘or
at least it was not, until it led to my being here--but that is not the
point in my thoughts. As we had given the name to the poor child, and as
the poor child took so lovingly to me, I think I should feel jealous of
calling another child by it. I think I should feel as if the name had
become endeared to me, and I had no right to use it so.’

‘And that’s your opinion?’ remarked Mr Boffin, observant of the
Secretary’s face and again addressing him.

‘I say again, it is a matter of feeling,’ returned the Secretary. ‘I
think Miss Wilfer’s feeling very womanly and pretty.’

‘Now, give us your opinion, Noddy,’ said Mrs Boffin.

‘My opinion, old lady,’ returned the Golden Dustman, ‘is your opinion.’

‘Then,’ said Mrs Boffin, ‘we agree not to revive John Harmon’s name, but
to let it rest in the grave. It is, as Mr Rokesmith says, a matter of
feeling, but Lor how many matters ARE matters of feeling! Well; and so
I come to the second thing I have thought of. You must know, Bella,
my dear, and Mr Rokesmith, that when I first named to my husband my
thoughts of adopting a little orphan boy in remembrance of John Harmon,
I further named to my husband that it was comforting to think that how
the poor boy would be benefited by John’s own money, and protected from
John’s own forlornness.’

‘Hear, hear!’ cried Mr Boffin. ‘So she did. Ancoar!’

‘No, not Ancoar, Noddy, my dear,’ returned Mrs Boffin, ‘because I am
going to say something else. I meant that, I am sure, as much as
I still mean it. But this little death has made me ask myself the
question, seriously, whether I wasn’t too bent upon pleasing myself.
Else why did I seek out so much for a pretty child, and a child quite to
my liking? Wanting to do good, why not do it for its own sake, and put
my tastes and likings by?’

‘Perhaps,’ said Bella; and perhaps she said it with some little
sensitiveness arising out of those old curious relations of hers towards
the murdered man; ‘perhaps, in reviving the name, you would not have
liked to give it to a less interesting child than the original. He
interested you very much.’

‘Well, my dear,’ returned Mrs Boffin, giving her a squeeze, ‘it’s kind
of you to find that reason out, and I hope it may have been so, and
indeed to a certain extent I believe it was so, but I am afraid not to
the whole extent. However, that don’t come in question now, because we
have done with the name.’

‘Laid it up as a remembrance,’ suggested Bella, musingly.

‘Much better said, my dear; laid it up as a remembrance. Well then; I
have been thinking if I take any orphan to provide for, let it not be
a pet and a plaything for me, but a creature to be helped for its own
sake.’

‘Not pretty then?’ said Bella.

‘No,’ returned Mrs Boffin, stoutly.

‘Nor prepossessing then?’ said Bella.

‘No,’ returned Mrs Boffin. ‘Not necessarily so. That’s as it may happen.
A well-disposed boy comes in my way who may be even a little wanting in
such advantages for getting on in life, but is honest and industrious
and requires a helping hand and deserves it. If I am very much in
earnest and quite determined to be unselfish, let me take care of HIM.’

Here the footman whose feelings had been hurt on the former occasion,
appeared, and crossing to Rokesmith apologetically announced the
objectionable Sloppy.

The four members of Council looked at one another, and paused. ‘Shall he
be brought here, ma’am?’ asked Rokesmith.

‘Yes,’ said Mrs Boffin. Whereupon the footman disappeared, reappeared
presenting Sloppy, and retired much disgusted.

The consideration of Mrs Boffin had clothed Mr Sloppy in a suit of
black, on which the tailor had received personal directions from
Rokesmith to expend the utmost cunning of his art, with a view to the
concealment of the cohering and sustaining buttons. But, so much
more powerful were the frailties of Sloppy’s form than the strongest
resources of tailoring science, that he now stood before the Council,
a perfect Argus in the way of buttons: shining and winking and gleaming
and twinkling out of a hundred of those eyes of bright metal, at the
dazzled spectators. The artistic taste of some unknown hatter had
furnished him with a hatband of wholesale capacity which was fluted
behind, from the crown of his hat to the brim, and terminated in a black
bunch, from which the imagination shrunk discomfited and the reason
revolted. Some special powers with which his legs were endowed, had
already hitched up his glossy trousers at the ankles, and bagged them at
the knees; while similar gifts in his arms had raised his coat-sleeves
from his wrists and accumulated them at his elbows. Thus set forth, with
the additional embellishments of a very little tail to his coat, and a
yawning gulf at his waistband, Sloppy stood confessed.

‘And how is Betty, my good fellow?’ Mrs Boffin asked him.

‘Thankee, mum,’ said Sloppy, ‘she do pretty nicely, and sending her
dooty and many thanks for the tea and all faviours and wishing to know
the family’s healths.’

‘Have you just come, Sloppy?’

‘Yes, mum.’

‘Then you have not had your dinner yet?’

‘No, mum. But I mean to it. For I ain’t forgotten your handsome orders
that I was never to go away without having had a good ‘un off of meat
and beer and pudding--no: there was four of ‘em, for I reckoned ‘em
up when I had ‘em; meat one, beer two, vegetables three, and which was
four?--Why, pudding, HE was four!’ Here Sloppy threw his head back,
opened his mouth wide, and laughed rapturously.

‘How are the two poor little Minders?’ asked Mrs Boffin.

‘Striking right out, mum, and coming round beautiful.’

Mrs Boffin looked on the other three members of Council, and then said,
beckoning with her finger:

‘Sloppy.’

‘Yes, mum.’

‘Come forward, Sloppy. Should you like to dine here every day?’

‘Off of all four on ‘em, mum? O mum!’ Sloppy’s feelings obliged him to
squeeze his hat, and contract one leg at the knee.

‘Yes. And should you like to be always taken care of here, if you were
industrious and deserving?’

‘Oh, mum!--But there’s Mrs Higden,’ said Sloppy, checking himself in his
raptures, drawing back, and shaking his head with very serious meaning.
‘There’s Mrs Higden. Mrs Higden goes before all. None can ever be better
friends to me than Mrs Higden’s been. And she must be turned for, must
Mrs Higden. Where would Mrs Higden be if she warn’t turned for!’ At the
mere thought of Mrs Higden in this inconceivable affliction, Mr Sloppy’s
countenance became pale, and manifested the most distressful emotions.

‘You are as right as right can be, Sloppy,’ said Mrs Boffin ‘and far be
it from me to tell you otherwise. It shall be seen to. If Betty Higden
can be turned for all the same, you shall come here and be taken care of
for life, and be made able to keep her in other ways than the turning.’

‘Even as to that, mum,’ answered the ecstatic Sloppy, ‘the turning might
be done in the night, don’t you see? I could be here in the day, and
turn in the night. I don’t want no sleep, I don’t. Or even if I any ways
should want a wink or two,’ added Sloppy, after a moment’s apologetic
reflection, ‘I could take ‘em turning. I’ve took ‘em turning many a
time, and enjoyed ‘em wonderful!’

On the grateful impulse of the moment, Mr Sloppy kissed Mrs Boffin’s
hand, and then detaching himself from that good creature that he might
have room enough for his feelings, threw back his head, opened his mouth
wide, and uttered a dismal howl. It was creditable to his tenderness of
heart, but suggested that he might on occasion give some offence to the
neighbours: the rather, as the footman looked in, and begged pardon,
finding he was not wanted, but excused himself; on the ground ‘that he
thought it was Cats.’



Chapter 11

SOME AFFAIRS OF THE HEART


Little Miss Peecher, from her little official dwelling-house, with its
little windows like the eyes in needles, and its little doors like the
covers of school-books, was very observant indeed of the object of her
quiet affections. Love, though said to be afflicted with blindness, is
a vigilant watchman, and Miss Peecher kept him on double duty over Mr
Bradley Headstone. It was not that she was naturally given to playing
the spy--it was not that she was at all secret, plotting, or mean--it
was simply that she loved the irresponsive Bradley with all the
primitive and homely stock of love that had never been examined or
certificated out of her. If her faithful slate had had the latent
qualities of sympathetic paper, and its pencil those of invisible ink,
many a little treatise calculated to astonish the pupils would have come
bursting through the dry sums in school-time under the warming influence
of Miss Peecher’s bosom. For, oftentimes when school was not, and her
calm leisure and calm little house were her own, Miss Peecher would
commit to the confidential slate an imaginary description of how, upon
a balmy evening at dusk, two figures might have been observed in the
market-garden ground round the corner, of whom one, being a manly form,
bent over the other, being a womanly form of short stature and some
compactness, and breathed in a low voice the words, ‘Emma Peecher, wilt
thou be my own?’ after which the womanly form’s head reposed upon the
manly form’s shoulder, and the nightingales tuned up. Though all unseen,
and unsuspected by the pupils, Bradley Headstone even pervaded the
school exercises. Was Geography in question? He would come triumphantly
flying out of Vesuvius and Aetna ahead of the lava, and would boil
unharmed in the hot springs of Iceland, and would float majestically
down the Ganges and the Nile. Did History chronicle a king of men?
Behold him in pepper-and-salt pantaloons, with his watch-guard round
his neck. Were copies to be written? In capital B’s and H’s most of the
girls under Miss Peecher’s tuition were half a year ahead of every other
letter in the alphabet. And Mental Arithmetic, administered by Miss
Peecher, often devoted itself to providing Bradley Headstone with a
wardrobe of fabulous extent: fourscore and four neck-ties at two and
ninepence-halfpenny, two gross of silver watches at four pounds fifteen
and sixpence, seventy-four black hats at eighteen shillings; and many
similar superfluities.

The vigilant watchman, using his daily opportunities of turning his eyes
in Bradley’s direction, soon apprized Miss Peecher that Bradley was more
preoccupied than had been his wont, and more given to strolling about
with a downcast and reserved face, turning something difficult in his
mind that was not in the scholastic syllabus. Putting this and that
together--combining under the head ‘this,’ present appearances and the
intimacy with Charley Hexam, and ranging under the head ‘that’ the
visit to his sister, the watchman reported to Miss Peecher his strong
suspicions that the sister was at the bottom of it.

‘I wonder,’ said Miss Peecher, as she sat making up her weekly report on
a half-holiday afternoon, ‘what they call Hexam’s sister?’

Mary Anne, at her needlework, attendant and attentive, held her arm up.

‘Well, Mary Anne?’

‘She is named Lizzie, ma’am.’

‘She can hardly be named Lizzie, I think, Mary Anne,’ returned Miss
Peecher, in a tunefully instructive voice. ‘Is Lizzie a Christian name,
Mary Anne?’

Mary Anne laid down her work, rose, hooked herself behind, as being
under catechization, and replied: ‘No, it is a corruption, Miss
Peecher.’

‘Who gave her that name?’ Miss Peecher was going on, from the mere force
of habit, when she checked herself; on Mary Anne’s evincing theological
impatience to strike in with her godfathers and her godmothers, and
said: ‘I mean of what name is it a corruption?’

‘Elizabeth, or Eliza, Miss Peecher.’

‘Right, Mary Anne. Whether there were any Lizzies in the early Christian
Church must be considered very doubtful, very doubtful.’ Miss Peecher
was exceedingly sage here. ‘Speaking correctly, we say, then, that
Hexam’s sister is called Lizzie; not that she is named so. Do we not,
Mary Anne?’

‘We do, Miss Peecher.’

‘And where,’ pursued Miss Peecher, complacent in her little transparent
fiction of conducting the examination in a semiofficial manner for Mary
Anne’s benefit, not her own, ‘where does this young woman, who is called
but not named Lizzie, live? Think, now, before answering.’

‘In Church Street, Smith Square, by Mill Bank, ma’am.’

‘In Church Street, Smith Square, by Mill Bank,’ repeated Miss Peecher,
as if possessed beforehand of the book in which it was written. Exactly
so. And what occupation does this young woman pursue, Mary Anne? Take
time.’

‘She has a place of trust at an outfitter’s in the City, ma’am.’

‘Oh!’ said Miss Peecher, pondering on it; but smoothly added, in a
confirmatory tone, ‘At an outfitter’s in the City. Ye-es?’

‘And Charley--’ Mary Anne was proceeding, when Miss Peecher stared.

‘I mean Hexam, Miss Peecher.’

‘I should think you did, Mary Anne. I am glad to hear you do. And
Hexam--’

‘Says,’ Mary Anne went on, ‘that he is not pleased with his sister, and
that his sister won’t be guided by his advice, and persists in being
guided by somebody else’s; and that--’

‘Mr Headstone coming across the garden!’ exclaimed Miss Peecher, with a
flushed glance at the looking-glass. ‘You have answered very well, Mary
Anne. You are forming an excellent habit of arranging your thoughts
clearly. That will do.’

The discreet Mary Anne resumed her seat and her silence, and stitched,
and stitched, and was stitching when the schoolmaster’s shadow came in
before him, announcing that he might be instantly expected.

‘Good evening, Miss Peecher,’ he said, pursuing the shadow, and taking
its place.

‘Good evening, Mr Headstone. Mary Anne, a chair.’

‘Thank you,’ said Bradley, seating himself in his constrained manner.
‘This is but a flying visit. I have looked in, on my way, to ask a
kindness of you as a neighbour.’

‘Did you say on your way, Mr Headstone?’ asked Miss Peecher.

‘On my way to--where I am going.’

‘Church Street, Smith Square, by Mill Bank,’ repeated Miss Peecher, in
her own thoughts.

‘Charley Hexam has gone to get a book or two he wants, and will probably
be back before me. As we leave my house empty, I took the liberty of
telling him I would leave the key here. Would you kindly allow me to do
so?’

‘Certainly, Mr Headstone. Going for an evening walk, sir?’

‘Partly for a walk, and partly for--on business.’

‘Business in Church Street, Smith Square, by Mill Bank,’ repeated Miss
Peecher to herself.

‘Having said which,’ pursued Bradley, laying his door-key on the table,
‘I must be already going. There is nothing I can do for you, Miss
Peecher?’

‘Thank you, Mr Headstone. In which direction?’

‘In the direction of Westminster.’

‘Mill Bank,’ Miss Peecher repeated in her own thoughts once again. ‘No,
thank you, Mr Headstone; I’ll not trouble you.’

‘You couldn’t trouble me,’ said the schoolmaster.

‘Ah!’ returned Miss Peecher, though not aloud; ‘but you can trouble
ME!’ And for all her quiet manner, and her quiet smile, she was full of
trouble as he went his way.

She was right touching his destination. He held as straight a course
for the house of the dolls’ dressmaker as the wisdom of his ancestors,
exemplified in the construction of the intervening streets, would let
him, and walked with a bent head hammering at one fixed idea. It had
been an immoveable idea since he first set eyes upon her. It seemed to
him as if all that he could suppress in himself he had suppressed, as
if all that he could restrain in himself he had restrained, and the time
had come--in a rush, in a moment--when the power of self-command had
departed from him. Love at first sight is a trite expression quite
sufficiently discussed; enough that in certain smouldering natures like
this man’s, that passion leaps into a blaze, and makes such head as fire
does in a rage of wind, when other passions, but for its mastery, could
be held in chains. As a multitude of weak, imitative natures are
always lying by, ready to go mad upon the next wrong idea that may be
broached--in these times, generally some form of tribute to Somebody
for something that never was done, or, if ever done, that was done by
Somebody Else--so these less ordinary natures may lie by for years,
ready on the touch of an instant to burst into flame.

The schoolmaster went his way, brooding and brooding, and a sense of
being vanquished in a struggle might have been pieced out of his worried
face. Truly, in his breast there lingered a resentful shame to find
himself defeated by this passion for Charley Hexam’s sister, though in
the very self-same moments he was concentrating himself upon the object
of bringing the passion to a successful issue.

He appeared before the dolls’ dressmaker, sitting alone at her work.
‘Oho!’ thought that sharp young personage, ‘it’s you, is it? I know your
tricks and your manners, my friend!’

‘Hexam’s sister,’ said Bradley Headstone, ‘is not come home yet?’

‘You are quite a conjuror,’ returned Miss Wren.

‘I will wait, if you please, for I want to speak to her.’

‘Do you?’ returned Miss Wren. ‘Sit down. I hope it’s mutual.’ Bradley
glanced distrustfully at the shrewd face again bending over the work,
and said, trying to conquer doubt and hesitation:

‘I hope you don’t imply that my visit will be unacceptable to Hexam’s
sister?’

‘There! Don’t call her that. I can’t bear you to call her that,’
returned Miss Wren, snapping her fingers in a volley of impatient snaps,
‘for I don’t like Hexam.’

‘Indeed?’

‘No.’ Miss Wren wrinkled her nose, to express dislike. ‘Selfish. Thinks
only of himself. The way with all of you.’

‘The way with all of us? Then you don’t like ME?’

‘So-so,’ replied Miss Wren, with a shrug and a laugh. ‘Don’t know much
about you.’

‘But I was not aware it was the way with all of us,’ said Bradley,
returning to the accusation, a little injured. ‘Won’t you say, some of
us?’

‘Meaning,’ returned the little creature, ‘every one of you, but you.
Hah! Now look this lady in the face. This is Mrs Truth. The Honourable.
Full-dressed.’

Bradley glanced at the doll she held up for his observation--which had
been lying on its face on her bench, while with a needle and thread she
fastened the dress on at the back--and looked from it to her.

‘I stand the Honourable Mrs T. on my bench in this corner against the
wall, where her blue eyes can shine upon you,’ pursued Miss Wren, doing
so, and making two little dabs at him in the air with her needle, as
if she pricked him with it in his own eyes; ‘and I defy you to tell me,
with Mrs T. for a witness, what you have come here for.’

‘To see Hexam’s sister.’

‘You don’t say so!’ retorted Miss Wren, hitching her chin. ‘But on whose
account?’

‘Her own.’

‘O Mrs T.!’ exclaimed Miss Wren. ‘You hear him!’

‘To reason with her,’ pursued Bradley, half humouring what was present,
and half angry with what was not present; ‘for her own sake.’

‘Oh Mrs T.!’ exclaimed the dressmaker.

‘For her own sake,’ repeated Bradley, warming, ‘and for her brother’s,
and as a perfectly disinterested person.’

‘Really, Mrs T.,’ remarked the dressmaker, ‘since it comes to this, we
must positively turn you with your face to the wall.’ She had hardly
done so, when Lizzie Hexam arrived, and showed some surprise on seeing
Bradley Headstone there, and Jenny shaking her little fist at him close
before her eyes, and the Honourable Mrs T. with her face to the wall.

‘Here’s a perfectly disinterested person, Lizzie dear,’ said the knowing
Miss Wren, ‘come to talk with you, for your own sake and your brother’s.
Think of that. I am sure there ought to be no third party present at
anything so very kind and so very serious; and so, if you’ll remove the
third party upstairs, my dear, the third party will retire.’

Lizzie took the hand which the dolls’ dressmaker held out to her for
the purpose of being supported away, but only looked at her with an
inquiring smile, and made no other movement.

‘The third party hobbles awfully, you know, when she’s left to herself;’
said Miss Wren, ‘her back being so bad, and her legs so queer; so she
can’t retire gracefully unless you help her, Lizzie.’

‘She can do no better than stay where she is,’ returned Lizzie,
releasing the hand, and laying her own lightly on Miss Jenny’s curls.
And then to Bradley: ‘From Charley, sir?’

In an irresolute way, and stealing a clumsy look at her, Bradley rose to
place a chair for her, and then returned to his own.

‘Strictly speaking,’ said he, ‘I come from Charley, because I left him
only a little while ago; but I am not commissioned by Charley. I come of
my own spontaneous act.’

With her elbows on her bench, and her chin upon her hands, Miss Jenny
Wren sat looking at him with a watchful sidelong look. Lizzie, in her
different way, sat looking at him too.

‘The fact is,’ began Bradley, with a mouth so dry that he had some
difficulty in articulating his words: the consciousness of which
rendered his manner still more ungainly and undecided; ‘the truth is,
that Charley, having no secrets from me (to the best of my belief), has
confided the whole of this matter to me.’

He came to a stop, and Lizzie asked: ‘what matter, sir?’

‘I thought,’ returned the schoolmaster, stealing another look at her,
and seeming to try in vain to sustain it; for the look dropped as it
lighted on her eyes, ‘that it might be so superfluous as to be almost
impertinent, to enter upon a definition of it. My allusion was to this
matter of your having put aside your brother’s plans for you, and
given the preference to those of Mr--I believe the name is Mr Eugene
Wrayburn.’

He made this point of not being certain of the name, with another uneasy
look at her, which dropped like the last.

Nothing being said on the other side, he had to begin again, and began
with new embarrassment.

‘Your brother’s plans were communicated to me when he first had them in
his thoughts. In point of fact he spoke to me about them when I was
last here--when we were walking back together, and when I--when the
impression was fresh upon me of having seen his sister.’

There might have been no meaning in it, but the little dressmaker here
removed one of her supporting hands from her chin, and musingly turned
the Honourable Mrs T. with her face to the company. That done, she fell
into her former attitude.

‘I approved of his idea,’ said Bradley, with his uneasy look wandering
to the doll, and unconsciously resting there longer than it had
rested on Lizzie, ‘both because your brother ought naturally to be the
originator of any such scheme, and because I hoped to be able to promote
it. I should have had inexpressible pleasure, I should have taken
inexpressible interest, in promoting it. Therefore I must acknowledge
that when your brother was disappointed, I too was disappointed. I wish
to avoid reservation or concealment, and I fully acknowledge that.’

He appeared to have encouraged himself by having got so far. At all
events he went on with much greater firmness and force of emphasis:
though with a curious disposition to set his teeth, and with a curious
tight-screwing movement of his right hand in the clenching palm of his
left, like the action of one who was being physically hurt, and was
unwilling to cry out.

‘I am a man of strong feelings, and I have strongly felt this
disappointment. I do strongly feel it. I don’t show what I feel; some
of us are obliged habitually to keep it down. To keep it down. But to
return to your brother. He has taken the matter so much to heart that
he has remonstrated (in my presence he remonstrated) with Mr Eugene
Wrayburn, if that be the name. He did so, quite ineffectually. As any
one not blinded to the real character of Mr--Mr Eugene Wrayburn--would
readily suppose.’

He looked at Lizzie again, and held the look. And his face turned from
burning red to white, and from white back to burning red, and so for the
time to lasting deadly white.

‘Finally, I resolved to come here alone, and appeal to you. I resolved
to come here alone, and entreat you to retract the course you have
chosen, and instead of confiding in a mere stranger--a person of most
insolent behaviour to your brother and others--to prefer your brother
and your brother’s friend.’

Lizzie Hexam had changed colour when those changes came over him, and
her face now expressed some anger, more dislike, and even a touch of
fear. But she answered him very steadily.

‘I cannot doubt, Mr Headstone, that your visit is well meant. You have
been so good a friend to Charley that I have no right to doubt it. I
have nothing to tell Charley, but that I accepted the help to which he
so much objects before he made any plans for me; or certainly before I
knew of any. It was considerately and delicately offered, and there were
reasons that had weight with me which should be as dear to Charley as to
me. I have no more to say to Charley on this subject.’

His lips trembled and stood apart, as he followed this repudiation of
himself; and limitation of her words to her brother.

‘I should have told Charley, if he had come to me,’ she resumed, as
though it were an after-thought, ‘that Jenny and I find our teacher very
able and very patient, and that she takes great pains with us. So much
so, that we have said to her we hope in a very little while to be able
to go on by ourselves. Charley knows about teachers, and I should also
have told him, for his satisfaction, that ours comes from an institution
where teachers are regularly brought up.’

‘I should like to ask you,’ said Bradley Headstone, grinding his words
slowly out, as though they came from a rusty mill; ‘I should like to
ask you, if I may without offence, whether you would have objected--no;
rather, I should like to say, if I may without offence, that I wish I
had had the opportunity of coming here with your brother and devoting my
poor abilities and experience to your service.’

‘Thank you, Mr Headstone.’

‘But I fear,’ he pursued, after a pause, furtively wrenching at the seat
of his chair with one hand, as if he would have wrenched the chair to
pieces, and gloomily observing her while her eyes were cast down, ‘that
my humble services would not have found much favour with you?’

She made no reply, and the poor stricken wretch sat contending with
himself in a heat of passion and torment. After a while he took out his
handkerchief and wiped his forehead and hands.

‘There is only one thing more I had to say, but it is the most
important. There is a reason against this matter, there is a personal
relation concerned in this matter, not yet explained to you. It might--I
don’t say it would--it might--induce you to think differently. To
proceed under the present circumstances is out of the question. Will you
please come to the understanding that there shall be another interview
on the subject?’

‘With Charley, Mr Headstone?’

‘With--well,’ he answered, breaking off, ‘yes! Say with him too.
Will you please come to the understanding that there must be another
interview under more favourable circumstances, before the whole case can
be submitted?’

‘I don’t,’ said Lizzie, shaking her head, ‘understand your meaning, Mr
Headstone.’

‘Limit my meaning for the present,’ he interrupted, ‘to the whole case
being submitted to you in another interview.’

‘What case, Mr Headstone? What is wanting to it?’

‘You--you shall be informed in the other interview.’ Then he said, as
if in a burst of irrepressible despair, ‘I--I leave it all incomplete!
There is a spell upon me, I think!’ And then added, almost as if he
asked for pity, ‘Good-night!’

He held out his hand. As she, with manifest hesitation, not to say
reluctance, touched it, a strange tremble passed over him, and his face,
so deadly white, was moved as by a stroke of pain. Then he was gone.

The dolls’ dressmaker sat with her attitude unchanged, eyeing the door
by which he had departed, until Lizzie pushed her bench aside and sat
down near her. Then, eyeing Lizzie as she had previously eyed Bradley
and the door, Miss Wren chopped that very sudden and keen chop in which
her jaws sometimes indulged, leaned back in her chair with folded arms,
and thus expressed herself:

‘Humph! If he--I mean, of course, my dear, the party who is coming to
court me when the time comes--should be THAT sort of man, he may spare
himself the trouble. HE wouldn’t do to be trotted about and made useful.
He’d take fire and blow up while he was about it.’

‘And so you would be rid of him,’ said Lizzie, humouring her.

‘Not so easily,’ returned Miss Wren. ‘He wouldn’t blow up alone. He’d
carry me up with him. I know his tricks and his manners.’

‘Would he want to hurt you, do you mean?’ asked Lizzie.

‘Mightn’t exactly want to do it, my dear,’ returned Miss Wren; ‘but a
lot of gunpowder among lighted lucifer-matches in the next room might
almost as well be here.’

‘He is a very strange man,’ said Lizzie, thoughtfully.

‘I wish he was so very strange a man as to be a total stranger,’
answered the sharp little thing.

It being Lizzie’s regular occupation when they were alone of an evening
to brush out and smooth the long fair hair of the dolls’ dressmaker, she
unfastened a ribbon that kept it back while the little creature was at
her work, and it fell in a beautiful shower over the poor shoulders that
were much in need of such adorning rain. ‘Not now, Lizzie, dear,’ said
Jenny; ‘let us have a talk by the fire.’ With those words, she in her
turn loosened her friend’s dark hair, and it dropped of its own weight
over her bosom, in two rich masses. Pretending to compare the colours
and admire the contrast, Jenny so managed a mere touch or two of her
nimble hands, as that she herself laying a cheek on one of the dark
folds, seemed blinded by her own clustering curls to all but the fire,
while the fine handsome face and brow of Lizzie were revealed without
obstruction in the sombre light.

‘Let us have a talk,’ said Jenny, ‘about Mr Eugene Wrayburn.’

Something sparkled down among the fair hair resting on the dark hair;
and if it were not a star--which it couldn’t be--it was an eye; and
if it were an eye, it was Jenny Wren’s eye, bright and watchful as the
bird’s whose name she had taken.

‘Why about Mr Wrayburn?’ Lizzie asked.

‘For no better reason than because I’m in the humour. I wonder whether
he’s rich!’

‘No, not rich.’

‘Poor?’

‘I think so, for a gentleman.’

‘Ah! To be sure! Yes, he’s a gentleman. Not of our sort; is he?’ A shake
of the head, a thoughtful shake of the head, and the answer, softly
spoken, ‘Oh no, oh no!’

The dolls’ dressmaker had an arm round her friend’s waist. Adjusting the
arm, she slyly took the opportunity of blowing at her own hair where
it fell over her face; then the eye down there, under lighter shadows
sparkled more brightly and appeared more watchful.

‘When He turns up, he shan’t be a gentleman; I’ll very soon send him
packing, if he is. However, he’s not Mr Wrayburn; I haven’t captivated
HIM. I wonder whether anybody has, Lizzie!’

‘It is very likely.’

‘Is it very likely? I wonder who!’

‘Is it not very likely that some lady has been taken by him, and that he
may love her dearly?’

‘Perhaps. I don’t know. What would you think of him, Lizzie, if you were
a lady?’

‘I a lady!’ she repeated, laughing. ‘Such a fancy!’

‘Yes. But say: just as a fancy, and for instance.’

‘I a lady! I, a poor girl who used to row poor father on the river. I,
who had rowed poor father out and home on the very night when I saw him
for the first time. I, who was made so timid by his looking at me, that
I got up and went out!’

[‘He did look at you, even that night, though you were not a lady!’
thought Miss Wren.)

‘I a lady!’ Lizzie went on in a low voice, with her eyes upon the fire.
‘I, with poor father’s grave not even cleared of undeserved stain and
shame, and he trying to clear it for me! I a lady!’

‘Only as a fancy, and for instance,’ urged Miss Wren.

‘Too much, Jenny, dear, too much! My fancy is not able to get that far.’
As the low fire gleamed upon her, it showed her smiling, mournfully and
abstractedly.

‘But I am in the humour, and I must be humoured, Lizzie, because after
all I am a poor little thing, and have had a hard day with my bad child.
Look in the fire, as I like to hear you tell how you used to do when you
lived in that dreary old house that had once been a windmill. Look in
the--what was its name when you told fortunes with your brother that I
DON’T like?’

‘The hollow down by the flare?’

‘Ah! That’s the name! You can find a lady there, I know.’

‘More easily than I can make one of such material as myself, Jenny.’

The sparkling eye looked steadfastly up, as the musing face looked
thoughtfully down. ‘Well?’ said the dolls’ dressmaker, ‘We have found
our lady?’

Lizzie nodded, and asked, ‘Shall she be rich?’

‘She had better be, as he’s poor.’

‘She is very rich. Shall she be handsome?’

‘Even you can be that, Lizzie, so she ought to be.’

‘She is very handsome.’

‘What does she say about him?’ asked Miss Jenny, in a low voice:
watchful, through an intervening silence, of the face looking down at
the fire.

‘She is glad, glad, to be rich, that he may have the money. She is glad,
glad, to be beautiful, that he may be proud of her. Her poor heart--’

‘Eh? Her poor heart?’ said Miss Wren.

‘Her heart--is given him, with all its love and truth. She would
joyfully die with him, or, better than that, die for him. She knows he
has failings, but she thinks they have grown up through his being like
one cast away, for the want of something to trust in, and care for, and
think well of. And she says, that lady rich and beautiful that I can
never come near, “Only put me in that empty place, only try how little
I mind myself, only prove what a world of things I will do and bear for
you, and I hope that you might even come to be much better than you are,
through me who am so much worse, and hardly worth the thinking of beside
you.”’

As the face looking at the fire had become exalted and forgetful in the
rapture of these words, the little creature, openly clearing away
her fair hair with her disengaged hand, had gazed at it with earnest
attention and something like alarm. Now that the speaker ceased, the
little creature laid down her head again, and moaned, ‘O me, O me, O
me!’

‘In pain, dear Jenny?’ asked Lizzie, as if awakened.

‘Yes, but not the old pain. Lay me down, lay me down. Don’t go out of
my sight to-night. Lock the door and keep close to me.’ Then turning away
her face, she said in a whisper to herself, ‘My Lizzie, my poor Lizzie!
O my blessed children, come back in the long bright slanting rows, and
come for her, not me. She wants help more than I, my blessed children!’

She had stretched her hands up with that higher and better look, and
now she turned again, and folded them round Lizzie’s neck, and rocked
herself on Lizzie’s breast.



Chapter 12

MORE BIRDS OF PREY


Rogue Riderhood dwelt deep and dark in Limehouse Hole, among the
riggers, and the mast, oar and block makers, and the boat-builders, and
the sail-lofts, as in a kind of ship’s hold stored full of waterside
characters, some no better than himself, some very much better, and
none much worse. The Hole, albeit in a general way not over nice in
its choice of company, was rather shy in reference to the honour of
cultivating the Rogue’s acquaintance; more frequently giving him the
cold shoulder than the warm hand, and seldom or never drinking with him
unless at his own expense. A part of the Hole, indeed, contained so
much public spirit and private virtue that not even this strong leverage
could move it to good fellowship with a tainted accuser. But, there may
have been the drawback on this magnanimous morality, that its exponents
held a true witness before Justice to be the next unneighbourly and
accursed character to a false one.

Had it not been for the daughter whom he often mentioned, Mr Riderhood
might have found the Hole a mere grave as to any means it would yield
him of getting a living. But Miss Pleasant Riderhood had some little
position and connection in Limehouse Hole. Upon the smallest of small
scales, she was an unlicensed pawnbroker, keeping what was popularly
called a Leaving Shop, by lending insignificant sums on insignificant
articles of property deposited with her as security. In her
four-and-twentieth year of life, Pleasant was already in her fifth year
of this way of trade. Her deceased mother had established the business,
and on that parent’s demise she had appropriated a secret capital of
fifteen shillings to establishing herself in it; the existence of
such capital in a pillow being the last intelligible confidential
communication made to her by the departed, before succumbing to
dropsical conditions of snuff and gin, incompatible equally with
coherence and existence.

Why christened Pleasant, the late Mrs Riderhood might possibly have
been at some time able to explain, and possibly not. Her daughter had no
information on that point. Pleasant she found herself, and she couldn’t
help it. She had not been consulted on the question, any more than on
the question of her coming into these terrestrial parts, to want a name.
Similarly, she found herself possessed of what is colloquially termed
a swivel eye (derived from her father), which she might perhaps have
declined if her sentiments on the subject had been taken. She was not
otherwise positively ill-looking, though anxious, meagre, of a muddy
complexion, and looking as old again as she really was.

As some dogs have it in the blood, or are trained, to worry certain
creatures to a certain point, so--not to make the comparison
disrespectfully--Pleasant Riderhood had it in the blood, or had been
trained, to regard seamen, within certain limits, as her prey. Show
her a man in a blue jacket, and, figuratively speaking, she pinned him
instantly. Yet, all things considered, she was not of an evil mind or an
unkindly disposition. For, observe how many things were to be considered
according to her own unfortunate experience. Show Pleasant Riderhood a
Wedding in the street, and she only saw two people taking out a regular
licence to quarrel and fight. Show her a Christening, and she saw a
little heathen personage having a quite superfluous name bestowed upon
it, inasmuch as it would be commonly addressed by some abusive epithet:
which little personage was not in the least wanted by anybody, and would
be shoved and banged out of everybody’s way, until it should grow
big enough to shove and bang. Show her a Funeral, and she saw an
unremunerative ceremony in the nature of a black masquerade, conferring
a temporary gentility on the performers, at an immense expense, and
representing the only formal party ever given by the deceased. Show her
a live father, and she saw but a duplicate of her own father, who from
her infancy had been taken with fits and starts of discharging his duty
to her, which duty was always incorporated in the form of a fist or a
leathern strap, and being discharged hurt her. All things considered,
therefore, Pleasant Riderhood was not so very, very bad. There was even
a touch of romance in her--of such romance as could creep into Limehouse
Hole--and maybe sometimes of a summer evening, when she stood with
folded arms at her shop-door, looking from the reeking street to the
sky where the sun was setting, she may have had some vaporous visions
of far-off islands in the southern seas or elsewhere (not being
geographically particular), where it would be good to roam with a
congenial partner among groves of bread-fruit, waiting for ships to be
wafted from the hollow ports of civilization. For, sailors to be got the
better of, were essential to Miss Pleasant’s Eden.

Not on a summer evening did she come to her little shop-door, when a
certain man standing over against the house on the opposite side of
the street took notice of her. That was on a cold shrewd windy evening,
after dark. Pleasant Riderhood shared with most of the lady inhabitants
of the Hole, the peculiarity that her hair was a ragged knot, constantly
coming down behind, and that she never could enter upon any undertaking
without first twisting it into place. At that particular moment, being
newly come to the threshold to take a look out of doors, she was winding
herself up with both hands after this fashion. And so prevalent was the
fashion, that on the occasion of a fight or other disturbance in the
Hole, the ladies would be seen flocking from all quarters universally
twisting their back-hair as they came along, and many of them, in the
hurry of the moment, carrying their back-combs in their mouths.

It was a wretched little shop, with a roof that any man standing in it
could touch with his hand; little better than a cellar or cave, down
three steps. Yet in its ill-lighted window, among a flaring handkerchief
or two, an old peacoat or so, a few valueless watches and compasses, a
jar of tobacco and two crossed pipes, a bottle of walnut ketchup, and
some horrible sweets these creature discomforts serving as a blind to
the main business of the Leaving Shop--was displayed the inscription
SEAMAN’S BOARDING-HOUSE.

Taking notice of Pleasant Riderhood at the door, the man crossed so
quickly that she was still winding herself up, when he stood close
before her.

‘Is your father at home?’ said he.

‘I think he is,’ returned Pleasant, dropping her arms; ‘come in.’

It was a tentative reply, the man having a seafaring appearance. Her
father was not at home, and Pleasant knew it. ‘Take a seat by the fire,’
were her hospitable words when she had got him in; ‘men of your calling
are always welcome here.’

‘Thankee,’ said the man.

His manner was the manner of a sailor, and his hands were the hands of
a sailor, except that they were smooth. Pleasant had an eye for sailors,
and she noticed the unused colour and texture of the hands, sunburnt
though they were, as sharply as she noticed their unmistakable looseness
and suppleness, as he sat himself down with his left arm carelessly
thrown across his left leg a little above the knee, and the right arm
as carelessly thrown over the elbow of the wooden chair, with the hand
curved, half open and half shut, as if it had just let go a rope.

‘Might you be looking for a Boarding-House?’ Pleasant inquired, taking
her observant stand on one side of the fire.

‘I don’t rightly know my plans yet,’ returned the man.

‘You ain’t looking for a Leaving Shop?’

‘No,’ said the man.

‘No,’ assented Pleasant, ‘you’ve got too much of an outfit on you for
that. But if you should want either, this is both.’

‘Ay, ay!’ said the man, glancing round the place. ‘I know. I’ve been
here before.’

‘Did you Leave anything when you were here before?’ asked Pleasant, with
a view to principal and interest.

‘No.’ The man shook his head.

‘I am pretty sure you never boarded here?’

‘No.’ The man again shook his head.

‘What DID you do here when you were here before?’ asked Pleasant. ‘For I
don’t remember you.’

‘It’s not at all likely you should. I only stood at the door, one
night--on the lower step there--while a shipmate of mine looked in to
speak to your father. I remember the place well.’ Looking very curiously
round it.

‘Might that have been long ago?’

‘Ay, a goodish bit ago. When I came off my last voyage.’

‘Then you have not been to sea lately?’

‘No. Been in the sick bay since then, and been employed ashore.’

‘Then, to be sure, that accounts for your hands.’

The man with a keen look, a quick smile, and a change of manner, caught
her up. ‘You’re a good observer. Yes. That accounts for my hands.’

Pleasant was somewhat disquieted by his look, and returned it
suspiciously. Not only was his change of manner, though very sudden,
quite collected, but his former manner, which he resumed, had a
certain suppressed confidence and sense of power in it that were half
threatening.

‘Will your father be long?’ he inquired.

‘I don’t know. I can’t say.’

‘As you supposed he was at home, it would seem that he has just gone
out? How’s that?’

‘I supposed he had come home,’ Pleasant explained.

‘Oh! You supposed he had come home? Then he has been some time out?
How’s that?’

‘I don’t want to deceive you. Father’s on the river in his boat.’

‘At the old work?’ asked the man.

‘I don’t know what you mean,’ said Pleasant, shrinking a step back.
‘What on earth d’ye want?’

‘I don’t want to hurt your father. I don’t want to say I might, if I
chose. I want to speak to him. Not much in that, is there? There shall
be no secrets from you; you shall be by. And plainly, Miss Riderhood,
there’s nothing to be got out of me, or made of me. I am not good for
the Leaving Shop, I am not good for the Boarding-House, I am not good
for anything in your way to the extent of sixpenn’orth of halfpence. Put
the idea aside, and we shall get on together.’

‘But you’re a seafaring man?’ argued Pleasant, as if that were a
sufficient reason for his being good for something in her way.

‘Yes and no. I have been, and I may be again. But I am not for you.
Won’t you take my word for it?’

The conversation had arrived at a crisis to justify Miss Pleasant’s hair
in tumbling down. It tumbled down accordingly, and she twisted it up,
looking from under her bent forehead at the man. In taking stock of his
familiarly worn rough-weather nautical clothes, piece by piece, she took
stock of a formidable knife in a sheath at his waist ready to his hand,
and of a whistle hanging round his neck, and of a short jagged knotted
club with a loaded head that peeped out of a pocket of his loose
outer jacket or frock. He sat quietly looking at her; but, with these
appendages partially revealing themselves, and with a quantity
of bristling oakum-coloured head and whisker, he had a formidable
appearance.

‘Won’t you take my word for it?’ he asked again.

Pleasant answered with a short dumb nod. He rejoined with another short
dumb nod. Then he got up and stood with his arms folded, in front of
the fire, looking down into it occasionally, as she stood with her arms
folded, leaning against the side of the chimney-piece.

‘To wile away the time till your father comes,’ he said,--‘pray is there
much robbing and murdering of seamen about the water-side now?’

‘No,’ said Pleasant.

‘Any?’

‘Complaints of that sort are sometimes made, about Ratcliffe and Wapping
and up that way. But who knows how many are true?’

‘To be sure. And it don’t seem necessary.’

‘That’s what I say,’ observed Pleasant. ‘Where’s the reason for it?
Bless the sailors, it ain’t as if they ever could keep what they have,
without it.’

‘You’re right. Their money may be soon got out of them, without
violence,’ said the man.

‘Of course it may,’ said Pleasant; ‘and then they ship again and get
more. And the best thing for ‘em, too, to ship again as soon as ever
they can be brought to it. They’re never so well off as when they’re
afloat.’

‘I’ll tell you why I ask,’ pursued the visitor, looking up from the
fire. ‘I was once beset that way myself, and left for dead.’

‘No?’ said Pleasant. ‘Where did it happen?’

‘It happened,’ returned the man, with a ruminative air, as he drew his
right hand across his chin, and dipped the other in the pocket of his
rough outer coat, ‘it happened somewhere about here as I reckon. I don’t
think it can have been a mile from here.’

‘Were you drunk?’ asked Pleasant.

‘I was muddled, but not with fair drinking. I had not been drinking, you
understand. A mouthful did it.’

Pleasant with a grave look shook her head; importing that she understood
the process, but decidedly disapproved.

‘Fair trade is one thing,’ said she, ‘but that’s another. No one has a
right to carry on with Jack in THAT way.’

‘The sentiment does you credit,’ returned the man, with a grim smile;
and added, in a mutter, ‘the more so, as I believe it’s not your
father’s.--Yes, I had a bad time of it, that time. I lost everything,
and had a sharp struggle for my life, weak as I was.’

‘Did you get the parties punished?’ asked Pleasant.

‘A tremendous punishment followed,’ said the man, more seriously; ‘but
it was not of my bringing about.’

‘Of whose, then?’ asked Pleasant.

The man pointed upward with his forefinger, and, slowly recovering that
hand, settled his chin in it again as he looked at the fire. Bringing
her inherited eye to bear upon him, Pleasant Riderhood felt more
and more uncomfortable, his manner was so mysterious, so stern, so
self-possessed.

‘Anyways,’ said the damsel, ‘I am glad punishment followed, and I say
so. Fair trade with seafaring men gets a bad name through deeds of
violence. I am as much against deeds of violence being done to seafaring
men, as seafaring men can be themselves. I am of the same opinion as my
mother was, when she was living. Fair trade, my mother used to say, but
no robbery and no blows.’ In the way of trade Miss Pleasant would have
taken--and indeed did take when she could--as much as thirty shillings
a week for board that would be dear at five, and likewise conducted the
Leaving business upon correspondingly equitable principles; yet she had
that tenderness of conscience and those feelings of humanity, that the
moment her ideas of trade were overstepped, she became the seaman’s
champion, even against her father whom she seldom otherwise resisted.

But, she was here interrupted by her father’s voice exclaiming angrily,
‘Now, Poll Parrot!’ and by her father’s hat being heavily flung from his
hand and striking her face. Accustomed to such occasional manifestations
of his sense of parental duty, Pleasant merely wiped her face on her
hair (which of course had tumbled down) before she twisted it up. This
was another common procedure on the part of the ladies of the Hole, when
heated by verbal or fistic altercation.

‘Blest if I believe such a Poll Parrot as you was ever learned to
speak!’ growled Mr Riderhood, stooping to pick up his hat, and making
a feint at her with his head and right elbow; for he took the delicate
subject of robbing seamen in extraordinary dudgeon, and was out of
humour too. ‘What are you Poll Parroting at now? Ain’t you got nothing
to do but fold your arms and stand a Poll Parroting all night?’

‘Let her alone,’ urged the man. ‘She was only speaking to me.’

‘Let her alone too!’ retorted Mr Riderhood, eyeing him all over. ‘Do you
know she’s my daughter?’

‘Yes.’

‘And don’t you know that I won’t have no Poll Parroting on the part of
my daughter? No, nor yet that I won’t take no Poll Parroting from no
man? And who may YOU be, and what may YOU want?’

‘How can I tell you until you are silent?’ returned the other fiercely.

‘Well,’ said Mr Riderhood, quailing a little, ‘I am willing to be silent
for the purpose of hearing. But don’t Poll Parrot me.’

‘Are you thirsty, you?’ the man asked, in the same fierce short way,
after returning his look.

‘Why nat’rally,’ said Mr Riderhood, ‘ain’t I always thirsty!’ (Indignant
at the absurdity of the question.)

‘What will you drink?’ demanded the man.

‘Sherry wine,’ returned Mr Riderhood, in the same sharp tone, ‘if you’re
capable of it.’

The man put his hand in his pocket, took out half a sovereign, and
begged the favour of Miss Pleasant that she would fetch a bottle. ‘With
the cork undrawn,’ he added, emphatically, looking at her father.

‘I’ll take my Alfred David,’ muttered Mr Riderhood, slowly relaxing into
a dark smile, ‘that you know a move. Do I know YOU? N--n--no, I don’t
know you.’

The man replied, ‘No, you don’t know me.’ And so they stood looking at
one another surlily enough, until Pleasant came back.

‘There’s small glasses on the shelf,’ said Riderhood to his daughter.
‘Give me the one without a foot. I gets my living by the sweat of my
brow, and it’s good enough for ME.’ This had a modest self-denying
appearance; but it soon turned out that as, by reason of the
impossibility of standing the glass upright while there was anything in
it, it required to be emptied as soon as filled, Mr Riderhood managed to
drink in the proportion of three to one.

With his Fortunatus’s goblet ready in his hand, Mr Riderhood sat down on
one side of the table before the fire, and the strange man on the other:
Pleasant occupying a stool between the latter and the fireside. The
background, composed of handkerchiefs, coats, shirts, hats, and other
old articles ‘On Leaving,’ had a general dim resemblance to human
listeners; especially where a shiny black sou’wester suit and hat hung,
looking very like a clumsy mariner with his back to the company, who
was so curious to overhear, that he paused for the purpose with his
coat half pulled on, and his shoulders up to his ears in the uncompleted
action.

The visitor first held the bottle against the light of the candle,
and next examined the top of the cork. Satisfied that it had not been
tampered with, he slowly took from his breastpocket a rusty clasp-knife,
and, with a corkscrew in the handle, opened the wine. That done,
he looked at the cork, unscrewed it from the corkscrew, laid each
separately on the table, and, with the end of the sailor’s knot of his
neckerchief, dusted the inside of the neck of the bottle. All this with
great deliberation.

At first Riderhood had sat with his footless glass extended at arm’s
length for filling, while the very deliberate stranger seemed absorbed
in his preparations. But, gradually his arm reverted home to him, and
his glass was lowered and lowered until he rested it upside down upon
the table. By the same degrees his attention became concentrated on
the knife. And now, as the man held out the bottle to fill all round,
Riderhood stood up, leaned over the table to look closer at the knife,
and stared from it to him.

‘What’s the matter?’ asked the man.

‘Why, I know that knife!’ said Riderhood.

‘Yes, I dare say you do.’

He motioned to him to hold up his glass, and filled it. Riderhood
emptied it to the last drop and began again.

‘That there knife--’

‘Stop,’ said the man, composedly. ‘I was going to drink to your
daughter. Your health, Miss Riderhood.’

‘That knife was the knife of a seaman named George Radfoot.’

‘It was.’

‘That seaman was well beknown to me.’

‘He was.’

‘What’s come to him?’

‘Death has come to him. Death came to him in an ugly shape. He looked,’
said the man, ‘very horrible after it.’

‘Arter what?’ said Riderhood, with a frowning stare.

‘After he was killed.’

‘Killed? Who killed him?’

Only answering with a shrug, the man filled the footless glass, and
Riderhood emptied it: looking amazedly from his daughter to his visitor.

‘You don’t mean to tell a honest man--’ he was recommencing with
his empty glass in his hand, when his eye became fascinated by the
stranger’s outer coat. He leaned across the table to see it nearer,
touched the sleeve, turned the cuff to look at the sleeve-lining (the
man, in his perfect composure, offering not the least objection), and
exclaimed, ‘It’s my belief as this here coat was George Radfoot’s too!’

‘You are right. He wore it the last time you ever saw him, and the last
time you ever will see him--in this world.’

‘It’s my belief you mean to tell me to my face you killed him!’
exclaimed Riderhood; but, nevertheless, allowing his glass to be filled
again.

The man only answered with another shrug, and showed no symptom of
confusion.

‘Wish I may die if I know what to be up to with this chap!’ said
Riderhood, after staring at him, and tossing his last glassful down his
throat. ‘Let’s know what to make of you. Say something plain.’

‘I will,’ returned the other, leaning forward across the table, and
speaking in a low impressive voice. ‘What a liar you are!’

The honest witness rose, and made as though he would fling his glass in
the man’s face. The man not wincing, and merely shaking his forefinger
half knowingly, half menacingly, the piece of honesty thought better of
it and sat down again, putting the glass down too.

‘And when you went to that lawyer yonder in the Temple with that
invented story,’ said the stranger, in an exasperatingly comfortable
sort of confidence, ‘you might have had your strong suspicions of a
friend of your own, you know. I think you had, you know.’

‘Me my suspicions? Of what friend?’

‘Tell me again whose knife was this?’ demanded the man.

‘It was possessed by, and was the property of--him as I have made
mention on,’ said Riderhood, stupidly evading the actual mention of the
name.

‘Tell me again whose coat was this?’

‘That there article of clothing likeways belonged to, and was wore
by--him as I have made mention on,’ was again the dull Old Bailey
evasion.

‘I suspect that you gave him the credit of the deed, and of keeping
cleverly out of the way. But there was small cleverness in HIS keeping
out of the way. The cleverness would have been, to have got back for one
single instant to the light of the sun.’

‘Things is come to a pretty pass,’ growled Mr Riderhood, rising to his
feet, goaded to stand at bay, ‘when bullyers as is wearing dead men’s
clothes, and bullyers as is armed with dead men’s knives, is to come
into the houses of honest live men, getting their livings by the sweats
of their brows, and is to make these here sort of charges with no rhyme
and no reason, neither the one nor yet the other! Why should I have had
my suspicions of him?’

‘Because you knew him,’ replied the man; ‘because you had been one with
him, and knew his real character under a fair outside; because on the
night which you had afterwards reason to believe to be the very night of
the murder, he came in here, within an hour of his having left his ship
in the docks, and asked you in what lodgings he could find room. Was
there no stranger with him?’

‘I’ll take my world-without-end everlasting Alfred David that you warn’t
with him,’ answered Riderhood. ‘You talk big, you do, but things look
pretty black against yourself, to my thinking. You charge again’ me that
George Radfoot got lost sight of, and was no more thought of. What’s
that for a sailor? Why there’s fifty such, out of sight and out of
mind, ten times as long as him--through entering in different names,
re-shipping when the out’ard voyage is made, and what not--a turning
up to light every day about here, and no matter made of it. Ask my
daughter. You could go on Poll Parroting enough with her, when I warn’t
come in: Poll Parrot a little with her on this pint. You and your
suspicions of my suspicions of him! What are my suspicions of you? You
tell me George Radfoot got killed. I ask you who done it and how you
know it. You carry his knife and you wear his coat. I ask you how you
come by ‘em? Hand over that there bottle!’ Here Mr Riderhood appeared
to labour under a virtuous delusion that it was his own property. ‘And
you,’ he added, turning to his daughter, as he filled the footless
glass, ‘if it warn’t wasting good sherry wine on you, I’d chuck this at
you, for Poll Parroting with this man. It’s along of Poll Parroting
that such like as him gets their suspicions, whereas I gets mine by
argueyment, and being nat’rally a honest man, and sweating away at the
brow as a honest man ought.’ Here he filled the footless goblet again,
and stood chewing one half of its contents and looking down into the
other as he slowly rolled the wine about in the glass; while Pleasant,
whose sympathetic hair had come down on her being apostrophised,
rearranged it, much in the style of the tail of a horse when proceeding
to market to be sold.

‘Well? Have you finished?’ asked the strange man.

‘No,’ said Riderhood, ‘I ain’t. Far from it. Now then! I want to know
how George Radfoot come by his death, and how you come by his kit?’

‘If you ever do know, you won’t know now.’

‘And next I want to know,’ proceeded Riderhood ‘whether you mean to
charge that what-you-may-call-it-murder--’

‘Harmon murder, father,’ suggested Pleasant.

‘No Poll Parroting!’ he vociferated, in return. ‘Keep your mouth
shut!--I want to know, you sir, whether you charge that there crime on
George Radfoot?’

‘If you ever do know, you won’t know now.’

‘Perhaps you done it yourself?’ said Riderhood, with a threatening
action.

‘I alone know,’ returned the man, sternly shaking his head, ‘the
mysteries of that crime. I alone know that your trumped-up story cannot
possibly be true. I alone know that it must be altogether false, and
that you must know it to be altogether false. I come here to-night to
tell you so much of what I know, and no more.’

Mr Riderhood, with his crooked eye upon his visitor, meditated for some
moments, and then refilled his glass, and tipped the contents down his
throat in three tips.

‘Shut the shop-door!’ he then said to his daughter, putting the glass
suddenly down. ‘And turn the key and stand by it! If you know all this,
you sir,’ getting, as he spoke, between the visitor and the door, ‘why
han’t you gone to Lawyer Lightwood?’

‘That, also, is alone known to myself,’ was the cool answer.

‘Don’t you know that, if you didn’t do the deed, what you say you could
tell is worth from five to ten thousand pound?’ asked Riderhood.

‘I know it very well, and when I claim the money you shall share it.’

The honest man paused, and drew a little nearer to the visitor, and a
little further from the door.

‘I know it,’ repeated the man, quietly, ‘as well as I know that you and
George Radfoot were one together in more than one dark business; and as
well as I know that you, Roger Riderhood, conspired against an innocent
man for blood-money; and as well as I know that I can--and that I swear
I will!--give you up on both scores, and be the proof against you in my
own person, if you defy me!’

‘Father!’ cried Pleasant, from the door. ‘Don’t defy him! Give way to
him! Don’t get into more trouble, father!’

‘Will you leave off a Poll Parroting, I ask you?’ cried Mr Riderhood,
half beside himself between the two. Then, propitiatingly and
crawlingly: ‘You sir! You han’t said what you want of me. Is it fair, is
it worthy of yourself, to talk of my defying you afore ever you say what
you want of me?’

‘I don’t want much,’ said the man. ‘This accusation of yours must not be
left half made and half unmade. What was done for the blood-money must
be thoroughly undone.’

‘Well; but Shipmate--’

‘Don’t call me Shipmate,’ said the man.

‘Captain, then,’ urged Mr Riderhood; ‘there! You won’t object to
Captain. It’s a honourable title, and you fully look it. Captain! Ain’t
the man dead? Now I ask you fair. Ain’t Gaffer dead?’

‘Well,’ returned the other, with impatience, ‘yes, he is dead. What
then?’

‘Can words hurt a dead man, Captain? I only ask you fair.’

‘They can hurt the memory of a dead man, and they can hurt his living
children. How many children had this man?’

‘Meaning Gaffer, Captain?’

‘Of whom else are we speaking?’ returned the other, with a movement of
his foot, as if Rogue Riderhood were beginning to sneak before him in
the body as well as the spirit, and he spurned him off. ‘I have heard
of a daughter, and a son. I ask for information; I ask YOUR daughter; I
prefer to speak to her. What children did Hexam leave?’

Pleasant, looking to her father for permission to reply, that honest man
exclaimed with great bitterness:

‘Why the devil don’t you answer the Captain? You can Poll Parrot enough
when you ain’t wanted to Poll Parrot, you perwerse jade!’

Thus encouraged, Pleasant explained that there were only Lizzie, the
daughter in question, and the youth. Both very respectable, she added.

‘It is dreadful that any stigma should attach to them,’ said the
visitor, whom the consideration rendered so uneasy that he rose, and
paced to and fro, muttering, ‘Dreadful! Unforeseen? How could it be
foreseen!’ Then he stopped, and asked aloud: ‘Where do they live?’

Pleasant further explained that only the daughter had resided with the
father at the time of his accidental death, and that she had immediately
afterwards quitted the neighbourhood.

‘I know that,’ said the man, ‘for I have been to the place they dwelt
in, at the time of the inquest. Could you quietly find out for me where
she lives now?’

Pleasant had no doubt she could do that. Within what time, did she
think? Within a day. The visitor said that was well, and he would return
for the information, relying on its being obtained. To this dialogue
Riderhood had attended in silence, and he now obsequiously bespake the
Captain.

‘Captain! Mentioning them unfort’net words of mine respecting Gaffer,
it is contrairily to be bore in mind that Gaffer always were a precious
rascal, and that his line were a thieving line. Likeways when I went to
them two Governors, Lawyer Lightwood and the t’other Governor, with
my information, I may have been a little over-eager for the cause of
justice, or (to put it another way) a little over-stimilated by them
feelings which rouses a man up, when a pot of money is going about,
to get his hand into that pot of money for his family’s sake. Besides
which, I think the wine of them two Governors was--I will not say
a hocussed wine, but fur from a wine as was elthy for the mind. And
there’s another thing to be remembered, Captain. Did I stick to them
words when Gaffer was no more, and did I say bold to them two Governors,
“Governors both, wot I informed I still inform; wot was took down I hold
to”? No. I says, frank and open--no shuffling, mind you, Captain!--“I
may have been mistook, I’ve been a thinking of it, it mayn’t have been
took down correct on this and that, and I won’t swear to thick and thin,
I’d rayther forfeit your good opinions than do it.” And so far as
I know,’ concluded Mr Riderhood, by way of proof and evidence to
character, ‘I HAVE actiwally forfeited the good opinions of several
persons--even your own, Captain, if I understand your words--but I’d
sooner do it than be forswore. There; if that’s conspiracy, call me
conspirator.’

‘You shall sign,’ said the visitor, taking very little heed of this
oration, ‘a statement that it was all utterly false, and the poor girl
shall have it. I will bring it with me for your signature, when I come
again.’

‘When might you be expected, Captain?’ inquired Riderhood, again
dubiously getting between him and door.

‘Quite soon enough for you. I shall not disappoint you; don’t be
afraid.’

‘Might you be inclined to leave any name, Captain?’

‘No, not at all. I have no such intention.’

‘“Shall” is summ’at of a hard word, Captain,’ urged Riderhood, still
feebly dodging between him and the door, as he advanced. ‘When you say a
man “shall” sign this and that and t’other, Captain, you order him about
in a grand sort of a way. Don’t it seem so to yourself?’

The man stood still, and angrily fixed him with his eyes.

‘Father, father!’ entreated Pleasant, from the door, with her disengaged
hand nervously trembling at her lips; ‘don’t! Don’t get into trouble any
more!’

‘Hear me out, Captain, hear me out! All I was wishing to mention,
Captain, afore you took your departer,’ said the sneaking Mr Riderhood,
falling out of his path, ‘was, your handsome words relating to the
reward.’

‘When I claim it,’ said the man, in a tone which seemed to leave some
such words as ‘you dog,’ very distinctly understood, ‘you shall share
it.’

Looking stedfastly at Riderhood, he once more said in a low voice, this
time with a grim sort of admiration of him as a perfect piece of evil,
‘What a liar you are!’ and, nodding his head twice or thrice over the
compliment, passed out of the shop. But, to Pleasant he said good-night
kindly.

The honest man who gained his living by the sweat of his brow remained
in a state akin to stupefaction, until the footless glass and the
unfinished bottle conveyed themselves into his mind. From his mind he
conveyed them into his hands, and so conveyed the last of the wine into
his stomach. When that was done, he awoke to a clear perception that
Poll Parroting was solely chargeable with what had passed. Therefore,
not to be remiss in his duty as a father, he threw a pair of sea-boots
at Pleasant, which she ducked to avoid, and then cried, poor thing,
using her hair for a pocket-handkerchief.



Chapter 13

A SOLO AND A DUETT


The wind was blowing so hard when the visitor came out at the shop-door
into the darkness and dirt of Limehouse Hole, that it almost blew him
in again. Doors were slamming violently, lamps were flickering or blown
out, signs were rocking in their frames, the water of the kennels,
wind-dispersed, flew about in drops like rain. Indifferent to the
weather, and even preferring it to better weather for its clearance of
the streets, the man looked about him with a scrutinizing glance. ‘Thus
much I know,’ he murmured. ‘I have never been here since that night, and
never was here before that night, but thus much I recognize. I wonder
which way did we take when we came out of that shop. We turned to the
right as I have turned, but I can recall no more. Did we go by this
alley? Or down that little lane?’

He tried both, but both confused him equally, and he came straying
back to the same spot. ‘I remember there were poles pushed out of upper
windows on which clothes were drying, and I remember a low public-house,
and the sound flowing down a narrow passage belonging to it of the
scraping of a fiddle and the shuffling of feet. But here are all these
things in the lane, and here are all these things in the alley. And I
have nothing else in my mind but a wall, a dark doorway, a flight of
stairs, and a room.’

He tried a new direction, but made nothing of it; walls, dark doorways,
flights of stairs and rooms, were too abundant. And, like most people so
puzzled, he again and again described a circle, and found himself at
the point from which he had begun. ‘This is like what I have read in
narratives of escape from prison,’ said he, ‘where the little track of
the fugitives in the night always seems to take the shape of the great
round world, on which they wander; as if it were a secret law.’

Here he ceased to be the oakum-headed, oakum-whiskered man on whom Miss
Pleasant Riderhood had looked, and, allowing for his being still wrapped
in a nautical overcoat, became as like that same lost wanted Mr Julius
Handford, as never man was like another in this world. In the breast of
the coat he stowed the bristling hair and whisker, in a moment, as the
favouring wind went with him down a solitary place that it had swept
clear of passengers. Yet in that same moment he was the Secretary also,
Mr Boffin’s Secretary. For John Rokesmith, too, was as like that same
lost wanted Mr Julius Handford as never man was like another in this
world.

‘I have no clue to the scene of my death,’ said he. ‘Not that it matters
now. But having risked discovery by venturing here at all, I should have
been glad to track some part of the way.’ With which singular words he
abandoned his search, came up out of Limehouse Hole, and took the way
past Limehouse Church. At the great iron gate of the churchyard he
stopped and looked in. He looked up at the high tower spectrally
resisting the wind, and he looked round at the white tombstones, like
enough to the dead in their winding-sheets, and he counted the nine
tolls of the clock-bell.

‘It is a sensation not experienced by many mortals,’ said he, ‘to be
looking into a churchyard on a wild windy night, and to feel that I no
more hold a place among the living than these dead do, and even to know
that I lie buried somewhere else, as they lie buried here. Nothing uses
me to it. A spirit that was once a man could hardly feel stranger or
lonelier, going unrecognized among mankind, than I feel.

‘But this is the fanciful side of the situation. It has a real side, so
difficult that, though I think of it every day, I never thoroughly think
it out. Now, let me determine to think it out as I walk home. I know
I evade it, as many men--perhaps most men--do evade thinking their way
through their greatest perplexity. I will try to pin myself to mine.
Don’t evade it, John Harmon; don’t evade it; think it out!


‘When I came to England, attracted to the country with which I had none
but most miserable associations, by the accounts of my fine inheritance
that found me abroad, I came back, shrinking from my father’s money,
shrinking from my father’s memory, mistrustful of being forced on a
mercenary wife, mistrustful of my father’s intention in thrusting that
marriage on me, mistrustful that I was already growing avaricious,
mistrustful that I was slackening in gratitude to the two dear noble
honest friends who had made the only sunlight in my childish life or
that of my heartbroken sister. I came back, timid, divided in my mind,
afraid of myself and everybody here, knowing of nothing but wretchedness
that my father’s wealth had ever brought about. Now, stop, and so far
think it out, John Harmon. Is that so? That is exactly so.

‘On board serving as third mate was George Radfoot. I knew nothing of
him. His name first became known to me about a week before we sailed,
through my being accosted by one of the ship-agent’s clerks as
“Mr Radfoot.” It was one day when I had gone aboard to look to my
preparations, and the clerk, coming behind me as I stood on deck, tapped
me on the shoulder, and said, “Mr Rad-foot, look here,” referring to
some papers that he had in his hand. And my name first became known to
Radfoot, through another clerk within a day or two, and while the ship
was yet in port, coming up behind him, tapping him on the shoulder and
beginning, “I beg your pardon, Mr Harmon--.” I believe we were alike
in bulk and stature but not otherwise, and that we were not strikingly
alike, even in those respects, when we were together and could be
compared.

‘However, a sociable word or two on these mistakes became an easy
introduction between us, and the weather was hot, and he helped me to a
cool cabin on deck alongside his own, and his first school had been at
Brussels as mine had been, and he had learnt French as I had learnt it,
and he had a little history of himself to relate--God only knows how
much of it true, and how much of it false--that had its likeness to
mine. I had been a seaman too. So we got to be confidential together,
and the more easily yet, because he and every one on board had known
by general rumour what I was making the voyage to England for. By such
degrees and means, he came to the knowledge of my uneasiness of mind,
and of its setting at that time in the direction of desiring to see and
form some judgment of my allotted wife, before she could possibly know
me for myself; also to try Mrs Boffin and give her a glad surprise. So
the plot was made out of our getting common sailors’ dresses (as he was
able to guide me about London), and throwing ourselves in Bella Wilfer’s
neighbourhood, and trying to put ourselves in her way, and doing
whatever chance might favour on the spot, and seeing what came of it. If
nothing came of it, I should be no worse off, and there would merely
be a short delay in my presenting myself to Lightwood. I have all these
facts right? Yes. They are all accurately right.

‘His advantage in all this was, that for a time I was to be lost. It
might be for a day or for two days, but I must be lost sight of on
landing, or there would be recognition, anticipation, and failure.
Therefore, I disembarked with my valise in my hand--as Potterson
the steward and Mr Jacob Kibble my fellow-passenger afterwards
remembered--and waited for him in the dark by that very Limehouse Church
which is now behind me.

‘As I had always shunned the port of London, I only knew the church
through his pointing out its spire from on board. Perhaps I might
recall, if it were any good to try, the way by which I went to it alone
from the river; but how we two went from it to Riderhood’s shop, I don’t
know--any more than I know what turns we took and doubles we made, after
we left it. The way was purposely confused, no doubt.

‘But let me go on thinking the facts out, and avoid confusing them with
my speculations. Whether he took me by a straight way or a crooked way,
what is that to the purpose now? Steady, John Harmon.

‘When we stopped at Riderhood’s, and he asked that scoundrel a question
or two, purporting to refer only to the lodging-houses in which there
was accommodation for us, had I the least suspicion of him? None.
Certainly none until afterwards when I held the clue. I think he must
have got from Riderhood in a paper, the drug, or whatever it was, that
afterwards stupefied me, but I am far from sure. All I felt safe in
charging on him to-night, was old companionship in villainy between
them. Their undisguised intimacy, and the character I now know Riderhood
to bear, made that not at all adventurous. But I am not clear about the
drug. Thinking out the circumstances on which I found my suspicion, they
are only two. One: I remember his changing a small folded paper from one
pocket to another, after we came out, which he had not touched before.
Two: I now know Riderhood to have been previously taken up for being
concerned in the robbery of an unlucky seaman, to whom some such poison
had been given.

‘It is my conviction that we cannot have gone a mile from that shop,
before we came to the wall, the dark doorway, the flight of stairs, and
the room. The night was particularly dark and it rained hard. As I think
the circumstances back, I hear the rain splashing on the stone pavement
of the passage, which was not under cover. The room overlooked the
river, or a dock, or a creek, and the tide was out. Being possessed of
the time down to that point, I know by the hour that it must have been
about low water; but while the coffee was getting ready, I drew back the
curtain (a dark-brown curtain), and, looking out, knew by the kind
of reflection below, of the few neighbouring lights, that they were
reflected in tidal mud.

‘He had carried under his arm a canvas bag, containing a suit of his
clothes. I had no change of outer clothes with me, as I was to buy
slops. “You are very wet, Mr Harmon,”--I can hear him saying--“and I am
quite dry under this good waterproof coat. Put on these clothes of
mine. You may find on trying them that they will answer your purpose
to-morrow, as well as the slops you mean to buy, or better. While you
change, I’ll hurry the hot coffee.” When he came back, I had his clothes
on, and there was a black man with him, wearing a linen jacket, like
a steward, who put the smoking coffee on the table in a tray and never
looked at me. I am so far literal and exact? Literal and exact, I am
certain.

‘Now, I pass to sick and deranged impressions; they are so strong, that
I rely upon them; but there are spaces between them that I know nothing
about, and they are not pervaded by any idea of time.

‘I had drank some coffee, when to my sense of sight he began to swell
immensely, and something urged me to rush at him. We had a struggle near
the door. He got from me, through my not knowing where to strike, in the
whirling round of the room, and the flashing of flames of fire between
us. I dropped down. Lying helpless on the ground, I was turned over by
a foot. I was dragged by the neck into a corner. I heard men speak
together. I was turned over by other feet. I saw a figure like myself
lying dressed in my clothes on a bed. What might have been, for anything
I knew, a silence of days, weeks, months, years, was broken by a violent
wrestling of men all over the room. The figure like myself was assailed,
and my valise was in its hand. I was trodden upon and fallen over. I
heard a noise of blows, and thought it was a wood-cutter cutting down
a tree. I could not have said that my name was John Harmon--I could not
have thought it--I didn’t know it--but when I heard the blows, I thought
of the wood-cutter and his axe, and had some dead idea that I was lying
in a forest.

‘This is still correct? Still correct, with the exception that I cannot
possibly express it to myself without using the word I. But it was not
I. There was no such thing as I, within my knowledge.

‘It was only after a downward slide through something like a tube, and
then a great noise and a sparkling and crackling as of fires, that the
consciousness came upon me, “This is John Harmon drowning! John Harmon,
struggle for your life. John Harmon, call on Heaven and save yourself!”
 I think I cried it out aloud in a great agony, and then a heavy horrid
unintelligible something vanished, and it was I who was struggling there
alone in the water.

‘I was very weak and faint, frightfully oppressed with drowsiness, and
driving fast with the tide. Looking over the black water, I saw the
lights racing past me on the two banks of the river, as if they were
eager to be gone and leave me dying in the dark. The tide was running
down, but I knew nothing of up or down then. When, guiding myself safely
with Heaven’s assistance before the fierce set of the water, I at last
caught at a boat moored, one of a tier of boats at a causeway, I was
sucked under her, and came up, only just alive, on the other side.

‘Was I long in the water? Long enough to be chilled to the heart, but
I don’t know how long. Yet the cold was merciful, for it was the cold
night air and the rain that restored me from a swoon on the stones of
the causeway. They naturally supposed me to have toppled in, drunk, when
I crept to the public-house it belonged to; for I had no notion where
I was, and could not articulate--through the poison that had made me
insensible having affected my speech--and I supposed the night to be
the previous night, as it was still dark and raining. But I had lost
twenty-four hours.

‘I have checked the calculation often, and it must have been two nights
that I lay recovering in that public-house. Let me see. Yes. I am sure
it was while I lay in that bed there, that the thought entered my head
of turning the danger I had passed through, to the account of being
for some time supposed to have disappeared mysteriously, and of proving
Bella. The dread of our being forced on one another, and perpetuating
the fate that seemed to have fallen on my father’s riches--the fate that
they should lead to nothing but evil--was strong upon the moral timidity
that dates from my childhood with my poor sister.

‘As to this hour I cannot understand that side of the river where I
recovered the shore, being the opposite side to that on which I was
ensnared, I shall never understand it now. Even at this moment, while I
leave the river behind me, going home, I cannot conceive that it rolls
between me and that spot, or that the sea is where it is. But this is
not thinking it out; this is making a leap to the present time.

‘I could not have done it, but for the fortune in the waterproof
belt round my body. Not a great fortune, forty and odd pounds for the
inheritor of a hundred and odd thousand! But it was enough. Without it I
must have disclosed myself. Without it, I could never have gone to that
Exchequer Coffee House, or taken Mrs Wilfer’s lodgings.

‘Some twelve days I lived at that hotel, before the night when I saw the
corpse of Radfoot at the Police Station. The inexpressible mental horror
that I laboured under, as one of the consequences of the poison, makes
the interval seem greatly longer, but I know it cannot have been longer.
That suffering has gradually weakened and weakened since, and has only
come upon me by starts, and I hope I am free from it now; but even now,
I have sometimes to think, constrain myself, and stop before speaking,
or I could not say the words I want to say.

‘Again I ramble away from thinking it out to the end. It is not so far
to the end that I need be tempted to break off. Now, on straight!

‘I examined the newspapers every day for tidings that I was missing, but
saw none. Going out that night to walk (for I kept retired while it was
light), I found a crowd assembled round a placard posted at Whitehall.
It described myself, John Harmon, as found dead and mutilated in the
river under circumstances of strong suspicion, described my dress,
described the papers in my pockets, and stated where I was lying for
recognition. In a wild incautious way I hurried there, and there--with
the horror of the death I had escaped, before my eyes in its most
appalling shape, added to the inconceivable horror tormenting me at
that time when the poisonous stuff was strongest on me--I perceived that
Radfoot had been murdered by some unknown hands for the money for which
he would have murdered me, and that probably we had both been shot into
the river from the same dark place into the same dark tide, when the
stream ran deep and strong.

‘That night I almost gave up my mystery, though I suspected no one,
could offer no information, knew absolutely nothing save that the
murdered man was not I, but Radfoot. Next day while I hesitated, and
next day while I hesitated, it seemed as if the whole country were
determined to have me dead. The Inquest declared me dead, the Government
proclaimed me dead; I could not listen at my fireside for five minutes
to the outer noises, but it was borne into my ears that I was dead.

‘So John Harmon died, and Julius Handford disappeared, and John
Rokesmith was born. John Rokesmith’s intent to-night has been to repair
a wrong that he could never have imagined possible, coming to his ears
through the Lightwood talk related to him, and which he is bound by
every consideration to remedy. In that intent John Rokesmith will
persevere, as his duty is.

‘Now, is it all thought out? All to this time? Nothing omitted? No,
nothing. But beyond this time? To think it out through the future, is a
harder though a much shorter task than to think it out through the past.
John Harmon is dead. Should John Harmon come to life?

‘If yes, why? If no, why?’

‘Take yes, first. To enlighten human Justice concerning the offence of
one far beyond it who may have a living mother. To enlighten it with the
lights of a stone passage, a flight of stairs, a brown window-curtain,
and a black man. To come into possession of my father’s money, and with
it sordidly to buy a beautiful creature whom I love--I cannot help it;
reason has nothing to do with it; I love her against reason--but who
would as soon love me for my own sake, as she would love the beggar at
the corner. What a use for the money, and how worthy of its old misuses!

‘Now, take no. The reasons why John Harmon should not come to life.
Because he has passively allowed these dear old faithful friends to pass
into possession of the property. Because he sees them happy with it,
making a good use of it, effacing the old rust and tarnish on the money.
Because they have virtually adopted Bella, and will provide for her.
Because there is affection enough in her nature, and warmth enough in
her heart, to develop into something enduringly good, under favourable
conditions. Because her faults have been intensified by her place in my
father’s will, and she is already growing better. Because her marriage
with John Harmon, after what I have heard from her own lips, would be a
shocking mockery, of which both she and I must always be conscious, and
which would degrade her in her mind, and me in mine, and each of us in
the other’s. Because if John Harmon comes to life and does not marry
her, the property falls into the very hands that hold it now.

‘What would I have? Dead, I have found the true friends of my lifetime
still as true as tender and as faithful as when I was alive, and making
my memory an incentive to good actions done in my name. Dead, I have
found them when they might have slighted my name, and passed
greedily over my grave to ease and wealth, lingering by the way, like
single-hearted children, to recall their love for me when I was a poor
frightened child. Dead, I have heard from the woman who would have been
my wife if I had lived, the revolting truth that I should have purchased
her, caring nothing for me, as a Sultan buys a slave.

‘What would I have? If the dead could know, or do know, how the living
use them, who among the hosts of dead has found a more disinterested
fidelity on earth than I? Is not that enough for me? If I had come back,
these noble creatures would have welcomed me, wept over me, given up
everything to me with joy. I did not come back, and they have passed
unspoiled into my place. Let them rest in it, and let Bella rest in
hers.

‘What course for me then? This. To live the same quiet Secretary life,
carefully avoiding chances of recognition, until they shall have become
more accustomed to their altered state, and until the great swarm of
swindlers under many names shall have found newer prey. By that time,
the method I am establishing through all the affairs, and with which I
will every day take new pains to make them both familiar, will be, I may
hope, a machine in such working order as that they can keep it going.
I know I need but ask of their generosity, to have. When the right time
comes, I will ask no more than will replace me in my former path of
life, and John Rokesmith shall tread it as contentedly as he may. But
John Harmon shall come back no more.

‘That I may never, in the days to come afar off, have any weak misgiving
that Bella might, in any contingency, have taken me for my own sake if
I had plainly asked her, I WILL plainly ask her: proving beyond all
question what I already know too well. And now it is all thought out,
from the beginning to the end, and my mind is easier.’


So deeply engaged had the living-dead man been, in thus communing with
himself, that he had regarded neither the wind nor the way, and had
resisted the former instinctively as he had pursued the latter. But
being now come into the City, where there was a coach-stand, he stood
irresolute whether to go to his lodgings, or to go first to Mr Boffin’s
house. He decided to go round by the house, arguing, as he carried his
overcoat upon his arm, that it was less likely to attract notice if left
there, than if taken to Holloway: both Mrs Wilfer and Miss Lavinia being
ravenously curious touching every article of which the lodger stood
possessed.

Arriving at the house, he found that Mr and Mrs Boffin were out, but
that Miss Wilfer was in the drawing-room. Miss Wilfer had remained at
home, in consequence of not feeling very well, and had inquired in the
evening if Mr Rokesmith were in his room.

‘Make my compliments to Miss Wilfer, and say I am here now.’

Miss Wilfer’s compliments came down in return, and, if it were not too
much trouble, would Mr Rokesmith be so kind as to come up before he
went?

It was not too much trouble, and Mr Rokesmith came up.

Oh she looked very pretty, she looked very, very pretty! If the father
of the late John Harmon had but left his money unconditionally to his
son, and if his son had but lighted on this loveable girl for himself,
and had the happiness to make her loving as well as loveable!

‘Dear me! Are you not well, Mr Rokesmith?’

‘Yes, quite well. I was sorry to hear, when I came in, that YOU were
not.’

‘A mere nothing. I had a headache--gone now--and was not quite fit for
a hot theatre, so I stayed at home. I asked you if you were not well,
because you look so white.’

‘Do I? I have had a busy evening.’

She was on a low ottoman before the fire, with a little shining jewel
of a table, and her book and her work, beside her. Ah! what a different
life the late John Harmon’s, if it had been his happy privilege to take
his place upon that ottoman, and draw his arm about that waist, and say,
‘I hope the time has been long without me? What a Home Goddess you look,
my darling!’

But, the present John Rokesmith, far removed from the late John Harmon,
remained standing at a distance. A little distance in respect of space,
but a great distance in respect of separation.

‘Mr Rokesmith,’ said Bella, taking up her work, and inspecting it all
round the corners, ‘I wanted to say something to you when I could have
the opportunity, as an explanation why I was rude to you the other day.
You have no right to think ill of me, sir.’

The sharp little way in which she darted a look at him, half sensitively
injured, and half pettishly, would have been very much admired by the
late John Harmon.

‘You don’t know how well I think of you, Miss Wilfer.’

‘Truly, you must have a very high opinion of me, Mr Rokesmith, when you
believe that in prosperity I neglect and forget my old home.’

‘Do I believe so?’

‘You DID, sir, at any rate,’ returned Bella.

‘I took the liberty of reminding you of a little omission into which you
had fallen--insensibly and naturally fallen. It was no more than that.’

‘And I beg leave to ask you, Mr Rokesmith,’ said Bella, ‘why you took
that liberty?--I hope there is no offence in the phrase; it is your own,
remember.’

‘Because I am truly, deeply, profoundly interested in you, Miss Wilfer.
Because I wish to see you always at your best. Because I--shall I go
on?’

‘No, sir,’ returned Bella, with a burning face, ‘you have said more than
enough. I beg that you will NOT go on. If you have any generosity, any
honour, you will say no more.’

The late John Harmon, looking at the proud face with the down-cast eyes,
and at the quick breathing as it stirred the fall of bright brown hair
over the beautiful neck, would probably have remained silent.

‘I wish to speak to you, sir,’ said Bella, ‘once for all, and I don’t
know how to do it. I have sat here all this evening, wishing to speak to
you, and determining to speak to you, and feeling that I must. I beg for
a moment’s time.’

He remained silent, and she remained with her face averted, sometimes
making a slight movement as if she would turn and speak. At length she
did so.

‘You know how I am situated here, sir, and you know how I am situated
at home. I must speak to you for myself, since there is no one about
me whom I could ask to do so. It is not generous in you, it is not
honourable in you, to conduct yourself towards me as you do.’

‘Is it ungenerous or dishonourable to be devoted to you; fascinated by
you?’

‘Preposterous!’ said Bella.

The late John Harmon might have thought it rather a contemptuous and
lofty word of repudiation.

‘I now feel obliged to go on,’ pursued the Secretary, ‘though it were
only in self-explanation and self-defence. I hope, Miss Wilfer, that
it is not unpardonable--even in me--to make an honest declaration of an
honest devotion to you.’

‘An honest declaration!’ repeated Bella, with emphasis.

‘Is it otherwise?’

‘I must request, sir,’ said Bella, taking refuge in a touch of timely
resentment, ‘that I may not be questioned. You must excuse me if I
decline to be cross-examined.’

‘Oh, Miss Wilfer, this is hardly charitable. I ask you nothing but what
your own emphasis suggests. However, I waive even that question. But
what I have declared, I take my stand by. I cannot recall the avowal of
my earnest and deep attachment to you, and I do not recall it.’

‘I reject it, sir,’ said Bella.

‘I should be blind and deaf if I were not prepared for the reply.
Forgive my offence, for it carries its punishment with it.’

‘What punishment?’ asked Bella.

‘Is my present endurance none? But excuse me; I did not mean to
cross-examine you again.’

‘You take advantage of a hasty word of mine,’ said Bella with a little
sting of self-reproach, ‘to make me seem--I don’t know what. I spoke
without consideration when I used it. If that was bad, I am sorry; but
you repeat it after consideration, and that seems to me to be at least
no better. For the rest, I beg it may be understood, Mr Rokesmith, that
there is an end of this between us, now and for ever.’

‘Now and for ever,’ he repeated.

‘Yes. I appeal to you, sir,’ proceeded Bella with increasing spirit,
‘not to pursue me. I appeal to you not to take advantage of your
position in this house to make my position in it distressing and
disagreeable. I appeal to you to discontinue your habit of making your
misplaced attentions as plain to Mrs Boffin as to me.’

‘Have I done so?’

‘I should think you have,’ replied Bella. ‘In any case it is not your
fault if you have not, Mr Rokesmith.’

‘I hope you are wrong in that impression. I should be very sorry to
have justified it. I think I have not. For the future there is no
apprehension. It is all over.’

‘I am much relieved to hear it,’ said Bella. ‘I have far other views in
life, and why should you waste your own?’

‘Mine!’ said the Secretary. ‘My life!’

His curious tone caused Bella to glance at the curious smile with which
he said it. It was gone as he glanced back. ‘Pardon me, Miss Wilfer,’
he proceeded, when their eyes met; ‘you have used some hard words, for
which I do not doubt you have a justification in your mind, that I do
not understand. Ungenerous and dishonourable. In what?’

‘I would rather not be asked,’ said Bella, haughtily looking down.

‘I would rather not ask, but the question is imposed upon me. Kindly
explain; or if not kindly, justly.’

‘Oh, sir!’ said Bella, raising her eyes to his, after a little struggle
to forbear, ‘is it generous and honourable to use the power here which
your favour with Mr and Mrs Boffin and your ability in your place give
you, against me?’

‘Against you?’

‘Is it generous and honourable to form a plan for gradually bringing
their influence to bear upon a suit which I have shown you that I do not
like, and which I tell you that I utterly reject?’

The late John Harmon could have borne a good deal, but he would have
been cut to the heart by such a suspicion as this.

‘Would it be generous and honourable to step into your place--if you did
so, for I don’t know that you did, and I hope you did not--anticipating,
or knowing beforehand, that I should come here, and designing to take me
at this disadvantage?’

‘This mean and cruel disadvantage,’ said the Secretary.

‘Yes,’ assented Bella.

The Secretary kept silence for a little while; then merely said, ‘You
are wholly mistaken, Miss Wilfer; wonderfully mistaken. I cannot say,
however, that it is your fault. If I deserve better things of you, you
do not know it.’

‘At least, sir,’ retorted Bella, with her old indignation rising, ‘you
know the history of my being here at all. I have heard Mr Boffin say
that you are master of every line and word of that will, as you are
master of all his affairs. And was it not enough that I should have been
willed away, like a horse, or a dog, or a bird; but must you too begin
to dispose of me in your mind, and speculate in me, as soon as I had
ceased to be the talk and the laugh of the town? Am I for ever to be
made the property of strangers?’

‘Believe me,’ returned the Secretary, ‘you are wonderfully mistaken.’

‘I should be glad to know it,’ answered Bella.

‘I doubt if you ever will. Good-night. Of course I shall be careful to
conceal any traces of this interview from Mr and Mrs Boffin, as long as
I remain here. Trust me, what you have complained of is at an end for
ever.’

‘I am glad I have spoken, then, Mr Rokesmith. It has been painful and
difficult, but it is done. If I have hurt you, I hope you will forgive
me. I am inexperienced and impetuous, and I have been a little spoilt;
but I really am not so bad as I dare say I appear, or as you think me.’

He quitted the room when Bella had said this, relenting in her wilful
inconsistent way. Left alone, she threw herself back on her ottoman, and
said, ‘I didn’t know the lovely woman was such a Dragon!’ Then, she
got up and looked in the glass, and said to her image, ‘You have been
positively swelling your features, you little fool!’ Then, she took an
impatient walk to the other end of the room and back, and said, ‘I
wish Pa was here to have a talk about an avaricious marriage; but he
is better away, poor dear, for I know I should pull his hair if he WAS
here.’ And then she threw her work away, and threw her book after
it, and sat down and hummed a tune, and hummed it out of tune, and
quarrelled with it.

And John Rokesmith, what did he?

He went down to his room, and buried John Harmon many additional fathoms
deep. He took his hat, and walked out, and, as he went to Holloway or
anywhere else--not at all minding where--heaped mounds upon mounds of
earth over John Harmon’s grave. His walking did not bring him home until
the dawn of day. And so busy had he been all night, piling and piling
weights upon weights of earth above John Harmon’s grave, that by that
time John Harmon lay buried under a whole Alpine range; and still the
Sexton Rokesmith accumulated mountains over him, lightening his labour
with the dirge, ‘Cover him, crush him, keep him down!’



Chapter 14

STRONG OF PURPOSE


The sexton-task of piling earth above John Harmon all night long, was
not conducive to sound sleep; but Rokesmith had some broken morning
rest, and rose strengthened in his purpose. It was all over now. No
ghost should trouble Mr and Mrs Boffin’s peace; invisible and voiceless,
the ghost should look on for a little while longer at the state of
existence out of which it had departed, and then should for ever cease
to haunt the scenes in which it had no place.

He went over it all again. He had lapsed into the condition in which
he found himself, as many a man lapses into many a condition, without
perceiving the accumulative power of its separate circumstances. When
in the distrust engendered by his wretched childhood and the action for
evil--never yet for good within his knowledge then--of his father and
his father’s wealth on all within their influence, he conceived the idea
of his first deception, it was meant to be harmless, it was to last
but a few hours or days, it was to involve in it only the girl so
capriciously forced upon him and upon whom he was so capriciously
forced, and it was honestly meant well towards her. For, if he had
found her unhappy in the prospect of that marriage (through her heart
inclining to another man or for any other cause), he would seriously
have said: ‘This is another of the old perverted uses of the
misery-making money. I will let it go to my and my sister’s only
protectors and friends.’ When the snare into which he fell so
outstripped his first intention as that he found himself placarded by
the police authorities upon the London walls for dead, he confusedly
accepted the aid that fell upon him, without considering how firmly it
must seem to fix the Boffins in their accession to the fortune. When he
saw them, and knew them, and even from his vantage-ground of inspection
could find no flaw in them, he asked himself, ‘And shall I come to life
to dispossess such people as these?’ There was no good to set against
the putting of them to that hard proof. He had heard from Bella’s own
lips when he stood tapping at the door on that night of his taking
the lodgings, that the marriage would have been on her part thoroughly
mercenary. He had since tried her, in his own unknown person and
supposed station, and she not only rejected his advances but resented
them. Was it for him to have the shame of buying her, or the meanness of
punishing her? Yet, by coming to life and accepting the condition of the
inheritance, he must do the former; and by coming to life and rejecting
it, he must do the latter.

Another consequence that he had never foreshadowed, was the implication
of an innocent man in his supposed murder. He would obtain complete
retraction from the accuser, and set the wrong right; but clearly the
wrong could never have been done if he had never planned a deception.
Then, whatever inconvenience or distress of mind the deception cost him,
it was manful repentantly to accept as among its consequences, and make
no complaint.

Thus John Rokesmith in the morning, and it buried John Harmon still many
fathoms deeper than he had been buried in the night.

Going out earlier than he was accustomed to do, he encountered the
cherub at the door. The cherub’s way was for a certain space his way,
and they walked together.

It was impossible not to notice the change in the cherub’s appearance.
The cherub felt very conscious of it, and modestly remarked:

‘A present from my daughter Bella, Mr Rokesmith.’

The words gave the Secretary a stroke of pleasure, for he remembered the
fifty pounds, and he still loved the girl. No doubt it was very weak--it
always IS very weak, some authorities hold--but he loved the girl.

‘I don’t know whether you happen to have read many books of African
Travel, Mr Rokesmith?’ said R. W.

‘I have read several.’

‘Well, you know, there’s usually a King George, or a King Boy, or a King
Sambo, or a King Bill, or Bull, or Rum, or Junk, or whatever name the
sailors may have happened to give him.’

‘Where?’ asked Rokesmith.

‘Anywhere. Anywhere in Africa, I mean. Pretty well everywhere, I may
say; for black kings are cheap--and I think’--said R. W., with an
apologetic air, ‘nasty’.

‘I am much of your opinion, Mr Wilfer. You were going to say--?’

‘I was going to say, the king is generally dressed in a London hat only,
or a Manchester pair of braces, or one epaulette, or an uniform coat
with his legs in the sleeves, or something of that kind.’

‘Just so,’ said the Secretary.

‘In confidence, I assure you, Mr Rokesmith,’ observed the cheerful
cherub, ‘that when more of my family were at home and to be provided
for, I used to remind myself immensely of that king. You have no idea,
as a single man, of the difficulty I have had in wearing more than one
good article at a time.’

‘I can easily believe it, Mr Wilfer.’

‘I only mention it,’ said R. W. in the warmth of his heart, ‘as a proof
of the amiable, delicate, and considerate affection of my daughter
Bella. If she had been a little spoilt, I couldn’t have thought so very
much of it, under the circumstances. But no, not a bit. And she is so
very pretty! I hope you agree with me in finding her very pretty, Mr
Rokesmith?’

‘Certainly I do. Every one must.’

‘I hope so,’ said the cherub. ‘Indeed, I have no doubt of it. This is a
great advancement for her in life, Mr Rokesmith. A great opening of her
prospects?’

‘Miss Wilfer could have no better friends than Mr and Mrs Boffin.’

‘Impossible!’ said the gratified cherub. ‘Really I begin to think things
are very well as they are. If Mr John Harmon had lived--’

‘He is better dead,’ said the Secretary.

‘No, I won’t go so far as to say that,’ urged the cherub, a little
remonstrant against the very decisive and unpitying tone; ‘but he
mightn’t have suited Bella, or Bella mightn’t have suited him, or fifty
things, whereas now I hope she can choose for herself.’

‘Has she--as you place the confidence in me of speaking on the subject,
you will excuse my asking--has she--perhaps--chosen?’ faltered the
Secretary.

‘Oh dear no!’ returned R. W.

‘Young ladies sometimes,’ Rokesmith hinted, ‘choose without mentioning
their choice to their fathers.’

‘Not in this case, Mr Rokesmith. Between my daughter Bella and me there
is a regular league and covenant of confidence. It was ratified only the
other day. The ratification dates from--these,’ said the cherub,
giving a little pull at the lappels of his coat and the pockets of his
trousers. ‘Oh no, she has not chosen. To be sure, young George Sampson,
in the days when Mr John Harmon--’

‘Who I wish had never been born!’ said the Secretary, with a gloomy
brow.

R. W. looked at him with surprise, as thinking he had contracted an
unaccountable spite against the poor deceased, and continued: ‘In the
days when Mr John Harmon was being sought out, young George Sampson
certainly was hovering about Bella, and Bella let him hover. But it
never was seriously thought of, and it’s still less than ever to be
thought of now. For Bella is ambitious, Mr Rokesmith, and I think I may
predict will marry fortune. This time, you see, she will have the person
and the property before her together, and will be able to make her
choice with her eyes open. This is my road. I am very sorry to part
company so soon. Good morning, sir!’

The Secretary pursued his way, not very much elevated in spirits by this
conversation, and, arriving at the Boffin mansion, found Betty Higden
waiting for him.

‘I should thank you kindly, sir,’ said Betty, ‘if I might make so bold
as have a word or two wi’ you.’

She should have as many words as she liked, he told her; and took her
into his room, and made her sit down.

‘’Tis concerning Sloppy, sir,’ said Betty. ‘And that’s how I come here
by myself. Not wishing him to know what I’m a-going to say to you, I got
the start of him early and walked up.’

‘You have wonderful energy,’ returned Rokesmith. ‘You are as young as I
am.’

Betty Higden gravely shook her head. ‘I am strong for my time of life,
sir, but not young, thank the Lord!’

‘Are you thankful for not being young?’

‘Yes, sir. If I was young, it would all have to be gone through again,
and the end would be a weary way off, don’t you see? But never mind me;
‘tis concerning Sloppy.’

‘And what about him, Betty?’

‘’Tis just this, sir. It can’t be reasoned out of his head by any powers
of mine but what that he can do right by your kind lady and gentleman
and do his work for me, both together. Now he can’t. To give himself up
to being put in the way of arning a good living and getting on, he must
give me up. Well; he won’t.’

‘I respect him for it,’ said Rokesmith.

‘DO ye, sir? I don’t know but what I do myself. Still that don’t make it
right to let him have his way. So as he won’t give me up, I’m a-going to
give him up.’

‘How, Betty?’

‘I’m a-going to run away from him.’

With an astonished look at the indomitable old face and the bright eyes,
the Secretary repeated, ‘Run away from him?’

‘Yes, sir,’ said Betty, with one nod. And in the nod and in the firm set
of her mouth, there was a vigour of purpose not to be doubted.

‘Come, come!’ said the Secretary. ‘We must talk about this. Let us take
our time over it, and try to get at the true sense of the case and the
true course, by degrees.’

‘Now, lookee here, by dear,’ returned old Betty--‘asking your excuse
for being so familiar, but being of a time of life a’most to be your
grandmother twice over. Now, lookee, here. ‘Tis a poor living and a
hard as is to be got out of this work that I’m a doing now, and but for
Sloppy I don’t know as I should have held to it this long. But it did
just keep us on, the two together. Now that I’m alone--with even Johnny
gone--I’d far sooner be upon my feet and tiring of myself out, than a
sitting folding and folding by the fire. And I’ll tell you why. There’s
a deadness steals over me at times, that the kind of life favours and I
don’t like. Now, I seem to have Johnny in my arms--now, his mother--now,
his mother’s mother--now, I seem to be a child myself, a lying once
again in the arms of my own mother--then I get numbed, thought and
sense, till I start out of my seat, afeerd that I’m a growing like the
poor old people that they brick up in the Unions, as you may sometimes
see when they let ‘em out of the four walls to have a warm in the sun,
crawling quite scared about the streets. I was a nimble girl, and have
always been a active body, as I told your lady, first time ever I see
her good face. I can still walk twenty mile if I am put to it. I’d far
better be a walking than a getting numbed and dreary. I’m a good fair
knitter, and can make many little things to sell. The loan from your
lady and gentleman of twenty shillings to fit out a basket with, would
be a fortune for me. Trudging round the country and tiring of myself
out, I shall keep the deadness off, and get my own bread by my own
labour. And what more can I want?’

‘And this is your plan,’ said the Secretary, ‘for running away?’

‘Show me a better! My deary, show me a better! Why, I know very well,’
said old Betty Higden, ‘and you know very well, that your lady and
gentleman would set me up like a queen for the rest of my life, if so be
that we could make it right among us to have it so. But we can’t make it
right among us to have it so. I’ve never took charity yet, nor yet has
any one belonging to me. And it would be forsaking of myself indeed, and
forsaking of my children dead and gone, and forsaking of their children
dead and gone, to set up a contradiction now at last.’

‘It might come to be justifiable and unavoidable at last,’ the Secretary
gently hinted, with a slight stress on the word.

‘I hope it never will! It ain’t that I mean to give offence by being
anyways proud,’ said the old creature simply, ‘but that I want to be of
a piece like, and helpful of myself right through to my death.’

‘And to be sure,’ added the Secretary, as a comfort for her, ‘Sloppy
will be eagerly looking forward to his opportunity of being to you what
you have been to him.’

‘Trust him for that, sir!’ said Betty, cheerfully. ‘Though he had need
to be something quick about it, for I’m a getting to be an old one. But
I’m a strong one too, and travel and weather never hurt me yet! Now, be
so kind as speak for me to your lady and gentleman, and tell ‘em what I
ask of their good friendliness to let me do, and why I ask it.’

The Secretary felt that there was no gainsaying what was urged by
this brave old heroine, and he presently repaired to Mrs Boffin and
recommended her to let Betty Higden have her way, at all events for the
time. ‘It would be far more satisfactory to your kind heart, I know,’
he said, ‘to provide for her, but it may be a duty to respect this
independent spirit.’ Mrs Boffin was not proof against the consideration
set before her. She and her husband had worked too, and had brought
their simple faith and honour clean out of dustheaps. If they owed a
duty to Betty Higden, of a surety that duty must be done.

‘But, Betty,’ said Mrs Boffin, when she accompanied John Rokesmith back
to his room, and shone upon her with the light of her radiant face,
‘granted all else, I think I wouldn’t run away’.

‘’Twould come easier to Sloppy,’ said Mrs Higden, shaking her head.
‘’Twould come easier to me too. But ‘tis as you please.’

‘When would you go?’

‘Now,’ was the bright and ready answer. ‘To-day, my deary, to-morrow.
Bless ye, I am used to it. I know many parts of the country well. When
nothing else was to be done, I have worked in many a market-garden afore
now, and in many a hop-garden too.’

‘If I give my consent to your going, Betty--which Mr Rokesmith thinks I
ought to do--’

Betty thanked him with a grateful curtsey.

‘--We must not lose sight of you. We must not let you pass out of our
knowledge. We must know all about you.’

‘Yes, my deary, but not through letter-writing, because
letter-writing--indeed, writing of most sorts hadn’t much come up for
such as me when I was young. But I shall be to and fro. No fear of
my missing a chance of giving myself a sight of your reviving face.
Besides,’ said Betty, with logical good faith, ‘I shall have a debt to
pay off, by littles, and naturally that would bring me back, if nothing
else would.’

‘MUST it be done?’ asked Mrs Boffin, still reluctant, of the Secretary.

‘I think it must.’

After more discussion it was agreed that it should be done, and Mrs
Boffin summoned Bella to note down the little purchases that were
necessary to set Betty up in trade. ‘Don’t ye be timorous for me, my
dear,’ said the stanch old heart, observant of Bella’s face: ‘when I
take my seat with my work, clean and busy and fresh, in a country
market-place, I shall turn a sixpence as sure as ever a farmer’s wife
there.’

The Secretary took that opportunity of touching on the practical
question of Mr Sloppy’s capabilities. He would have made a wonderful
cabinet-maker, said Mrs Higden, ‘if there had been the money to put him
to it.’ She had seen him handle tools that he had borrowed to mend
the mangle, or to knock a broken piece of furniture together, in a
surprising manner. As to constructing toys for the Minders, out of
nothing, he had done that daily. And once as many as a dozen people had
got together in the lane to see the neatness with which he fitted the
broken pieces of a foreign monkey’s musical instrument. ‘That’s well,’
said the Secretary. ‘It will not be hard to find a trade for him.’

John Harmon being buried under mountains now, the Secretary that very
same day set himself to finish his affairs and have done with him. He
drew up an ample declaration, to be signed by Rogue Riderhood (knowing
he could get his signature to it, by making him another and much shorter
evening call), and then considered to whom should he give the document?
To Hexam’s son, or daughter? Resolved speedily, to the daughter. But it
would be safer to avoid seeing the daughter, because the son had seen
Julius Handford, and--he could not be too careful--there might possibly
be some comparison of notes between the son and daughter, which would
awaken slumbering suspicion, and lead to consequences. ‘I might even,’
he reflected, ‘be apprehended as having been concerned in my own
murder!’ Therefore, best to send it to the daughter under cover by the
post. Pleasant Riderhood had undertaken to find out where she lived,
and it was not necessary that it should be attended by a single word of
explanation. So far, straight.

But, all that he knew of the daughter he derived from Mrs Boffin’s
accounts of what she heard from Mr Lightwood, who seemed to have a
reputation for his manner of relating a story, and to have made this
story quite his own. It interested him, and he would like to have
the means of knowing more--as, for instance, that she received the
exonerating paper, and that it satisfied her--by opening some channel
altogether independent of Lightwood: who likewise had seen Julius
Handford, who had publicly advertised for Julius Handford, and whom
of all men he, the Secretary, most avoided. ‘But with whom the common
course of things might bring me in a moment face to face, any day in the
week or any hour in the day.’

Now, to cast about for some likely means of opening such a channel. The
boy, Hexam, was training for and with a schoolmaster. The Secretary knew
it, because his sister’s share in that disposal of him seemed to be
the best part of Lightwood’s account of the family. This young fellow,
Sloppy, stood in need of some instruction. If he, the Secretary, engaged
that schoolmaster to impart it to him, the channel might be opened. The
next point was, did Mrs Boffin know the schoolmaster’s name? No, but she
knew where the school was. Quite enough. Promptly the Secretary wrote
to the master of that school, and that very evening Bradley Headstone
answered in person.

The Secretary stated to the schoolmaster how the object was, to send to
him for certain occasional evening instruction, a youth whom Mr and Mrs
Boffin wished to help to an industrious and useful place in life. The
schoolmaster was willing to undertake the charge of such a pupil. The
Secretary inquired on what terms? The schoolmaster stated on what terms.
Agreed and disposed of.

‘May I ask, sir,’ said Bradley Headstone, ‘to whose good opinion I owe a
recommendation to you?’

‘You should know that I am not the principal here. I am Mr Boffin’s
Secretary. Mr Boffin is a gentleman who inherited a property of which
you may have heard some public mention; the Harmon property.’

‘Mr Harmon,’ said Bradley: who would have been a great deal more at a
loss than he was, if he had known to whom he spoke: ‘was murdered and
found in the river.’

‘Was murdered and found in the river.’

‘It was not--’

‘No,’ interposed the Secretary, smiling, ‘it was not he who recommended
you. Mr Boffin heard of you through a certain Mr Lightwood. I think you
know Mr Lightwood, or know of him?’

‘I know as much of him as I wish to know, sir. I have no acquaintance
with Mr Lightwood, and I desire none. I have no objection to Mr
Lightwood, but I have a particular objection to some of Mr Lightwood’s
friends--in short, to one of Mr Lightwood’s friends. His great friend.’

He could hardly get the words out, even then and there, so fierce did
he grow (though keeping himself down with infinite pains of repression),
when the careless and contemptuous bearing of Eugene Wrayburn rose
before his mind.

The Secretary saw there was a strong feeling here on some sore point,
and he would have made a diversion from it, but for Bradley’s holding to
it in his cumbersome way.

‘I have no objection to mention the friend by name,’ he said, doggedly.
‘The person I object to, is Mr Eugene Wrayburn.’

The Secretary remembered him. In his disturbed recollection of that
night when he was striving against the drugged drink, there was but a
dim image of Eugene’s person; but he remembered his name, and his manner
of speaking, and how he had gone with them to view the body, and where
he had stood, and what he had said.

‘Pray, Mr Headstone, what is the name,’ he asked, again trying to make a
diversion, ‘of young Hexam’s sister?’

‘Her name is Lizzie,’ said the schoolmaster, with a strong contraction
of his whole face.

‘She is a young woman of a remarkable character; is she not?’

‘She is sufficiently remarkable to be very superior to Mr Eugene
Wrayburn--though an ordinary person might be that,’ said the
schoolmaster; ‘and I hope you will not think it impertinent in me, sir,
to ask why you put the two names together?’

‘By mere accident,’ returned the Secretary. ‘Observing that Mr Wrayburn
was a disagreeable subject with you, I tried to get away from it: though
not very successfully, it would appear.’

‘Do you know Mr Wrayburn, sir?’

‘No.’

‘Then perhaps the names cannot be put together on the authority of any
representation of his?’

‘Certainly not.’

‘I took the liberty to ask,’ said Bradley, after casting his eyes on
the ground, ‘because he is capable of making any representation, in the
swaggering levity of his insolence. I--I hope you will not misunderstand
me, sir. I--I am much interested in this brother and sister, and the
subject awakens very strong feelings within me. Very, very, strong
feelings.’ With a shaking hand, Bradley took out his handkerchief and
wiped his brow.

The Secretary thought, as he glanced at the schoolmaster’s face, that he
had opened a channel here indeed, and that it was an unexpectedly dark
and deep and stormy one, and difficult to sound. All at once, in the
midst of his turbulent emotions, Bradley stopped and seemed to challenge
his look. Much as though he suddenly asked him, ‘What do you see in me?’

‘The brother, young Hexam, was your real recommendation here,’ said the
Secretary, quietly going back to the point; ‘Mr and Mrs Boffin happening
to know, through Mr Lightwood, that he was your pupil. Anything that
I ask respecting the brother and sister, or either of them, I ask for
myself out of my own interest in the subject, and not in my official
character, or on Mr Boffin’s behalf. How I come to be interested, I need
not explain. You know the father’s connection with the discovery of Mr
Harmon’s body.’

‘Sir,’ replied Bradley, very restlessly indeed, ‘I know all the
circumstances of that case.’

‘Pray tell me, Mr Headstone,’ said the Secretary. ‘Does the sister
suffer under any stigma because of the impossible accusation--groundless
would be a better word--that was made against the father, and
substantially withdrawn?’

‘No, sir,’ returned Bradley, with a kind of anger.

‘I am very glad to hear it.’

‘The sister,’ said Bradley, separating his words over-carefully, and
speaking as if he were repeating them from a book, ‘suffers under no
reproach that repels a man of unimpeachable character who had made
for himself every step of his way in life, from placing her in his own
station. I will not say, raising her to his own station; I say, placing
her in it. The sister labours under no reproach, unless she should
unfortunately make it for herself. When such a man is not deterred from
regarding her as his equal, and when he has convinced himself that
there is no blemish on her, I think the fact must be taken to be pretty
expressive.’

‘And there is such a man?’ said the Secretary.

Bradley Headstone knotted his brows, and squared his large lower jaw,
and fixed his eyes on the ground with an air of determination that
seemed unnecessary to the occasion, as he replied: ‘And there is such a
man.’

The Secretary had no reason or excuse for prolonging the conversation,
and it ended here. Within three hours the oakum-headed apparition once
more dived into the Leaving Shop, and that night Rogue Riderhood’s
recantation lay in the post office, addressed under cover to Lizzie
Hexam at her right address.

All these proceedings occupied John Rokesmith so much, that it was not
until the following day that he saw Bella again. It seemed then to be
tacitly understood between them that they were to be as distantly easy
as they could, without attracting the attention of Mr and Mrs Boffin to
any marked change in their manner. The fitting out of old Betty Higden
was favourable to this, as keeping Bella engaged and interested, and as
occupying the general attention.

‘I think,’ said Rokesmith, when they all stood about her, while she
packed her tidy basket--except Bella, who was busily helping on her
knees at the chair on which it stood; ‘that at least you might keep a
letter in your pocket, Mrs Higden, which I would write for you and date
from here, merely stating, in the names of Mr and Mrs Boffin, that they
are your friends;--I won’t say patrons, because they wouldn’t like it.’

‘No, no, no,’ said Mr Boffin; ‘no patronizing! Let’s keep out of THAT,
whatever we come to.’

‘There’s more than enough of that about, without us; ain’t there,
Noddy?’ said Mrs Boffin.

‘I believe you, old lady!’ returned the Golden Dustman. ‘Overmuch
indeed!’

‘But people sometimes like to be patronized; don’t they, sir?’ asked
Bella, looking up.

‘I don’t. And if THEY do, my dear, they ought to learn better,’ said Mr
Boffin. ‘Patrons and Patronesses, and Vice-Patrons and Vice-Patronesses,
and Deceased Patrons and Deceased Patronesses, and Ex-Vice-Patrons and
Ex-Vice-Patronesses, what does it all mean in the books of the Charities
that come pouring in on Rokesmith as he sits among ‘em pretty well up to
his neck! If Mr Tom Noakes gives his five shillings ain’t he a Patron,
and if Mrs Jack Styles gives her five shillings ain’t she a Patroness?
What the deuce is it all about? If it ain’t stark staring impudence,
what do you call it?’

‘Don’t be warm, Noddy,’ Mrs Boffin urged.

‘Warm!’ cried Mr Boffin. ‘It’s enough to make a man smoking hot. I can’t
go anywhere without being Patronized. I don’t want to be Patronized. If
I buy a ticket for a Flower Show, or a Music Show, or any sort of Show,
and pay pretty heavy for it, why am I to be Patroned and Patronessed as
if the Patrons and Patronesses treated me? If there’s a good thing to be
done, can’t it be done on its own merits? If there’s a bad thing to
be done, can it ever be Patroned and Patronessed right? Yet when a new
Institution’s going to be built, it seems to me that the bricks and
mortar ain’t made of half so much consequence as the Patrons and
Patronesses; no, nor yet the objects. I wish somebody would tell me
whether other countries get Patronized to anything like the extent of
this one! And as to the Patrons and Patronesses themselves, I wonder
they’re not ashamed of themselves. They ain’t Pills, or Hair-Washes, or
Invigorating Nervous Essences, to be puffed in that way!’

Having delivered himself of these remarks, Mr Boffin took a trot,
according to his usual custom, and trotted back to the spot from which
he had started.

‘As to the letter, Rokesmith,’ said Mr Boffin, ‘you’re as right as a
trivet. Give her the letter, make her take the letter, put it in her
pocket by violence. She might fall sick. You know you might fall sick,’
said Mr Boffin. ‘Don’t deny it, Mrs Higden, in your obstinacy; you know
you might.’

Old Betty laughed, and said that she would take the letter and be
thankful.

‘That’s right!’ said Mr Boffin. ‘Come! That’s sensible. And don’t be
thankful to us (for we never thought of it), but to Mr Rokesmith.’

The letter was written, and read to her, and given to her.

‘Now, how do you feel?’ said Mr Boffin. ‘Do you like it?’

‘The letter, sir?’ said Betty. ‘Ay, it’s a beautiful letter!’

‘No, no, no; not the letter,’ said Mr Boffin; ‘the idea. Are you sure
you’re strong enough to carry out the idea?’

‘I shall be stronger, and keep the deadness off better, this way, than
any way left open to me, sir.’

‘Don’t say than any way left open, you know,’ urged Mr Boffin; ‘because
there are ways without end. A housekeeper would be acceptable over
yonder at the Bower, for instance. Wouldn’t you like to see the
Bower, and know a retired literary man of the name of Wegg that lives
there--WITH a wooden leg?’

Old Betty was proof even against this temptation, and fell to adjusting
her black bonnet and shawl.

‘I wouldn’t let you go, now it comes to this, after all,’ said Mr
Boffin, ‘if I didn’t hope that it may make a man and a workman of
Sloppy, in as short a time as ever a man and workman was made yet. Why,
what have you got there, Betty? Not a doll?’

It was the man in the Guards who had been on duty over Johnny’s bed.
The solitary old woman showed what it was, and put it up quietly in her
dress. Then, she gratefully took leave of Mrs Boffin, and of Mr Boffin,
and of Rokesmith, and then put her old withered arms round Bella’s young
and blooming neck, and said, repeating Johnny’s words: ‘A kiss for the
boofer lady.’

The Secretary looked on from a doorway at the boofer lady thus
encircled, and still looked on at the boofer lady standing alone there,
when the determined old figure with its steady bright eyes was trudging
through the streets, away from paralysis and pauperism.



Chapter 15

THE WHOLE CASE SO FAR


Bradley Headstone held fast by that other interview he was to have with
Lizzie Hexam. In stipulating for it, he had been impelled by a feeling
little short of desperation, and the feeling abided by him. It was very
soon after his interview with the Secretary, that he and Charley Hexam
set out one leaden evening, not unnoticed by Miss Peecher, to have this
desperate interview accomplished.

‘That dolls’ dressmaker,’ said Bradley, ‘is favourable neither to me nor
to you, Hexam.’

‘A pert crooked little chit, Mr Headstone! I knew she would put herself
in the way, if she could, and would be sure to strike in with something
impertinent. It was on that account that I proposed our going to the
City to-night and meeting my sister.’

‘So I supposed,’ said Bradley, getting his gloves on his nervous hands
as he walked. ‘So I supposed.’

‘Nobody but my sister,’ pursued Charley, ‘would have found out such an
extraordinary companion. She has done it in a ridiculous fancy of giving
herself up to another. She told me so, that night when we went there.’

‘Why should she give herself up to the dressmaker?’ asked Bradley.

‘Oh!’ said the boy, colouring. ‘One of her romantic ideas! I tried to
convince her so, but I didn’t succeed. However, what we have got to do,
is, to succeed to-night, Mr Headstone, and then all the rest follows.’

‘You are still sanguine, Hexam.’

‘Certainly I am, sir. Why, we have everything on our side.’

‘Except your sister, perhaps,’ thought Bradley. But he only gloomily
thought it, and said nothing.

‘Everything on our side,’ repeated the boy with boyish confidence.
‘Respectability, an excellent connexion for me, common sense,
everything!’

‘To be sure, your sister has always shown herself a devoted sister,’
said Bradley, willing to sustain himself on even that low ground of
hope.

‘Naturally, Mr Headstone, I have a good deal of influence with her.
And now that you have honoured me with your confidence and spoken to me
first, I say again, we have everything on our side.’

And Bradley thought again, ‘Except your sister, perhaps.’

A grey dusty withered evening in London city has not a hopeful aspect.
The closed warehouses and offices have an air of death about them, and
the national dread of colour has an air of mourning. The towers and
steeples of the many house-encompassed churches, dark and dingy as the
sky that seems descending on them, are no relief to the general gloom;
a sun-dial on a church-wall has the look, in its useless black shade, of
having failed in its business enterprise and stopped payment for ever;
melancholy waifs and strays of housekeepers and porter sweep melancholy
waifs and strays of papers and pins into the kennels, and other more
melancholy waifs and strays explore them, searching and stooping and
poking for anything to sell. The set of humanity outward from the City
is as a set of prisoners departing from gaol, and dismal Newgate
seems quite as fit a stronghold for the mighty Lord Mayor as his own
state-dwelling.

On such an evening, when the city grit gets into the hair and eyes and
skin, and when the fallen leaves of the few unhappy city trees grind
down in corners under wheels of wind, the schoolmaster and the pupil
emerged upon the Leadenhall Street region, spying eastward for Lizzie.
Being something too soon in their arrival, they lurked at a corner,
waiting for her to appear. The best-looking among us will not look very
well, lurking at a corner, and Bradley came out of that disadvantage
very poorly indeed.

‘Here she comes, Mr Headstone! Let us go forward and meet her.’

As they advanced, she saw them coming, and seemed rather troubled. But
she greeted her brother with the usual warmth, and touched the extended
hand of Bradley.

‘Why, where are you going, Charley, dear?’ she asked him then.

‘Nowhere. We came on purpose to meet you.’

‘To meet me, Charley?’

‘Yes. We are going to walk with you. But don’t let us take the great
leading streets where every one walks, and we can’t hear ourselves
speak. Let us go by the quiet backways. Here’s a large paved court by
this church, and quiet, too. Let us go up here.’

‘But it’s not in the way, Charley.’

‘Yes it is,’ said the boy, petulantly. ‘It’s in my way, and my way is
yours.’

She had not released his hand, and, still holding it, looked at him with
a kind of appeal. He avoided her eyes, under pretence of saying, ‘Come
along, Mr Headstone.’ Bradley walked at his side--not at hers--and the
brother and sister walked hand in hand. The court brought them to a
churchyard; a paved square court, with a raised bank of earth about
breast high, in the middle, enclosed by iron rails. Here, conveniently
and healthfully elevated above the level of the living, were the dead,
and the tombstones; some of the latter droopingly inclined from the
perpendicular, as if they were ashamed of the lies they told.

They paced the whole of this place once, in a constrained and
uncomfortable manner, when the boy stopped and said:

‘Lizzie, Mr Headstone has something to say to you. I don’t wish to be an
interruption either to him or to you, and so I’ll go and take a little
stroll and come back. I know in a general way what Mr Headstone intends
to say, and I very highly approve of it, as I hope--and indeed I do
not doubt--you will. I needn’t tell you, Lizzie, that I am under great
obligations to Mr Headstone, and that I am very anxious for Mr Headstone
to succeed in all he undertakes. As I hope--and as, indeed, I don’t
doubt--you must be.’

‘Charley,’ returned his sister, detaining his hand as he withdrew it, ‘I
think you had better stay. I think Mr Headstone had better not say what
he thinks of saying.’

‘Why, how do you know what it is?’ returned the boy.

‘Perhaps I don’t, but--’

‘Perhaps you don’t? No, Liz, I should think not. If you knew what
it was, you would give me a very different answer. There; let go; be
sensible. I wonder you don’t remember that Mr Headstone is looking on.’

She allowed him to separate himself from her, and he, after saying, ‘Now
Liz, be a rational girl and a good sister,’ walked away. She remained
standing alone with Bradley Headstone, and it was not until she raised
her eyes, that he spoke.

‘I said,’ he began, ‘when I saw you last, that there was something
unexplained, which might perhaps influence you. I have come this evening
to explain it. I hope you will not judge of me by my hesitating manner
when I speak to you. You see me at my greatest disadvantage. It is most
unfortunate for me that I wish you to see me at my best, and that I know
you see me at my worst.’

She moved slowly on when he paused, and he moved slowly on beside her.

‘It seems egotistical to begin by saying so much about myself,’ he
resumed, ‘but whatever I say to you seems, even in my own ears, below
what I want to say, and different from what I want to say. I can’t help
it. So it is. You are the ruin of me.’

She started at the passionate sound of the last words, and at the
passionate action of his hands, with which they were accompanied.

‘Yes! you are the ruin--the ruin--the ruin--of me. I have no resources
in myself, I have no confidence in myself, I have no government of
myself when you are near me or in my thoughts. And you are always in my
thoughts now. I have never been quit of you since I first saw you. Oh,
that was a wretched day for me! That was a wretched, miserable day!’

A touch of pity for him mingled with her dislike of him, and she said:
‘Mr Headstone, I am grieved to have done you any harm, but I have never
meant it.’

‘There!’ he cried, despairingly. ‘Now, I seem to have reproached you,
instead of revealing to you the state of my own mind! Bear with me. I am
always wrong when you are in question. It is my doom.’

Struggling with himself, and by times looking up at the deserted windows
of the houses as if there could be anything written in their grimy panes
that would help him, he paced the whole pavement at her side, before he
spoke again.

‘I must try to give expression to what is in my mind; it shall and must
be spoken. Though you see me so confounded--though you strike me so
helpless--I ask you to believe that there are many people who think well
of me; that there are some people who highly esteem me; that I have in
my way won a Station which is considered worth winning.’

‘Surely, Mr Headstone, I do believe it. Surely I have always known it
from Charley.’

‘I ask you to believe that if I were to offer my home such as it is, my
station such as it is, my affections such as they are, to any one of the
best considered, and best qualified, and most distinguished, among the
young women engaged in my calling, they would probably be accepted. Even
readily accepted.’

‘I do not doubt it,’ said Lizzie, with her eyes upon the ground.

‘I have sometimes had it in my thoughts to make that offer and to settle
down as many men of my class do: I on the one side of a school, my wife
on the other, both of us interested in the same work.’

‘Why have you not done so?’ asked Lizzie Hexam. ‘Why do you not do so?’

‘Far better that I never did! The only one grain of comfort I have had
these many weeks,’ he said, always speaking passionately, and, when
most emphatic, repeating that former action of his hands, which was
like flinging his heart’s blood down before her in drops upon the
pavement-stones; ‘the only one grain of comfort I have had these many
weeks is, that I never did. For if I had, and if the same spell had come
upon me for my ruin, I know I should have broken that tie asunder as if
it had been thread.’

She glanced at him with a glance of fear, and a shrinking gesture. He
answered, as if she had spoken.

‘No! It would not have been voluntary on my part, any more than it is
voluntary in me to be here now. You draw me to you. If I were shut up in
a strong prison, you would draw me out. I should break through the wall
to come to you. If I were lying on a sick bed, you would draw me up--to
stagger to your feet and fall there.’

The wild energy of the man, now quite let loose, was absolutely
terrible. He stopped and laid his hand upon a piece of the coping of the
burial-ground enclosure, as if he would have dislodged the stone.

‘No man knows till the time comes, what depths are within him. To some
men it never comes; let them rest and be thankful! To me, you brought
it; on me, you forced it; and the bottom of this raging sea,’ striking
himself upon the breast, ‘has been heaved up ever since.’

‘Mr Headstone, I have heard enough. Let me stop you here. It will be
better for you and better for me. Let us find my brother.’

‘Not yet. It shall and must be spoken. I have been in torments ever
since I stopped short of it before. You are alarmed. It is another of my
miseries that I cannot speak to you or speak of you without stumbling at
every syllable, unless I let the check go altogether and run mad. Here
is a man lighting the lamps. He will be gone directly. I entreat of you
let us walk round this place again. You have no reason to look alarmed;
I can restrain myself, and I will.’

She yielded to the entreaty--how could she do otherwise!--and they paced
the stones in silence. One by one the lights leaped up making the cold
grey church tower more remote, and they were alone again. He said no
more until they had regained the spot where he had broken off; there, he
again stood still, and again grasped the stone. In saying what he said
then, he never looked at her; but looked at it and wrenched at it.

‘You know what I am going to say. I love you. What other men may mean
when they use that expression, I cannot tell; what I mean is, that I am
under the influence of some tremendous attraction which I have resisted
in vain, and which overmasters me. You could draw me to fire, you could
draw me to water, you could draw me to the gallows, you could draw me to
any death, you could draw me to anything I have most avoided, you could
draw me to any exposure and disgrace. This and the confusion of my
thoughts, so that I am fit for nothing, is what I mean by your being the
ruin of me. But if you would return a favourable answer to my offer
of myself in marriage, you could draw me to any good--every good--with
equal force. My circumstances are quite easy, and you would want for
nothing. My reputation stands quite high, and would be a shield for
yours. If you saw me at my work, able to do it well and respected in
it, you might even come to take a sort of pride in me;--I would try hard
that you should. Whatever considerations I may have thought of against
this offer, I have conquered, and I make it with all my heart. Your
brother favours me to the utmost, and it is likely that we might live
and work together; anyhow, it is certain that he would have my best
influence and support. I don’t know what I could say more if I tried. I
might only weaken what is ill enough said as it is. I only add that
if it is any claim on you to be in earnest, I am in thorough earnest,
dreadful earnest.’

The powdered mortar from under the stone at which he wrenched, rattled
on the pavement to confirm his words.

‘Mr Headstone--’

‘Stop! I implore you, before you answer me, to walk round this place
once more. It will give you a minute’s time to think, and me a minute’s
time to get some fortitude together.’

Again she yielded to the entreaty, and again they came back to the same
place, and again he worked at the stone.

‘Is it,’ he said, with his attention apparently engrossed by it, ‘yes,
or no?’

‘Mr Headstone, I thank you sincerely, I thank you gratefully, and hope
you may find a worthy wife before long and be very happy. But it is no.’

‘Is no short time necessary for reflection; no weeks or days?’ he asked,
in the same half-suffocated way.

‘None whatever.’

‘Are you quite decided, and is there no chance of any change in my
favour?’

‘I am quite decided, Mr Headstone, and I am bound to answer I am certain
there is none.’

‘Then,’ said he, suddenly changing his tone and turning to her, and
bringing his clenched hand down upon the stone with a force that laid
the knuckles raw and bleeding; ‘then I hope that I may never kill him!’

The dark look of hatred and revenge with which the words broke from his
livid lips, and with which he stood holding out his smeared hand as
if it held some weapon and had just struck a mortal blow, made her so
afraid of him that she turned to run away. But he caught her by the arm.

‘Mr Headstone, let me go. Mr Headstone, I must call for help!’

‘It is I who should call for help,’ he said; ‘you don’t know yet how
much I need it.’

The working of his face as she shrank from it, glancing round for her
brother and uncertain what to do, might have extorted a cry from her in
another instant; but all at once he sternly stopped it and fixed it, as
if Death itself had done so.

‘There! You see I have recovered myself. Hear me out.’

With much of the dignity of courage, as she recalled her self-reliant
life and her right to be free from accountability to this man, she
released her arm from his grasp and stood looking full at him. She had
never been so handsome, in his eyes. A shade came over them while
he looked back at her, as if she drew the very light out of them to
herself.

‘This time, at least, I will leave nothing unsaid,’ he went on, folding
his hands before him, clearly to prevent his being betrayed into any
impetuous gesture; ‘this last time at least I will not be tortured with
after-thoughts of a lost opportunity. Mr Eugene Wrayburn.’

‘Was it of him you spoke in your ungovernable rage and violence?’ Lizzie
Hexam demanded with spirit.

He bit his lip, and looked at her, and said never a word.

‘Was it Mr Wrayburn that you threatened?’

He bit his lip again, and looked at her, and said never a word.

‘You asked me to hear you out, and you will not speak. Let me find my
brother.’

‘Stay! I threatened no one.’

Her look dropped for an instant to his bleeding hand. He lifted it to
his mouth, wiped it on his sleeve, and again folded it over the other.
‘Mr Eugene Wrayburn,’ he repeated.

‘Why do you mention that name again and again, Mr Headstone?’

‘Because it is the text of the little I have left to say. Observe! There
are no threats in it. If I utter a threat, stop me, and fasten it upon
me. Mr Eugene Wrayburn.’

A worse threat than was conveyed in his manner of uttering the name,
could hardly have escaped him.

‘He haunts you. You accept favours from him. You are willing enough to
listen to HIM. I know it, as well as he does.’

‘Mr Wrayburn has been considerate and good to me, sir,’ said Lizzie,
proudly, ‘in connexion with the death and with the memory of my poor
father.’

‘No doubt. He is of course a very considerate and a very good man, Mr
Eugene Wrayburn.’

‘He is nothing to you, I think,’ said Lizzie, with an indignation she
could not repress.

‘Oh yes, he is. There you mistake. He is much to me.’

‘What can he be to you?’

‘He can be a rival to me among other things,’ said Bradley.

‘Mr Headstone,’ returned Lizzie, with a burning face, ‘it is cowardly in
you to speak to me in this way. But it makes me able to tell you that
I do not like you, and that I never have liked you from the first, and
that no other living creature has anything to do with the effect you
have produced upon me for yourself.’

His head bent for a moment, as if under a weight, and he then looked up
again, moistening his lips. ‘I was going on with the little I had left
to say. I knew all this about Mr Eugene Wrayburn, all the while you were
drawing me to you. I strove against the knowledge, but quite in vain. It
made no difference in me. With Mr Eugene Wrayburn in my mind, I went
on. With Mr Eugene Wrayburn in my mind, I spoke to you just now. With Mr
Eugene Wrayburn in my mind, I have been set aside and I have been cast
out.’

‘If you give those names to my thanking you for your proposal
and declining it, is it my fault, Mr Headstone?’ said Lizzie,
compassionating the bitter struggle he could not conceal, almost as much
as she was repelled and alarmed by it.

‘I am not complaining,’ he returned, ‘I am only stating the case. I had
to wrestle with my self-respect when I submitted to be drawn to you in
spite of Mr Wrayburn. You may imagine how low my self-respect lies now.’

She was hurt and angry; but repressed herself in consideration of his
suffering, and of his being her brother’s friend.

‘And it lies under his feet,’ said Bradley, unfolding his hands in spite
of himself, and fiercely motioning with them both towards the stones of
the pavement. ‘Remember that! It lies under that fellow’s feet, and he
treads upon it and exults above it.’

‘He does not!’ said Lizzie.

‘He does!’ said Bradley. ‘I have stood before him face to face, and he
crushed me down in the dirt of his contempt, and walked over me. Why?
Because he knew with triumph what was in store for me to-night.’

‘O, Mr Headstone, you talk quite wildly.’

‘Quite collectedly. I know what I say too well. Now I have said all. I
have used no threat, remember; I have done no more than show you how the
case stands;--how the case stands, so far.’

At this moment her brother sauntered into view close by. She darted to
him, and caught him by the hand. Bradley followed, and laid his heavy
hand on the boy’s opposite shoulder.

‘Charley Hexam, I am going home. I must walk home by myself to-night,
and get shut up in my room without being spoken to. Give me half an
hour’s start, and let me be, till you find me at my work in the morning.
I shall be at my work in the morning just as usual.’

Clasping his hands, he uttered a short unearthly broken cry, and went
his way. The brother and sister were left looking at one another near
a lamp in the solitary churchyard, and the boy’s face clouded and
darkened, as he said in a rough tone: ‘What is the meaning of this? What
have you done to my best friend? Out with the truth!’

‘Charley!’ said his sister. ‘Speak a little more considerately!’

‘I am not in the humour for consideration, or for nonsense of any sort,’
replied the boy. ‘What have you been doing? Why has Mr Headstone gone
from us in that way?’

‘He asked me--you know he asked me--to be his wife, Charley.’

‘Well?’ said the boy, impatiently.

‘And I was obliged to tell him that I could not be his wife.’

‘You were obliged to tell him,’ repeated the boy angrily, between his
teeth, and rudely pushing her away. ‘You were obliged to tell him! Do
you know that he is worth fifty of you?’

‘It may easily be so, Charley, but I cannot marry him.’

‘You mean that you are conscious that you can’t appreciate him, and
don’t deserve him, I suppose?’

‘I mean that I do not like him, Charley, and that I will never marry
him.’

‘Upon my soul,’ exclaimed the boy, ‘you are a nice picture of a sister!
Upon my soul, you are a pretty piece of disinterestedness! And so all my
endeavours to cancel the past and to raise myself in the world, and to
raise you with me, are to be beaten down by YOUR low whims; are they?’

‘I will not reproach you, Charley.’

‘Hear her!’ exclaimed the boy, looking round at the darkness. ‘She won’t
reproach me! She does her best to destroy my fortunes and her own,
and she won’t reproach me! Why, you’ll tell me, next, that you won’t
reproach Mr Headstone for coming out of the sphere to which he is an
ornament, and putting himself at YOUR feet, to be rejected by YOU!’

‘No, Charley; I will only tell you, as I told himself, that I thank him
for doing so, that I am sorry he did so, and that I hope he will do much
better, and be happy.’

Some touch of compunction smote the boy’s hardening heart as he looked
upon her, his patient little nurse in infancy, his patient friend,
adviser, and reclaimer in boyhood, the self-forgetting sister who had
done everything for him. His tone relented, and he drew her arm through
his.

‘Now, come, Liz; don’t let us quarrel: let us be reasonable and talk
this over like brother and sister. Will you listen to me?’

‘Oh, Charley!’ she replied through her starting tears; ‘do I not listen
to you, and hear many hard things!’

‘Then I am sorry. There, Liz! I am unfeignedly sorry. Only you do put me
out so. Now see. Mr Headstone is perfectly devoted to you. He has told
me in the strongest manner that he has never been his old self for one
single minute since I first brought him to see you. Miss Peecher, our
schoolmistress--pretty and young, and all that--is known to be very much
attached to him, and he won’t so much as look at her or hear of her.
Now, his devotion to you must be a disinterested one; mustn’t it? If he
married Miss Peecher, he would be a great deal better off in all worldly
respects, than in marrying you. Well then; he has nothing to get by it,
has he?’

‘Nothing, Heaven knows!’

‘Very well then,’ said the boy; ‘that’s something in his favour, and a
great thing. Then I come in. Mr Headstone has always got me on, and he
has a good deal in his power, and of course if he was my brother-in-law
he wouldn’t get me on less, but would get me on more. Mr Headstone
comes and confides in me, in a very delicate way, and says, “I hope my
marrying your sister would be agreeable to you, Hexam, and useful to
you?” I say, “There’s nothing in the world, Mr Headstone, that I could
be better pleased with.” Mr Headstone says, “Then I may rely upon your
intimate knowledge of me for your good word with your sister, Hexam?”
 And I say, “Certainly, Mr Headstone, and naturally I have a good deal of
influence with her.” So I have; haven’t I, Liz?’

‘Yes, Charley.’

‘Well said! Now, you see, we begin to get on, the moment we begin to
be really talking it over, like brother and sister. Very well. Then
YOU come in. As Mr Headstone’s wife you would be occupying a most
respectable station, and you would be holding a far better place in
society than you hold now, and you would at length get quit of the
river-side and the old disagreeables belonging to it, and you would be
rid for good of dolls’ dressmakers and their drunken fathers, and the
like of that. Not that I want to disparage Miss Jenny Wren: I dare
say she is all very well in her way; but her way is not your way as
Mr Headstone’s wife. Now, you see, Liz, on all three accounts--on
Mr Headstone’s, on mine, on yours--nothing could be better or more
desirable.’

They were walking slowly as the boy spoke, and here he stood still, to
see what effect he had made. His sister’s eyes were fixed upon him; but
as they showed no yielding, and as she remained silent, he walked her on
again. There was some discomfiture in his tone as he resumed, though he
tried to conceal it.

‘Having so much influence with you, Liz, as I have, perhaps I should
have done better to have had a little chat with you in the first
instance, before Mr Headstone spoke for himself. But really all this in
his favour seemed so plain and undeniable, and I knew you to have always
been so reasonable and sensible, that I didn’t consider it worth while.
Very likely that was a mistake of mine. However, it’s soon set right.
All that need be done to set it right, is for you to tell me at once
that I may go home and tell Mr Headstone that what has taken place is
not final, and that it will all come round by-and-by.’

He stopped again. The pale face looked anxiously and lovingly at him,
but she shook her head.

‘Can’t you speak?’ said the boy sharply.

‘I am very unwilling to speak, Charley. If I must, I must. I cannot
authorize you to say any such thing to Mr Headstone: I cannot allow you
to say any such thing to Mr Headstone. Nothing remains to be said to him
from me, after what I have said for good and all, to-night.’

‘And this girl,’ cried the boy, contemptuously throwing her off again,
‘calls herself a sister!’

‘Charley, dear, that is the second time that you have almost struck
me. Don’t be hurt by my words. I don’t mean--Heaven forbid!--that you
intended it; but you hardly know with what a sudden swing you removed
yourself from me.’

‘However!’ said the boy, taking no heed of the remonstrance, and
pursuing his own mortified disappointment, ‘I know what this means, and
you shall not disgrace me.’

‘It means what I have told you, Charley, and nothing more.’

‘That’s not true,’ said the boy in a violent tone, ‘and you know it’s
not. It means your precious Mr Wrayburn; that’s what it means.’

‘Charley! If you remember any old days of ours together, forbear!’

‘But you shall not disgrace me,’ doggedly pursued the boy. ‘I am
determined that after I have climbed up out of the mire, you shall not
pull me down. You can’t disgrace me if I have nothing to do with you,
and I will have nothing to do with you for the future.’

‘Charley! On many a night like this, and many a worse night, I have sat
on the stones of the street, hushing you in my arms. Unsay those words
without even saying you are sorry for them, and my arms are open to you
still, and so is my heart.’

‘I’ll not unsay them. I’ll say them again. You are an inveterately bad
girl, and a false sister, and I have done with you. For ever, I have
done with you!’

He threw up his ungrateful and ungracious hand as if it set up a barrier
between them, and flung himself upon his heel and left her. She remained
impassive on the same spot, silent and motionless, until the striking
of the church clock roused her, and she turned away. But then, with the
breaking up of her immobility came the breaking up of the waters that
the cold heart of the selfish boy had frozen. And ‘O that I were lying
here with the dead!’ and ‘O Charley, Charley, that this should be the
end of our pictures in the fire!’ were all the words she said, as she
laid her face in her hands on the stone coping.

A figure passed by, and passed on, but stopped and looked round at
her. It was the figure of an old man with a bowed head, wearing a large
brimmed low-crowned hat, and a long-skirted coat. After hesitating a
little, the figure turned back, and, advancing with an air of gentleness
and compassion, said:

‘Pardon me, young woman, for speaking to you, but you are under some
distress of mind. I cannot pass upon my way and leave you weeping here
alone, as if there was nothing in the place. Can I help you? Can I do
anything to give you comfort?’

She raised her head at the sound of these kind words, and answered
gladly, ‘O, Mr Riah, is it you?’

‘My daughter,’ said the old man, ‘I stand amazed! I spoke as to a
stranger. Take my arm, take my arm. What grieves you? Who has done this?
Poor girl, poor girl!’

‘My brother has quarrelled with me,’ sobbed Lizzie, ‘and renounced me.’

‘He is a thankless dog,’ said the Jew, angrily. ‘Let him go. Shake the
dust from thy feet and let him go. Come, daughter! Come home with me--it
is but across the road--and take a little time to recover your peace and
to make your eyes seemly, and then I will bear you company through the
streets. For it is past your usual time, and will soon be late, and the
way is long, and there is much company out of doors to-night.’

She accepted the support he offered her, and they slowly passed out
of the churchyard. They were in the act of emerging into the main
thoroughfare, when another figure loitering discontentedly by, and
looking up the street and down it, and all about, started and exclaimed,
‘Lizzie! why, where have you been? Why, what’s the matter?’

As Eugene Wrayburn thus addressed her, she drew closer to the Jew, and
bent her head. The Jew having taken in the whole of Eugene at one sharp
glance, cast his eyes upon the ground, and stood mute.

‘Lizzie, what is the matter?’

‘Mr Wrayburn, I cannot tell you now. I cannot tell you to-night, if I
ever can tell you. Pray leave me.’

‘But, Lizzie, I came expressly to join you. I came to walk home with
you, having dined at a coffee-house in this neighbourhood and knowing
your hour. And I have been lingering about,’ added Eugene, ‘like a
bailiff; or,’ with a look at Riah, ‘an old clothesman.’

The Jew lifted up his eyes, and took in Eugene once more, at another
glance.

‘Mr Wrayburn, pray, pray, leave me with this protector. And one thing
more. Pray, pray be careful of yourself.’

‘Mysteries of Udolpho!’ said Eugene, with a look of wonder. ‘May I be
excused for asking, in the elderly gentleman’s presence, who is this
kind protector?’

‘A trustworthy friend,’ said Lizzie.

‘I will relieve him of his trust,’ returned Eugene. ‘But you must tell
me, Lizzie, what is the matter?’

‘Her brother is the matter,’ said the old man, lifting up his eyes
again.

‘Our brother the matter?’ returned Eugene, with airy contempt. ‘Our
brother is not worth a thought, far less a tear. What has our brother
done?’

The old man lifted up his eyes again, with one grave look at Wrayburn,
and one grave glance at Lizzie, as she stood looking down. Both were so
full of meaning that even Eugene was checked in his light career, and
subsided into a thoughtful ‘Humph!’

With an air of perfect patience the old man, remaining mute and keeping
his eyes cast down, stood, retaining Lizzie’s arm, as though in his
habit of passive endurance, it would be all one to him if he had stood
there motionless all night.

‘If Mr Aaron,’ said Eugene, who soon found this fatiguing, ‘will be good
enough to relinquish his charge to me, he will be quite free for any
engagement he may have at the Synagogue. Mr Aaron, will you have the
kindness?’

But the old man stood stock still.

‘Good evening, Mr Aaron,’ said Eugene, politely; ‘we need not detain
you.’ Then turning to Lizzie, ‘Is our friend Mr Aaron a little deaf?’

‘My hearing is very good, Christian gentleman,’ replied the old man,
calmly; ‘but I will hear only one voice to-night, desiring me to leave
this damsel before I have conveyed her to her home. If she requests it,
I will do it. I will do it for no one else.’

‘May I ask why so, Mr Aaron?’ said Eugene, quite undisturbed in his
ease.

‘Excuse me. If she asks me, I will tell her,’ replied the old man. ‘I
will tell no one else.’

‘I do not ask you,’ said Lizzie, ‘and I beg you to take me home. Mr
Wrayburn, I have had a bitter trial to-night, and I hope you will not
think me ungrateful, or mysterious, or changeable. I am neither; I am
wretched. Pray remember what I said to you. Pray, pray, take care.’

‘My dear Lizzie,’ he returned, in a low voice, bending over her on the
other side; ‘of what? Of whom?’

‘Of any one you have lately seen and made angry.’

He snapped his fingers and laughed. ‘Come,’ said he, ‘since no better
may be, Mr Aaron and I will divide this trust, and see you home
together. Mr Aaron on that side; I on this. If perfectly agreeable to Mr
Aaron, the escort will now proceed.’

He knew his power over her. He knew that she would not insist upon his
leaving her. He knew that, her fears for him being aroused, she would
be uneasy if he were out of her sight. For all his seeming levity and
carelessness, he knew whatever he chose to know of the thoughts of her
heart.

And going on at her side, so gaily, regardless of all that had been
urged against him; so superior in his sallies and self-possession to
the gloomy constraint of her suitor and the selfish petulance of her
brother; so faithful to her, as it seemed, when her own stock was
faithless; what an immense advantage, what an overpowering influence,
were his that night! Add to the rest, poor girl, that she had heard him
vilified for her sake, and that she had suffered for his, and where the
wonder that his occasional tones of serious interest (setting off his
carelessness, as if it were assumed to calm her), that his lightest
touch, his lightest look, his very presence beside her in the dark
common street, were like glimpses of an enchanted world, which it was
natural for jealousy and malice and all meanness to be unable to bear
the brightness of, and to gird at as bad spirits might.

Nothing more being said of repairing to Riah’s, they went direct to
Lizzie’s lodging. A little short of the house-door she parted from them,
and went in alone.

‘Mr Aaron,’ said Eugene, when they were left together in the street,
‘with many thanks for your company, it remains for me unwillingly to say
Farewell.’

‘Sir,’ returned the other, ‘I give you good night, and I wish that you
were not so thoughtless.’

‘Mr Aaron,’ returned Eugene, ‘I give you good night, and I wish (for you
are a little dull) that you were not so thoughtful.’

But now, that his part was played out for the evening, and when in
turning his back upon the Jew he came off the stage, he was thoughtful
himself. ‘How did Lightwood’s catechism run?’ he murmured, as he stopped
to light his cigar. ‘What is to come of it? What are you doing? Where
are you going? We shall soon know now. Ah!’ with a heavy sigh.

The heavy sigh was repeated as if by an echo, an hour afterwards, when
Riah, who had been sitting on some dark steps in a corner over against
the house, arose and went his patient way; stealing through the streets
in his ancient dress, like the ghost of a departed Time.



Chapter 16

AN ANNIVERSARY OCCASION


The estimable Twemlow, dressing himself in his lodgings over the
stable-yard in Duke Street, Saint James’s, and hearing the horses at
their toilette below, finds himself on the whole in a disadvantageous
position as compared with the noble animals at livery. For whereas, on
the one hand, he has no attendant to slap him soundingly and require him
in gruff accents to come up and come over, still, on the other hand,
he has no attendant at all; and the mild gentleman’s finger-joints and
other joints working rustily in the morning, he could deem it agreeable
even to be tied up by the countenance at his chamber-door, so he were
there skilfully rubbed down and slushed and sluiced and polished and
clothed, while himself taking merely a passive part in these trying
transactions.

How the fascinating Tippins gets on when arraying herself for the
bewilderment of the senses of men, is known only to the Graces and her
maid; but perhaps even that engaging creature, though not reduced to
the self-dependence of Twemlow could dispense with a good deal of the
trouble attendant on the daily restoration of her charms, seeing that
as to her face and neck this adorable divinity is, as it were, a diurnal
species of lobster--throwing off a shell every forenoon, and needing to
keep in a retired spot until the new crust hardens.

Howbeit, Twemlow doth at length invest himself with collar and cravat
and wristbands to his knuckles, and goeth forth to breakfast. And to
breakfast with whom but his near neighbours, the Lammles of Sackville
Street, who have imparted to him that he will meet his distant kinsman,
Mr Fledgely. The awful Snigsworth might taboo and prohibit Fledgely, but
the peaceable Twemlow reasons, If he IS my kinsman I didn’t make him so,
and to meet a man is not to know him.’

It is the first anniversary of the happy marriage of Mr and Mrs Lammle,
and the celebration is a breakfast, because a dinner on the desired
scale of sumptuosity cannot be achieved within less limits than those
of the non-existent palatial residence of which so many people are
madly envious. So, Twemlow trips with not a little stiffness across
Piccadilly, sensible of having once been more upright in figure and less
in danger of being knocked down by swift vehicles. To be sure that was
in the days when he hoped for leave from the dread Snigsworth to do
something, or be something, in life, and before that magnificent Tartar
issued the ukase, ‘As he will never distinguish himself, he must be a
poor gentleman-pensioner of mine, and let him hereby consider himself
pensioned.’

Ah! my Twemlow! Say, little feeble grey personage, what thoughts are in
thy breast to-day, of the Fancy--so still to call her who bruised thy
heart when it was green and thy head brown--and whether it be better or
worse, more painful or less, to believe in the Fancy to this hour, than
to know her for a greedy armour-plated crocodile, with no more capacity
of imagining the delicate and sensitive and tender spot behind thy
waistcoat, than of going straight at it with a knitting-needle. Say
likewise, my Twemlow, whether it be the happier lot to be a poor
relation of the great, or to stand in the wintry slush giving the hack
horses to drink out of the shallow tub at the coach-stand, into which
thou has so nearly set thy uncertain foot. Twemlow says nothing, and
goes on.

As he approaches the Lammles’ door, drives up a little one-horse
carriage, containing Tippins the divine. Tippins, letting down the
window, playfully extols the vigilance of her cavalier in being in
waiting there to hand her out. Twemlow hands her out with as much polite
gravity as if she were anything real, and they proceed upstairs. Tippins
all abroad about the legs, and seeking to express that those unsteady
articles are only skipping in their native buoyancy.

And dear Mrs Lammle and dear Mr Lammle, how do you do, and when are
you going down to what’s-its-name place--Guy, Earl of Warwick, you
know--what is it?--Dun Cow--to claim the flitch of bacon? And Mortimer,
whose name is for ever blotted out from my list of lovers, by reason
first of fickleness and then of base desertion, how do YOU do, wretch?
And Mr Wrayburn, YOU here! What can YOU come for, because we are all
very sure before-hand that you are not going to talk! And Veneering,
M.P., how are things going on down at the house, and when will you turn
out those terrible people for us? And Mrs Veneering, my dear, can it
positively be true that you go down to that stifling place night after
night, to hear those men prose? Talking of which, Veneering, why don’t
you prose, for you haven’t opened your lips there yet, and we are dying
to hear what you have got to say to us! Miss Podsnap, charmed to see
you. Pa, here? No! Ma, neither? Oh! Mr Boots! Delighted. Mr Brewer!
This IS a gathering of the clans. Thus Tippins, and surveys Fledgeby and
outsiders through golden glass, murmuring as she turns about and about,
in her innocent giddy way, Anybody else I know? No, I think not. Nobody
there. Nobody THERE. Nobody anywhere!

Mr Lammle, all a-glitter, produces his friend Fledgeby, as dying for the
honour of presentation to Lady Tippins. Fledgeby presented, has the air
of going to say something, has the air of going to say nothing, has an
air successively of meditation, of resignation, and of desolation,
backs on Brewer, makes the tour of Boots, and fades into the extreme
background, feeling for his whisker, as if it might have turned up since
he was there five minutes ago.

But Lammle has him out again before he has so much as completely
ascertained the bareness of the land. He would seem to be in a bad way,
Fledgeby; for Lammle represents him as dying again. He is dying now, of
want of presentation to Twemlow.

Twemlow offers his hand. Glad to see him. ‘Your mother, sir, was a
connexion of mine.’

‘I believe so,’ says Fledgeby, ‘but my mother and her family were two.’

‘Are you staying in town?’ asks Twemlow.

‘I always am,’ says Fledgeby.

‘You like town,’ says Twemlow. But is felled flat by Fledgeby’s taking
it quite ill, and replying, No, he don’t like town. Lammle tries to
break the force of the fall, by remarking that some people do not like
town. Fledgeby retorting that he never heard of any such case but his
own, Twemlow goes down again heavily.

‘There is nothing new this morning, I suppose?’ says Twemlow, returning
to the mark with great spirit.

Fledgeby has not heard of anything.

‘No, there’s not a word of news,’ says Lammle.

‘Not a particle,’ adds Boots.

‘Not an atom,’ chimes in Brewer.

Somehow the execution of this little concerted piece appears to raise
the general spirits as with a sense of duty done, and sets the company a
going. Everybody seems more equal than before, to the calamity of being
in the society of everybody else. Even Eugene standing in a window,
moodily swinging the tassel of a blind, gives it a smarter jerk now, as
if he found himself in better case.

Breakfast announced. Everything on table showy and gaudy, but with
a self-assertingly temporary and nomadic air on the decorations, as
boasting that they will be much more showy and gaudy in the palatial
residence. Mr Lammle’s own particular servant behind his chair; the
Analytical behind Veneering’s chair; instances in point that
such servants fall into two classes: one mistrusting the master’s
acquaintances, and the other mistrusting the master. Mr Lammle’s
servant, of the second class. Appearing to be lost in wonder and low
spirits because the police are so long in coming to take his master up
on some charge of the first magnitude.

Veneering, M.P., on the right of Mrs Lammle; Twemlow on her left; Mrs
Veneering, W.M.P. (wife of Member of Parliament), and Lady Tippins on Mr
Lammle’s right and left. But be sure that well within the fascination of
Mr Lammle’s eye and smile sits little Georgiana. And be sure that
close to little Georgiana, also under inspection by the same gingerous
gentleman, sits Fledgeby.

Oftener than twice or thrice while breakfast is in progress, Mr Twemlow
gives a little sudden turn towards Mrs Lammle, and then says to her, ‘I
beg your pardon!’ This not being Twemlow’s usual way, why is it his
way to-day? Why, the truth is, Twemlow repeatedly labours under the
impression that Mrs Lammle is going to speak to him, and turning finds
that it is not so, and mostly that she has her eyes upon Veneering.
Strange that this impression so abides by Twemlow after being corrected,
yet so it is.

Lady Tippins partaking plentifully of the fruits of the earth (including
grape-juice in the category) becomes livelier, and applies herself to
elicit sparks from Mortimer Lightwood. It is always understood among the
initiated, that that faithless lover must be planted at table opposite
to Lady Tippins, who will then strike conversational fire out of him.
In a pause of mastication and deglutition, Lady Tippins, contemplating
Mortimer, recalls that it was at our dear Veneerings, and in the
presence of a party who are surely all here, that he told them his
story of the man from somewhere, which afterwards became so horribly
interesting and vulgarly popular.

‘Yes, Lady Tippins,’ assents Mortimer; ‘as they say on the stage, “Even
so!”’

‘Then we expect you,’ retorts the charmer, ‘to sustain your reputation,
and tell us something else.’

‘Lady Tippins, I exhausted myself for life that day, and there is
nothing more to be got out of me.’

Mortimer parries thus, with a sense upon him that elsewhere it is Eugene
and not he who is the jester, and that in these circles where Eugene
persists in being speechless, he, Mortimer, is but the double of the
friend on whom he has founded himself.

‘But,’ quoth the fascinating Tippins, ‘I am resolved on getting
something more out of you. Traitor! what is this I hear about another
disappearance?’

‘As it is you who have heard it,’ returns Lightwood, ‘perhaps you’ll
tell us.’

‘Monster, away!’ retorts Lady Tippins. ‘Your own Golden Dustman referred
me to you.’

Mr Lammle, striking in here, proclaims aloud that there is a sequel
to the story of the man from somewhere. Silence ensues upon the
proclamation.

‘I assure you,’ says Lightwood, glancing round the table, ‘I have
nothing to tell.’ But Eugene adding in a low voice, ‘There, tell
it, tell it!’ he corrects himself with the addition, ‘Nothing worth
mentioning.’

Boots and Brewer immediately perceive that it is immensely worth
mentioning, and become politely clamorous. Veneering is also visited by
a perception to the same effect. But it is understood that his attention
is now rather used up, and difficult to hold, that being the tone of the
House of Commons.

‘Pray don’t be at the trouble of composing yourselves to listen,’ says
Mortimer Lightwood, ‘because I shall have finished long before you have
fallen into comfortable attitudes. It’s like--’

‘It’s like,’ impatiently interrupts Eugene, ‘the children’s narrative:

     “I’ll tell you a story
     Of Jack a Manory,
     And now my story’s begun;
     I’ll tell you another
     Of Jack and his brother,
     And now my story is done.”

--Get on, and get it over!’

Eugene says this with a sound of vexation in his voice, leaning back in
his chair and looking balefully at Lady Tippins, who nods to him as
her dear Bear, and playfully insinuates that she (a self-evident
proposition) is Beauty, and he Beast.

‘The reference,’ proceeds Mortimer, ‘which I suppose to be made by my
honourable and fair enslaver opposite, is to the following circumstance.
Very lately, the young woman, Lizzie Hexam, daughter of the late Jesse
Hexam, otherwise Gaffer, who will be remembered to have found the body
of the man from somewhere, mysteriously received, she knew not from
whom, an explicit retraction of the charges made against her father, by
another water-side character of the name of Riderhood. Nobody believed
them, because little Rogue Riderhood--I am tempted into the paraphrase
by remembering the charming wolf who would have rendered society a great
service if he had devoured Mr Riderhood’s father and mother in their
infancy--had previously played fast and loose with the said charges,
and, in fact, abandoned them. However, the retraction I have mentioned
found its way into Lizzie Hexam’s hands, with a general flavour on it
of having been favoured by some anonymous messenger in a dark cloak and
slouched hat, and was by her forwarded, in her father’s vindication, to
Mr Boffin, my client. You will excuse the phraseology of the shop, but
as I never had another client, and in all likelihood never shall have, I
am rather proud of him as a natural curiosity probably unique.’

Although as easy as usual on the surface, Lightwood is not quite as easy
as usual below it. With an air of not minding Eugene at all, he feels
that the subject is not altogether a safe one in that connexion.

‘The natural curiosity which forms the sole ornament of my professional
museum,’ he resumes, ‘hereupon desires his Secretary--an individual
of the hermit-crab or oyster species, and whose name, I think, is
Chokesmith--but it doesn’t in the least matter--say Artichoke--to put
himself in communication with Lizzie Hexam. Artichoke professes his
readiness so to do, endeavours to do so, but fails.’

‘Why fails?’ asks Boots.

‘How fails?’ asks Brewer.

‘Pardon me,’ returns Lightwood, ‘I must postpone the reply for one
moment, or we shall have an anti-climax. Artichoke failing signally, my
client refers the task to me: his purpose being to advance the interests
of the object of his search. I proceed to put myself in communication
with her; I even happen to possess some special means,’ with a glance
at Eugene, ‘of putting myself in communication with her; but I fail too,
because she has vanished.’

‘Vanished!’ is the general echo.

‘Disappeared,’ says Mortimer. ‘Nobody knows how, nobody knows when,
nobody knows where. And so ends the story to which my honourable and
fair enslaver opposite referred.’

Tippins, with a bewitching little scream, opines that we shall every one
of us be murdered in our beds. Eugene eyes her as if some of us would
be enough for him. Mrs Veneering, W.M.P., remarks that these social
mysteries make one afraid of leaving Baby. Veneering, M.P., wishes to
be informed (with something of a second-hand air of seeing the Right
Honourable Gentleman at the head of the Home Department in his place)
whether it is intended to be conveyed that the vanished person has been
spirited away or otherwise harmed? Instead of Lightwood’s answering,
Eugene answers, and answers hastily and vexedly: ‘No, no, no; he doesn’t
mean that; he means voluntarily vanished--but utterly--completely.’

However, the great subject of the happiness of Mr and Mrs Lammle must
not be allowed to vanish with the other vanishments--with the vanishing
of the murderer, the vanishing of Julius Handford, the vanishing of
Lizzie Hexam,--and therefore Veneering must recall the present sheep
to the pen from which they have strayed. Who so fit to discourse of
the happiness of Mr and Mrs Lammle, they being the dearest and oldest
friends he has in the world; or what audience so fit for him to take
into his confidence as that audience, a noun of multitude or signifying
many, who are all the oldest and dearest friends he has in the world?
So Veneering, without the formality of rising, launches into a familiar
oration, gradually toning into the Parliamentary sing-song, in which he
sees at that board his dear friend Twemlow who on that day twelvemonth
bestowed on his dear friend Lammle the fair hand of his dear friend
Sophronia, and in which he also sees at that board his dear friends
Boots and Brewer whose rallying round him at a period when his dear
friend Lady Tippins likewise rallied round him--ay, and in the foremost
rank--he can never forget while memory holds her seat. But he is free
to confess that he misses from that board his dear old friend Podsnap,
though he is well represented by his dear young friend Georgiana. And he
further sees at that board (this he announces with pomp, as if exulting
in the powers of an extraordinary telescope) his friend Mr Fledgeby, if
he will permit him to call him so. For all of these reasons, and many
more which he right well knows will have occurred to persons of your
exceptional acuteness, he is here to submit to you that the time has
arrived when, with our hearts in our glasses, with tears in our eyes,
with blessings on our lips, and in a general way with a profusion of
gammon and spinach in our emotional larders, we should one and all drink
to our dear friends the Lammles, wishing them many years as happy as
the last, and many many friends as congenially united as themselves. And
this he will add; that Anastatia Veneering (who is instantly heard to
weep) is formed on the same model as her old and chosen friend Sophronia
Lammle, in respect that she is devoted to the man who wooed and won her,
and nobly discharges the duties of a wife.

Seeing no better way out of it, Veneering here pulls up his oratorical
Pegasus extremely short, and plumps down, clean over his head, with:
‘Lammle, God bless you!’

Then Lammle. Too much of him every way; pervadingly too much nose of a
coarse wrong shape, and his nose in his mind and his manners; too much
smile to be real; too much frown to be false; too many large teeth to be
visible at once without suggesting a bite. He thanks you, dear friends,
for your kindly greeting, and hopes to receive you--it may be on the
next of these delightful occasions--in a residence better suited to
your claims on the rites of hospitality. He will never forget that at
Veneering’s he first saw Sophronia. Sophronia will never forget that at
Veneering’s she first saw him. ‘They spoke of it soon after they
were married, and agreed that they would never forget it. In fact, to
Veneering they owe their union. They hope to show their sense of this
some day [‘No, no, from Veneering)--oh yes, yes, and let him rely
upon it, they will if they can! His marriage with Sophronia was not a
marriage of interest on either side: she had her little fortune, he had
his little fortune: they joined their little fortunes: it was a marriage
of pure inclination and suitability. Thank you! Sophronia and he are
fond of the society of young people; but he is not sure that their house
would be a good house for young people proposing to remain single, since
the contemplation of its domestic bliss might induce them to change
their minds. He will not apply this to any one present; certainly not
to their darling little Georgiana. Again thank you! Neither, by-the-by,
will he apply it to his friend Fledgeby. He thanks Veneering for the
feeling manner in which he referred to their common friend Fledgeby, for
he holds that gentleman in the highest estimation. Thank you. In fact
(returning unexpectedly to Fledgeby), the better you know him, the more
you find in him that you desire to know. Again thank you! In his dear
Sophronia’s name and in his own, thank you!

Mrs Lammle has sat quite still, with her eyes cast down upon the
table-cloth. As Mr Lammle’s address ends, Twemlow once more turns to her
involuntarily, not cured yet of that often recurring impression that she
is going to speak to him. This time she really is going to speak to him.
Veneering is talking with his other next neighbour, and she speaks in a
low voice.

‘Mr Twemlow.’

He answers, ‘I beg your pardon? Yes?’ Still a little doubtful, because
of her not looking at him.

‘You have the soul of a gentleman, and I know I may trust you. Will you
give me the opportunity of saying a few words to you when you come up
stairs?’

‘Assuredly. I shall be honoured.’

‘Don’t seem to do so, if you please, and don’t think it inconsistent if
my manner should be more careless than my words. I may be watched.’

Intensely astonished, Twemlow puts his hand to his forehead, and sinks
back in his chair meditating. Mrs Lammle rises. All rise. The ladies go
up stairs. The gentlemen soon saunter after them. Fledgeby has devoted
the interval to taking an observation of Boots’s whiskers, Brewer’s
whiskers, and Lammle’s whiskers, and considering which pattern of
whisker he would prefer to produce out of himself by friction, if the
Genie of the cheek would only answer to his rubbing.

In the drawing-room, groups form as usual. Lightwood, Boots, and Brewer,
flutter like moths around that yellow wax candle--guttering down,
and with some hint of a winding-sheet in it--Lady Tippins. Outsiders
cultivate Veneering, M P., and Mrs Veneering, W.M.P. Lammle stands with
folded arms, Mephistophelean in a corner, with Georgiana and Fledgeby.
Mrs Lammle, on a sofa by a table, invites Mr Twemlow’s attention to a
book of portraits in her hand.

Mr Twemlow takes his station on a settee before her, and Mrs Lammle
shows him a portrait.

‘You have reason to be surprised,’ she says softly, ‘but I wish you
wouldn’t look so.’

Disturbed Twemlow, making an effort not to look so, looks much more so.

‘I think, Mr Twemlow, you never saw that distant connexion of yours
before to-day?’

‘No, never.’

‘Now that you do see him, you see what he is. You are not proud of him?’

‘To say the truth, Mrs Lammle, no.’

‘If you knew more of him, you would be less inclined to acknowledge him.
Here is another portrait. What do you think of it?’

Twemlow has just presence of mind enough to say aloud: ‘Very like!
Uncommonly like!’

‘You have noticed, perhaps, whom he favours with his attentions? You
notice where he is now, and how engaged?’

‘Yes. But Mr Lammle--’

She darts a look at him which he cannot comprehend, and shows him
another portrait.

‘Very good; is it not?’

‘Charming!’ says Twemlow.

‘So like as to be almost a caricature?--Mr Twemlow, it is impossible
to tell you what the struggle in my mind has been, before I could bring
myself to speak to you as I do now. It is only in the conviction that I
may trust you never to betray me, that I can proceed. Sincerely promise
me that you never will betray my confidence--that you will respect it,
even though you may no longer respect me,--and I shall be as satisfied
as if you had sworn it.’

‘Madam, on the honour of a poor gentleman--’

‘Thank you. I can desire no more. Mr Twemlow, I implore you to save that
child!’

‘That child?’

‘Georgiana. She will be sacrificed. She will be inveigled and married
to that connexion of yours. It is a partnership affair, a
money-speculation. She has no strength of will or character to help
herself and she is on the brink of being sold into wretchedness for
life.’

‘Amazing! But what can I do to prevent it?’ demands Twemlow, shocked and
bewildered to the last degree.

‘Here is another portrait. And not good, is it?’

Aghast at the light manner of her throwing her head back to look at it
critically, Twemlow still dimly perceives the expediency of throwing his
own head back, and does so. Though he no more sees the portrait than if
it were in China.

‘Decidedly not good,’ says Mrs Lammle. ‘Stiff and exaggerated!’

‘And ex--’ But Twemlow, in his demolished state, cannot command the
word, and trails off into ‘--actly so.’

‘Mr Twemlow, your word will have weight with her pompous, self-blinded
father. You know how much he makes of your family. Lose no time. Warn
him.’

‘But warn him against whom?’

‘Against me.’

By great good fortune Twemlow receives a stimulant at this critical
instant. The stimulant is Lammle’s voice.

‘Sophronia, my dear, what portraits are you showing Twemlow?’

‘Public characters, Alfred.’

‘Show him the last of me.’

‘Yes, Alfred.’

She puts the book down, takes another book up, turns the leaves, and
presents the portrait to Twemlow.

‘That is the last of Mr Lammle. Do you think it good?--Warn her father
against me. I deserve it, for I have been in the scheme from the first.
It is my husband’s scheme, your connexion’s, and mine. I tell you this,
only to show you the necessity of the poor little foolish affectionate
creature’s being befriended and rescued. You will not repeat this to her
father. You will spare me so far, and spare my husband. For, though this
celebration of to-day is all a mockery, he is my husband, and we must
live.--Do you think it like?’

Twemlow, in a stunned condition, feigns to compare the portrait in his
hand with the original looking towards him from his Mephistophelean
corner.

‘Very well indeed!’ are at length the words which Twemlow with great
difficulty extracts from himself.

‘I am glad you think so. On the whole, I myself consider it the best.
The others are so dark. Now here, for instance, is another of Mr
Lammle--’

‘But I don’t understand; I don’t see my way,’ Twemlow stammers, as he
falters over the book with his glass at his eye. ‘How warn her father,
and not tell him? Tell him how much? Tell him how little? I--I--am
getting lost.’

‘Tell him I am a match-maker; tell him I am an artful and designing
woman; tell him you are sure his daughter is best out of my house and my
company. Tell him any such things of me; they will all be true. You know
what a puffed-up man he is, and how easily you can cause his vanity to
take the alarm. Tell him as much as will give him the alarm and make
him careful of her, and spare me the rest. Mr Twemlow, I feel my sudden
degradation in your eyes; familiar as I am with my degradation in my own
eyes, I keenly feel the change that must have come upon me in yours,
in these last few moments. But I trust to your good faith with me as
implicitly as when I began. If you knew how often I have tried to speak
to you to-day, you would almost pity me. I want no new promise from you
on my own account, for I am satisfied, and I always shall be satisfied,
with the promise you have given me. I can venture to say no more, for
I see that I am watched. If you would set my mind at rest with the
assurance that you will interpose with the father and save this harmless
girl, close that book before you return it to me, and I shall know what
you mean, and deeply thank you in my heart.--Alfred, Mr Twemlow thinks
the last one the best, and quite agrees with you and me.’

Alfred advances. The groups break up. Lady Tippins rises to go, and Mrs
Veneering follows her leader. For the moment, Mrs Lammle does not turn
to them, but remains looking at Twemlow looking at Alfred’s portrait
through his eyeglass. The moment past, Twemlow drops his eyeglass at its
ribbon’s length, rises, and closes the book with an emphasis which makes
that fragile nursling of the fairies, Tippins, start.

Then good-bye and good-bye, and charming occasion worthy of the Golden
Age, and more about the flitch of bacon, and the like of that; and
Twemlow goes staggering across Piccadilly with his hand to his forehead,
and is nearly run down by a flushed lettercart, and at last drops
safe in his easy-chair, innocent good gentleman, with his hand to his
forehead still, and his head in a whirl.





