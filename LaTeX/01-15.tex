% REV00 Tue 04 May 2021 13:55:16 WIB
% START Tue 04 May 2021 13:55:16 WIB

\chapter{TWO NEW SERVANTS}

Mr and Mrs Boffin sat after breakfast, in the Bower, a prey to
prosperity. Mr Boffin’s face denoted Care and Complication. Many
disordered papers were before him, and he looked at them about as
hopefully as an innocent civilian might look at a crowd of troops whom
he was required at five minutes’ notice to manoeuvre and review. He had
been engaged in some attempts to make notes of these papers; but being
troubled (as men of his stamp often are) with an exceedingly distrustful
and corrective thumb, that busy member had so often interposed to
smear his notes, that they were little more legible than the various
impressions of itself; which blurred his nose and forehead. It is
curious to consider, in such a case as Mr Boffin’s, what a cheap article
ink is, and how far it may be made to go. As a grain of musk will scent
a drawer for many years, and still lose nothing appreciable of its
original weight, so a halfpenny-worth of ink would blot Mr Boffin to the
roots of his hair and the calves of his legs, without inscribing a line
on the paper before him, or appearing to diminish in the inkstand.

Mr Boffin was in such severe literary difficulties that his eyes were
prominent and fixed, and his breathing was stertorous, when, to the
great relief of Mrs Boffin, who observed these symptoms with alarm, the
yard bell rang.

‘Who’s that, I wonder!’ said Mrs Boffin.

Mr Boffin drew a long breath, laid down his pen, looked at his notes
as doubting whether he had the pleasure of their acquaintance, and
appeared, on a second perusal of their countenances, to be confirmed
in his impression that he had not, when there was announced by the
hammer-headed young man:

‘Mr Rokesmith.’

‘Oh!’ said Mr Boffin. ‘Oh indeed! Our and the Wilfers’ Mutual Friend, my
dear. Yes. Ask him to come in.’

Mr Rokesmith appeared.

‘Sit down, sir,’ said Mr Boffin, shaking hands with him. ‘Mrs Boffin
you’re already acquainted with. Well, sir, I am rather unprepared to see
you, for, to tell you the truth, I’ve been so busy with one thing and
another, that I’ve not had time to turn your offer over.’

‘That’s apology for both of us: for Mr Boffin, and for me as well,’ said
the smiling Mrs Boffin. ‘But Lor! we can talk it over now; can’t us?’

Mr Rokesmith bowed, thanked her, and said he hoped so.

‘Let me see then,’ resumed Mr Boffin, with his hand to his chin. ‘It was
Secretary that you named; wasn’t it?’

‘I said Secretary,’ assented Mr Rokesmith.

‘It rather puzzled me at the time,’ said Mr Boffin, ‘and it rather
puzzled me and Mrs Boffin when we spoke of it afterwards, because (not
to make a mystery of our belief) we have always believed a Secretary to
be a piece of furniture, mostly of mahogany, lined with green baize or
leather, with a lot of little drawers in it. Now, you won’t think I take
a liberty when I mention that you certainly ain’t THAT.’

Certainly not, said Mr Rokesmith. But he had used the word in the sense
of Steward.

‘Why, as to Steward, you see,’ returned Mr Boffin, with his hand still
to his chin, ‘the odds are that Mrs Boffin and me may never go upon the
water. Being both bad sailors, we should want a Steward if we did; but
there’s generally one provided.’

Mr Rokesmith again explained; defining the duties he sought to
undertake, as those of general superintendent, or manager, or
overlooker, or man of business.

‘Now, for instance--come!’ said Mr Boffin, in his pouncing way. ‘If you
entered my employment, what would you do?’

‘I would keep exact accounts of all the expenditure you sanctioned,
Mr Boffin. I would write your letters, under your direction. I would
transact your business with people in your pay or employment. I would,’
with a glance and a half-smile at the table, ‘arrange your papers--’

Mr Boffin rubbed his inky ear, and looked at his wife.

‘--And so arrange them as to have them always in order for immediate
reference, with a note of the contents of each outside it.’

‘I tell you what,’ said Mr Boffin, slowly crumpling his own blotted note
in his hand; ‘if you’ll turn to at these present papers, and see what
you can make of ‘em, I shall know better what I can make of you.’

No sooner said than done. Relinquishing his hat and gloves, Mr Rokesmith
sat down quietly at the table, arranged the open papers into an orderly
heap, cast his eyes over each in succession, folded it, docketed it on
the outside, laid it in a second heap, and, when that second heap was
complete and the first gone, took from his pocket a piece of string and
tied it together with a remarkably dexterous hand at a running curve and
a loop.

‘Good!’ said Mr Boffin. ‘Very good! Now let us hear what they’re all
about; will you be so good?’

John Rokesmith read his abstracts aloud. They were all about the new
house. Decorator’s estimate, so much. Furniture estimate, so much.
Estimate for furniture of offices, so much. Coach-maker’s estimate, so
much. Horse-dealer’s estimate, so much. Harness-maker’s estimate, so
much. Goldsmith’s estimate, so much. Total, so very much. Then came
correspondence. Acceptance of Mr Boffin’s offer of such a date, and to
such an effect. Rejection of Mr Boffin’s proposal of such a date and to
such an effect. Concerning Mr Boffin’s scheme of such another date to
such another effect. All compact and methodical.

‘Apple-pie order!’ said Mr Boffin, after checking off each inscription
with his hand, like a man beating time. ‘And whatever you do with your
ink, I can’t think, for you’re as clean as a whistle after it. Now, as
to a letter. Let’s,’ said Mr Boffin, rubbing his hands in his pleasantly
childish admiration, ‘let’s try a letter next.’

‘To whom shall it be addressed, Mr Boffin?’

‘Anyone. Yourself.’

Mr Rokesmith quickly wrote, and then read aloud:

‘“Mr Boffin presents his compliments to Mr John Rokesmith, and begs
to say that he has decided on giving Mr John Rokesmith a trial in the
capacity he desires to fill. Mr Boffin takes Mr John Rokesmith at his
word, in postponing to some indefinite period, the consideration of
salary. It is quite understood that Mr Boffin is in no way committed
on that point. Mr Boffin has merely to add, that he relies on Mr John
Rokesmith’s assurance that he will be faithful and serviceable. Mr John
Rokesmith will please enter on his duties immediately.”’

‘Well! Now, Noddy!’ cried Mrs Boffin, clapping her hands, ‘That IS a
good one!’

Mr Boffin was no less delighted; indeed, in his own bosom, he regarded
both the composition itself and the device that had given birth to it,
as a very remarkable monument of human ingenuity.

‘And I tell you, my deary,’ said Mrs Boffin, ‘that if you don’t close
with Mr Rokesmith now at once, and if you ever go a muddling yourself
again with things never meant nor made for you, you’ll have an
apoplexy--besides iron-moulding your linen--and you’ll break my heart.’

Mr Boffin embraced his spouse for these words of wisdom, and then,
congratulating John Rokesmith on the brilliancy of his achievements,
gave him his hand in pledge of their new relations. So did Mrs Boffin.

‘Now,’ said Mr Boffin, who, in his frankness, felt that it did not
become him to have a gentleman in his employment five minutes, without
reposing some confidence in him, ‘you must be let a little more into our
affairs, Rokesmith. I mentioned to you, when I made your acquaintance,
or I might better say when you made mine, that Mrs Boffin’s inclinations
was setting in the way of Fashion, but that I didn’t know how
fashionable we might or might not grow. Well! Mrs Boffin has carried the
day, and we’re going in neck and crop for Fashion.’

‘I rather inferred that, sir,’ replied John Rokesmith, ‘from the scale
on which your new establishment is to be maintained.’

‘Yes,’ said Mr Boffin, ‘it’s to be a Spanker. The fact is, my
literary man named to me that a house with which he is, as I may say,
connected--in which he has an interest--’

‘As property?’ inquired John Rokesmith.

‘Why no,’ said Mr Boffin, ‘not exactly that; a sort of a family tie.’

‘Association?’ the Secretary suggested.

‘Ah!’ said Mr Boffin. ‘Perhaps. Anyhow, he named to me that the house
had a board up, “This Eminently Aristocratic Mansion to be let or sold.”
 Me and Mrs Boffin went to look at it, and finding it beyond a doubt
Eminently Aristocratic (though a trifle high and dull, which after all
may be part of the same thing) took it. My literary man was so friendly
as to drop into a charming piece of poetry on that occasion, in which he
complimented Mrs Boffin on coming into possession of--how did it go, my
dear?’

Mrs Boffin replied:

\begin{verbatim}
     ‘"The gay, the gay and festive scene,
     The halls, the halls of dazzling light."'
\end{verbatim}

‘That’s it! And it was made neater by there really being two halls
in the house, a front ‘un and a back ‘un, besides the servants’.
He likewise dropped into a very pretty piece of poetry to be sure,
respecting the extent to which he would be willing to put himself out
of the way to bring Mrs Boffin round, in case she should ever get low
in her spirits in the house. Mrs Boffin has a wonderful memory. Will you
repeat it, my dear?’

Mrs Boffin complied, by reciting the verses in which this obliging offer
had been made, exactly as she had received them.

\begin{verbatim}
     ‘"I’ll tell thee how the maiden wept, Mrs Boffin,
     When her true love was slain ma’am,
     And how her broken spirit slept, Mrs Boffin,
     And never woke again ma’am.
     I’ll tell thee (if agreeable to Mr Boffin) how the steed drew
     nigh,
     And left his lord afar;
     And if my tale (which I hope Mr Boffin might excuse) should
     make you sigh,
     I’ll strike the light guitar.”’
\end{verbatim}

‘Correct to the letter!’ said Mr Boffin. ‘And I consider that the poetry
brings us both in, in a beautiful manner.’

The effect of the poem on the Secretary being evidently to astonish
him, Mr Boffin was confirmed in his high opinion of it, and was greatly
pleased.

‘Now, you see, Rokesmith,’ he went on, ‘a literary man--WITH a wooden
leg--is liable to jealousy. I shall therefore cast about for comfortable
ways and means of not calling up Wegg’s jealousy, but of keeping you in
your department, and keeping him in his.’

‘Lor!’ cried Mrs Boffin. ‘What I say is, the world’s wide enough for all
of us!’

‘So it is, my dear,’ said Mr Boffin, ‘when not literary. But when so,
not so. And I am bound to bear in mind that I took Wegg on, at a time
when I had no thought of being fashionable or of leaving the Bower. To
let him feel himself anyways slighted now, would be to be guilty of
a meanness, and to act like having one’s head turned by the halls of
dazzling light. Which Lord forbid! Rokesmith, what shall we say about
your living in the house?’

‘In this house?’

‘No, no. I have got other plans for this house. In the new house?’

‘That will be as you please, Mr Boffin. I hold myself quite at your
disposal. You know where I live at present.’

‘Well!’ said Mr Boffin, after considering the point; ‘suppose you keep
as you are for the present, and we’ll decide by-and-by. You’ll begin to
take charge at once, of all that’s going on in the new house, will you?’

‘Most willingly. I will begin this very day. Will you give me the
address?’

Mr Boffin repeated it, and the Secretary wrote it down in his
pocket-book. Mrs Boffin took the opportunity of his being so engaged,
to get a better observation of his face than she had yet taken. It
impressed her in his favour, for she nodded aside to Mr Boffin, ‘I like
him.’

‘I will see directly that everything is in train, Mr Boffin.’

‘Thank’ee. Being here, would you care at all to look round the Bower?’

‘I should greatly like it. I have heard so much of its story.’

‘Come!’ said Mr Boffin. And he and Mrs Boffin led the way.

A gloomy house the Bower, with sordid signs on it of having been,
through its long existence as Harmony Jail, in miserly holding. Bare of
paint, bare of paper on the walls, bare of furniture, bare of experience
of human life. Whatever is built by man for man’s occupation, must,
like natural creations, fulfil the intention of its existence, or soon
perish. This old house had wasted--more from desuetude than it would
have wasted from use, twenty years for one.

A certain leanness falls upon houses not sufficiently imbued with life
(as if they were nourished upon it), which was very noticeable here.
The staircase, balustrades, and rails, had a spare look--an air of being
denuded to the bone--which the panels of the walls and the jambs of the
doors and windows also bore. The scanty moveables partook of it; save
for the cleanliness of the place, the dust--into which they were all
resolving would have lain thick on the floors; and those, both in colour
and in grain, were worn like old faces that had kept much alone.

The bedroom where the clutching old man had lost his grip on life, was
left as he had left it. There was the old grisly four-post bedstead,
without hangings, and with a jail-like upper rim of iron and spikes; and
there was the old patch-work counterpane. There was the tight-clenched
old bureau, receding atop like a bad and secret forehead; there was the
cumbersome old table with twisted legs, at the bed-side; and there
was the box upon it, in which the will had lain. A few old chairs with
patch-work covers, under which the more precious stuff to be preserved
had slowly lost its quality of colour without imparting pleasure to any
eye, stood against the wall. A hard family likeness was on all these
things.

‘The room was kept like this, Rokesmith,’ said Mr Boffin, ‘against the
son’s return. In short, everything in the house was kept exactly as it
came to us, for him to see and approve. Even now, nothing is changed
but our own room below-stairs that you have just left. When the son came
home for the last time in his life, and for the last time in his life
saw his father, it was most likely in this room that they met.’

As the Secretary looked all round it, his eyes rested on a side door in
a corner.

‘Another staircase,’ said Mr Boffin, unlocking the door, ‘leading down
into the yard. We’ll go down this way, as you may like to see the yard,
and it’s all in the road. When the son was a little child, it was up
and down these stairs that he mostly came and went to his father. He was
very timid of his father. I’ve seen him sit on these stairs, in his
shy way, poor child, many a time. Mr and Mrs Boffin have comforted him,
sitting with his little book on these stairs, often.’

‘Ah! And his poor sister too,’ said Mrs Boffin. ‘And here’s the sunny
place on the white wall where they one day measured one another. Their
own little hands wrote up their names here, only with a pencil; but the
names are here still, and the poor dears gone for ever.’

‘We must take care of the names, old lady,’ said Mr Boffin. ‘We must
take care of the names. They shan’t be rubbed out in our time, nor yet,
if we can help it, in the time after us. Poor little children!’

‘Ah, poor little children!’ said Mrs Boffin.

They had opened the door at the bottom of the staircase giving on the
yard, and they stood in the sunlight, looking at the scrawl of the two
unsteady childish hands two or three steps up the staircase. There was
something in this simple memento of a blighted childhood, and in the
tenderness of Mrs Boffin, that touched the Secretary.

Mr Boffin then showed his new man of business the Mounds, and his own
particular Mound which had been left him as his legacy under the will
before he acquired the whole estate.

‘It would have been enough for us,’ said Mr Boffin, ‘in case it had
pleased God to spare the last of those two young lives and sorrowful
deaths. We didn’t want the rest.’

At the treasures of the yard, and at the outside of the house, and at
the detached building which Mr Boffin pointed out as the residence
of himself and his wife during the many years of their service, the
Secretary looked with interest. It was not until Mr Boffin had shown
him every wonder of the Bower twice over, that he remembered his having
duties to discharge elsewhere.

‘You have no instructions to give me, Mr Boffin, in reference to this
place?’

‘Not any, Rokesmith. No.’

‘Might I ask, without seeming impertinent, whether you have any
intention of selling it?’

‘Certainly not. In remembrance of our old master, our old master’s
children, and our old service, me and Mrs Boffin mean to keep it up as
it stands.’

The Secretary’s eyes glanced with so much meaning in them at the Mounds,
that Mr Boffin said, as if in answer to a remark:

‘Ay, ay, that’s another thing. I may sell THEM, though I should be sorry
to see the neighbourhood deprived of ‘em too. It’ll look but a poor dead
flat without the Mounds. Still I don’t say that I’m going to keep ‘em
always there, for the sake of the beauty of the landscape. There’s no
hurry about it; that’s all I say at present. I ain’t a scholar in much,
Rokesmith, but I’m a pretty fair scholar in dust. I can price the Mounds
to a fraction, and I know how they can be best disposed of; and likewise
that they take no harm by standing where they do. You’ll look in
to-morrow, will you be so kind?’

‘Every day. And the sooner I can get you into your new house, complete,
the better you will be pleased, sir?’

‘Well, it ain’t that I’m in a mortal hurry,’ said Mr Boffin; ‘only when
you DO pay people for looking alive, it’s as well to know that they ARE
looking alive. Ain’t that your opinion?’

‘Quite!’ replied the Secretary; and so withdrew.

‘Now,’ said Mr Boffin to himself; subsiding into his regular series of
turns in the yard, ‘if I can make it comfortable with Wegg, my affairs
will be going smooth.’

The man of low cunning had, of course, acquired a mastery over the man
of high simplicity. The mean man had, of course, got the better of the
generous man. How long such conquests last, is another matter; that they
are achieved, is every-day experience, not even to be flourished away by
Podsnappery itself. The undesigning Boffin had become so far immeshed
by the wily Wegg that his mind misgave him he was a very designing man
indeed in purposing to do more for Wegg. It seemed to him (so skilful
was Wegg) that he was plotting darkly, when he was contriving to do the
very thing that Wegg was plotting to get him to do. And thus, while he
was mentally turning the kindest of kind faces on Wegg this morning, he
was not absolutely sure but that he might somehow deserve the charge of
turning his back on him.

For these reasons Mr Boffin passed but anxious hours until evening came,
and with it Mr Wegg, stumping leisurely to the Roman Empire. At about
this period Mr Boffin had become profoundly interested in the fortunes
of a great military leader known to him as Bully Sawyers, but perhaps
better known to fame and easier of identification by the classical
student, under the less Britannic name of Belisarius. Even this
general’s career paled in interest for Mr Boffin before the clearing of
his conscience with Wegg; and hence, when that literary gentleman had
according to custom eaten and drunk until he was all a-glow, and when
he took up his book with the usual chirping introduction, ‘And now, Mr
Boffin, sir, we’ll decline and we’ll fall!’ Mr Boffin stopped him.

‘You remember, Wegg, when I first told you that I wanted to make a sort
of offer to you?’

‘Let me get on my considering cap, sir,’ replied that gentleman, turning
the open book face downward. ‘When you first told me that you wanted
to make a sort of offer to me? Now let me think.’ (as if there were the
least necessity) ‘Yes, to be sure I do, Mr Boffin. It was at my corner.
To be sure it was! You had first asked me whether I liked your name,
and Candour had compelled a reply in the negative case. I little thought
then, sir, how familiar that name would come to be!’

‘I hope it will be more familiar still, Wegg.’

‘Do you, Mr Boffin? Much obliged to you, I’m sure. Is it your pleasure,
sir, that we decline and we fall?’ with a feint of taking up the book.

‘Not just yet awhile, Wegg. In fact, I have got another offer to make
you.’

Mr Wegg (who had had nothing else in his mind for several nights) took
off his spectacles with an air of bland surprise.

‘And I hope you’ll like it, Wegg.’

‘Thank you, sir,’ returned that reticent individual. ‘I hope it may
prove so. On all accounts, I am sure.’ (This, as a philanthropic
aspiration.)

‘What do you think,’ said Mr Boffin, ‘of not keeping a stall, Wegg?’

‘I think, sir,’ replied Wegg, ‘that I should like to be shown the
gentleman prepared to make it worth my while!’

‘Here he is,’ said Mr Boffin.

Mr Wegg was going to say, My Benefactor, and had said My Bene, when a
grandiloquent change came over him.

‘No, Mr Boffin, not you sir. Anybody but you. Do not fear, Mr Boffin,
that I shall contaminate the premises which your gold has bought, with
MY lowly pursuits. I am aware, sir, that it would not become me to carry
on my little traffic under the windows of your mansion. I have already
thought of that, and taken my measures. No need to be bought out, sir.
Would Stepney Fields be considered intrusive? If not remote enough, I
can go remoter. In the words of the poet’s song, which I do not quite
remember:

\begin{verbatim}
     Thrown on the wide world, doom’d to wander and roam,
     Bereft of my parents, bereft of a home,
     A stranger to something and what’s his name joy,
     Behold little Edmund the poor Peasant boy.
\end{verbatim}

--And equally,’ said Mr Wegg, repairing the want of direct application
in the last line, ‘behold myself on a similar footing!’

‘Now, Wegg, Wegg, Wegg,’ remonstrated the excellent Boffin. ‘You are too
sensitive.’

‘I know I am, sir,’ returned Wegg, with obstinate magnanimity. ‘I am
acquainted with my faults. I always was, from a child, too sensitive.’

‘But listen,’ pursued the Golden Dustman; ‘hear me out, Wegg. You have
taken it into your head that I mean to pension you off.’

‘True, sir,’ returned Wegg, still with an obstinate magnanimity. ‘I am
acquainted with my faults. Far be it from me to deny them. I HAVE taken
it into my head.’

‘But I DON’T mean it.’

The assurance seemed hardly as comforting to Mr Wegg, as Mr Boffin
intended it to be. Indeed, an appreciable elongation of his visage might
have been observed as he replied:

‘Don’t you, indeed, sir?’

‘No,’ pursued Mr Boffin; ‘because that would express, as I understand
it, that you were not going to do anything to deserve your money. But
you are; you are.’

‘That, sir,’ replied Mr Wegg, cheering up bravely, ‘is quite another
pair of shoes. Now, my independence as a man is again elevated. Now, I
no longer

\begin{verbatim}
     Weep for the hour,
     When to Boffinses bower,
     The Lord of the valley with offers came;
     Neither does the moon hide her light
     From the heavens to-night,
     And weep behind her clouds o’er any individual in the present
     Company’s shame.
\end{verbatim}

--Please to proceed, Mr Boffin.’

‘Thank’ee, Wegg, both for your confidence in me and for your frequent
dropping into poetry; both of which is friendly. Well, then; my idea is,
that you should give up your stall, and that I should put you into the
Bower here, to keep it for us. It’s a pleasant spot; and a man with
coals and candles and a pound a week might be in clover here.’

‘Hem! Would that man, sir--we will say that man, for the purposes of
argueyment;’ Mr Wegg made a smiling demonstration of great perspicuity
here; ‘would that man, sir, be expected to throw any other capacity in,
or would any other capacity be considered extra? Now let us (for the
purposes of argueyment) suppose that man to be engaged as a reader: say
(for the purposes of argueyment) in the evening. Would that man’s pay as
a reader in the evening, be added to the other amount, which, adopting
your language, we will call clover; or would it merge into that amount,
or clover?’

‘Well,’ said Mr Boffin, ‘I suppose it would be added.’

‘I suppose it would, sir. You are right, sir. Exactly my own views,
Mr Boffin.’ Here Wegg rose, and balancing himself on his wooden leg,
fluttered over his prey with extended hand. ‘Mr Boffin, consider it
done. Say no more, sir, not a word more. My stall and I are for ever
parted. The collection of ballads will in future be reserved for private
study, with the object of making poetry tributary’--Wegg was so proud
of having found this word, that he said it again, with a capital
letter--‘Tributary, to friendship. Mr Boffin, don’t allow yourself to
be made uncomfortable by the pang it gives me to part from my stock and
stall. Similar emotion was undergone by my own father when promoted
for his merits from his occupation as a waterman to a situation under
Government. His Christian name was Thomas. His words at the time (I was
then an infant, but so deep was their impression on me, that I committed
them to memory) were:

\begin{verbatim}
     Then farewell my trim-built wherry,
     Oars and coat and badge farewell!
     Never more at Chelsea Ferry,
     Shall your Thomas take a spell!
\end{verbatim}

--My father got over it, Mr Boffin, and so shall I.’

While delivering these valedictory observations, Wegg continually
disappointed Mr Boffin of his hand by flourishing it in the air. He now
darted it at his patron, who took it, and felt his mind relieved of a
great weight: observing that as they had arranged their joint affairs
so satisfactorily, he would now be glad to look into those of Bully
Sawyers. Which, indeed, had been left over-night in a very unpromising
posture, and for whose impending expedition against the Persians the
weather had been by no means favourable all day.

Mr Wegg resumed his spectacles therefore. But Sawyers was not to be of
the party that night; for, before Wegg had found his place, Mrs Boffin’s
tread was heard upon the stairs, so unusually heavy and hurried, that Mr
Boffin would have started up at the sound, anticipating some occurrence
much out of the common course, even though she had not also called to
him in an agitated tone.

Mr Boffin hurried out, and found her on the dark staircase, panting,
with a lighted candle in her hand.

‘What’s the matter, my dear?’

‘I don’t know; I don’t know; but I wish you’d come up-stairs.’

Much surprised, Mr Boffin went up stairs and accompanied Mrs Boffin into
their own room: a second large room on the same floor as the room in
which the late proprietor had died. Mr Boffin looked all round him,
and saw nothing more unusual than various articles of folded linen on a
large chest, which Mrs Boffin had been sorting.

‘What is it, my dear? Why, you’re frightened! YOU frightened?’

‘I am not one of that sort certainly,’ said Mrs Boffin, as she sat down
in a chair to recover herself, and took her husband’s arm; ‘but it’s
very strange!’

‘What is, my dear?’

‘Noddy, the faces of the old man and the two children are all over the
house to-night.’

‘My dear?’ exclaimed Mr Boffin. But not without a certain uncomfortable
sensation gliding down his back.

‘I know it must sound foolish, and yet it is so.’

‘Where did you think you saw them?’

‘I don’t know that I think I saw them anywhere. I felt them.’

‘Touched them?’

‘No. Felt them in the air. I was sorting those things on the chest, and
not thinking of the old man or the children, but singing to myself, when
all in a moment I felt there was a face growing out of the dark.’

‘What face?’ asked her husband, looking about him.

‘For a moment it was the old man’s, and then it got younger. For a
moment it was both the children’s, and then it got older. For a moment
it was a strange face, and then it was all the faces.’

‘And then it was gone?’

‘Yes; and then it was gone.’

‘Where were you then, old lady?’

‘Here, at the chest. Well; I got the better of it, and went on sorting,
and went on singing to myself. “Lor!” I says, “I’ll think of something
else--something comfortable--and put it out of my head.” So I thought
of the new house and Miss Bella Wilfer, and was thinking at a great rate
with that sheet there in my hand, when all of a sudden, the faces seemed
to be hidden in among the folds of it and I let it drop.’

As it still lay on the floor where it had fallen, Mr Boffin picked it up
and laid it on the chest.

‘And then you ran down stairs?’

‘No. I thought I’d try another room, and shake it off. I says to myself,
“I’ll go and walk slowly up and down the old man’s room three times,
from end to end, and then I shall have conquered it.” I went in with the
candle in my hand; but the moment I came near the bed, the air got thick
with them.’

‘With the faces?’

‘Yes, and I even felt that they were in the dark behind the side-door,
and on the little staircase, floating away into the yard. Then, I called
you.’

Mr Boffin, lost in amazement, looked at Mrs Boffin. Mrs Boffin, lost in
her own fluttered inability to make this out, looked at Mr Boffin.

‘I think, my dear,’ said the Golden Dustman, ‘I’ll at once get rid of
Wegg for the night, because he’s coming to inhabit the Bower, and it
might be put into his head or somebody else’s, if he heard this and it
got about that the house is haunted. Whereas we know better. Don’t we?’

‘I never had the feeling in the house before,’ said Mrs Boffin; ‘and I
have been about it alone at all hours of the night. I have been in the
house when Death was in it, and I have been in the house when Murder was
a new part of its adventures, and I never had a fright in it yet.’

‘And won’t again, my dear,’ said Mr Boffin. ‘Depend upon it, it comes of
thinking and dwelling on that dark spot.’

‘Yes; but why didn’t it come before?’ asked Mrs Boffin.

This draft on Mr Boffin’s philosophy could only be met by that gentleman
with the remark that everything that is at all, must begin at some time.
Then, tucking his wife’s arm under his own, that she might not be left
by herself to be troubled again, he descended to release Wegg. Who,
being something drowsy after his plentiful repast, and constitutionally
of a shirking temperament, was well enough pleased to stump away,
without doing what he had come to do, and was paid for doing.

Mr Boffin then put on his hat, and Mrs Boffin her shawl; and the pair,
further provided with a bunch of keys and a lighted lantern, went
all over the dismal house--dismal everywhere, but in their own two
rooms--from cellar to cock-loft. Not resting satisfied with giving that
much chace to Mrs Boffin’s fancies, they pursued them into the yard and
outbuildings, and under the Mounds. And setting the lantern, when all
was done, at the foot of one of the Mounds, they comfortably trotted to
and fro for an evening walk, to the end that the murky cobwebs in Mrs
Boffin’s brain might be blown away.

‘There, my dear!’ said Mr Boffin when they came in to supper. ‘That was
the treatment, you see. Completely worked round, haven’t you?’

‘Yes, deary,’ said Mrs Boffin, laying aside her shawl. ‘I’m not nervous
any more. I’m not a bit troubled now. I’d go anywhere about the house
the same as ever. But--’

‘Eh!’ said Mr Boffin.

‘But I’ve only to shut my eyes.’

‘And what then?’

‘Why then,’ said Mrs Boffin, speaking with her eyes closed, and her
left hand thoughtfully touching her brow, ‘then, there they are! The old
man’s face, and it gets younger. The two children’s faces, and they get
older. A face that I don’t know. And then all the faces!’

Opening her eyes again, and seeing her husband’s face across the table,
she leaned forward to give it a pat on the cheek, and sat down to
supper, declaring it to be the best face in the world.


