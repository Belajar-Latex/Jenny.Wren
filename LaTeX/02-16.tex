% REV01 Thu 24 Jun 2021 05:29:40 WIB
% START Tue 04 May 2021 13:55:16 WIB

\chapter{AN ANNIVERSARY OCCASION}

The estimable Twemlow, dressing himself in his lodgings over the
stable-yard in Duke Street, Saint James’s, and hearing the horses at
their toilette below, finds himself on the whole in a disadvantageous
position as compared with the noble animals at livery. For whereas, on
the one hand, he has no attendant to slap him soundingly and require him
in gruff accents to come up and come over, still, on the other hand,
he has no attendant at all; and the mild gentleman’s finger-joints and
other joints working rustily in the morning, he could deem it agreeable
even to be tied up by the countenance at his chamber-door, so he were
there skilfully rubbed down and slushed and sluiced and polished and
clothed, while himself taking merely a passive part in these trying
transactions.

How the fascinating Tippins gets on when arraying herself for the
bewilderment of the senses of men, is known only to the Graces and her
maid; but perhaps even that engaging creature, though not reduced to
the self-dependence of Twemlow could dispense with a good deal of the
trouble attendant on the daily restoration of her charms, seeing that
as to her face and neck this adorable divinity is, as it were, a diurnal
species of lobster--throwing off a shell every forenoon, and needing to
keep in a retired spot until the new crust hardens.

Howbeit, Twemlow doth at length invest himself with collar and cravat
and wristbands to his knuckles, and goeth forth to breakfast. And to
breakfast with whom but his near neighbours, the Lammles of Sackville
Street, who have imparted to him that he will meet his distant kinsman,
Mr Fledgely. The awful Snigsworth might taboo and prohibit Fledgely, but
the peaceable Twemlow reasons, If he IS my kinsman I didn’t make him so,
and to meet a man is not to know him.’

It is the first anniversary of the happy marriage of Mr and Mrs Lammle,
and the celebration is a breakfast, because a dinner on the desired
scale of sumptuosity cannot be achieved within less limits than those
of the non-existent palatial residence of which so many people are
madly envious. So, Twemlow trips with not a little stiffness across
Piccadilly, sensible of having once been more upright in figure and less
in danger of being knocked down by swift vehicles. To be sure that was
in the days when he hoped for leave from the dread Snigsworth to do
something, or be something, in life, and before that magnificent Tartar
issued the ukase, ‘As he will never distinguish himself, he must be a
poor gentleman-pensioner of mine, and let him hereby consider himself
pensioned.’

Ah! my Twemlow! Say, little feeble grey personage, what thoughts are in
thy breast to-day, of the Fancy--so still to call her who bruised thy
heart when it was green and thy head brown--and whether it be better or
worse, more painful or less, to believe in the Fancy to this hour, than
to know her for a greedy armour-plated crocodile, with no more capacity
of imagining the delicate and sensitive and tender spot behind thy
waistcoat, than of going straight at it with a knitting-needle. Say
likewise, my Twemlow, whether it be the happier lot to be a poor
relation of the great, or to stand in the wintry slush giving the hack
horses to drink out of the shallow tub at the coach-stand, into which
thou has so nearly set thy uncertain foot. Twemlow says nothing, and
goes on.

As he approaches the Lammles’ door, drives up a little one-horse
carriage, containing Tippins the divine. Tippins, letting down the
window, playfully extols the vigilance of her cavalier in being in
waiting there to hand her out. Twemlow hands her out with as much polite
gravity as if she were anything real, and they proceed upstairs. Tippins
all abroad about the legs, and seeking to express that those unsteady
articles are only skipping in their native buoyancy.

And dear Mrs Lammle and dear Mr Lammle, how do you do, and when are
you going down to what’s-its-name place--Guy, Earl of Warwick, you
know--what is it?--Dun Cow--to claim the flitch of bacon? And Mortimer,
whose name is for ever blotted out from my list of lovers, by reason
first of fickleness and then of base desertion, how do YOU do, wretch?
And Mr Wrayburn, YOU here! What can YOU come for, because we are all
very sure before-hand that you are not going to talk! And Veneering,
M.P., how are things going on down at the house, and when will you turn
out those terrible people for us? And Mrs Veneering, my dear, can it
positively be true that you go down to that stifling place night after
night, to hear those men prose? Talking of which, Veneering, why don’t
you prose, for you haven’t opened your lips there yet, and we are dying
to hear what you have got to say to us! Miss Podsnap, charmed to see
you. Pa, here? No! Ma, neither? Oh! Mr Boots! Delighted. Mr Brewer!
This IS a gathering of the clans. Thus Tippins, and surveys Fledgeby and
outsiders through golden glass, murmuring as she turns about and about,
in her innocent giddy way, Anybody else I know? No, I think not. Nobody
there. Nobody THERE. Nobody anywhere!

Mr Lammle, all a-glitter, produces his friend Fledgeby, as dying for the
honour of presentation to Lady Tippins. Fledgeby presented, has the air
of going to say something, has the air of going to say nothing, has an
air successively of meditation, of resignation, and of desolation,
backs on Brewer, makes the tour of Boots, and fades into the extreme
background, feeling for his whisker, as if it might have turned up since
he was there five minutes ago.

But Lammle has him out again before he has so much as completely
ascertained the bareness of the land. He would seem to be in a bad way,
Fledgeby; for Lammle represents him as dying again. He is dying now, of
want of presentation to Twemlow.

Twemlow offers his hand. Glad to see him. ‘Your mother, sir, was a
connexion of mine.’

‘I believe so,’ says Fledgeby, ‘but my mother and her family were two.’

‘Are you staying in town?’ asks Twemlow.

‘I always am,’ says Fledgeby.

‘You like town,’ says Twemlow. But is felled flat by Fledgeby’s taking
it quite ill, and replying, No, he don’t like town. Lammle tries to
break the force of the fall, by remarking that some people do not like
town. Fledgeby retorting that he never heard of any such case but his
own, Twemlow goes down again heavily.

‘There is nothing new this morning, I suppose?’ says Twemlow, returning
to the mark with great spirit.

Fledgeby has not heard of anything.

‘No, there’s not a word of news,’ says Lammle.

‘Not a particle,’ adds Boots.

‘Not an atom,’ chimes in Brewer.

Somehow the execution of this little concerted piece appears to raise
the general spirits as with a sense of duty done, and sets the company a
going. Everybody seems more equal than before, to the calamity of being
in the society of everybody else. Even Eugene standing in a window,
moodily swinging the tassel of a blind, gives it a smarter jerk now, as
if he found himself in better case.

Breakfast announced. Everything on table showy and gaudy, but with
a self-assertingly temporary and nomadic air on the decorations, as
boasting that they will be much more showy and gaudy in the palatial
residence. Mr Lammle’s own particular servant behind his chair; the
Analytical behind Veneering’s chair; instances in point that
such servants fall into two classes: one mistrusting the master’s
acquaintances, and the other mistrusting the master. Mr Lammle’s
servant, of the second class. Appearing to be lost in wonder and low
spirits because the police are so long in coming to take his master up
on some charge of the first magnitude.

Veneering, M.P., on the right of Mrs Lammle; Twemlow on her left; Mrs
Veneering, W.M.P. (wife of Member of Parliament), and Lady Tippins on Mr
Lammle’s right and left. But be sure that well within the fascination of
Mr Lammle’s eye and smile sits little Georgiana. And be sure that
close to little Georgiana, also under inspection by the same gingerous
gentleman, sits Fledgeby.

Oftener than twice or thrice while breakfast is in progress, Mr Twemlow
gives a little sudden turn towards Mrs Lammle, and then says to her, ‘I
beg your pardon!’ This not being Twemlow’s usual way, why is it his
way to-day? Why, the truth is, Twemlow repeatedly labours under the
impression that Mrs Lammle is going to speak to him, and turning finds
that it is not so, and mostly that she has her eyes upon Veneering.
Strange that this impression so abides by Twemlow after being corrected,
yet so it is.

Lady Tippins partaking plentifully of the fruits of the earth (including
grape-juice in the category) becomes livelier, and applies herself to
elicit sparks from Mortimer Lightwood. It is always understood among the
initiated, that that faithless lover must be planted at table opposite
to Lady Tippins, who will then strike conversational fire out of him.
In a pause of mastication and deglutition, Lady Tippins, contemplating
Mortimer, recalls that it was at our dear Veneerings, and in the
presence of a party who are surely all here, that he told them his
story of the man from somewhere, which afterwards became so horribly
interesting and vulgarly popular.

‘Yes, Lady Tippins,’ assents Mortimer; ‘as they say on the stage, “Even
so!”’

‘Then we expect you,’ retorts the charmer, ‘to sustain your reputation,
and tell us something else.’

‘Lady Tippins, I exhausted myself for life that day, and there is
nothing more to be got out of me.’

Mortimer parries thus, with a sense upon him that elsewhere it is Eugene
and not he who is the jester, and that in these circles where Eugene
persists in being speechless, he, Mortimer, is but the double of the
friend on whom he has founded himself.

‘But,’ quoth the fascinating Tippins, ‘I am resolved on getting
something more out of you. Traitor! what is this I hear about another
disappearance?’

‘As it is you who have heard it,’ returns Lightwood, ‘perhaps you’ll
tell us.’

‘Monster, away!’ retorts Lady Tippins. ‘Your own Golden Dustman referred
me to you.’

Mr Lammle, striking in here, proclaims aloud that there is a sequel
to the story of the man from somewhere. Silence ensues upon the
proclamation.

‘I assure you,’ says Lightwood, glancing round the table, ‘I have
nothing to tell.’ But Eugene adding in a low voice, ‘There, tell
it, tell it!’ he corrects himself with the addition, ‘Nothing worth
mentioning.’

Boots and Brewer immediately perceive that it is immensely worth
mentioning, and become politely clamorous. Veneering is also visited by
a perception to the same effect. But it is understood that his attention
is now rather used up, and difficult to hold, that being the tone of the
House of Commons.

‘Pray don’t be at the trouble of composing yourselves to listen,’ says
Mortimer Lightwood, ‘because I shall have finished long before you have
fallen into comfortable attitudes. It’s like--’

‘It’s like,’ impatiently interrupts Eugene, ‘the children’s narrative:

\begin{verbatim}
     "I’ll tell you a story
     Of Jack a Manory,
     And now my story’s begun;
     I’ll tell you another
     Of Jack and his brother,
     And now my story is done."
\end{verbatim}

--Get on, and get it over!’

Eugene says this with a sound of vexation in his voice, leaning back in
his chair and looking balefully at Lady Tippins, who nods to him as
her dear Bear, and playfully insinuates that she (a self-evident
proposition) is Beauty, and he Beast.

‘The reference,’ proceeds Mortimer, ‘which I suppose to be made by my
honourable and fair enslaver opposite, is to the following circumstance.
Very lately, the young woman, Lizzie Hexam, daughter of the late Jesse
Hexam, otherwise Gaffer, who will be remembered to have found the body
of the man from somewhere, mysteriously received, she knew not from
whom, an explicit retraction of the charges made against her father, by
another water-side character of the name of Riderhood. Nobody believed
them, because little Rogue Riderhood--I am tempted into the paraphrase
by remembering the charming wolf who would have rendered society a great
service if he had devoured Mr Riderhood’s father and mother in their
infancy--had previously played fast and loose with the said charges,
and, in fact, abandoned them. However, the retraction I have mentioned
found its way into Lizzie Hexam’s hands, with a general flavour on it
of having been favoured by some anonymous messenger in a dark cloak and
slouched hat, and was by her forwarded, in her father’s vindication, to
Mr Boffin, my client. You will excuse the phraseology of the shop, but
as I never had another client, and in all likelihood never shall have, I
am rather proud of him as a natural curiosity probably unique.’

Although as easy as usual on the surface, Lightwood is not quite as easy
as usual below it. With an air of not minding Eugene at all, he feels
that the subject is not altogether a safe one in that connexion.

‘The natural curiosity which forms the sole ornament of my professional
museum,’ he resumes, ‘hereupon desires his Secretary--an individual
of the hermit-crab or oyster species, and whose name, I think, is
Chokesmith--but it doesn’t in the least matter--say Artichoke--to put
himself in communication with Lizzie Hexam. Artichoke professes his
readiness so to do, endeavours to do so, but fails.’

‘Why fails?’ asks Boots.

‘How fails?’ asks Brewer.

‘Pardon me,’ returns Lightwood, ‘I must postpone the reply for one
moment, or we shall have an anti-climax. Artichoke failing signally, my
client refers the task to me: his purpose being to advance the interests
of the object of his search. I proceed to put myself in communication
with her; I even happen to possess some special means,’ with a glance
at Eugene, ‘of putting myself in communication with her; but I fail too,
because she has vanished.’

‘Vanished!’ is the general echo.

‘Disappeared,’ says Mortimer. ‘Nobody knows how, nobody knows when,
nobody knows where. And so ends the story to which my honourable and
fair enslaver opposite referred.’

Tippins, with a bewitching little scream, opines that we shall every one
of us be murdered in our beds. Eugene eyes her as if some of us would
be enough for him. Mrs Veneering, W.M.P., remarks that these social
mysteries make one afraid of leaving Baby. Veneering, M.P., wishes to
be informed (with something of a second-hand air of seeing the Right
Honourable Gentleman at the head of the Home Department in his place)
whether it is intended to be conveyed that the vanished person has been
spirited away or otherwise harmed? Instead of Lightwood’s answering,
Eugene answers, and answers hastily and vexedly: ‘No, no, no; he doesn’t
mean that; he means voluntarily vanished--but utterly--completely.’

However, the great subject of the happiness of Mr and Mrs Lammle must
not be allowed to vanish with the other vanishments--with the vanishing
of the murderer, the vanishing of Julius Handford, the vanishing of
Lizzie Hexam,--and therefore Veneering must recall the present sheep
to the pen from which they have strayed. Who so fit to discourse of
the happiness of Mr and Mrs Lammle, they being the dearest and oldest
friends he has in the world; or what audience so fit for him to take
into his confidence as that audience, a noun of multitude or signifying
many, who are all the oldest and dearest friends he has in the world?
So Veneering, without the formality of rising, launches into a familiar
oration, gradually toning into the Parliamentary sing-song, in which he
sees at that board his dear friend Twemlow who on that day twelvemonth
bestowed on his dear friend Lammle the fair hand of his dear friend
Sophronia, and in which he also sees at that board his dear friends
Boots and Brewer whose rallying round him at a period when his dear
friend Lady Tippins likewise rallied round him--ay, and in the foremost
rank--he can never forget while memory holds her seat. But he is free
to confess that he misses from that board his dear old friend Podsnap,
though he is well represented by his dear young friend Georgiana. And he
further sees at that board (this he announces with pomp, as if exulting
in the powers of an extraordinary telescope) his friend Mr Fledgeby, if
he will permit him to call him so. For all of these reasons, and many
more which he right well knows will have occurred to persons of your
exceptional acuteness, he is here to submit to you that the time has
arrived when, with our hearts in our glasses, with tears in our eyes,
with blessings on our lips, and in a general way with a profusion of
gammon and spinach in our emotional larders, we should one and all drink
to our dear friends the Lammles, wishing them many years as happy as
the last, and many many friends as congenially united as themselves. And
this he will add; that Anastatia Veneering (who is instantly heard to
weep) is formed on the same model as her old and chosen friend Sophronia
Lammle, in respect that she is devoted to the man who wooed and won her,
and nobly discharges the duties of a wife.

Seeing no better way out of it, Veneering here pulls up his oratorical
Pegasus extremely short, and plumps down, clean over his head, with:
‘Lammle, God bless you!’

Then Lammle. Too much of him every way; pervadingly too much nose of a
coarse wrong shape, and his nose in his mind and his manners; too much
smile to be real; too much frown to be false; too many large teeth to be
visible at once without suggesting a bite. He thanks you, dear friends,
for your kindly greeting, and hopes to receive you--it may be on the
next of these delightful occasions--in a residence better suited to
your claims on the rites of hospitality. He will never forget that at
Veneering’s he first saw Sophronia. Sophronia will never forget that at
Veneering’s she first saw him. ‘They spoke of it soon after they
were married, and agreed that they would never forget it. In fact, to
Veneering they owe their union. They hope to show their sense of this
some day [‘No, no, from Veneering)--oh yes, yes, and let him rely
upon it, they will if they can! His marriage with Sophronia was not a
marriage of interest on either side: she had her little fortune, he had
his little fortune: they joined their little fortunes: it was a marriage
of pure inclination and suitability. Thank you! Sophronia and he are
fond of the society of young people; but he is not sure that their house
would be a good house for young people proposing to remain single, since
the contemplation of its domestic bliss might induce them to change
their minds. He will not apply this to any one present; certainly not
to their darling little Georgiana. Again thank you! Neither, by-the-by,
will he apply it to his friend Fledgeby. He thanks Veneering for the
feeling manner in which he referred to their common friend Fledgeby, for
he holds that gentleman in the highest estimation. Thank you. In fact
(returning unexpectedly to Fledgeby), the better you know him, the more
you find in him that you desire to know. Again thank you! In his dear
Sophronia’s name and in his own, thank you!

Mrs Lammle has sat quite still, with her eyes cast down upon the
table-cloth. As Mr Lammle’s address ends, Twemlow once more turns to her
involuntarily, not cured yet of that often recurring impression that she
is going to speak to him. This time she really is going to speak to him.
Veneering is talking with his other next neighbour, and she speaks in a
low voice.

‘Mr Twemlow.’

He answers, ‘I beg your pardon? Yes?’ Still a little doubtful, because
of her not looking at him.

‘You have the soul of a gentleman, and I know I may trust you. Will you
give me the opportunity of saying a few words to you when you come up
stairs?’

‘Assuredly. I shall be honoured.’

‘Don’t seem to do so, if you please, and don’t think it inconsistent if
my manner should be more careless than my words. I may be watched.’

Intensely astonished, Twemlow puts his hand to his forehead, and sinks
back in his chair meditating. Mrs Lammle rises. All rise. The ladies go
up stairs. The gentlemen soon saunter after them. Fledgeby has devoted
the interval to taking an observation of Boots’s whiskers, Brewer’s
whiskers, and Lammle’s whiskers, and considering which pattern of
whisker he would prefer to produce out of himself by friction, if the
Genie of the cheek would only answer to his rubbing.

In the drawing-room, groups form as usual. Lightwood, Boots, and Brewer,
flutter like moths around that yellow wax candle--guttering down,
and with some hint of a winding-sheet in it--Lady Tippins. Outsiders
cultivate Veneering, M P., and Mrs Veneering, W.M.P. Lammle stands with
folded arms, Mephistophelean in a corner, with Georgiana and Fledgeby.
Mrs Lammle, on a sofa by a table, invites Mr Twemlow’s attention to a
book of portraits in her hand.

Mr Twemlow takes his station on a settee before her, and Mrs Lammle
shows him a portrait.

‘You have reason to be surprised,’ she says softly, ‘but I wish you
wouldn’t look so.’

Disturbed Twemlow, making an effort not to look so, looks much more so.

‘I think, Mr Twemlow, you never saw that distant connexion of yours
before to-day?’

‘No, never.’

‘Now that you do see him, you see what he is. You are not proud of him?’

‘To say the truth, Mrs Lammle, no.’

‘If you knew more of him, you would be less inclined to acknowledge him.
Here is another portrait. What do you think of it?’

Twemlow has just presence of mind enough to say aloud: ‘Very like!
Uncommonly like!’

‘You have noticed, perhaps, whom he favours with his attentions? You
notice where he is now, and how engaged?’

‘Yes. But Mr Lammle--’

She darts a look at him which he cannot comprehend, and shows him
another portrait.

‘Very good; is it not?’

‘Charming!’ says Twemlow.

‘So like as to be almost a caricature?--Mr Twemlow, it is impossible
to tell you what the struggle in my mind has been, before I could bring
myself to speak to you as I do now. It is only in the conviction that I
may trust you never to betray me, that I can proceed. Sincerely promise
me that you never will betray my confidence--that you will respect it,
even though you may no longer respect me,--and I shall be as satisfied
as if you had sworn it.’

‘Madam, on the honour of a poor gentleman--’

‘Thank you. I can desire no more. Mr Twemlow, I implore you to save that
child!’

‘That child?’

‘Georgiana. She will be sacrificed. She will be inveigled and married
to that connexion of yours. It is a partnership affair, a
money-speculation. She has no strength of will or character to help
herself and she is on the brink of being sold into wretchedness for
life.’

‘Amazing! But what can I do to prevent it?’ demands Twemlow, shocked and
bewildered to the last degree.

‘Here is another portrait. And not good, is it?’

Aghast at the light manner of her throwing her head back to look at it
critically, Twemlow still dimly perceives the expediency of throwing his
own head back, and does so. Though he no more sees the portrait than if
it were in China.

‘Decidedly not good,’ says Mrs Lammle. ‘Stiff and exaggerated!’

‘And ex--’ But Twemlow, in his demolished state, cannot command the
word, and trails off into ‘--actly so.’

‘Mr Twemlow, your word will have weight with her pompous, self-blinded
father. You know how much he makes of your family. Lose no time. Warn
him.’

‘But warn him against whom?’

‘Against me.’

By great good fortune Twemlow receives a stimulant at this critical
instant. The stimulant is Lammle’s voice.

‘Sophronia, my dear, what portraits are you showing Twemlow?’

‘Public characters, Alfred.’

‘Show him the last of me.’

‘Yes, Alfred.’

She puts the book down, takes another book up, turns the leaves, and
presents the portrait to Twemlow.

‘That is the last of Mr Lammle. Do you think it good?--Warn her father
against me. I deserve it, for I have been in the scheme from the first.
It is my husband’s scheme, your connexion’s, and mine. I tell you this,
only to show you the necessity of the poor little foolish affectionate
creature’s being befriended and rescued. You will not repeat this to her
father. You will spare me so far, and spare my husband. For, though this
celebration of to-day is all a mockery, he is my husband, and we must
live.--Do you think it like?’

Twemlow, in a stunned condition, feigns to compare the portrait in his
hand with the original looking towards him from his Mephistophelean
corner.

‘Very well indeed!’ are at length the words which Twemlow with great
difficulty extracts from himself.

‘I am glad you think so. On the whole, I myself consider it the best.
The others are so dark. Now here, for instance, is another of Mr
Lammle--’

‘But I don’t understand; I don’t see my way,’ Twemlow stammers, as he
falters over the book with his glass at his eye. ‘How warn her father,
and not tell him? Tell him how much? Tell him how little? I--I--am
getting lost.’

‘Tell him I am a match-maker; tell him I am an artful and designing
woman; tell him you are sure his daughter is best out of my house and my
company. Tell him any such things of me; they will all be true. You know
what a puffed-up man he is, and how easily you can cause his vanity to
take the alarm. Tell him as much as will give him the alarm and make
him careful of her, and spare me the rest. Mr Twemlow, I feel my sudden
degradation in your eyes; familiar as I am with my degradation in my own
eyes, I keenly feel the change that must have come upon me in yours,
in these last few moments. But I trust to your good faith with me as
implicitly as when I began. If you knew how often I have tried to speak
to you to-day, you would almost pity me. I want no new promise from you
on my own account, for I am satisfied, and I always shall be satisfied,
with the promise you have given me. I can venture to say no more, for
I see that I am watched. If you would set my mind at rest with the
assurance that you will interpose with the father and save this harmless
girl, close that book before you return it to me, and I shall know what
you mean, and deeply thank you in my heart.--Alfred, Mr Twemlow thinks
the last one the best, and quite agrees with you and me.’

Alfred advances. The groups break up. Lady Tippins rises to go, and Mrs
Veneering follows her leader. For the moment, Mrs Lammle does not turn
to them, but remains looking at Twemlow looking at Alfred’s portrait
through his eyeglass. The moment past, Twemlow drops his eyeglass at its
ribbon’s length, rises, and closes the book with an emphasis which makes
that fragile nursling of the fairies, Tippins, start.

Then good-bye and good-bye, and charming occasion worthy of the Golden
Age, and more about the flitch of bacon, and the like of that; and
Twemlow goes staggering across Piccadilly with his hand to his forehead,
and is nearly run down by a flushed lettercart, and at last drops
safe in his easy-chair, innocent good gentleman, with his hand to his
forehead still, and his head in a whirl.





