% REV01 Wed 23 Jun 2021 06:35:26 WIB
% START Tue 04 May 2021 13:55:16 WIB

\chapter{SOME AFFAIRS OF THE HEART}

Little Miss Peecher, from her little official dwelling-house, with its
little windows like the eyes in needles, and its little doors like the
covers of school-books, was very observant indeed of the object of her
quiet affections. Love, though said to be afflicted with blindness, is
a vigilant watchman, and Miss Peecher kept him on double duty over Mr
Bradley Headstone. It was not that she was naturally given to playing
the spy--it was not that she was at all secret, plotting, or mean--it
was simply that she loved the irresponsive Bradley with all the
primitive and homely stock of love that had never been examined or
certificated out of her. If her faithful slate had had the latent
qualities of sympathetic paper, and its pencil those of invisible ink,
many a little treatise calculated to astonish the pupils would have come
bursting through the dry sums in school-time under the warming influence
of Miss Peecher’s bosom. For, oftentimes when school was not, and her
calm leisure and calm little house were her own, Miss Peecher would
commit to the confidential slate an imaginary description of how, upon
a balmy evening at dusk, two figures might have been observed in the
market-garden ground round the corner, of whom one, being a manly form,
bent over the other, being a womanly form of short stature and some
compactness, and breathed in a low voice the words, ‘Emma Peecher, wilt
thou be my own?’ after which the womanly form’s head reposed upon the
manly form’s shoulder, and the nightingales tuned up. Though all unseen,
and unsuspected by the pupils, Bradley Headstone even pervaded the
school exercises. Was Geography in question? He would come triumphantly
flying out of Vesuvius and Aetna ahead of the lava, and would boil
unharmed in the hot springs of Iceland, and would float majestically
down the Ganges and the Nile. Did History chronicle a king of men?
Behold him in pepper-and-salt pantaloons, with his watch-guard round
his neck. Were copies to be written? In capital B’s and H’s most of the
girls under Miss Peecher’s tuition were half a year ahead of every other
letter in the alphabet. And Mental Arithmetic, administered by Miss
Peecher, often devoted itself to providing Bradley Headstone with a
wardrobe of fabulous extent: fourscore and four neck-ties at two and
ninepence-halfpenny, two gross of silver watches at four pounds fifteen
and sixpence, seventy-four black hats at eighteen shillings; and many
similar superfluities.

The vigilant watchman, using his daily opportunities of turning his eyes
in Bradley’s direction, soon apprized Miss Peecher that Bradley was more
preoccupied than had been his wont, and more given to strolling about
with a downcast and reserved face, turning something difficult in his
mind that was not in the scholastic syllabus. Putting this and that
together--combining under the head ‘this,’ present appearances and the
intimacy with Charley Hexam, and ranging under the head ‘that’ the
visit to his sister, the watchman reported to Miss Peecher his strong
suspicions that the sister was at the bottom of it.

‘I wonder,’ said Miss Peecher, as she sat making up her weekly report on
a half-holiday afternoon, ‘what they call Hexam’s sister?’

Mary Anne, at her needlework, attendant and attentive, held her arm up.

‘Well, Mary Anne?’

‘She is named Lizzie, ma’am.’

‘She can hardly be named Lizzie, I think, Mary Anne,’ returned Miss
Peecher, in a tunefully instructive voice. ‘Is Lizzie a Christian name,
Mary Anne?’

Mary Anne laid down her work, rose, hooked herself behind, as being
under catechization, and replied: ‘No, it is a corruption, Miss
Peecher.’

‘Who gave her that name?’ Miss Peecher was going on, from the mere force
of habit, when she checked herself; on Mary Anne’s evincing theological
impatience to strike in with her godfathers and her godmothers, and
said: ‘I mean of what name is it a corruption?’

‘Elizabeth, or Eliza, Miss Peecher.’

‘Right, Mary Anne. Whether there were any Lizzies in the early Christian
Church must be considered very doubtful, very doubtful.’ Miss Peecher
was exceedingly sage here. ‘Speaking correctly, we say, then, that
Hexam’s sister is called Lizzie; not that she is named so. Do we not,
Mary Anne?’

‘We do, Miss Peecher.’

‘And where,’ pursued Miss Peecher, complacent in her little transparent
fiction of conducting the examination in a semiofficial manner for Mary
Anne’s benefit, not her own, ‘where does this young woman, who is called
but not named Lizzie, live? Think, now, before answering.’

‘In Church Street, Smith Square, by Mill Bank, ma’am.’

‘In Church Street, Smith Square, by Mill Bank,’ repeated Miss Peecher,
as if possessed beforehand of the book in which it was written. Exactly
so. And what occupation does this young woman pursue, Mary Anne? Take
time.’

‘She has a place of trust at an outfitter’s in the City, ma’am.’

‘Oh!’ said Miss Peecher, pondering on it; but smoothly added, in a
confirmatory tone, ‘At an outfitter’s in the City. Ye-es?’

‘And Charley--’ Mary Anne was proceeding, when Miss Peecher stared.

‘I mean Hexam, Miss Peecher.’

‘I should think you did, Mary Anne. I am glad to hear you do. And
Hexam--’

‘Says,’ Mary Anne went on, ‘that he is not pleased with his sister, and
that his sister won’t be guided by his advice, and persists in being
guided by somebody else’s; and that--’

‘Mr Headstone coming across the garden!’ exclaimed Miss Peecher, with a
flushed glance at the looking-glass. ‘You have answered very well, Mary
Anne. You are forming an excellent habit of arranging your thoughts
clearly. That will do.’

The discreet Mary Anne resumed her seat and her silence, and stitched,
and stitched, and was stitching when the schoolmaster’s shadow came in
before him, announcing that he might be instantly expected.

‘Good evening, Miss Peecher,’ he said, pursuing the shadow, and taking
its place.

‘Good evening, Mr Headstone. Mary Anne, a chair.’

‘Thank you,’ said Bradley, seating himself in his constrained manner.
‘This is but a flying visit. I have looked in, on my way, to ask a
kindness of you as a neighbour.’

‘Did you say on your way, Mr Headstone?’ asked Miss Peecher.

‘On my way to--where I am going.’

‘Church Street, Smith Square, by Mill Bank,’ repeated Miss Peecher, in
her own thoughts.

‘Charley Hexam has gone to get a book or two he wants, and will probably
be back before me. As we leave my house empty, I took the liberty of
telling him I would leave the key here. Would you kindly allow me to do
so?’

‘Certainly, Mr Headstone. Going for an evening walk, sir?’

‘Partly for a walk, and partly for--on business.’

‘Business in Church Street, Smith Square, by Mill Bank,’ repeated Miss
Peecher to herself.

‘Having said which,’ pursued Bradley, laying his door-key on the table,
‘I must be already going. There is nothing I can do for you, Miss
Peecher?’

‘Thank you, Mr Headstone. In which direction?’

‘In the direction of Westminster.’

‘Mill Bank,’ Miss Peecher repeated in her own thoughts once again. ‘No,
thank you, Mr Headstone; I’ll not trouble you.’

‘You couldn’t trouble me,’ said the schoolmaster.

‘Ah!’ returned Miss Peecher, though not aloud; ‘but you can trouble
ME!’ And for all her quiet manner, and her quiet smile, she was full of
trouble as he went his way.

She was right touching his destination. He held as straight a course
for the house of the dolls’ dressmaker as the wisdom of his ancestors,
exemplified in the construction of the intervening streets, would let
him, and walked with a bent head hammering at one fixed idea. It had
been an immoveable idea since he first set eyes upon her. It seemed to
him as if all that he could suppress in himself he had suppressed, as
if all that he could restrain in himself he had restrained, and the time
had come--in a rush, in a moment--when the power of self-command had
departed from him. Love at first sight is a trite expression quite
sufficiently discussed; enough that in certain smouldering natures like
this man’s, that passion leaps into a blaze, and makes such head as fire
does in a rage of wind, when other passions, but for its mastery, could
be held in chains. As a multitude of weak, imitative natures are
always lying by, ready to go mad upon the next wrong idea that may be
broached--in these times, generally some form of tribute to Somebody
for something that never was done, or, if ever done, that was done by
Somebody Else--so these less ordinary natures may lie by for years,
ready on the touch of an instant to burst into flame.

The schoolmaster went his way, brooding and brooding, and a sense of
being vanquished in a struggle might have been pieced out of his worried
face. Truly, in his breast there lingered a resentful shame to find
himself defeated by this passion for Charley Hexam’s sister, though in
the very self-same moments he was concentrating himself upon the object
of bringing the passion to a successful issue.

He appeared before the dolls’ dressmaker, sitting alone at her work.
‘Oho!’ thought that sharp young personage, ‘it’s you, is it? I know your
tricks and your manners, my friend!’

‘Hexam’s sister,’ said Bradley Headstone, ‘is not come home yet?’

‘You are quite a conjuror,’ returned Miss Wren.

‘I will wait, if you please, for I want to speak to her.’

‘Do you?’ returned Miss Wren. ‘Sit down. I hope it’s mutual.’ Bradley
glanced distrustfully at the shrewd face again bending over the work,
and said, trying to conquer doubt and hesitation:

‘I hope you don’t imply that my visit will be unacceptable to Hexam’s
sister?’

‘There! Don’t call her that. I can’t bear you to call her that,’
returned Miss Wren, snapping her fingers in a volley of impatient snaps,
‘for I don’t like Hexam.’

‘Indeed?’

‘No.’ Miss Wren wrinkled her nose, to express dislike. ‘Selfish. Thinks
only of himself. The way with all of you.’

‘The way with all of us? Then you don’t like ME?’

‘So-so,’ replied Miss Wren, with a shrug and a laugh. ‘Don’t know much
about you.’

‘But I was not aware it was the way with all of us,’ said Bradley,
returning to the accusation, a little injured. ‘Won’t you say, some of
us?’

‘Meaning,’ returned the little creature, ‘every one of you, but you.
Hah! Now look this lady in the face. This is Mrs Truth. The Honourable.
Full-dressed.’

Bradley glanced at the doll she held up for his observation--which had
been lying on its face on her bench, while with a needle and thread she
fastened the dress on at the back--and looked from it to her.

‘I stand the Honourable Mrs T. on my bench in this corner against the
wall, where her blue eyes can shine upon you,’ pursued Miss Wren, doing
so, and making two little dabs at him in the air with her needle, as
if she pricked him with it in his own eyes; ‘and I defy you to tell me,
with Mrs T. for a witness, what you have come here for.’

‘To see Hexam’s sister.’

‘You don’t say so!’ retorted Miss Wren, hitching her chin. ‘But on whose
account?’

‘Her own.’

‘O Mrs T.!’ exclaimed Miss Wren. ‘You hear him!’

‘To reason with her,’ pursued Bradley, half humouring what was present,
and half angry with what was not present; ‘for her own sake.’

‘Oh Mrs T.!’ exclaimed the dressmaker.

‘For her own sake,’ repeated Bradley, warming, ‘and for her brother’s,
and as a perfectly disinterested person.’

‘Really, Mrs T.,’ remarked the dressmaker, ‘since it comes to this, we
must positively turn you with your face to the wall.’ She had hardly
done so, when Lizzie Hexam arrived, and showed some surprise on seeing
Bradley Headstone there, and Jenny shaking her little fist at him close
before her eyes, and the Honourable Mrs T. with her face to the wall.

‘Here’s a perfectly disinterested person, Lizzie dear,’ said the knowing
Miss Wren, ‘come to talk with you, for your own sake and your brother’s.
Think of that. I am sure there ought to be no third party present at
anything so very kind and so very serious; and so, if you’ll remove the
third party upstairs, my dear, the third party will retire.’

Lizzie took the hand which the dolls’ dressmaker held out to her for
the purpose of being supported away, but only looked at her with an
inquiring smile, and made no other movement.

‘The third party hobbles awfully, you know, when she’s left to herself;’
said Miss Wren, ‘her back being so bad, and her legs so queer; so she
can’t retire gracefully unless you help her, Lizzie.’

‘She can do no better than stay where she is,’ returned Lizzie,
releasing the hand, and laying her own lightly on Miss Jenny’s curls.
And then to Bradley: ‘From Charley, sir?’

In an irresolute way, and stealing a clumsy look at her, Bradley rose to
place a chair for her, and then returned to his own.

‘Strictly speaking,’ said he, ‘I come from Charley, because I left him
only a little while ago; but I am not commissioned by Charley. I come of
my own spontaneous act.’

With her elbows on her bench, and her chin upon her hands, Miss Jenny
Wren sat looking at him with a watchful sidelong look. Lizzie, in her
different way, sat looking at him too.

‘The fact is,’ began Bradley, with a mouth so dry that he had some
difficulty in articulating his words: the consciousness of which
rendered his manner still more ungainly and undecided; ‘the truth is,
that Charley, having no secrets from me (to the best of my belief), has
confided the whole of this matter to me.’

He came to a stop, and Lizzie asked: ‘what matter, sir?’

‘I thought,’ returned the schoolmaster, stealing another look at her,
and seeming to try in vain to sustain it; for the look dropped as it
lighted on her eyes, ‘that it might be so superfluous as to be almost
impertinent, to enter upon a definition of it. My allusion was to this
matter of your having put aside your brother’s plans for you, and
given the preference to those of Mr--I believe the name is Mr Eugene
Wrayburn.’

He made this point of not being certain of the name, with another uneasy
look at her, which dropped like the last.

Nothing being said on the other side, he had to begin again, and began
with new embarrassment.

‘Your brother’s plans were communicated to me when he first had them in
his thoughts. In point of fact he spoke to me about them when I was
last here--when we were walking back together, and when I--when the
impression was fresh upon me of having seen his sister.’

There might have been no meaning in it, but the little dressmaker here
removed one of her supporting hands from her chin, and musingly turned
the Honourable Mrs T. with her face to the company. That done, she fell
into her former attitude.

‘I approved of his idea,’ said Bradley, with his uneasy look wandering
to the doll, and unconsciously resting there longer than it had
rested on Lizzie, ‘both because your brother ought naturally to be the
originator of any such scheme, and because I hoped to be able to promote
it. I should have had inexpressible pleasure, I should have taken
inexpressible interest, in promoting it. Therefore I must acknowledge
that when your brother was disappointed, I too was disappointed. I wish
to avoid reservation or concealment, and I fully acknowledge that.’

He appeared to have encouraged himself by having got so far. At all
events he went on with much greater firmness and force of emphasis:
though with a curious disposition to set his teeth, and with a curious
tight-screwing movement of his right hand in the clenching palm of his
left, like the action of one who was being physically hurt, and was
unwilling to cry out.

‘I am a man of strong feelings, and I have strongly felt this
disappointment. I do strongly feel it. I don’t show what I feel; some
of us are obliged habitually to keep it down. To keep it down. But to
return to your brother. He has taken the matter so much to heart that
he has remonstrated (in my presence he remonstrated) with Mr Eugene
Wrayburn, if that be the name. He did so, quite ineffectually. As any
one not blinded to the real character of Mr--Mr Eugene Wrayburn--would
readily suppose.’

He looked at Lizzie again, and held the look. And his face turned from
burning red to white, and from white back to burning red, and so for the
time to lasting deadly white.

‘Finally, I resolved to come here alone, and appeal to you. I resolved
to come here alone, and entreat you to retract the course you have
chosen, and instead of confiding in a mere stranger--a person of most
insolent behaviour to your brother and others--to prefer your brother
and your brother’s friend.’

Lizzie Hexam had changed colour when those changes came over him, and
her face now expressed some anger, more dislike, and even a touch of
fear. But she answered him very steadily.

‘I cannot doubt, Mr Headstone, that your visit is well meant. You have
been so good a friend to Charley that I have no right to doubt it. I
have nothing to tell Charley, but that I accepted the help to which he
so much objects before he made any plans for me; or certainly before I
knew of any. It was considerately and delicately offered, and there were
reasons that had weight with me which should be as dear to Charley as to
me. I have no more to say to Charley on this subject.’

His lips trembled and stood apart, as he followed this repudiation of
himself; and limitation of her words to her brother.

‘I should have told Charley, if he had come to me,’ she resumed, as
though it were an after-thought, ‘that Jenny and I find our teacher very
able and very patient, and that she takes great pains with us. So much
so, that we have said to her we hope in a very little while to be able
to go on by ourselves. Charley knows about teachers, and I should also
have told him, for his satisfaction, that ours comes from an institution
where teachers are regularly brought up.’

‘I should like to ask you,’ said Bradley Headstone, grinding his words
slowly out, as though they came from a rusty mill; ‘I should like to
ask you, if I may without offence, whether you would have objected--no;
rather, I should like to say, if I may without offence, that I wish I
had had the opportunity of coming here with your brother and devoting my
poor abilities and experience to your service.’

‘Thank you, Mr Headstone.’

‘But I fear,’ he pursued, after a pause, furtively wrenching at the seat
of his chair with one hand, as if he would have wrenched the chair to
pieces, and gloomily observing her while her eyes were cast down, ‘that
my humble services would not have found much favour with you?’

She made no reply, and the poor stricken wretch sat contending with
himself in a heat of passion and torment. After a while he took out his
handkerchief and wiped his forehead and hands.

‘There is only one thing more I had to say, but it is the most
important. There is a reason against this matter, there is a personal
relation concerned in this matter, not yet explained to you. It might--I
don’t say it would--it might--induce you to think differently. To
proceed under the present circumstances is out of the question. Will you
please come to the understanding that there shall be another interview
on the subject?’

‘With Charley, Mr Headstone?’

‘With--well,’ he answered, breaking off, ‘yes! Say with him too.
Will you please come to the understanding that there must be another
interview under more favourable circumstances, before the whole case can
be submitted?’

‘I don’t,’ said Lizzie, shaking her head, ‘understand your meaning, Mr
Headstone.’

‘Limit my meaning for the present,’ he interrupted, ‘to the whole case
being submitted to you in another interview.’

‘What case, Mr Headstone? What is wanting to it?’

‘You--you shall be informed in the other interview.’ Then he said, as
if in a burst of irrepressible despair, ‘I--I leave it all incomplete!
There is a spell upon me, I think!’ And then added, almost as if he
asked for pity, ‘Good-night!’

He held out his hand. As she, with manifest hesitation, not to say
reluctance, touched it, a strange tremble passed over him, and his face,
so deadly white, was moved as by a stroke of pain. Then he was gone.

The dolls’ dressmaker sat with her attitude unchanged, eyeing the door
by which he had departed, until Lizzie pushed her bench aside and sat
down near her. Then, eyeing Lizzie as she had previously eyed Bradley
and the door, Miss Wren chopped that very sudden and keen chop in which
her jaws sometimes indulged, leaned back in her chair with folded arms,
and thus expressed herself:

‘Humph! If he--I mean, of course, my dear, the party who is coming to
court me when the time comes--should be THAT sort of man, he may spare
himself the trouble. HE wouldn’t do to be trotted about and made useful.
He’d take fire and blow up while he was about it.’

‘And so you would be rid of him,’ said Lizzie, humouring her.

‘Not so easily,’ returned Miss Wren. ‘He wouldn’t blow up alone. He’d
carry me up with him. I know his tricks and his manners.’

‘Would he want to hurt you, do you mean?’ asked Lizzie.

‘Mightn’t exactly want to do it, my dear,’ returned Miss Wren; ‘but a
lot of gunpowder among lighted lucifer-matches in the next room might
almost as well be here.’

‘He is a very strange man,’ said Lizzie, thoughtfully.

‘I wish he was so very strange a man as to be a total stranger,’
answered the sharp little thing.

It being Lizzie’s regular occupation when they were alone of an evening
to brush out and smooth the long fair hair of the dolls’ dressmaker, she
unfastened a ribbon that kept it back while the little creature was at
her work, and it fell in a beautiful shower over the poor shoulders that
were much in need of such adorning rain. ‘Not now, Lizzie, dear,’ said
Jenny; ‘let us have a talk by the fire.’ With those words, she in her
turn loosened her friend’s dark hair, and it dropped of its own weight
over her bosom, in two rich masses. Pretending to compare the colours
and admire the contrast, Jenny so managed a mere touch or two of her
nimble hands, as that she herself laying a cheek on one of the dark
folds, seemed blinded by her own clustering curls to all but the fire,
while the fine handsome face and brow of Lizzie were revealed without
obstruction in the sombre light.

‘Let us have a talk,’ said Jenny, ‘about Mr Eugene Wrayburn.’

Something sparkled down among the fair hair resting on the dark hair;
and if it were not a star--which it couldn’t be--it was an eye; and
if it were an eye, it was Jenny Wren’s eye, bright and watchful as the
bird’s whose name she had taken.

‘Why about Mr Wrayburn?’ Lizzie asked.

‘For no better reason than because I’m in the humour. I wonder whether
he’s rich!’

‘No, not rich.’

‘Poor?’

‘I think so, for a gentleman.’

‘Ah! To be sure! Yes, he’s a gentleman. Not of our sort; is he?’ A shake
of the head, a thoughtful shake of the head, and the answer, softly
spoken, ‘Oh no, oh no!’

The dolls’ dressmaker had an arm round her friend’s waist. Adjusting the
arm, she slyly took the opportunity of blowing at her own hair where
it fell over her face; then the eye down there, under lighter shadows
sparkled more brightly and appeared more watchful.

‘When He turns up, he shan’t be a gentleman; I’ll very soon send him
packing, if he is. However, he’s not Mr Wrayburn; I haven’t captivated
HIM. I wonder whether anybody has, Lizzie!’

‘It is very likely.’

‘Is it very likely? I wonder who!’

‘Is it not very likely that some lady has been taken by him, and that he
may love her dearly?’

‘Perhaps. I don’t know. What would you think of him, Lizzie, if you were
a lady?’

‘I a lady!’ she repeated, laughing. ‘Such a fancy!’

‘Yes. But say: just as a fancy, and for instance.’

‘I a lady! I, a poor girl who used to row poor father on the river. I,
who had rowed poor father out and home on the very night when I saw him
for the first time. I, who was made so timid by his looking at me, that
I got up and went out!’

[‘He did look at you, even that night, though you were not a lady!’
thought Miss Wren.)

‘I a lady!’ Lizzie went on in a low voice, with her eyes upon the fire.
‘I, with poor father’s grave not even cleared of undeserved stain and
shame, and he trying to clear it for me! I a lady!’

‘Only as a fancy, and for instance,’ urged Miss Wren.

‘Too much, Jenny, dear, too much! My fancy is not able to get that far.’
As the low fire gleamed upon her, it showed her smiling, mournfully and
abstractedly.

‘But I am in the humour, and I must be humoured, Lizzie, because after
all I am a poor little thing, and have had a hard day with my bad child.
Look in the fire, as I like to hear you tell how you used to do when you
lived in that dreary old house that had once been a windmill. Look in
the--what was its name when you told fortunes with your brother that I
DON’T like?’

‘The hollow down by the flare?’

‘Ah! That’s the name! You can find a lady there, I know.’

‘More easily than I can make one of such material as myself, Jenny.’

The sparkling eye looked steadfastly up, as the musing face looked
thoughtfully down. ‘Well?’ said the dolls’ dressmaker, ‘We have found
our lady?’

Lizzie nodded, and asked, ‘Shall she be rich?’

‘She had better be, as he’s poor.’

‘She is very rich. Shall she be handsome?’

‘Even you can be that, Lizzie, so she ought to be.’

‘She is very handsome.’

‘What does she say about him?’ asked Miss Jenny, in a low voice:
watchful, through an intervening silence, of the face looking down at
the fire.

‘She is glad, glad, to be rich, that he may have the money. She is glad,
glad, to be beautiful, that he may be proud of her. Her poor heart--’

‘Eh? Her poor heart?’ said Miss Wren.

‘Her heart--is given him, with all its love and truth. She would
joyfully die with him, or, better than that, die for him. She knows he
has failings, but she thinks they have grown up through his being like
one cast away, for the want of something to trust in, and care for, and
think well of. And she says, that lady rich and beautiful that I can
never come near, “Only put me in that empty place, only try how little
I mind myself, only prove what a world of things I will do and bear for
you, and I hope that you might even come to be much better than you are,
through me who am so much worse, and hardly worth the thinking of beside
you.”’

As the face looking at the fire had become exalted and forgetful in the
rapture of these words, the little creature, openly clearing away
her fair hair with her disengaged hand, had gazed at it with earnest
attention and something like alarm. Now that the speaker ceased, the
little creature laid down her head again, and moaned, ‘O me, O me, O
me!’

‘In pain, dear Jenny?’ asked Lizzie, as if awakened.

‘Yes, but not the old pain. Lay me down, lay me down. Don’t go out of
my sight to-night. Lock the door and keep close to me.’ Then turning away
her face, she said in a whisper to herself, ‘My Lizzie, my poor Lizzie!
O my blessed children, come back in the long bright slanting rows, and
come for her, not me. She wants help more than I, my blessed children!’

She had stretched her hands up with that higher and better look, and
now she turned again, and folded them round Lizzie’s neck, and rocked
herself on Lizzie’s breast.



