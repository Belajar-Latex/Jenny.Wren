% REV00 Fri 26 Mar 2021 18:30:59 WIB
% START Fri 26 Mar 2021 18:30:49 WIB

\chapter{THE MAN FROM SOMEWHERE}

Mr and Mrs Veneering were bran-new people in a bran-new house in a
bran-new quarter of London. Everything about the Veneerings was spick
and span new. All their furniture was new, all their friends were new,
all their servants were new, their plate was new, their carriage was
new, their harness was new, their horses were new, their pictures
were new, they themselves were new, they were as newly married as was
lawfully compatible with their having a bran-new baby, and if they had
set up a great-grandfather, he would have come home in matting from the
Pantechnicon, without a scratch upon him, French polished to the crown
of his head.

For, in the Veneering establishment, from the hall-chairs with the new
coat of arms, to the grand pianoforte with the new action, and upstairs
again to the new fire-escape, all things were in a state of high varnish
and polish. And what was observable in the furniture, was observable in
the Veneerings--the surface smelt a little too much of the workshop and
was a trifle sticky.

There was an innocent piece of dinner-furniture that went upon easy
castors and was kept over a livery stable-yard in Duke Street, Saint
James’s, when not in use, to whom the Veneerings were a source of blind
confusion. The name of this article was Twemlow. Being first cousin
to Lord Snigsworth, he was in frequent requisition, and at many houses
might be said to represent the dining-table in its normal state. Mr and
Mrs Veneering, for example, arranging a dinner, habitually started with
Twemlow, and then put leaves in him, or added guests to him. Sometimes,
the table consisted of Twemlow and half a dozen leaves; sometimes, of
Twemlow and a dozen leaves; sometimes, Twemlow was pulled out to his
utmost extent of twenty leaves. Mr and Mrs Veneering on occasions of
ceremony faced each other in the centre of the board, and thus the
parallel still held; for, it always happened that the more Twemlow was
pulled out, the further he found himself from the center, and nearer
to the sideboard at one end of the room, or the window-curtains at the
other.

But, it was not this which steeped the feeble soul of Twemlow in
confusion. This he was used to, and could take soundings of. The abyss
to which he could find no bottom, and from which started forth the
engrossing and ever-swelling difficulty of his life, was the insoluble
question whether he was Veneering’s oldest friend, or newest friend.
To the excogitation of this problem, the harmless gentleman had devoted
many anxious hours, both in his lodgings over the livery stable-yard,
and in the cold gloom, favourable to meditation, of Saint James’s
Square. Thus. Twemlow had first known Veneering at his club, where
Veneering then knew nobody but the man who made them known to one
another, who seemed to be the most intimate friend he had in the world,
and whom he had known two days--the bond of union between their souls,
the nefarious conduct of the committee respecting the cookery of
a fillet of veal, having been accidentally cemented at that date.
Immediately upon this, Twemlow received an invitation to dine with
Veneering, and dined: the man being of the party. Immediately upon
that, Twemlow received an invitation to dine with the man, and dined:
Veneering being of the party. At the man’s were a Member, an Engineer, a
Payer-off of the National Debt, a Poem on Shakespeare, a Grievance, and
a Public Office, who all seem to be utter strangers to Veneering. And
yet immediately after that, Twemlow received an invitation to dine at
Veneerings, expressly to meet the Member, the Engineer, the Payer-off
of the National Debt, the Poem on Shakespeare, the Grievance, and the
Public Office, and, dining, discovered that all of them were the most
intimate friends Veneering had in the world, and that the wives of all
of them (who were all there) were the objects of Mrs Veneering’s most
devoted affection and tender confidence.

Thus it had come about, that Mr Twemlow had said to himself in his
lodgings, with his hand to his forehead: ‘I must not think of this. This
is enough to soften any man’s brain,’--and yet was always thinking of
it, and could never form a conclusion.

This evening the Veneerings give a banquet. Eleven leaves in the
Twemlow; fourteen in company all told. Four pigeon-breasted retainers in
plain clothes stand in line in the hall. A fifth retainer, proceeding up
the staircase with a mournful air--as who should say, ‘Here is another
wretched creature come to dinner; such is life!’--announces, ‘Mis-ter
Twemlow!’

Mrs Veneering welcomes her sweet Mr Twemlow. Mr Veneering welcomes
his dear Twemlow. Mrs Veneering does not expect that Mr Twemlow can in
nature care much for such insipid things as babies, but so old a friend
must please to look at baby. ‘Ah! You will know the friend of your
family better, Tootleums,’ says Mr Veneering, nodding emotionally at
that new article, ‘when you begin to take notice.’ He then begs to make
his dear Twemlow known to his two friends, Mr Boots and Mr Brewer--and
clearly has no distinct idea which is which.

But now a fearful circumstance occurs.

‘Mis-ter and Mis-sus Podsnap!’

‘My dear,’ says Mr Veneering to Mrs Veneering, with an air of much
friendly interest, while the door stands open, ‘the Podsnaps.’

A too, too smiling large man, with a fatal freshness on him, appearing
with his wife, instantly deserts his wife and darts at Twemlow with:

‘How do you do? So glad to know you. Charming house you have here. I
hope we are not late. So glad of the opportunity, I am sure!’

When the first shock fell upon him, Twemlow twice skipped back in
his neat little shoes and his neat little silk stockings of a bygone
fashion, as if impelled to leap over a sofa behind him; but the large
man closed with him and proved too strong.

‘Let me,’ says the large man, trying to attract the attention of his
wife in the distance, ‘have the pleasure of presenting Mrs Podsnap
to her host. She will be,’ in his fatal freshness he seems to find
perpetual verdure and eternal youth in the phrase, ‘she will be so glad
of the opportunity, I am sure!’

In the meantime, Mrs Podsnap, unable to originate a mistake on her own
account, because Mrs Veneering is the only other lady there, does her
best in the way of handsomely supporting her husband’s, by looking
towards Mr Twemlow with a plaintive countenance and remarking to Mrs
Veneering in a feeling manner, firstly, that she fears he has been
rather bilious of late, and, secondly, that the baby is already very
like him.

It is questionable whether any man quite relishes being mistaken for
any other man; but, Mr Veneering having this very evening set up the
shirt-front of the young Antinous in new worked cambric just come home,
is not at all complimented by being supposed to be Twemlow, who is dry
and weazen and some thirty years older. Mrs Veneering equally resents
the imputation of being the wife of Twemlow. As to Twemlow, he is
so sensible of being a much better bred man than Veneering, that he
considers the large man an offensive ass.

In this complicated dilemma, Mr Veneering approaches the large man with
extended hand and, smilingly assures that incorrigible personage that he
is delighted to see him: who in his fatal freshness instantly replies:

‘Thank you. I am ashamed to say that I cannot at this moment recall
where we met, but I am so glad of this opportunity, I am sure!’

Then pouncing upon Twemlow, who holds back with all his feeble might, he
is haling him off to present him, as Veneering, to Mrs Podsnap, when the
arrival of more guests unravels the mistake. Whereupon, having re-shaken
hands with Veneering as Veneering, he re-shakes hands with Twemlow as
Twemlow, and winds it all up to his own perfect satisfaction by saying
to the last-named, ‘Ridiculous opportunity--but so glad of it, I am
sure!’

Now, Twemlow having undergone this terrific experience, having likewise
noted the fusion of Boots in Brewer and Brewer in Boots, and having
further observed that of the remaining seven guests four discrete
characters enter with wandering eyes and wholly declined to commit
themselves as to which is Veneering, until Veneering has them in his
grasp;--Twemlow having profited by these studies, finds his brain
wholesomely hardening as he approaches the conclusion that he really is
Veneering’s oldest friend, when his brain softens again and all is
lost, through his eyes encountering Veneering and the large man linked
together as twin brothers in the back drawing-room near the conservatory
door, and through his ears informing him in the tones of Mrs Veneering
that the same large man is to be baby’s godfather.

‘Dinner is on the table!’

Thus the melancholy retainer, as who should say, ‘Come down and be
poisoned, ye unhappy children of men!’

Twemlow, having no lady assigned him, goes down in the rear, with
his hand to his forehead. Boots and Brewer, thinking him indisposed,
whisper, ‘Man faint. Had no lunch.’ But he is only stunned by the
unvanquishable difficulty of his existence.

Revived by soup, Twemlow discourses mildly of the Court Circular with
Boots and Brewer. Is appealed to, at the fish stage of the banquet, by
Veneering, on the disputed question whether his cousin Lord Snigsworth
is in or out of town? Gives it that his cousin is out of town. ‘At
Snigsworthy Park?’ Veneering inquires. ‘At Snigsworthy,’ Twemlow
rejoins. Boots and Brewer regard this as a man to be cultivated; and
Veneering is clear that he is a remunerative article. Meantime the
retainer goes round, like a gloomy Analytical Chemist: always seeming
to say, after ‘Chablis, sir?’--‘You wouldn’t if you knew what it’s made
of.’

The great looking-glass above the sideboard, reflects the table and the
company. Reflects the new Veneering crest, in gold and eke in silver,
frosted and also thawed, a camel of all work. The Heralds’ College found
out a Crusading ancestor for Veneering who bore a camel on his shield
(or might have done it if he had thought of it), and a caravan of camels
take charge of the fruits and flowers and candles, and kneel down be
loaded with the salt. Reflects Veneering; forty, wavy-haired, dark,
tending to corpulence, sly, mysterious, filmy--a kind of sufficiently
well-looking veiled-prophet, not prophesying. Reflects Mrs Veneering;
fair, aquiline-nosed and fingered, not so much light hair as she might
have, gorgeous in raiment and jewels, enthusiastic, propitiatory,
conscious that a corner of her husband’s veil is over herself. Reflects
Podsnap; prosperously feeding, two little light-coloured wiry wings, one
on either side of his else bald head, looking as like his hairbrushes as
his hair, dissolving view of red beads on his forehead, large allowance
of crumpled shirt-collar up behind. Reflects Mrs Podsnap; fine woman
for Professor Owen, quantity of bone, neck and nostrils like a
rocking-horse, hard features, majestic head-dress in which Podsnap has
hung golden offerings. Reflects Twemlow; grey, dry, polite, susceptible
to east wind, First-Gentleman-in-Europe collar and cravat, cheeks drawn
in as if he had made a great effort to retire into himself some years
ago, and had got so far and had never got any farther. Reflects mature
young lady; raven locks, and complexion that lights up well when well
powdered--as it is--carrying on considerably in the captivation of
mature young gentleman; with too much nose in his face, too much ginger
in his whiskers, too much torso in his waistcoat, too much sparkle in
his studs, his eyes, his buttons, his talk, and his teeth. Reflects
charming old Lady Tippins on Veneering’s right; with an immense obtuse
drab oblong face, like a face in a tablespoon, and a dyed Long Walk up
the top of her head, as a convenient public approach to the bunch of
false hair behind, pleased to patronize Mrs Veneering opposite, who
is pleased to be patronized. Reflects a certain ‘Mortimer’, another
of Veneering’s oldest friends; who never was in the house before,
and appears not to want to come again, who sits disconsolate on Mrs
Veneering’s left, and who was inveigled by Lady Tippins (a friend of
his boyhood) to come to these people’s and talk, and who won’t talk.
Reflects Eugene, friend of Mortimer; buried alive in the back of his
chair, behind a shoulder--with a powder-epaulette on it--of the mature
young lady, and gloomily resorting to the champagne chalice whenever
proffered by the Analytical Chemist. Lastly, the looking-glass reflects
Boots and Brewer, and two other stuffed Buffers interposed between the
rest of the company and possible accidents.

The Veneering dinners are excellent dinners--or new people wouldn’t
come--and all goes well. Notably, Lady Tippins has made a series of
experiments on her digestive functions, so extremely complicated and
daring, that if they could be published with their results it might
benefit the human race. Having taken in provisions from all parts of the
world, this hardy old cruiser has last touched at the North Pole, when,
as the ice-plates are being removed, the following words fall from her:

‘I assure you, my dear Veneering--’

(Poor Twemlow’s hand approaches his forehead, for it would seem now,
that Lady Tippins is going to be the oldest friend.)

‘I assure you, my dear Veneering, that it is the oddest affair! Like
the advertising people, I don’t ask you to trust me, without offering
a respectable reference. Mortimer there, is my reference, and knows all
about it.’

Mortimer raises his drooping eyelids, and slightly opens his mouth. But
a faint smile, expressive of ‘What’s the use!’ passes over his face, and
he drops his eyelids and shuts his mouth.

‘Now, Mortimer,’ says Lady Tippins, rapping the sticks of her closed
green fan upon the knuckles of her left hand--which is particularly rich
in knuckles, ‘I insist upon your telling all that is to be told about
the man from Jamaica.’

‘Give you my honour I never heard of any man from Jamaica, except the
man who was a brother,’ replies Mortimer.

‘Tobago, then.’

‘Nor yet from Tobago.’

‘Except,’ Eugene strikes in: so unexpectedly that the mature young lady,
who has forgotten all about him, with a start takes the epaulette out
of his way: ‘except our friend who long lived on rice-pudding and
isinglass, till at length to his something or other, his physician said
something else, and a leg of mutton somehow ended in daygo.’

A reviving impression goes round the table that Eugene is coming out. An
unfulfilled impression, for he goes in again.

‘Now, my dear Mrs Veneering,’ quoth Lady Tippins, I appeal to you
whether this is not the basest conduct ever known in this world? I carry
my lovers about, two or three at a time, on condition that they are very
obedient and devoted; and here is my oldest lover-in-chief, the head of
all my slaves, throwing off his allegiance before company! And here is
another of my lovers, a rough Cymon at present certainly, but of whom
I had most hopeful expectations as to his turning out well in course of
time, pretending that he can’t remember his nursery rhymes! On purpose
to annoy me, for he knows how I doat upon them!’

A grisly little fiction concerning her lovers is Lady Tippins’s point.
She is always attended by a lover or two, and she keeps a little list
of her lovers, and she is always booking a new lover, or striking out an
old lover, or putting a lover in her black list, or promoting a lover to
her blue list, or adding up her lovers, or otherwise posting her book.
Mrs Veneering is charmed by the humour, and so is Veneering. Perhaps it
is enhanced by a certain yellow play in Lady Tippins’s throat, like the
legs of scratching poultry.

‘I banish the false wretch from this moment, and I strike him out of
my Cupidon (my name for my Ledger, my dear,) this very night. But I am
resolved to have the account of the man from Somewhere, and I beg you
to elicit it for me, my love,’ to Mrs Veneering, ‘as I have lost my own
influence. Oh, you perjured man!’ This to Mortimer, with a rattle of her
fan.

‘We are all very much interested in the man from Somewhere,’ Veneering
observes.

Then the four Buffers, taking heart of grace all four at once, say:

‘Deeply interested!’

‘Quite excited!’

‘Dramatic!’

‘Man from Nowhere, perhaps!’

And then Mrs Veneering--for the Lady Tippins’s winning wiles are
contagious--folds her hands in the manner of a supplicating child, turns
to her left neighbour, and says, ‘Tease! Pay! Man from Tumwhere!’ At
which the four Buffers, again mysteriously moved all four at once,
explain, ‘You can’t resist!’

‘Upon my life,’ says Mortimer languidly, ‘I find it immensely
embarrassing to have the eyes of Europe upon me to this extent, and my
only consolation is that you will all of you execrate Lady Tippins in
your secret hearts when you find, as you inevitably will, the man from
Somewhere a bore. Sorry to destroy romance by fixing him with a local
habitation, but he comes from the place, the name of which escapes me,
but will suggest itself to everybody else here, where they make the
wine.’

Eugene suggests ‘Day and Martin’s.’

‘No, not that place,’ returns the unmoved Mortimer, ‘that’s where they
make the Port. My man comes from the country where they make the Cape
Wine. But look here, old fellow; its not at all statistical and it’s
rather odd.’

It is always noticeable at the table of the Veneerings, that no man
troubles himself much about the Veneerings themselves, and that any
one who has anything to tell, generally tells it to anybody else in
preference.

‘The man,’ Mortimer goes on, addressing Eugene, ‘whose name is Harmon,
was only son of a tremendous old rascal who made his money by Dust.’

‘Red velveteens and a bell?’ the gloomy Eugene inquires.

‘And a ladder and basket if you like. By which means, or by others, he
grew rich as a Dust Contractor, and lived in a hollow in a hilly country
entirely composed of Dust. On his own small estate the growling old
vagabond threw up his own mountain range, like an old volcano, and its
geological formation was Dust. Coal-dust, vegetable-dust, bone-dust,
crockery dust, rough dust and sifted dust,--all manner of Dust.’

A passing remembrance of Mrs Veneering, here induces Mortimer to address
his next half-dozen words to her; after which he wanders away again,
tries Twemlow and finds he doesn’t answer, ultimately takes up with the
Buffers who receive him enthusiastically.

‘The moral being--I believe that’s the right expression--of this
exemplary person, derived its highest gratification from anathematizing
his nearest relations and turning them out of doors. Having begun (as
was natural) by rendering these attentions to the wife of his bosom,
he next found himself at leisure to bestow a similar recognition on the
claims of his daughter. He chose a husband for her, entirely to his own
satisfaction and not in the least to hers, and proceeded to settle upon
her, as her marriage portion, I don’t know how much Dust, but something
immense. At this stage of the affair the poor girl respectfully
intimated that she was secretly engaged to that popular character whom
the novelists and versifiers call Another, and that such a marriage
would make Dust of her heart and Dust of her life--in short, would
set her up, on a very extensive scale, in her father’s business.
Immediately, the venerable parent--on a cold winter’s night, it is
said--anathematized and turned her out.’

Here, the Analytical Chemist (who has evidently formed a very low
opinion of Mortimer’s story) concedes a little claret to the Buffers;
who, again mysteriously moved all four at once, screw it slowly into
themselves with a peculiar twist of enjoyment, as they cry in chorus,
‘Pray go on.’

‘The pecuniary resources of Another were, as they usually are, of a very
limited nature. I believe I am not using too strong an expression when
I say that Another was hard up. However, he married the young lady, and
they lived in a humble dwelling, probably possessing a porch ornamented
with honeysuckle and woodbine twining, until she died. I must refer
you to the Registrar of the District in which the humble dwelling was
situated, for the certified cause of death; but early sorrow and anxiety
may have had to do with it, though they may not appear in the ruled
pages and printed forms. Indisputably this was the case with Another,
for he was so cut up by the loss of his young wife that if he outlived
her a year it was as much as he did.’

There is that in the indolent Mortimer, which seems to hint that if good
society might on any account allow itself to be impressible, he, one of
good society, might have the weakness to be impressed by what he here
relates. It is hidden with great pains, but it is in him. The gloomy
Eugene too, is not without some kindred touch; for, when that appalling
Lady Tippins declares that if Another had survived, he should have gone
down at the head of her list of lovers--and also when the mature young
lady shrugs her epaulettes, and laughs at some private and confidential
comment from the mature young gentleman--his gloom deepens to that
degree that he trifles quite ferociously with his dessert-knife.

Mortimer proceeds.

‘We must now return, as novelists say, and as we all wish they wouldn’t,
to the man from Somewhere. Being a boy of fourteen, cheaply educated
at Brussels when his sister’s expulsion befell, it was some little time
before he heard of it--probably from herself, for the mother was dead;
but that I don’t know. Instantly, he absconded, and came over here. He
must have been a boy of spirit and resource, to get here on a stopped
allowance of five sous a week; but he did it somehow, and he burst in
on his father, and pleaded his sister’s cause. Venerable parent promptly
resorts to anathematization, and turns him out. Shocked and terrified
boy takes flight, seeks his fortune, gets aboard ship, ultimately
turns up on dry land among the Cape wine: small proprietor, farmer,
grower--whatever you like to call it.’

At this juncture, shuffling is heard in the hall, and tapping is heard
at the dining-room door. Analytical Chemist goes to the door, confers
angrily with unseen tapper, appears to become mollified by descrying
reason in the tapping, and goes out.

‘So he was discovered, only the other day, after having been expatriated
about fourteen years.’

A Buffer, suddenly astounding the other three, by detaching himself, and
asserting individuality, inquires: ‘How discovered, and why?’

‘Ah! To be sure. Thank you for reminding me. Venerable parent dies.’

Same Buffer, emboldened by success, says: ‘When?’

‘The other day. Ten or twelve months ago.’

Same Buffer inquires with smartness, ‘What of?’ But herein perishes a
melancholy example; being regarded by the three other Buffers with a
stony stare, and attracting no further attention from any mortal.

‘Venerable parent,’ Mortimer repeats with a passing remembrance that
there is a Veneering at table, and for the first time addressing
him--‘dies.’

The gratified Veneering repeats, gravely, ‘dies’; and folds his arms,
and composes his brow to hear it out in a judicial manner, when he finds
himself again deserted in the bleak world.

‘His will is found,’ said Mortimer, catching Mrs Podsnap’s
rocking-horse’s eye. ‘It is dated very soon after the son’s flight. It
leaves the lowest of the range of dust-mountains, with some sort of a
dwelling-house at its foot, to an old servant who is sole executor, and
all the rest of the property--which is very considerable--to the son.
He directs himself to be buried with certain eccentric ceremonies and
precautions against his coming to life, with which I need not bore you,
and that’s all--except--’ and this ends the story.

The Analytical Chemist returning, everybody looks at him. Not because
anybody wants to see him, but because of that subtle influence in nature
which impels humanity to embrace the slightest opportunity of looking at
anything, rather than the person who addresses it.

‘--Except that the son’s inheriting is made conditional on his marrying
a girl, who at the date of the will, was a child of four or five years
old, and who is now a marriageable young woman. Advertisement and
inquiry discovered the son in the man from Somewhere, and at the present
moment, he is on his way home from there--no doubt, in a state of great
astonishment--to succeed to a very large fortune, and to take a wife.’

Mrs Podsnap inquires whether the young person is a young person of
personal charms? Mortimer is unable to report.

Mr Podsnap inquires what would become of the very large fortune, in the
event of the marriage condition not being fulfilled? Mortimer replies,
that by special testamentary clause it would then go to the old servant
above mentioned, passing over and excluding the son; also, that if
the son had not been living, the same old servant would have been sole
residuary legatee.

Mrs Veneering has just succeeded in waking Lady Tippins from a snore, by
dexterously shunting a train of plates and dishes at her knuckles across
the table; when everybody but Mortimer himself becomes aware that the
Analytical Chemist is, in a ghostly manner, offering him a folded paper.
Curiosity detains Mrs Veneering a few moments.

Mortimer, in spite of all the arts of the chemist, placidly refreshes
himself with a glass of Madeira, and remains unconscious of the Document
which engrosses the general attention, until Lady Tippins (who has a
habit of waking totally insensible), having remembered where she is, and
recovered a perception of surrounding objects, says: ‘Falser man than
Don Juan; why don’t you take the note from the commendatore?’ Upon
which, the chemist advances it under the nose of Mortimer, who looks
round at him, and says:

‘What’s this?’

Analytical Chemist bends and whispers.

‘WHO?’ Says Mortimer.

Analytical Chemist again bends and whispers.

Mortimer stares at him, and unfolds the paper. Reads it, reads it twice,
turns it over to look at the blank outside, reads it a third time.

‘This arrives in an extraordinarily opportune manner,’ says Mortimer
then, looking with an altered face round the table: ‘this is the
conclusion of the story of the identical man.’

‘Already married?’ one guesses.

‘Declines to marry?’ another guesses.

‘Codicil among the dust?’ another guesses.

‘Why, no,’ says Mortimer; ‘remarkable thing, you are all wrong. The
story is completer and rather more exciting than I supposed. Man’s
drowned!’


