% REV01 Wed 23 Jun 2021 06:29:52 WIB
% START Tue 04 May 2021 13:55:16 WIB

\chapter{A SUCCESSOR}

Some of the Reverend Frank Milvey’s brethren had found themselves
exceedingly uncomfortable in their minds, because they were required to
bury the dead too hopefully. But, the Reverend Frank, inclining to the
belief that they were required to do one or two other things (say out of
nine-and-thirty) calculated to trouble their consciences rather more if
they would think as much about them, held his peace.

Indeed, the Reverend Frank Milvey was a forbearing man, who noticed many
sad warps and blights in the vineyard wherein he worked, and did not
profess that they made him savagely wise. He only learned that the more
he himself knew, in his little limited human way, the better he could
distantly imagine what Omniscience might know.

Wherefore, if the Reverend Frank had had to read the words that troubled
some of his brethren, and profitably touched innumerable hearts, in
a worse case than Johnny’s, he would have done so out of the pity and
humility of his soul. Reading them over Johnny, he thought of his own
six children, but not of his poverty, and read them with dimmed eyes.
And very seriously did he and his bright little wife, who had been
listening, look down into the small grave and walk home arm-in-arm.

There was grief in the aristocratic house, and there was joy in the
Bower. Mr Wegg argued, if an orphan were wanted, was he not an orphan
himself; and could a better be desired? And why go beating about
Brentford bushes, seeking orphans forsooth who had established no claims
upon you and made no sacrifices for you, when here was an orphan ready
to your hand who had given up in your cause, Miss Elizabeth, Master
George, Aunt Jane, and Uncle Parker?

Mr Wegg chuckled, consequently, when he heard the tidings. Nay, it was
afterwards affirmed by a witness who shall at present be nameless,
that in the seclusion of the Bower he poked out his wooden leg, in the
stage-ballet manner, and executed a taunting or triumphant pirouette on
the genuine leg remaining to him.

John Rokesmith’s manner towards Mrs Boffin at this time, was more the
manner of a young man towards a mother, than that of a Secretary towards
his employer’s wife. It had always been marked by a subdued affectionate
deference that seemed to have sprung up on the very day of his
engagement; whatever was odd in her dress or her ways had seemed to have
no oddity for him; he had sometimes borne a quietly-amused face in her
company, but still it had seemed as if the pleasure her genial temper
and radiant nature yielded him, could have been quite as naturally
expressed in a tear as in a smile. The completeness of his sympathy with
her fancy for having a little John Harmon to protect and rear, he
had shown in every act and word, and now that the kind fancy was
disappointed, he treated it with a manly tenderness and respect for
which she could hardly thank him enough.

‘But I do thank you, Mr Rokesmith,’ said Mrs Boffin, ‘and I thank you
most kindly. You love children.’

‘I hope everybody does.’

‘They ought,’ said Mrs Boffin; ‘but we don’t all of us do what we ought,
do us?’

John Rokesmith replied, ‘Some among us supply the short-comings of the
rest. You have loved children well, Mr Boffin has told me.’

‘Not a bit better than he has, but that’s his way; he puts all the good
upon me. You speak rather sadly, Mr Rokesmith.’

‘Do I?’

‘It sounds to me so. Were you one of many children?’ He shook his head.

‘An only child?’

‘No there was another. Dead long ago.’

‘Father or mother alive?’

‘Dead.’--

‘And the rest of your relations?’

‘Dead--if I ever had any living. I never heard of any.’

At this point of the dialogue Bella came in with a light step. She
paused at the door a moment, hesitating whether to remain or retire;
perplexed by finding that she was not observed.

‘Now, don’t mind an old lady’s talk,’ said Mrs Boffin, ‘but tell me. Are
you quite sure, Mr Rokesmith, that you have never had a disappointment
in love?’

‘Quite sure. Why do you ask me?’

‘Why, for this reason. Sometimes you have a kind of kept-down manner
with you, which is not like your age. You can’t be thirty?’

‘I am not yet thirty.’

Deeming it high time to make her presence known, Bella coughed here to
attract attention, begged pardon, and said she would go, fearing that
she interrupted some matter of business.

‘No, don’t go,’ rejoined Mrs Boffin, ‘because we are coming to business,
instead of having begun it, and you belong to it as much now, my dear
Bella, as I do. But I want my Noddy to consult with us. Would somebody
be so good as find my Noddy for me?’

Rokesmith departed on that errand, and presently returned accompanied by
Mr Boffin at his jog-trot. Bella felt a little vague trepidation as to
the subject-matter of this same consultation, until Mrs Boffin announced
it.

‘Now, you come and sit by me, my dear,’ said that worthy soul, taking
her comfortable place on a large ottoman in the centre of the room,
and drawing her arm through Bella’s; ‘and Noddy, you sit here, and Mr
Rokesmith you sit there. Now, you see, what I want to talk about, is
this. Mr and Mrs Milvey have sent me the kindest note possible (which
Mr Rokesmith just now read to me out aloud, for I ain’t good at
handwritings), offering to find me another little child to name and
educate and bring up. Well. This has set me thinking.’

[‘And she is a steam-ingein at it,’ murmured Mr Boffin, in an admiring
parenthesis, ‘when she once begins. It mayn’t be so easy to start her;
but once started, she’s a ingein.’)

‘--This has set me thinking, I say,’ repeated Mrs Boffin, cordially
beaming under the influence of her husband’s compliment, ‘and I have
thought two things. First of all, that I have grown timid of reviving
John Harmon’s name. It’s an unfortunate name, and I fancy I should
reproach myself if I gave it to another dear child, and it proved again
unlucky.’

‘Now, whether,’ said Mr Boffin, gravely propounding a case for his
Secretary’s opinion; ‘whether one might call that a superstition?’

‘It is a matter of feeling with Mrs Boffin,’ said Rokesmith, gently.
‘The name has always been unfortunate. It has now this new unfortunate
association connected with it. The name has died out. Why revive it?
Might I ask Miss Wilfer what she thinks?’

‘It has not been a fortunate name for me,’ said Bella, colouring--‘or
at least it was not, until it led to my being here--but that is not the
point in my thoughts. As we had given the name to the poor child, and as
the poor child took so lovingly to me, I think I should feel jealous of
calling another child by it. I think I should feel as if the name had
become endeared to me, and I had no right to use it so.’

‘And that’s your opinion?’ remarked Mr Boffin, observant of the
Secretary’s face and again addressing him.

‘I say again, it is a matter of feeling,’ returned the Secretary. ‘I
think Miss Wilfer’s feeling very womanly and pretty.’

‘Now, give us your opinion, Noddy,’ said Mrs Boffin.

‘My opinion, old lady,’ returned the Golden Dustman, ‘is your opinion.’

‘Then,’ said Mrs Boffin, ‘we agree not to revive John Harmon’s name, but
to let it rest in the grave. It is, as Mr Rokesmith says, a matter of
feeling, but Lor how many matters ARE matters of feeling! Well; and so
I come to the second thing I have thought of. You must know, Bella,
my dear, and Mr Rokesmith, that when I first named to my husband my
thoughts of adopting a little orphan boy in remembrance of John Harmon,
I further named to my husband that it was comforting to think that how
the poor boy would be benefited by John’s own money, and protected from
John’s own forlornness.’

‘Hear, hear!’ cried Mr Boffin. ‘So she did. Ancoar!’

‘No, not Ancoar, Noddy, my dear,’ returned Mrs Boffin, ‘because I am
going to say something else. I meant that, I am sure, as much as
I still mean it. But this little death has made me ask myself the
question, seriously, whether I wasn’t too bent upon pleasing myself.
Else why did I seek out so much for a pretty child, and a child quite to
my liking? Wanting to do good, why not do it for its own sake, and put
my tastes and likings by?’

‘Perhaps,’ said Bella; and perhaps she said it with some little
sensitiveness arising out of those old curious relations of hers towards
the murdered man; ‘perhaps, in reviving the name, you would not have
liked to give it to a less interesting child than the original. He
interested you very much.’

‘Well, my dear,’ returned Mrs Boffin, giving her a squeeze, ‘it’s kind
of you to find that reason out, and I hope it may have been so, and
indeed to a certain extent I believe it was so, but I am afraid not to
the whole extent. However, that don’t come in question now, because we
have done with the name.’

‘Laid it up as a remembrance,’ suggested Bella, musingly.

‘Much better said, my dear; laid it up as a remembrance. Well then; I
have been thinking if I take any orphan to provide for, let it not be
a pet and a plaything for me, but a creature to be helped for its own
sake.’

‘Not pretty then?’ said Bella.

‘No,’ returned Mrs Boffin, stoutly.

‘Nor prepossessing then?’ said Bella.

‘No,’ returned Mrs Boffin. ‘Not necessarily so. That’s as it may happen.
A well-disposed boy comes in my way who may be even a little wanting in
such advantages for getting on in life, but is honest and industrious
and requires a helping hand and deserves it. If I am very much in
earnest and quite determined to be unselfish, let me take care of HIM.’

Here the footman whose feelings had been hurt on the former occasion,
appeared, and crossing to Rokesmith apologetically announced the
objectionable Sloppy.

The four members of Council looked at one another, and paused. ‘Shall he
be brought here, ma’am?’ asked Rokesmith.

‘Yes,’ said Mrs Boffin. Whereupon the footman disappeared, reappeared
presenting Sloppy, and retired much disgusted.

The consideration of Mrs Boffin had clothed Mr Sloppy in a suit of
black, on which the tailor had received personal directions from
Rokesmith to expend the utmost cunning of his art, with a view to the
concealment of the cohering and sustaining buttons. But, so much
more powerful were the frailties of Sloppy’s form than the strongest
resources of tailoring science, that he now stood before the Council,
a perfect Argus in the way of buttons: shining and winking and gleaming
and twinkling out of a hundred of those eyes of bright metal, at the
dazzled spectators. The artistic taste of some unknown hatter had
furnished him with a hatband of wholesale capacity which was fluted
behind, from the crown of his hat to the brim, and terminated in a black
bunch, from which the imagination shrunk discomfited and the reason
revolted. Some special powers with which his legs were endowed, had
already hitched up his glossy trousers at the ankles, and bagged them at
the knees; while similar gifts in his arms had raised his coat-sleeves
from his wrists and accumulated them at his elbows. Thus set forth, with
the additional embellishments of a very little tail to his coat, and a
yawning gulf at his waistband, Sloppy stood confessed.

‘And how is Betty, my good fellow?’ Mrs Boffin asked him.

‘Thankee, mum,’ said Sloppy, ‘she do pretty nicely, and sending her
dooty and many thanks for the tea and all faviours and wishing to know
the family’s healths.’

‘Have you just come, Sloppy?’

‘Yes, mum.’

‘Then you have not had your dinner yet?’

‘No, mum. But I mean to it. For I ain’t forgotten your handsome orders
that I was never to go away without having had a good ‘un off of meat
and beer and pudding--no: there was four of ‘em, for I reckoned ‘em
up when I had ‘em; meat one, beer two, vegetables three, and which was
four?--Why, pudding, HE was four!’ Here Sloppy threw his head back,
opened his mouth wide, and laughed rapturously.

‘How are the two poor little Minders?’ asked Mrs Boffin.

‘Striking right out, mum, and coming round beautiful.’

Mrs Boffin looked on the other three members of Council, and then said,
beckoning with her finger:

‘Sloppy.’

‘Yes, mum.’

‘Come forward, Sloppy. Should you like to dine here every day?’

‘Off of all four on ‘em, mum? O mum!’ Sloppy’s feelings obliged him to
squeeze his hat, and contract one leg at the knee.

‘Yes. And should you like to be always taken care of here, if you were
industrious and deserving?’

‘Oh, mum!--But there’s Mrs Higden,’ said Sloppy, checking himself in his
raptures, drawing back, and shaking his head with very serious meaning.
‘There’s Mrs Higden. Mrs Higden goes before all. None can ever be better
friends to me than Mrs Higden’s been. And she must be turned for, must
Mrs Higden. Where would Mrs Higden be if she warn’t turned for!’ At the
mere thought of Mrs Higden in this inconceivable affliction, Mr Sloppy’s
countenance became pale, and manifested the most distressful emotions.

‘You are as right as right can be, Sloppy,’ said Mrs Boffin ‘and far be
it from me to tell you otherwise. It shall be seen to. If Betty Higden
can be turned for all the same, you shall come here and be taken care of
for life, and be made able to keep her in other ways than the turning.’

‘Even as to that, mum,’ answered the ecstatic Sloppy, ‘the turning might
be done in the night, don’t you see? I could be here in the day, and
turn in the night. I don’t want no sleep, I don’t. Or even if I any ways
should want a wink or two,’ added Sloppy, after a moment’s apologetic
reflection, ‘I could take ‘em turning. I’ve took ‘em turning many a
time, and enjoyed ‘em wonderful!’

On the grateful impulse of the moment, Mr Sloppy kissed Mrs Boffin’s
hand, and then detaching himself from that good creature that he might
have room enough for his feelings, threw back his head, opened his mouth
wide, and uttered a dismal howl. It was creditable to his tenderness of
heart, but suggested that he might on occasion give some offence to the
neighbours: the rather, as the footman looked in, and begged pardon,
finding he was not wanted, but excused himself; on the ground ‘that he
thought it was Cats.’



