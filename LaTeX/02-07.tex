% REV01 Tue 22 Jun 2021 19:25:58 WIB
% START Tue 04 May 2021 13:55:16 WIB

\chapter{IN WHICH A FRIENDLY MOVE IS ORIGINATED}

The arrangement between Mr Boffin and his literary man, Mr Silas Wegg,
so far altered with the altered habits of Mr Boffin’s life, as that
the Roman Empire usually declined in the morning and in the eminently
aristocratic family mansion, rather than in the evening, as of yore,
and in Boffin’s Bower. There were occasions, however, when Mr Boffin,
seeking a brief refuge from the blandishments of fashion, would present
himself at the Bower after dark, to anticipate the next sallying
forth of Wegg, and would there, on the old settle, pursue the downward
fortunes of those enervated and corrupted masters of the world who were
by this time on their last legs. If Wegg had been worse paid for his
office, or better qualified to discharge it, he would have considered
these visits complimentary and agreeable; but, holding the position of
a handsomely-remunerated humbug, he resented them. This was quite
according to rule, for the incompetent servant, by whomsoever employed,
is always against his employer. Even those born governors, noble and
right honourable creatures, who have been the most imbecile in high
places, have uniformly shown themselves the most opposed (sometimes in
belying distrust, sometimes in vapid insolence) to THEIR employer. What
is in such wise true of the public master and servant, is equally true
of the private master and servant all the world over.

When Mr Silas Wegg did at last obtain free access to ‘Our House’, as he
had been wont to call the mansion outside which he had sat shelterless
so long, and when he did at last find it in all particulars as different
from his mental plans of it as according to the nature of things it
well could be, that far-seeing and far-reaching character, by way of
asserting himself and making out a case for compensation, affected to
fall into a melancholy strain of musing over the mournful past; as if
the house and he had had a fall in life together.

‘And this, sir,’ Silas would say to his patron, sadly nodding his head
and musing, ‘was once Our House! This, sir, is the building from which I
have so often seen those great creatures, Miss Elizabeth, Master
George, Aunt Jane, and Uncle Parker’--whose very names were of his own
inventing--‘pass and repass! And has it come to this, indeed! Ah dear
me, dear me!’

So tender were his lamentations, that the kindly Mr Boffin was quite
sorry for him, and almost felt mistrustful that in buying the house he
had done him an irreparable injury.

Two or three diplomatic interviews, the result of great subtlety on Mr
Wegg’s part, but assuming the mask of careless yielding to a fortuitous
combination of circumstances impelling him towards Clerkenwell, had
enabled him to complete his bargain with Mr Venus.

‘Bring me round to the Bower,’ said Silas, when the bargain was closed,
‘next Saturday evening, and if a sociable glass of old Jamaikey warm
should meet your views, I am not the man to begrudge it.’

‘You are aware of my being poor company, sir,’ replied Mr Venus, ‘but be
it so.’

It being so, here is Saturday evening come, and here is Mr Venus come,
and ringing at the Bower-gate.

Mr Wegg opens the gate, descries a sort of brown paper truncheon under
Mr Venus’s arm, and remarks, in a dry tone: ‘Oh! I thought perhaps you
might have come in a cab.’

‘No, Mr Wegg,’ replies Venus. ‘I am not above a parcel.’

‘Above a parcel! No!’ says Wegg, with some dissatisfaction. But does not
openly growl, ‘a certain sort of parcel might be above you.’

‘Here is your purchase, Mr Wegg,’ says Venus, politely handing it over,
‘and I am glad to restore it to the source from whence it--flowed.’

‘Thankee,’ says Wegg. ‘Now this affair is concluded, I may mention to
you in a friendly way that I’ve my doubts whether, if I had consulted a
lawyer, you could have kept this article back from me. I only throw it
out as a legal point.’

‘Do you think so, Mr Wegg? I bought you in open contract.’

‘You can’t buy human flesh and blood in this country, sir; not alive,
you can’t,’ says Wegg, shaking his head. ‘Then query, bone?’

‘As a legal point?’ asks Venus.

‘As a legal point.’

‘I am not competent to speak upon that, Mr Wegg,’ says Venus, reddening
and growing something louder; ‘but upon a point of fact I think myself
competent to speak; and as a point of fact I would have seen you--will
you allow me to say, further?’

‘I wouldn’t say more than further, if I was you,’ Mr Wegg suggests,
pacifically.

--‘Before I’d have given that packet into your hand without being paid
my price for it. I don’t pretend to know how the point of law may stand,
but I’m thoroughly confident upon the point of fact.’

As Mr Venus is irritable (no doubt owing to his disappointment in love),
and as it is not the cue of Mr Wegg to have him out of temper, the
latter gentleman soothingly remarks, ‘I only put it as a little case; I
only put it ha’porthetically.’

‘Then I’d rather, Mr Wegg, you put it another time, penn’orth-etically,’
is Mr Venus’s retort, ‘for I tell you candidly I don’t like your little
cases.’

Arrived by this time in Mr Wegg’s sitting-room, made bright on the
chilly evening by gaslight and fire, Mr Venus softens and compliments
him on his abode; profiting by the occasion to remind Wegg that he
(Venus) told him he had got into a good thing.

‘Tolerable,’ Wegg rejoins. ‘But bear in mind, Mr Venus, that there’s
no gold without its alloy. Mix for yourself and take a seat in the
chimbley-corner. Will you perform upon a pipe, sir?’

‘I am but an indifferent performer, sir,’ returns the other; ‘but I’ll
accompany you with a whiff or two at intervals.’

So, Mr Venus mixes, and Wegg mixes; and Mr Venus lights and puffs, and
Wegg lights and puffs.

‘And there’s alloy even in this metal of yours, Mr Wegg, you was
remarking?’

‘Mystery,’ returns Wegg. ‘I don’t like it, Mr Venus. I don’t like to
have the life knocked out of former inhabitants of this house, in the
gloomy dark, and not know who did it.’

‘Might you have any suspicions, Mr Wegg?’

‘No,’ returns that gentleman. ‘I know who profits by it. But I’ve no
suspicions.’

Having said which, Mr Wegg smokes and looks at the fire with a most
determined expression of Charity; as if he had caught that cardinal
virtue by the skirts as she felt it her painful duty to depart from him,
and held her by main force.

‘Similarly,’ resumes Wegg, ‘I have observations as I can offer upon
certain points and parties; but I make no objections, Mr Venus. Here
is an immense fortune drops from the clouds upon a person that shall be
nameless. Here is a weekly allowance, with a certain weight of coals,
drops from the clouds upon me. Which of us is the better man? Not the
person that shall be nameless. That’s an observation of mine, but I
don’t make it an objection. I take my allowance and my certain weight of
coals. He takes his fortune. That’s the way it works.’

‘It would be a good thing for me, if I could see things in the calm
light you do, Mr Wegg.’

‘Again look here,’ pursues Silas, with an oratorical flourish of his
pipe and his wooden leg: the latter having an undignified tendency
to tilt him back in his chair; ‘here’s another observation, Mr Venus,
unaccompanied with an objection. Him that shall be nameless is liable to
be talked over. He gets talked over. Him that shall be nameless, having
me at his right hand, naturally looking to be promoted higher, and you
may perhaps say meriting to be promoted higher--’

(Mr Venus murmurs that he does say so.)

‘--Him that shall be nameless, under such circumstances passes me by,
and puts a talking-over stranger above my head. Which of us two is the
better man? Which of us two can repeat most poetry? Which of us two has,
in the service of him that shall be nameless, tackled the Romans, both
civil and military, till he has got as husky as if he’d been weaned and
ever since brought up on sawdust? Not the talking-over stranger. Yet the
house is as free to him as if it was his, and he has his room, and is
put upon a footing, and draws about a thousand a year. I am banished to
the Bower, to be found in it like a piece of furniture whenever wanted.
Merit, therefore, don’t win. That’s the way it works. I observe it,
because I can’t help observing it, being accustomed to take a powerful
sight of notice; but I don’t object. Ever here before, Mr Venus?’

‘Not inside the gate, Mr Wegg.’

‘You’ve been as far as the gate then, Mr Venus?’

‘Yes, Mr Wegg, and peeped in from curiosity.’

‘Did you see anything?’

‘Nothing but the dust-yard.’

Mr Wegg rolls his eyes all round the room, in that ever unsatisfied
quest of his, and then rolls his eyes all round Mr Venus; as if
suspicious of his having something about him to be found out.

‘And yet, sir,’ he pursues, ‘being acquainted with old Mr Harmon, one
would have thought it might have been polite in you, too, to give him a
call. And you’re naturally of a polite disposition, you are.’ This last
clause as a softening compliment to Mr Venus.

‘It is true, sir,’ replies Venus, winking his weak eyes, and running
his fingers through his dusty shock of hair, ‘that I was so, before a
certain observation soured me. You understand to what I allude, Mr Wegg?
To a certain written statement respecting not wishing to be regarded in
a certain light. Since that, all is fled, save gall.’

‘Not all,’ says Mr Wegg, in a tone of sentimental condolence.

‘Yes, sir,’ returns Venus, ‘all! The world may deem it harsh, but I’d
quite as soon pitch into my best friend as not. Indeed, I’d sooner!’

Involuntarily making a pass with his wooden leg to guard himself as Mr
Venus springs up in the emphasis of this unsociable declaration, Mr Wegg
tilts over on his back, chair and all, and is rescued by that harmless
misanthrope, in a disjointed state and ruefully rubbing his head.

‘Why, you lost your balance, Mr Wegg,’ says Venus, handing him his pipe.

‘And about time to do it,’ grumbles Silas, ‘when a man’s visitors,
without a word of notice, conduct themselves with the sudden wiciousness
of Jacks-in-boxes! Don’t come flying out of your chair like that, Mr
Venus!’

‘I ask your pardon, Mr Wegg. I am so soured.’

‘Yes, but hang it,’ says Wegg argumentatively, ‘a well-governed mind can
be soured sitting! And as to being regarded in lights, there’s bumpey
lights as well as bony. IN which,’ again rubbing his head, ‘I object to
regard myself.’

‘I’ll bear it in memory, sir.’

‘If you’ll be so good.’ Mr Wegg slowly subdues his ironical tone and his
lingering irritation, and resumes his pipe. ‘We were talking of old Mr
Harmon being a friend of yours.’

‘Not a friend, Mr Wegg. Only known to speak to, and to have a little
deal with now and then. A very inquisitive character, Mr Wegg, regarding
what was found in the dust. As inquisitive as secret.’

‘Ah! You found him secret?’ returns Wegg, with a greedy relish.

‘He had always the look of it, and the manner of it.’

‘Ah!’ with another roll of his eyes. ‘As to what was found in the dust
now. Did you ever hear him mention how he found it, my dear friend?
Living on the mysterious premises, one would like to know. For instance,
where he found things? Or, for instance, how he set about it? Whether
he began at the top of the mounds, or whether he began at the bottom.
Whether he prodded’; Mr Wegg’s pantomime is skilful and expressive here;
‘or whether he scooped? Should you say scooped, my dear Mr Venus; or
should you as a man--say prodded?’

‘I should say neither, Mr Wegg.’

‘As a fellow-man, Mr Venus--mix again--why neither?’

‘Because I suppose, sir, that what was found, was found in the sorting
and sifting. All the mounds are sorted and sifted?’

‘You shall see ‘em and pass your opinion. Mix again.’

On each occasion of his saying ‘mix again’, Mr Wegg, with a hop on
his wooden leg, hitches his chair a little nearer; more as if he were
proposing that himself and Mr Venus should mix again, than that they
should replenish their glasses.

‘Living (as I said before) on the mysterious premises,’ says Wegg when
the other has acted on his hospitable entreaty, ‘one likes to know.
Would you be inclined to say now--as a brother--that he ever hid things
in the dust, as well as found ‘em?’

‘Mr Wegg, on the whole I should say he might.’

Mr Wegg claps on his spectacles, and admiringly surveys Mr Venus from
head to foot.

‘As a mortal equally with myself, whose hand I take in mine for the
first time this day, having unaccountably overlooked that act so full of
boundless confidence binding a fellow-creetur TO a fellow creetur,’ says
Wegg, holding Mr Venus’s palm out, flat and ready for smiting, and now
smiting it; ‘as such--and no other--for I scorn all lowlier ties betwixt
myself and the man walking with his face erect that alone I call my
Twin--regarded and regarding in this trustful bond--what do you think he
might have hid?’

‘It is but a supposition, Mr Wegg.’

‘As a Being with his hand upon his heart,’ cries Wegg; and the
apostrophe is not the less impressive for the Being’s hand being
actually upon his rum and water; ‘put your supposition into language,
and bring it out, Mr Venus!’

‘He was the species of old gentleman, sir,’ slowly returns that
practical anatomist, after drinking, ‘that I should judge likely to
take such opportunities as this place offered, of stowing away money,
valuables, maybe papers.’

‘As one that was ever an ornament to human life,’ says Mr Wegg, again
holding out Mr Venus’s palm as if he were going to tell his fortune by
chiromancy, and holding his own up ready for smiting it when the time
should come; ‘as one that the poet might have had his eye on, in writing
the national naval words:

\begin{verbatim}
     Helm a-weather, now lay her close,
            Yard arm and yard arm she lies;
     Again, cried I, Mr Venus, give her t’other dose,
            Man shrouds and grapple, sir, or she flies!
\end{verbatim}

--that is to say, regarded in the light of true British Oak, for such
you are explain, Mr Venus, the expression “papers”!’

‘Seeing that the old gentleman was generally cutting off some near
relation, or blocking out some natural affection,’ Mr Venus rejoins, ‘he
most likely made a good many wills and codicils.’

The palm of Silas Wegg descends with a sounding smack upon the palm
of Venus, and Wegg lavishly exclaims, ‘Twin in opinion equally with
feeling! Mix a little more!’

Having now hitched his wooden leg and his chair close in front of Mr
Venus, Mr Wegg rapidly mixes for both, gives his visitor his glass,
touches its rim with the rim of his own, puts his own to his lips, puts
it down, and spreading his hands on his visitor’s knees thus addresses
him:

‘Mr Venus. It ain’t that I object to being passed over for a stranger,
though I regard the stranger as a more than doubtful customer. It ain’t
for the sake of making money, though money is ever welcome. It ain’t for
myself, though I am not so haughty as to be above doing myself a good
turn. It’s for the cause of the right.’

Mr Venus, passively winking his weak eyes both at once, demands: ‘What
is, Mr Wegg?’

‘The friendly move, sir, that I now propose. You see the move, sir?’

‘Till you have pointed it out, Mr Wegg, I can’t say whether I do or
not.’

‘If there IS anything to be found on these premises, let us find it
together. Let us make the friendly move of agreeing to look for it
together. Let us make the friendly move of agreeing to share the
profits of it equally betwixt us. In the cause of the right.’ Thus Silas
assuming a noble air.

‘Then,’ says Mr Venus, looking up, after meditating with his hair held
in his hands, as if he could only fix his attention by fixing his head;
‘if anything was to be unburied from under the dust, it would be kept a
secret by you and me? Would that be it, Mr Wegg?’

‘That would depend upon what it was, Mr Venus. Say it was money, or
plate, or jewellery, it would be as much ours as anybody else’s.’

Mr Venus rubs an eyebrow, interrogatively.

‘In the cause of the right it would. Because it would be unknowingly
sold with the mounds else, and the buyer would get what he was never
meant to have, and never bought. And what would that be, Mr Venus, but
the cause of the wrong?’

‘Say it was papers,’ Mr Venus propounds.

‘According to what they contained we should offer to dispose of ‘em to
the parties most interested,’ replies Wegg, promptly.

‘In the cause of the right, Mr Wegg?’

‘Always so, Mr Venus. If the parties should use them in the cause of the
wrong, that would be their act and deed. Mr Venus. I have an opinion of
you, sir, to which it is not easy to give mouth. Since I called upon you
that evening when you were, as I may say, floating your powerful mind in
tea, I have felt that you required to be roused with an object. In this
friendly move, sir, you will have a glorious object to rouse you.’

Mr Wegg then goes on to enlarge upon what throughout has been uppermost
in his crafty mind:--the qualifications of Mr Venus for such a search.
He expatiates on Mr Venus’s patient habits and delicate manipulation; on
his skill in piecing little things together; on his knowledge of various
tissues and textures; on the likelihood of small indications leading him
on to the discovery of great concealments. ‘While as to myself,’ says
Wegg, ‘I am not good at it. Whether I gave myself up to prodding,
or whether I gave myself up to scooping, I couldn’t do it with that
delicate touch so as not to show that I was disturbing the mounds.
Quite different with YOU, going to work (as YOU would) in the light of
a fellow-man, holily pledged in a friendly move to his brother man.’ Mr
Wegg next modestly remarks on the want of adaptation in a wooden leg
to ladders and such like airy perches, and also hints at an inherent
tendency in that timber fiction, when called into action for the
purposes of a promenade on an ashey slope, to stick itself into the
yielding foothold, and peg its owner to one spot. Then, leaving this
part of the subject, he remarks on the special phenomenon that before
his installation in the Bower, it was from Mr Venus that he first heard
of the legend of hidden wealth in the Mounds: ‘which’, he observes with
a vaguely pious air, ‘was surely never meant for nothing.’ Lastly,
he returns to the cause of the right, gloomily foreshadowing the
possibility of something being unearthed to criminate Mr Boffin (of whom
he once more candidly admits it cannot be denied that he profits by a
murder), and anticipating his denunciation by the friendly movers to
avenging justice. And this, Mr Wegg expressly points out, not at all for
the sake of the reward--though it would be a want of principle not to
take it.

To all this, Mr Venus, with his shock of dusty hair cocked after the
manner of a terrier’s ears, attends profoundly. When Mr Wegg, having
finished, opens his arms wide, as if to show Mr Venus how bare his
breast is, and then folds them pending a reply, Mr Venus winks at him
with both eyes some little time before speaking.

‘I see you have tried it by yourself, Mr Wegg,’ he says when he does
speak. ‘You have found out the difficulties by experience.’

‘No, it can hardly be said that I have tried it,’ replies Wegg, a little
dashed by the hint. ‘I have just skimmed it. Skimmed it.’

‘And found nothing besides the difficulties?’

Wegg shakes his head.

‘I scarcely know what to say to this, Mr Wegg,’ observes Venus, after
ruminating for a while.

‘Say yes,’ Wegg naturally urges.

‘If I wasn’t soured, my answer would be no. But being soured, Mr Wegg,
and driven to reckless madness and desperation, I suppose it’s Yes.’

Wegg joyfully reproduces the two glasses, repeats the ceremony of
clinking their rims, and inwardly drinks with great heartiness to the
health and success in life of the young lady who has reduced Mr Venus to
his present convenient state of mind.

The articles of the friendly move are then severally recited and agreed
upon. They are but secrecy, fidelity, and perseverance. The Bower to
be always free of access to Mr Venus for his researches, and every
precaution to be taken against their attracting observation in the
neighbourhood.

‘There’s a footstep!’ exclaims Venus.

‘Where?’ cries Wegg, starting.

‘Outside. St!’

They are in the act of ratifying the treaty of friendly move, by shaking
hands upon it. They softly break off, light their pipes which have gone
out, and lean back in their chairs. No doubt, a footstep. It approaches
the window, and a hand taps at the glass. ‘Come in!’ calls Wegg; meaning
come round by the door. But the heavy old-fashioned sash is slowly
raised, and a head slowly looks in out of the dark background of night.

‘Pray is Mr Silas Wegg here? Oh! I see him!’

The friendly movers might not have been quite at their ease, even
though the visitor had entered in the usual manner. But, leaning on the
breast-high window, and staring in out of the darkness, they find the
visitor extremely embarrassing. Especially Mr Venus: who removes his
pipe, draws back his head, and stares at the starer, as if it were his
own Hindoo baby come to fetch him home.

‘Good evening, Mr Wegg. The yard gate-lock should be looked to, if you
please; it don’t catch.’

‘Is it Mr Rokesmith?’ falters Wegg.

‘It is Mr Rokesmith. Don’t let me disturb you. I am not coming in. I
have only a message for you, which I undertook to deliver on my way home
to my lodgings. I was in two minds about coming beyond the gate without
ringing: not knowing but you might have a dog about.’

‘I wish I had,’ mutters Wegg, with his back turned as he rose from his
chair. ‘St! Hush! The talking-over stranger, Mr Venus.’

‘Is that any one I know?’ inquires the staring Secretary.

‘No, Mr Rokesmith. Friend of mine. Passing the evening with me.’

‘Oh! I beg his pardon. Mr Boffin wishes you to know that he does not
expect you to stay at home any evening, on the chance of his coming. It
has occurred to him that he may, without intending it, have been a tie
upon you. In future, if he should come without notice, he will take his
chance of finding you, and it will be all the same to him if he does
not. I undertook to tell you on my way. That’s all.’

With that, and ‘Good night,’ the Secretary lowers the window, and
disappears. They listen, and hear his footsteps go back to the gate, and
hear the gate close after him.

‘And for that individual, Mr Venus,’ remarks Wegg, when he is fully
gone, ‘I have been passed over! Let me ask you what you think of him?’

Apparently, Mr Venus does not know what to think of him, for he makes
sundry efforts to reply, without delivering himself of any other
articulate utterance than that he has ‘a singular look’.

‘A double look, you mean, sir,’ rejoins Wegg, playing bitterly upon the
word. ‘That’s HIS look. Any amount of singular look for me, but not a
double look! That’s an under-handed mind, sir.’

‘Do you say there’s something against him?’ Venus asks.

‘Something against him?’ repeats Wegg. ‘Something? What would the relief
be to my feelings--as a fellow-man--if I wasn’t the slave of truth, and
didn’t feel myself compelled to answer, Everything!’

See into what wonderful maudlin refuges, featherless ostriches plunge
their heads! It is such unspeakable moral compensation to Wegg, to be
overcome by the consideration that Mr Rokesmith has an underhanded mind!

‘On this starlight night, Mr Venus,’ he remarks, when he is showing that
friendly mover out across the yard, and both are something the worse
for mixing again and again: ‘on this starlight night to think that
talking-over strangers, and underhanded minds, can go walking home under
the sky, as if they was all square!’

‘The spectacle of those orbs,’ says Mr Venus, gazing upward with his hat
tumbling off; ‘brings heavy on me her crushing words that she did not
wish to regard herself nor yet to be regarded in that--’

‘I know! I know! You needn’t repeat ‘em,’ says Wegg, pressing his hand.
‘But think how those stars steady me in the cause of the right against
some that shall be nameless. It isn’t that I bear malice. But see how
they glisten with old remembrances! Old remembrances of what, sir?’

Mr Venus begins drearily replying, ‘Of her words, in her own
handwriting, that she does not wish to regard herself, nor yet--’ when
Silas cuts him short with dignity.

‘No, sir! Remembrances of Our House, of Master George, of Aunt Jane, of
Uncle Parker, all laid waste! All offered up sacrifices to the minion of
fortune and the worm of the hour!’



