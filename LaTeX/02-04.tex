% REV01 Tue 22 Jun 2021 17:39:52 WIB
% START Tue 04 May 2021 13:55:16 WIB

\chapter{CUPID PROMPTED}

To use the cold language of the world, Mrs Alfred Lammle rapidly
improved the acquaintance of Miss Podsnap. To use the warm language of
Mrs Lammle, she and her sweet Georgiana soon became one: in heart, in
mind, in sentiment, in soul.

Whenever Georgiana could escape from the thraldom of Podsnappery; could
throw off the bedclothes of the custard-coloured phaeton, and get up;
could shrink out of the range of her mother’s rocking, and (so to speak)
rescue her poor little frosty toes from being rocked over; she repaired
to her friend, Mrs Alfred Lammle. Mrs Podsnap by no means objected. As
a consciously ‘splendid woman,’ accustomed to overhear herself so
denominated by elderly osteologists pursuing their studies in dinner
society, Mrs Podsnap could dispense with her daughter. Mr Podsnap, for
his part, on being informed where Georgiana was, swelled with patronage
of the Lammles. That they, when unable to lay hold of him, should
respectfully grasp at the hem of his mantle; that they, when they could
not bask in the glory of him the sun, should take up with the pale
reflected light of the watery young moon his daughter; appeared quite
natural, becoming, and proper. It gave him a better opinion of the
discretion of the Lammles than he had heretofore held, as showing that
they appreciated the value of the connexion. So, Georgiana repairing
to her friend, Mr Podsnap went out to dinner, and to dinner, and yet to
dinner, arm in arm with Mrs Podsnap: settling his obstinate head in his
cravat and shirt-collar, much as if he were performing on the Pandean
pipes, in his own honour, the triumphal march, See the conquering
Podsnap comes, Sound the trumpets, beat the drums!

It was a trait in Mr Podsnap’s character (and in one form or other
it will be generally seen to pervade the depths and shallows of
Podsnappery), that he could not endure a hint of disparagement of any
friend or acquaintance of his. ‘How dare you?’ he would seem to say, in
such a case. ‘What do you mean? I have licensed this person. This person
has taken out MY certificate. Through this person you strike at me,
Podsnap the Great. And it is not that I particularly care for the
person’s dignity, but that I do most particularly care for Podsnap’s.’
Hence, if any one in his presence had presumed to doubt the
responsibility of the Lammles, he would have been mightily huffed. Not
that any one did, for Veneering, M.P., was always the authority for
their being very rich, and perhaps believed it. As indeed he might, if
he chose, for anything he knew of the matter.

Mr and Mrs Lammle’s house in Sackville Street, Piccadilly, was but
a temporary residence. It has done well enough, they informed their
friends, for Mr Lammle when a bachelor, but it would not do now. So,
they were always looking at palatial residences in the best situations,
and always very nearly taking or buying one, but never quite concluding
the bargain. Hereby they made for themselves a shining little reputation
apart. People said, on seeing a vacant palatial residence, ‘The very
thing for the Lammles!’ and wrote to the Lammles about it, and the
Lammles always went to look at it, but unfortunately it never exactly
answered. In short, they suffered so many disappointments, that they
began to think it would be necessary to build a palatial residence.
And hereby they made another shining reputation; many persons of their
acquaintance becoming by anticipation dissatisfied with their own
houses, and envious of the non-existent Lammle structure.

The handsome fittings and furnishings of the house in Sackville Street
were piled thick and high over the skeleton up-stairs, and if it ever
whispered from under its load of upholstery, ‘Here I am in the closet!’
it was to very few ears, and certainly never to Miss Podsnap’s. What
Miss Podsnap was particularly charmed with, next to the graces of
her friend, was the happiness of her friend’s married life. This was
frequently their theme of conversation.

‘I am sure,’ said Miss Podsnap, ‘Mr Lammle is like a lover. At least
I--I should think he was.’

‘Georgiana, darling!’ said Mrs Lammle, holding up a forefinger, ‘Take
care!’

‘Oh my goodness me!’ exclaimed Miss Podsnap, reddening. ‘What have I
said now?’

‘Alfred, you know,’ hinted Mrs Lammle, playfully shaking her head. ‘You
were never to say Mr Lammle any more, Georgiana.’

‘Oh! Alfred, then. I am glad it’s no worse. I was afraid I had said
something shocking. I am always saying something wrong to ma.’

‘To me, Georgiana dearest?’

‘No, not to you; you are not ma. I wish you were.’

Mrs Lammle bestowed a sweet and loving smile upon her friend, which Miss
Podsnap returned as she best could. They sat at lunch in Mrs Lammle’s
own boudoir.

‘And so, dearest Georgiana, Alfred is like your notion of a lover?’

‘I don’t say that, Sophronia,’ Georgiana replied, beginning to conceal
her elbows. ‘I haven’t any notion of a lover. The dreadful wretches that
ma brings up at places to torment me, are not lovers. I only mean that
Mr--’

‘Again, dearest Georgiana?’

‘That Alfred--’

‘Sounds much better, darling.’

‘--Loves you so. He always treats you with such delicate gallantry and
attention. Now, don’t he?’

‘Truly, my dear,’ said Mrs Lammle, with a rather singular expression
crossing her face. ‘I believe that he loves me, fully as much as I love
him.’

‘Oh, what happiness!’ exclaimed Miss Podsnap.

‘But do you know, my Georgiana,’ Mrs Lammle resumed presently, ‘that
there is something suspicious in your enthusiastic sympathy with
Alfred’s tenderness?’

‘Good gracious no, I hope not!’

‘Doesn’t it rather suggest,’ said Mrs Lammle archly, ‘that my
Georgiana’s little heart is--’

‘Oh don’t!’ Miss Podsnap blushingly besought her. ‘Please don’t! I
assure you, Sophronia, that I only praise Alfred, because he is your
husband and so fond of you.’

Sophronia’s glance was as if a rather new light broke in upon her. It
shaded off into a cool smile, as she said, with her eyes upon her lunch,
and her eyebrows raised:

‘You are quite wrong, my love, in your guess at my meaning. What I
insinuated was, that my Georgiana’s little heart was growing conscious
of a vacancy.’

‘No, no, no,’ said Georgiana. ‘I wouldn’t have anybody say anything to
me in that way for I don’t know how many thousand pounds.’

‘In what way, my Georgiana?’ inquired Mrs Lammle, still smiling coolly
with her eyes upon her lunch, and her eyebrows raised.

‘YOU know,’ returned poor little Miss Podsnap. ‘I think I should go out
of my mind, Sophronia, with vexation and shyness and detestation, if
anybody did. It’s enough for me to see how loving you and your husband
are. That’s a different thing. I couldn’t bear to have anything of that
sort going on with myself. I should beg and pray to--to have the person
taken away and trampled upon.’

Ah! here was Alfred. Having stolen in unobserved, he playfully leaned on
the back of Sophronia’s chair, and, as Miss Podsnap saw him, put one
of Sophronia’s wandering locks to his lips, and waved a kiss from it
towards Miss Podsnap.

‘What is this about husbands and detestations?’ inquired the captivating
Alfred.

‘Why, they say,’ returned his wife, ‘that listeners never hear any good
of themselves; though you--but pray how long have you been here, sir?’

‘This instant arrived, my own.’

‘Then I may go on--though if you had been here but a moment or two
sooner, you would have heard your praises sounded by Georgiana.’

‘Only, if they were to be called praises at all which I really don’t
think they were,’ explained Miss Podsnap in a flutter, ‘for being so
devoted to Sophronia.’

‘Sophronia!’ murmured Alfred. ‘My life!’ and kissed her hand. In return
for which she kissed his watch-chain.

‘But it was not I who was to be taken away and trampled upon, I hope?’
said Alfred, drawing a seat between them.

‘Ask Georgiana, my soul,’ replied his wife.

Alfred touchingly appealed to Georgiana.

‘Oh, it was nobody,’ replied Miss Podsnap. ‘It was nonsense.’

‘But if you are determined to know, Mr Inquisitive Pet, as I suppose you
are,’ said the happy and fond Sophronia, smiling, ‘it was any one who
should venture to aspire to Georgiana.’

‘Sophronia, my love,’ remonstrated Mr Lammle, becoming graver, ‘you are
not serious?’

‘Alfred, my love,’ returned his wife, ‘I dare say Georgiana was not, but
I am.’

‘Now this,’ said Mr Lammle, ‘shows the accidental combinations that
there are in things! Could you believe, my Ownest, that I came in here
with the name of an aspirant to our Georgiana on my lips?’

‘Of course I could believe, Alfred,’ said Mrs Lammle, ‘anything that YOU
told me.’

‘You dear one! And I anything that YOU told me.’

How delightful those interchanges, and the looks accompanying them! Now,
if the skeleton up-stairs had taken that opportunity, for instance, of
calling out ‘Here I am, suffocating in the closet!’

‘I give you my honour, my dear Sophronia--’

‘And I know what that is, love,’ said she.

‘You do, my darling--that I came into the room all but uttering young
Fledgeby’s name. Tell Georgiana, dearest, about young Fledgeby.’

‘Oh no, don’t! Please don’t!’ cried Miss Podsnap, putting her fingers in
her ears. ‘I’d rather not.’

Mrs Lammle laughed in her gayest manner, and, removing her Georgiana’s
unresisting hands, and playfully holding them in her own at arms’
length, sometimes near together and sometimes wide apart, went on:

‘You must know, you dearly beloved little goose, that once upon a
time there was a certain person called young Fledgeby. And this young
Fledgeby, who was of an excellent family and rich, was known to two
other certain persons, dearly attached to one another and called Mr and
Mrs Alfred Lammle. So this young Fledgeby, being one night at the play,
there sees with Mr and Mrs Alfred Lammle, a certain heroine called--’

‘No, don’t say Georgiana Podsnap!’ pleaded that young lady almost in
tears. ‘Please don’t. Oh do do do say somebody else! Not Georgiana
Podsnap. Oh don’t, don’t, don’t!’

‘No other,’ said Mrs Lammle, laughing airily, and, full of affectionate
blandishments, opening and closing Georgiana’s arms like a pair of
compasses, ‘than my little Georgiana Podsnap. So this young Fledgeby goes
to that Alfred Lammle and says--’

‘Oh ple-e-e-ease don’t!’ Georgiana, as if the supplication were being
squeezed out of her by powerful compression. ‘I so hate him for saying
it!’

‘For saying what, my dear?’ laughed Mrs Lammle.

‘Oh, I don’t know what he said,’ cried Georgiana wildly, ‘but I hate him
all the same for saying it.’

‘My dear,’ said Mrs Lammle, always laughing in her most captivating way,
‘the poor young fellow only says that he is stricken all of a heap.’

‘Oh, what shall I ever do!’ interposed Georgiana. ‘Oh my goodness what a
Fool he must be!’

‘--And implores to be asked to dinner, and to make a fourth at the play
another time. And so he dines to-morrow and goes to the Opera with
us. That’s all. Except, my dear Georgiana--and what will you think of
this!--that he is infinitely shyer than you, and far more afraid of you
than you ever were of any one in all your days!’

In perturbation of mind Miss Podsnap still fumed and plucked at her
hands a little, but could not help laughing at the notion of anybody’s
being afraid of her. With that advantage, Sophronia flattered her and
rallied her more successfully, and then the insinuating Alfred flattered
her and rallied her, and promised that at any moment when she might
require that service at his hands, he would take young Fledgeby out and
trample on him. Thus it remained amicably understood that young Fledgeby
was to come to admire, and that Georgiana was to come to be admired; and
Georgiana with the entirely new sensation in her breast of having that
prospect before her, and with many kisses from her dear Sophronia in
present possession, preceded six feet one of discontented footman (an
amount of the article that always came for her when she walked home) to
her father’s dwelling.

The happy pair being left together, Mrs Lammle said to her husband:

‘If I understand this girl, sir, your dangerous fascinations have
produced some effect upon her. I mention the conquest in good time
because I apprehend your scheme to be more important to you than your
vanity.’

There was a mirror on the wall before them, and her eyes just caught
him smirking in it. She gave the reflected image a look of the deepest
disdain, and the image received it in the glass. Next moment they
quietly eyed each other, as if they, the principals, had had no part in
that expressive transaction.

It may have been that Mrs Lammle tried in some manner to excuse her
conduct to herself by depreciating the poor little victim of whom she
spoke with acrimonious contempt. It may have been too that in this she
did not quite succeed, for it is very difficult to resist confidence,
and she knew she had Georgiana’s.

Nothing more was said between the happy pair. Perhaps conspirators
who have once established an understanding, may not be over-fond of
repeating the terms and objects of their conspiracy. Next day came; came
Georgiana; and came Fledgeby.

Georgiana had by this time seen a good deal of the house and its
frequenters. As there was a certain handsome room with a billiard table
in it--on the ground floor, eating out a backyard--which might have
been Mr Lammle’s office, or library, but was called by neither name, but
simply Mr Lammle’s room, so it would have been hard for stronger female
heads than Georgiana’s to determine whether its frequenters were men
of pleasure or men of business. Between the room and the men there were
strong points of general resemblance. Both were too gaudy, too slangey,
too odorous of cigars, and too much given to horseflesh; the latter
characteristic being exemplified in the room by its decorations, and in
the men by their conversation. High-stepping horses seemed necessary to
all Mr Lammle’s friends--as necessary as their transaction of business
together in a gipsy way at untimely hours of the morning and evening,
and in rushes and snatches. There were friends who seemed to be always
coming and going across the Channel, on errands about the Bourse, and
Greek and Spanish and India and Mexican and par and premium and discount
and three quarters and seven eighths. There were other friends who
seemed to be always lolling and lounging in and out of the City, on
questions of the Bourse, and Greek and Spanish and India and Mexican and
par and premium and discount and three quarters and seven eighths. They
were all feverish, boastful, and indefinably loose; and they all ate and
drank a great deal; and made bets in eating and drinking. They all spoke
of sums of money, and only mentioned the sums and left the money to
be understood; as ‘five and forty thousand Tom,’ or ‘Two hundred and
twenty-two on every individual share in the lot Joe.’ They seemed to
divide the world into two classes of people; people who were making
enormous fortunes, and people who were being enormously ruined. They
were always in a hurry, and yet seemed to have nothing tangible to do;
except a few of them (these, mostly asthmatic and thick-lipped) who were
for ever demonstrating to the rest, with gold pencil-cases which they
could hardly hold because of the big rings on their forefingers, how
money was to be made. Lastly, they all swore at their grooms, and the
grooms were not quite as respectful or complete as other men’s grooms;
seeming somehow to fall short of the groom point as their masters fell
short of the gentleman point.

Young Fledgeby was none of these. Young Fledgeby had a peachy cheek,
or a cheek compounded of the peach and the red red red wall on which
it grows, and was an awkward, sandy-haired, small-eyed youth, exceeding
slim (his enemies would have said lanky), and prone to self-examination
in the articles of whisker and moustache. While feeling for the whisker
that he anxiously expected, Fledgeby underwent remarkable fluctuations
of spirits, ranging along the whole scale from confidence to despair.
There were times when he started, as exclaiming ‘By Jupiter here it is
at last!’ There were other times when, being equally depressed, he would
be seen to shake his head, and give up hope. To see him at those periods
leaning on a chimneypiece, like as on an urn containing the ashes of his
ambition, with the cheek that would not sprout, upon the hand on which
that cheek had forced conviction, was a distressing sight.

Not so was Fledgeby seen on this occasion. Arrayed in superb raiment,
with his opera hat under his arm, he concluded his self-examination
hopefully, awaited the arrival of Miss Podsnap, and talked small-talk
with Mrs Lammle. In facetious homage to the smallness of his talk, and
the jerky nature of his manners, Fledgeby’s familiars had agreed to
confer upon him (behind his back) the honorary title of Fascination
Fledgeby.

‘Warm weather, Mrs Lammle,’ said Fascination Fledgeby. Mrs Lammle
thought it scarcely as warm as it had been yesterday. ‘Perhaps not,’
said Fascination Fledgeby, with great quickness of repartee; ‘but I
expect it will be devilish warm to-morrow.’

He threw off another little scintillation. ‘Been out to-day, Mrs
Lammle?’

Mrs Lammle answered, for a short drive.

‘Some people,’ said Fascination Fledgeby, ‘are accustomed to take long
drives; but it generally appears to me that if they make ‘em too long,
they overdo it.’

Being in such feather, he might have surpassed himself in his next
sally, had not Miss Podsnap been announced. Mrs Lammle flew to embrace
her darling little Georgy, and when the first transports were over,
presented Mr Fledgeby. Mr Lammle came on the scene last, for he was
always late, and so were the frequenters always late; all hands being
bound to be made late, by private information about the Bourse, and
Greek and Spanish and India and Mexican and par and premium and discount
and three quarters and seven eighths.

A handsome little dinner was served immediately, and Mr Lammle sat
sparkling at his end of the table, with his servant behind his chair,
and HIS ever-lingering doubts upon the subject of his wages behind
himself. Mr Lammle’s utmost powers of sparkling were in requisition
to-day, for Fascination Fledgeby and Georgiana not only struck each
other speechless, but struck each other into astonishing attitudes;
Georgiana, as she sat facing Fledgeby, making such efforts to conceal
her elbows as were totally incompatible with the use of a knife and
fork; and Fledgeby, as he sat facing Georgiana, avoiding her countenance
by every possible device, and betraying the discomposure of his mind in
feeling for his whiskers with his spoon, his wine glass, and his bread.

So, Mr and Mrs Alfred Lammle had to prompt, and this is how they
prompted.

‘Georgiana,’ said Mr Lammle, low and smiling, and sparkling all over,
like a harlequin; ‘you are not in your usual spirits. Why are you not in
your usual spirits, Georgiana?’

Georgiana faltered that she was much the same as she was in general; she
was not aware of being different.

‘Not aware of being different!’ retorted Mr Alfred Lammle. ‘You, my dear
Georgiana! Who are always so natural and unconstrained with us! Who are
such a relief from the crowd that are all alike! Who are the embodiment
of gentleness, simplicity, and reality!’

Miss Podsnap looked at the door, as if she entertained confused thoughts
of taking refuge from these compliments in flight.

‘Now, I will be judged,’ said Mr Lammle, raising his voice a little, ‘by
my friend Fledgeby.’

‘Oh DON’T!’ Miss Podsnap faintly ejaculated: when Mrs Lammle took the
prompt-book.

‘I beg your pardon, Alfred, my dear, but I cannot part with Mr Fledgeby
quite yet; you must wait for him a moment. Mr Fledgeby and I are engaged
in a personal discussion.’

Fledgeby must have conducted it on his side with immense art, for no
appearance of uttering one syllable had escaped him.

‘A personal discussion, Sophronia, my love? What discussion? Fledgeby, I
am jealous. What discussion, Fledgeby?’

‘Shall I tell him, Mr Fledgeby?’ asked Mrs Lammle.

Trying to look as if he knew anything about it, Fascination replied,
‘Yes, tell him.’

‘We were discussing then,’ said Mrs Lammle, ‘if you MUST know, Alfred,
whether Mr Fledgeby was in his usual flow of spirits.’

‘Why, that is the very point, Sophronia, that Georgiana and I were
discussing as to herself! What did Fledgeby say?’

‘Oh, a likely thing, sir, that I am going to tell you everything, and be
told nothing! What did Georgiana say?’

‘Georgiana said she was doing her usual justice to herself to-day, and I
said she was not.’

‘Precisely,’ exclaimed Mrs Lammle, ‘what I said to Mr Fledgeby.’ Still,
it wouldn’t do. They would not look at one another. No, not even
when the sparkling host proposed that the quartette should take an
appropriately sparkling glass of wine. Georgiana looked from her wine
glass at Mr Lammle and at Mrs Lammle; but mightn’t, couldn’t, shouldn’t,
wouldn’t, look at Mr Fledgeby. Fascination looked from his wine glass
at Mrs Lammle and at Mr Lammle; but mightn’t, couldn’t, shouldn’t,
wouldn’t, look at Georgiana.

More prompting was necessary. Cupid must be brought up to the mark. The
manager had put him down in the bill for the part, and he must play it.

‘Sophronia, my dear,’ said Mr Lammle, ‘I don’t like the colour of your
dress.’

‘I appeal,’ said Mrs Lammle, ‘to Mr Fledgeby.’

‘And I,’ said Mr Lammle, ‘to Georgiana.’

‘Georgy, my love,’ remarked Mrs Lammle aside to her dear girl, ‘I rely
upon you not to go over to the opposition. Now, Mr Fledgeby.’

Fascination wished to know if the colour were not called rose-colour?
Yes, said Mr Lammle; actually he knew everything; it was really
rose-colour. Fascination took rose-colour to mean the colour of roses.
(In this he was very warmly supported by Mr and Mrs Lammle.) Fascination
had heard the term Queen of Flowers applied to the Rose. Similarly, it
might be said that the dress was the Queen of Dresses. [‘Very happy,
Fledgeby!’ from Mr Lammle.) Notwithstanding, Fascination’s opinion
was that we all had our eyes--or at least a large majority of us--and
that--and--and his farther opinion was several ands, with nothing beyond
them.

‘Oh, Mr Fledgeby,’ said Mrs Lammle, ‘to desert me in that way! Oh, Mr
Fledgeby, to abandon my poor dear injured rose and declare for blue!’

‘Victory, victory!’ cried Mr Lammle; ‘your dress is condemned, my dear.’

‘But what,’ said Mrs Lammle, stealing her affectionate hand towards her
dear girl’s, ‘what does Georgy say?’

‘She says,’ replied Mr Lammle, interpreting for her, ‘that in her eyes
you look well in any colour, Sophronia, and that if she had expected to
be embarrassed by so pretty a compliment as she has received, she would
have worn another colour herself. Though I tell her, in reply, that it
would not have saved her, for whatever colour she had worn would have
been Fledgeby’s colour. But what does Fledgeby say?’

‘He says,’ replied Mrs Lammle, interpreting for him, and patting the
back of her dear girl’s hand, as if it were Fledgeby who was patting it,
‘that it was no compliment, but a little natural act of homage that
he couldn’t resist. And,’ expressing more feeling as if it were more
feeling on the part of Fledgeby, ‘he is right, he is right!’

Still, no not even now, would they look at one another. Seeming to gnash
his sparkling teeth, studs, eyes, and buttons, all at once, Mr Lammle
secretly bent a dark frown on the two, expressive of an intense desire
to bring them together by knocking their heads together.

‘Have you heard this opera of to-night, Fledgeby?’ he asked, stopping
very short, to prevent himself from running on into ‘confound you.’

‘Why no, not exactly,’ said Fledgeby. ‘In fact I don’t know a note of
it.’

‘Neither do you know it, Georgy?’ said Mrs Lammle. ‘N-no,’ replied
Georgiana, faintly, under the sympathetic coincidence.

‘Why, then,’ said Mrs Lammle, charmed by the discovery which flowed from
the premises, ‘you neither of you know it! How charming!’

Even the craven Fledgeby felt that the time was now come when he must
strike a blow. He struck it by saying, partly to Mrs Lammle and partly
to the circumambient air, ‘I consider myself very fortunate in being
reserved by--’

As he stopped dead, Mr Lammle, making that gingerous bush of his
whiskers to look out of, offered him the word ‘Destiny.’

‘No, I wasn’t going to say that,’ said Fledgeby. ‘I was going to say
Fate. I consider it very fortunate that Fate has written in the book
of--in the book which is its own property--that I should go to that
opera for the first time under the memorable circumstances of going with
Miss Podsnap.’

To which Georgiana replied, hooking her two little fingers in one
another, and addressing the tablecloth, ‘Thank you, but I generally go
with no one but you, Sophronia, and I like that very much.’

Content perforce with this success for the time, Mr Lammle let Miss
Podsnap out of the room, as if he were opening her cage door, and Mrs
Lammle followed. Coffee being presently served up stairs, he kept a
watch on Fledgeby until Miss Podsnap’s cup was empty, and then directed
him with his finger (as if that young gentleman were a slow Retriever)
to go and fetch it. This feat he performed, not only without failure,
but even with the original embellishment of informing Miss Podsnap that
green tea was considered bad for the nerves. Though there Miss Podsnap
unintentionally threw him out by faltering, ‘Oh, is it indeed? How does
it act?’ Which he was not prepared to elucidate.

The carriage announced, Mrs Lammle said; ‘Don’t mind me, Mr Fledgeby, my
skirts and cloak occupy both my hands, take Miss Podsnap.’ And he
took her, and Mrs Lammle went next, and Mr Lammle went last, savagely
following his little flock, like a drover.

But he was all sparkle and glitter in the box at the Opera, and there he
and his dear wife made a conversation between Fledgeby and Georgiana in
the following ingenious and skilful manner. They sat in this order:
Mrs Lammle, Fascination Fledgeby, Georgiana, Mr Lammle. Mrs Lammle made
leading remarks to Fledgeby, only requiring monosyllabic replies. Mr
Lammle did the like with Georgiana. At times Mrs Lammle would lean
forward to address Mr Lammle to this purpose.

‘Alfred, my dear, Mr Fledgeby very justly says, apropos of the last
scene, that true constancy would not require any such stimulant as the
stage deems necessary.’ To which Mr Lammle would reply, ‘Ay, Sophronia,
my love, but as Georgiana has observed to me, the lady had no sufficient
reason to know the state of the gentleman’s affections.’ To which Mrs
Lammle would rejoin, ‘Very true, Alfred; but Mr Fledgeby points
out,’ this. To which Alfred would demur: ‘Undoubtedly, Sophronia, but
Georgiana acutely remarks,’ that. Through this device the two young
people conversed at great length and committed themselves to a variety
of delicate sentiments, without having once opened their lips, save to
say yes or no, and even that not to one another.

Fledgeby took his leave of Miss Podsnap at the carriage door, and the
Lammles dropped her at her own home, and on the way Mrs Lammle archly
rallied her, in her fond and protecting manner, by saying at intervals,
‘Oh little Georgiana, little Georgiana!’ Which was not much; but the
tone added, ‘You have enslaved your Fledgeby.’

And thus the Lammles got home at last, and the lady sat down moody and
weary, looking at her dark lord engaged in a deed of violence with a
bottle of soda-water as though he were wringing the neck of some unlucky
creature and pouring its blood down his throat. As he wiped his dripping
whiskers in an ogreish way, he met her eyes, and pausing, said, with no
very gentle voice:

‘Well?’

‘Was such an absolute Booby necessary to the purpose?’

‘I know what I am doing. He is no such dolt as you suppose.’

‘A genius, perhaps?’

‘You sneer, perhaps; and you take a lofty air upon yourself perhaps!
But I tell you this:--when that young fellow’s interest is concerned,
he holds as tight as a horse-leech. When money is in question with that
young fellow, he is a match for the Devil.’

‘Is he a match for you?’

‘He is. Almost as good a one as you thought me for you. He has no
quality of youth in him, but such as you have seen to-day. Touch him
upon money, and you touch no booby then. He really is a dolt, I suppose,
in other things; but it answers his one purpose very well.’

‘Has she money in her own right in any case?’

‘Ay! she has money in her own right in any case. You have done so well
to-day, Sophronia, that I answer the question, though you know I object
to any such questions. You have done so well to-day, Sophronia, that you
must be tired. Get to bed.’


