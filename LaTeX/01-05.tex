% REV00 Sun 28 Mar 2021 11:03:13 WIB
% START Sun 28 Mar 2021 11:03:13 WIB

\chapter{BOFFIN’S BOWER}

Over against a London house, a corner house not far from Cavendish
Square, a man with a wooden leg had sat for some years, with his
remaining foot in a basket in cold weather, picking up a living on
this wise:--Every morning at eight o’clock, he stumped to the corner,
carrying a chair, a clothes-horse, a pair of trestles, a board, a
basket, and an umbrella, all strapped together. Separating these, the
board and trestles became a counter, the basket supplied the few small
lots of fruit and sweets that he offered for sale upon it and became a
foot-warmer, the unfolded clothes-horse displayed a choice collection of
halfpenny ballads and became a screen, and the stool planted within it
became his post for the rest of the day. All weathers saw the man at the
post. This is to be accepted in a double sense, for he contrived a
back to his wooden stool, by placing it against the lamp-post. When the
weather was wet, he put up his umbrella over his stock in trade, not
over himself; when the weather was dry, he furled that faded article,
tied it round with a piece of yarn, and laid it cross-wise under the
trestles: where it looked like an unwholesomely-forced lettuce that had
lost in colour and crispness what it had gained in size.

He had established his right to the corner, by imperceptible
prescription. He had never varied his ground an inch, but had in the
beginning diffidently taken the corner upon which the side of the house
gave. A howling corner in the winter time, a dusty corner in the summer
time, an undesirable corner at the best of times. Shelterless fragments
of straw and paper got up revolving storms there, when the main street
was at peace; and the water-cart, as if it were drunk or short-sighted,
came blundering and jolting round it, making it muddy when all else was
clean.

On the front of his sale-board hung a little placard, like a
kettle-holder, bearing the inscription in his own small text:

\begin{verbatim}
     Errands gone
     On with fi
     Delity By
     Ladies and Gentlemen
     I remain
     Your humble Servt:
     Silas Wegg
\end{verbatim}

He had not only settled it with himself in course of time, that he
was errand-goer by appointment to the house at the corner (though he
received such commissions not half a dozen times in a year, and then
only as some servant’s deputy), but also that he was one of the house’s
retainers and owed vassalage to it and was bound to leal and loyal
interest in it. For this reason, he always spoke of it as ‘Our House,’
and, though his knowledge of its affairs was mostly speculative and
all wrong, claimed to be in its confidence. On similar grounds he never
beheld an inmate at any one of its windows but he touched his hat. Yet,
he knew so little about the inmates that he gave them names of his own
invention: as ‘Miss Elizabeth’, ‘Master George’, ‘Aunt Jane’, ‘Uncle
Parker ‘--having no authority whatever for any such designations, but
particularly the last--to which, as a natural consequence, he stuck with
great obstinacy.

Over the house itself, he exercised the same imaginary power as over its
inhabitants and their affairs. He had never been in it, the length of
a piece of fat black water-pipe which trailed itself over the area-door
into a damp stone passage, and had rather the air of a leech on the
house that had ‘taken’ wonderfully; but this was no impediment to his
arranging it according to a plan of his own. It was a great dingy house
with a quantity of dim side window and blank back premises, and it
cost his mind a world of trouble so to lay it out as to account for
everything in its external appearance. But, this once done, was quite
satisfactory, and he rested persuaded, that he knew his way about the
house blindfold: from the barred garrets in the high roof, to the two
iron extinguishers before the main door--which seemed to request all
lively visitors to have the kindness to put themselves out, before
entering.

Assuredly, this stall of Silas Wegg’s was the hardest little stall of
all the sterile little stalls in London. It gave you the face-ache
to look at his apples, the stomach-ache to look at his oranges, the
tooth-ache to look at his nuts. Of the latter commodity he had always
a grim little heap, on which lay a little wooden measure which had
no discernible inside, and was considered to represent the penn’orth
appointed by Magna Charta. Whether from too much east wind or no--it was
an easterly corner--the stall, the stock, and the keeper, were all as
dry as the Desert. Wegg was a knotty man, and a close-grained, with a
face carved out of very hard material, that had just as much play
of expression as a watchman’s rattle. When he laughed, certain jerks
occurred in it, and the rattle sprung. Sooth to say, he was so wooden
a man that he seemed to have taken his wooden leg naturally, and rather
suggested to the fanciful observer, that he might be expected--if his
development received no untimely check--to be completely set up with a
pair of wooden legs in about six months.

Mr Wegg was an observant person, or, as he himself said, ‘took a
powerful sight of notice’. He saluted all his regular passers-by every
day, as he sat on his stool backed up by the lamp-post; and on the
adaptable character of these salutes he greatly plumed himself. Thus,
to the rector, he addressed a bow, compounded of lay deference, and
a slight touch of the shady preliminary meditation at church; to the
doctor, a confidential bow, as to a gentleman whose acquaintance with
his inside he begged respectfully to acknowledge; before the Quality he
delighted to abase himself; and for Uncle Parker, who was in the army
(at least, so he had settled it), he put his open hand to the side
of his hat, in a military manner which that angry-eyed buttoned-up
inflammatory-faced old gentleman appeared but imperfectly to appreciate.

The only article in which Silas dealt, that was not hard, was
gingerbread. On a certain day, some wretched infant having purchased the
damp gingerbread-horse (fearfully out of condition), and the adhesive
bird-cage, which had been exposed for the day’s sale, he had taken a tin
box from under his stool to produce a relay of those dreadful specimens,
and was going to look in at the lid, when he said to himself, pausing:
‘Oh! Here you are again!’

The words referred to a broad, round-shouldered, one-sided old fellow in
mourning, coming comically ambling towards the corner, dressed in a pea
over-coat, and carrying a large stick. He wore thick shoes, and thick
leather gaiters, and thick gloves like a hedger’s. Both as to his dress
and to himself, he was of an overlapping rhinoceros build, with folds
in his cheeks, and his forehead, and his eyelids, and his lips, and his
ears; but with bright, eager, childishly-inquiring, grey eyes, under his
ragged eyebrows, and broad-brimmed hat. A very odd-looking old fellow
altogether.

‘Here you are again,’ repeated Mr Wegg, musing. ‘And what are you now?
Are you in the Funns, or where are you? Have you lately come to settle
in this neighbourhood, or do you own to another neighbourhood? Are you
in independent circumstances, or is it wasting the motions of a bow on
you? Come! I’ll speculate! I’ll invest a bow in you.’

Which Mr Wegg, having replaced his tin box, accordingly did, as he rose
to bait his gingerbread-trap for some other devoted infant. The salute
was acknowledged with:

‘Morning, sir! Morning! Morning!’

(‘Calls me Sir!’ said Mr Wegg, to himself; ‘HE won’t answer. A bow
gone!’)

‘Morning, morning, morning!’

‘Appears to be rather a ‘arty old cock, too,’ said Mr Wegg, as before;
‘Good morning to YOU, sir.’

‘Do you remember me, then?’ asked his new acquaintance, stopping in
his amble, one-sided, before the stall, and speaking in a pounding way,
though with great good-humour.

‘I have noticed you go past our house, sir, several times in the course
of the last week or so.’

‘Our house,’ repeated the other. ‘Meaning--?’

‘Yes,’ said Mr Wegg, nodding, as the other pointed the clumsy forefinger
of his right glove at the corner house.

‘Oh! Now, what,’ pursued the old fellow, in an inquisitive manner,
carrying his knotted stick in his left arm as if it were a baby, ‘what
do they allow you now?’

‘It’s job work that I do for our house,’ returned Silas, drily, and with
reticence; ‘it’s not yet brought to an exact allowance.’

‘Oh! It’s not yet brought to an exact allowance? No! It’s not yet
brought to an exact allowance. Oh!--Morning, morning, morning!’

‘Appears to be rather a cracked old cock,’ thought Silas, qualifying his
former good opinion, as the other ambled off. But, in a moment he was
back again with the question:

‘How did you get your wooden leg?’

Mr Wegg replied, (tartly to this personal inquiry), ‘In an accident.’

‘Do you like it?’

‘Well! I haven’t got to keep it warm,’ Mr Wegg made answer, in a sort of
desperation occasioned by the singularity of the question.

‘He hasn’t,’ repeated the other to his knotted stick, as he gave it a
hug; ‘he hasn’t got--ha!--ha!--to keep it warm! Did you ever hear of the
name of Boffin?’

‘No,’ said Mr Wegg, who was growing restive under this examination. ‘I
never did hear of the name of Boffin.’

‘Do you like it?’

‘Why, no,’ retorted Mr Wegg, again approaching desperation; ‘I can’t say
I do.’

‘Why don’t you like it?’

‘I don’t know why I don’t,’ retorted Mr Wegg, approaching frenzy, ‘but I
don’t at all.’

‘Now, I’ll tell you something that’ll make you sorry for that,’ said the
stranger, smiling. ‘My name’s Boffin.’

‘I can’t help it!’ returned Mr Wegg. Implying in his manner the
offensive addition, ‘and if I could, I wouldn’t.’

‘But there’s another chance for you,’ said Mr Boffin, smiling still, ‘Do
you like the name of Nicodemus? Think it over. Nick, or Noddy.’

‘It is not, sir,’ Mr Wegg rejoined, as he sat down on his stool, with an
air of gentle resignation, combined with melancholy candour; ‘it is not
a name as I could wish any one that I had a respect for, to call ME
by; but there may be persons that would not view it with the same
objections.--I don’t know why,’ Mr Wegg added, anticipating another
question.

‘Noddy Boffin,’ said that gentleman. ‘Noddy. That’s my name. Noddy--or
Nick--Boffin. What’s your name?’

‘Silas Wegg.--I don’t,’ said Mr Wegg, bestirring himself to take the
same precaution as before, ‘I don’t know why Silas, and I don’t know why
Wegg.’

‘Now, Wegg,’ said Mr Boffin, hugging his stick closer, ‘I want to make a
sort of offer to you. Do you remember when you first see me?’

The wooden Wegg looked at him with a meditative eye, and also with a
softened air as descrying possibility of profit. ‘Let me think. I ain’t
quite sure, and yet I generally take a powerful sight of notice, too.
Was it on a Monday morning, when the butcher-boy had been to our house
for orders, and bought a ballad of me, which, being unacquainted with
the tune, I run it over to him?’

‘Right, Wegg, right! But he bought more than one.’

‘Yes, to be sure, sir; he bought several; and wishing to lay out his
money to the best, he took my opinion to guide his choice, and we went
over the collection together. To be sure we did. Here was him as it
might be, and here was myself as it might be, and there was you, Mr
Boffin, as you identically are, with your self-same stick under your
very same arm, and your very same back towards us. To--be--sure!’ added
Mr Wegg, looking a little round Mr Boffin, to take him in the rear,
and identify this last extraordinary coincidence, ‘your wery self-same
back!’

‘What do you think I was doing, Wegg?’

‘I should judge, sir, that you might be glancing your eye down the
street.’

‘No, Wegg. I was a listening.’

‘Was you, indeed?’ said Mr Wegg, dubiously.

‘Not in a dishonourable way, Wegg, because you was singing to the
butcher; and you wouldn’t sing secrets to a butcher in the street, you
know.’

‘It never happened that I did so yet, to the best of my remembrance,’
said Mr Wegg, cautiously. ‘But I might do it. A man can’t say what he
might wish to do some day or another.’ (This, not to release any little
advantage he might derive from Mr Boffin’s avowal.)

‘Well,’ repeated Boffin, ‘I was a listening to you and to him. And what
do you--you haven’t got another stool, have you? I’m rather thick in my
breath.’

‘I haven’t got another, but you’re welcome to this,’ said Wegg,
resigning it. ‘It’s a treat to me to stand.’

‘Lard!’ exclaimed Mr Boffin, in a tone of great enjoyment, as he settled
himself down, still nursing his stick like a baby, ‘it’s a pleasant
place, this! And then to be shut in on each side, with these ballads,
like so many book-leaf blinkers! Why, its delightful!’

‘If I am not mistaken, sir,’ Mr Wegg delicately hinted, resting a hand
on his stall, and bending over the discursive Boffin, ‘you alluded to
some offer or another that was in your mind?’

‘I’m coming to it! All right. I’m coming to it! I was going to say that
when I listened that morning, I listened with hadmiration amounting to
haw. I thought to myself, “Here’s a man with a wooden leg--a literary
man with--“’

‘N--not exactly so, sir,’ said Mr Wegg.

‘Why, you know every one of these songs by name and by tune, and if you
want to read or to sing any one on ‘em off straight, you’ve only to whip
on your spectacles and do it!’ cried Mr Boffin. ‘I see you at it!’

‘Well, sir,’ returned Mr Wegg, with a conscious inclination of the head;
‘we’ll say literary, then.’

‘“A literary man--WITH a wooden leg--and all Print is open to him!”
 That’s what I thought to myself, that morning,’ pursued Mr Boffin,
leaning forward to describe, uncramped by the clotheshorse, as large an
arc as his right arm could make; ‘“all Print is open to him!” And it is,
ain’t it?’

‘Why, truly, sir,’ Mr Wegg admitted, with modesty; ‘I believe you
couldn’t show me the piece of English print, that I wouldn’t be equal to
collaring and throwing.’

‘On the spot?’ said Mr Boffin.

‘On the spot.’

‘I know’d it! Then consider this. Here am I, a man without a wooden leg,
and yet all print is shut to me.’

‘Indeed, sir?’ Mr Wegg returned with increasing self-complacency.
‘Education neglected?’

‘Neg--lected!’ repeated Boffin, with emphasis. ‘That ain’t no word for
it. I don’t mean to say but what if you showed me a B, I could so far
give you change for it, as to answer Boffin.’

‘Come, come, sir,’ said Mr Wegg, throwing in a little encouragement,
‘that’s something, too.’

‘It’s something,’ answered Mr Boffin, ‘but I’ll take my oath it ain’t
much.’

‘Perhaps it’s not as much as could be wished by an inquiring mind, sir,’
Mr Wegg admitted.

‘Now, look here. I’m retired from business. Me and Mrs
Boffin--Henerietty Boffin--which her father’s name was Henery, and her
mother’s name was Hetty, and so you get it--we live on a compittance,
under the will of a diseased governor.’

‘Gentleman dead, sir?’

‘Man alive, don’t I tell you? A diseased governor? Now, it’s too late
for me to begin shovelling and sifting at alphabeds and grammar-books.
I’m getting to be a old bird, and I want to take it easy. But I want
some reading--some fine bold reading, some splendid book in a gorging
Lord-Mayor’s-Show of wollumes’ (probably meaning gorgeous, but misled
by association of ideas); ‘as’ll reach right down your pint of view, and
take time to go by you. How can I get that reading, Wegg? By,’ tapping
him on the breast with the head of his thick stick, ‘paying a man truly
qualified to do it, so much an hour (say twopence) to come and do it.’

‘Hem! Flattered, sir, I am sure,’ said Wegg, beginning to regard himself
in quite a new light. ‘Hew! This is the offer you mentioned, sir?’

‘Yes. Do you like it?’

‘I am considering of it, Mr Boffin.’

‘I don’t,’ said Boffin, in a free-handed manner, ‘want to tie a literary
man--WITH a wooden leg--down too tight. A halfpenny an hour shan’t part
us. The hours are your own to choose, after you’ve done for the day
with your house here. I live over Maiden-Lane way--out Holloway
direction--and you’ve only got to go East-and-by-North when you’ve
finished here, and you’re there. Twopence halfpenny an hour,’ said
Boffin, taking a piece of chalk from his pocket and getting off the
stool to work the sum on the top of it in his own way; ‘two long’uns and
a short’un--twopence halfpenny; two short’uns is a long’un and two two
long’uns is four long’uns--making five long’uns; six nights a week at
five long’uns a night,’ scoring them all down separately, ‘and you mount
up to thirty long’uns. A round’un! Half a crown!’

Pointing to this result as a large and satisfactory one, Mr Boffin
smeared it out with his moistened glove, and sat down on the remains.

‘Half a crown,’ said Wegg, meditating. ‘Yes. (It ain’t much, sir.) Half
a crown.’

‘Per week, you know.’

‘Per week. Yes. As to the amount of strain upon the intellect now. Was
you thinking at all of poetry?’ Mr Wegg inquired, musing.

‘Would it come dearer?’ Mr Boffin asked.

‘It would come dearer,’ Mr Wegg returned. ‘For when a person comes to
grind off poetry night after night, it is but right he should expect to
be paid for its weakening effect on his mind.’

‘To tell you the truth Wegg,’ said Boffin, ‘I wasn’t thinking of poetry,
except in so fur as this:--If you was to happen now and then to feel
yourself in the mind to tip me and Mrs Boffin one of your ballads, why
then we should drop into poetry.’

‘I follow you, sir,’ said Wegg. ‘But not being a regular musical
professional, I should be loath to engage myself for that; and therefore
when I dropped into poetry, I should ask to be considered so fur, in the
light of a friend.’

At this, Mr Boffin’s eyes sparkled, and he shook Silas earnestly by the
hand: protesting that it was more than he could have asked, and that he
took it very kindly indeed.

‘What do you think of the terms, Wegg?’ Mr Boffin then demanded, with
unconcealed anxiety.

Silas, who had stimulated this anxiety by his hard reserve of manner,
and who had begun to understand his man very well, replied with an air;
as if he were saying something extraordinarily generous and great:

‘Mr Boffin, I never bargain.’

‘So I should have thought of you!’ said Mr Boffin, admiringly. ‘No, sir.
I never did ‘aggle and I never will ‘aggle. Consequently I meet you at
once, free and fair, with--Done, for double the money!’

Mr Boffin seemed a little unprepared for this conclusion, but assented,
with the remark, ‘You know better what it ought to be than I do, Wegg,’
and again shook hands with him upon it.

‘Could you begin to night, Wegg?’ he then demanded.

‘Yes, sir,’ said Mr Wegg, careful to leave all the eagerness to him.
‘I see no difficulty if you wish it. You are provided with the needful
implement--a book, sir?’

‘Bought him at a sale,’ said Mr Boffin. ‘Eight wollumes. Red and gold.
Purple ribbon in every wollume, to keep the place where you leave off.
Do you know him?’

‘The book’s name, sir?’ inquired Silas.

‘I thought you might have know’d him without it,’ said Mr
Boffin slightly disappointed. ‘His name is
Decline-And-Fall-Off-The-Rooshan-Empire.’ (Mr Boffin went over these
stones slowly and with much caution.)

‘Ay indeed!’ said Mr Wegg, nodding his head with an air of friendly
recognition.

‘You know him, Wegg?’

‘I haven’t been not to say right slap through him, very lately,’ Mr Wegg
made answer, ‘having been otherways employed, Mr Boffin. But know him?
Old familiar declining and falling off the Rooshan? Rather, sir! Ever
since I was not so high as your stick. Ever since my eldest brother left
our cottage to enlist into the army. On which occasion, as the ballad
that was made about it describes:

\begin{verbatim}
     ‘Beside that cottage door, Mr Boffin,
             A girl was on her knees;
     She held aloft a snowy scarf, Sir,
             Which (my eldest brother noticed) fluttered in the breeze.
     She breathed a prayer for him, Mr Boffin;
             A prayer he coold not hear.
     And my eldest brother lean’d upon his sword, Mr Boffin,
              And wiped away a tear.’
\end{verbatim}

Much impressed by this family circumstance, and also by the friendly
disposition of Mr Wegg, as exemplified in his so soon dropping into
poetry, Mr Boffin again shook hands with that ligneous sharper, and
besought him to name his hour. Mr Wegg named eight.

‘Where I live,’ said Mr Boffin, ‘is called The Bower. Boffin’s Bower is
the name Mrs Boffin christened it when we come into it as a property.
If you should meet with anybody that don’t know it by that name (which
hardly anybody does), when you’ve got nigh upon about a odd mile, or
say and a quarter if you like, up Maiden Lane, Battle Bridge, ask for
Harmony Jail, and you’ll be put right. I shall expect you, Wegg,’ said
Mr Boffin, clapping him on the shoulder with the greatest enthusiasm,
‘most joyfully. I shall have no peace or patience till you come. Print
is now opening ahead of me. This night, a literary man--WITH a wooden
leg--’ he bestowed an admiring look upon that decoration, as if it
greatly enhanced the relish of Mr Wegg’s attainments--‘will begin to
lead me a new life! My fist again, Wegg. Morning, morning, morning!’

Left alone at his stall as the other ambled off, Mr Wegg subsided
into his screen, produced a small pocket-handkerchief of a
penitentially-scrubbing character, and took himself by the nose with
a thoughtful aspect. Also, while he still grasped that feature, he
directed several thoughtful looks down the street, after the retiring
figure of Mr Boffin. But, profound gravity sat enthroned on Wegg’s
countenance. For, while he considered within himself that this was
an old fellow of rare simplicity, that this was an opportunity to
be improved, and that here might be money to be got beyond present
calculation, still he compromised himself by no admission that his new
engagement was at all out of his way, or involved the least element of
the ridiculous. Mr Wegg would even have picked a handsome quarrel with
any one who should have challenged his deep acquaintance with those
aforesaid eight volumes of Decline and Fall. His gravity was unusual,
portentous, and immeasurable, not because he admitted any doubt of
himself but because he perceived it necessary to forestall any doubt of
himself in others. And herein he ranged with that very numerous class
of impostors, who are quite as determined to keep up appearances to
themselves, as to their neighbours.

A certain loftiness, likewise, took possession of Mr Wegg; a
condescending sense of being in request as an official expounder of
mysteries. It did not move him to commercial greatness, but rather to
littleness, insomuch that if it had been within the possibilities of
things for the wooden measure to hold fewer nuts than usual, it would
have done so that day. But, when night came, and with her veiled eyes
beheld him stumping towards Boffin’s Bower, he was elated too.

The Bower was as difficult to find, as Fair Rosamond’s without the clue.
Mr Wegg, having reached the quarter indicated, inquired for the Bower
half a dozen times without the least success, until he remembered to
ask for Harmony Jail. This occasioned a quick change in the spirits of a
hoarse gentleman and a donkey, whom he had much perplexed.

‘Why, yer mean Old Harmon’s, do yer?’ said the hoarse gentleman, who was
driving his donkey in a truck, with a carrot for a whip. ‘Why didn’t yer
niver say so? Eddard and me is a goin’ by HIM! Jump in.’

Mr Wegg complied, and the hoarse gentleman invited his attention to the
third person in company, thus;

‘Now, you look at Eddard’s ears. What was it as you named, agin?
Whisper.’

Mr Wegg whispered, ‘Boffin’s Bower.’

‘Eddard! (keep yer hi on his ears) cut away to Boffin’s Bower!’

Edward, with his ears lying back, remained immoveable.

‘Eddard! (keep yer hi on his ears) cut away to Old Harmon’s.’ Edward
instantly pricked up his ears to their utmost, and rattled off at such
a pace that Mr Wegg’s conversation was jolted out of him in a most
dislocated state.

‘Was-it-Ev-verajail?’ asked Mr Wegg, holding on.

‘Not a proper jail, wot you and me would get committed to,’ returned
his escort; ‘they giv’ it the name, on accounts of Old Harmon living
solitary there.’

‘And-why-did-they-callitharm-Ony?’ asked Wegg.

‘On accounts of his never agreeing with nobody. Like a speeches of
chaff. Harmon’s Jail; Harmony Jail. Working it round like.’

‘Doyouknow-Mist-Erboff-in?’ asked Wegg.

‘I should think so! Everybody do about here. Eddard knows him. (Keep yer
hi on his ears.) Noddy Boffin, Eddard!’

The effect of the name was so very alarming, in respect of causing a
temporary disappearance of Edward’s head, casting his hind hoofs in the
air, greatly accelerating the pace and increasing the jolting, that Mr
Wegg was fain to devote his attention exclusively to holding on, and to
relinquish his desire of ascertaining whether this homage to Boffin was
to be considered complimentary or the reverse.

Presently, Edward stopped at a gateway, and Wegg discreetly lost no time
in slipping out at the back of the truck. The moment he was landed, his
late driver with a wave of the carrot, said ‘Supper, Eddard!’ and he,
the hind hoofs, the truck, and Edward, all seemed to fly into the air
together, in a kind of apotheosis.

Pushing the gate, which stood ajar, Wegg looked into an enclosed space
where certain tall dark mounds rose high against the sky, and where the
pathway to the Bower was indicated, as the moonlight showed, between two
lines of broken crockery set in ashes. A white figure advancing along
this path, proved to be nothing more ghostly than Mr Boffin, easily
attired for the pursuit of knowledge, in an undress garment of short
white smock-frock. Having received his literary friend with great
cordiality, he conducted him to the interior of the Bower and there
presented him to Mrs Boffin:--a stout lady of a rubicund and cheerful
aspect, dressed (to Mr Wegg’s consternation) in a low evening-dress of
sable satin, and a large black velvet hat and feathers.

‘Mrs Boffin, Wegg,’ said Boffin, ‘is a highflyer at Fashion. And her
make is such, that she does it credit. As to myself I ain’t yet as
Fash’nable as I may come to be. Henerietty, old lady, this is the
gentleman that’s a going to decline and fall off the Rooshan Empire.’

‘And I am sure I hope it’ll do you both good,’ said Mrs Boffin.

It was the queerest of rooms, fitted and furnished more like a luxurious
amateur tap-room than anything else within the ken of Silas Wegg. There
were two wooden settles by the fire, one on either side of it, with
a corresponding table before each. On one of these tables, the eight
volumes were ranged flat, in a row, like a galvanic battery; on the
other, certain squat case-bottles of inviting appearance seemed to stand
on tiptoe to exchange glances with Mr Wegg over a front row of tumblers
and a basin of white sugar. On the hob, a kettle steamed; on the hearth,
a cat reposed. Facing the fire between the settles, a sofa, a footstool,
and a little table, formed a centrepiece devoted to Mrs Boffin.
They were garish in taste and colour, but were expensive articles of
drawing-room furniture that had a very odd look beside the settles
and the flaring gaslight pendent from the ceiling. There was a flowery
carpet on the floor; but, instead of reaching to the fireside, its
glowing vegetation stopped short at Mrs Boffin’s footstool, and gave
place to a region of sand and sawdust. Mr Wegg also noticed, with
admiring eyes, that, while the flowery land displayed such hollow
ornamentation as stuffed birds and waxen fruits under glass-shades,
there were, in the territory where vegetation ceased, compensatory
shelves on which the best part of a large pie and likewise of a cold
joint were plainly discernible among other solids. The room itself was
large, though low; and the heavy frames of its old-fashioned windows,
and the heavy beams in its crooked ceiling, seemed to indicate that it
had once been a house of some mark standing alone in the country.

‘Do you like it, Wegg?’ asked Mr Boffin, in his pouncing manner.

‘I admire it greatly, sir,’ said Wegg. ‘Peculiar comfort at this
fireside, sir.’

‘Do you understand it, Wegg?’

‘Why, in a general way, sir,’ Mr Wegg was beginning slowly and
knowingly, with his head stuck on one side, as evasive people do begin,
when the other cut him short:

‘You DON’T understand it, Wegg, and I’ll explain it. These arrangements
is made by mutual consent between Mrs Boffin and me. Mrs Boffin, as I’ve
mentioned, is a highflyer at Fashion; at present I’m not. I don’t go
higher than comfort, and comfort of the sort that I’m equal to the
enjoyment of. Well then. Where would be the good of Mrs Boffin and me
quarrelling over it? We never did quarrel, before we come into Boffin’s
Bower as a property; why quarrel when we HAVE come into Boffin’s Bower
as a property? So Mrs Boffin, she keeps up her part of the room, in her
way; I keep up my part of the room in mine. In consequence of which
we have at once, Sociability (I should go melancholy mad without Mrs
Boffin), Fashion, and Comfort. If I get by degrees to be a higher-flyer
at Fashion, then Mrs Boffin will by degrees come for’arder. If Mrs
Boffin should ever be less of a dab at Fashion than she is at the
present time, then Mrs Boffin’s carpet would go back’arder. If we should
both continny as we are, why then HERE we are, and give us a kiss, old
lady.’

Mrs Boffin who, perpetually smiling, had approached and drawn her plump
arm through her lord’s, most willingly complied. Fashion, in the form
of her black velvet hat and feathers, tried to prevent it; but got
deservedly crushed in the endeavour.

‘So now, Wegg,’ said Mr Boffin, wiping his mouth with an air of much
refreshment, ‘you begin to know us as we are. This is a charming spot,
is the Bower, but you must get to apprechiate it by degrees. It’s a spot
to find out the merits of; little by little, and a new’un every day.
There’s a serpentining walk up each of the mounds, that gives you the
yard and neighbourhood changing every moment. When you get to the top,
there’s a view of the neighbouring premises, not to be surpassed. The
premises of Mrs Boffin’s late father (Canine Provision Trade), you look
down into, as if they was your own. And the top of the High Mound is
crowned with a lattice-work Arbour, in which, if you don’t read out loud
many a book in the summer, ay, and as a friend, drop many a time into
poetry too, it shan’t be my fault. Now, what’ll you read on?’

‘Thank you, sir,’ returned Wegg, as if there were nothing new in his
reading at all. ‘I generally do it on gin and water.’

‘Keeps the organ moist, does it, Wegg?’ asked Mr Boffin, with innocent
eagerness.

‘N-no, sir,’ replied Wegg, coolly, ‘I should hardly describe it so, sir.
I should say, mellers it. Mellers it, is the word I should employ, Mr
Boffin.’

His wooden conceit and craft kept exact pace with the delighted
expectation of his victim. The visions rising before his mercenary mind,
of the many ways in which this connexion was to be turned to account,
never obscured the foremost idea natural to a dull overreaching man,
that he must not make himself too cheap.

Mrs Boffin’s Fashion, as a less inexorable deity than the idol usually
worshipped under that name, did not forbid her mixing for her literary
guest, or asking if he found the result to his liking. On his returning
a gracious answer and taking his place at the literary settle, Mr Boffin
began to compose himself as a listener, at the opposite settle, with
exultant eyes.

‘Sorry to deprive you of a pipe, Wegg,’ he said, filling his own, ‘but
you can’t do both together. Oh! and another thing I forgot to name! When
you come in here of an evening, and look round you, and notice anything
on a shelf that happens to catch your fancy, mention it.’

Wegg, who had been going to put on his spectacles, immediately laid them
down, with the sprightly observation:

‘You read my thoughts, sir. DO my eyes deceive me, or is that object up
there a--a pie? It can’t be a pie.’

‘Yes, it’s a pie, Wegg,’ replied Mr Boffin, with a glance of some little
discomfiture at the Decline and Fall.

‘HAVE I lost my smell for fruits, or is it a apple pie, sir?’ asked
Wegg.

‘It’s a veal and ham pie,’ said Mr Boffin.

‘Is it indeed, sir? And it would be hard, sir, to name the pie that is
a better pie than a weal and hammer,’ said Mr Wegg, nodding his head
emotionally.

‘Have some, Wegg?’

‘Thank you, Mr Boffin, I think I will, at your invitation. I wouldn’t
at any other party’s, at the present juncture; but at yours, sir!--And
meaty jelly too, especially when a little salt, which is the case where
there’s ham, is mellering to the organ, is very mellering to the organ.’
Mr Wegg did not say what organ, but spoke with a cheerful generality.

So, the pie was brought down, and the worthy Mr Boffin exercised his
patience until Wegg, in the exercise of his knife and fork, had finished
the dish: only profiting by the opportunity to inform Wegg that although
it was not strictly Fashionable to keep the contents of a larder thus
exposed to view, he (Mr Boffin) considered it hospitable; for the
reason, that instead of saying, in a comparatively unmeaning manner, to
a visitor, ‘There are such and such edibles down stairs; will you have
anything up?’ you took the bold practical course of saying, ‘Cast your
eye along the shelves, and, if you see anything you like there, have it
down.’

And now, Mr Wegg at length pushed away his plate and put on his
spectacles, and Mr Boffin lighted his pipe and looked with beaming
eyes into the opening world before him, and Mrs Boffin reclined in a
fashionable manner on her sofa: as one who would be part of the audience
if she found she could, and would go to sleep if she found she couldn’t.

‘Hem!’ began Wegg, ‘This, Mr Boffin and Lady, is the first chapter of
the first wollume of the Decline and Fall off--’ here he looked hard at
the book, and stopped.

‘What’s the matter, Wegg?’

‘Why, it comes into my mind, do you know, sir,’ said Wegg with an air
of insinuating frankness (having first again looked hard at the book),
‘that you made a little mistake this morning, which I had meant to set
you right in, only something put it out of my head. I think you said
Rooshan Empire, sir?’

‘It is Rooshan; ain’t it, Wegg?’

‘No, sir. Roman. Roman.’

‘What’s the difference, Wegg?’

‘The difference, sir?’ Mr Wegg was faltering and in danger of breaking
down, when a bright thought flashed upon him. ‘The difference, sir?
There you place me in a difficulty, Mr Boffin. Suffice it to observe,
that the difference is best postponed to some other occasion when Mrs
Boffin does not honour us with her company. In Mrs Boffin’s presence,
sir, we had better drop it.’

Mr Wegg thus came out of his disadvantage with quite a chivalrous air,
and not only that, but by dint of repeating with a manly delicacy,
‘In Mrs Boffin’s presence, sir, we had better drop it!’ turned the
disadvantage on Boffin, who felt that he had committed himself in a very
painful manner.

Then, Mr Wegg, in a dry unflinching way, entered on his task; going
straight across country at everything that came before him; taking all
the hard words, biographical and geographical; getting rather shaken by
Hadrian, Trajan, and the Antonines; stumbling at Polybius (pronounced
Polly Beeious, and supposed by Mr Boffin to be a Roman virgin, and by
Mrs Boffin to be responsible for that necessity of dropping it); heavily
unseated by Titus Antoninus Pius; up again and galloping smoothly with
Augustus; finally, getting over the ground well with Commodus: who,
under the appellation of Commodious, was held by Mr Boffin to have been
quite unworthy of his English origin, and ‘not to have acted up to his
name’ in his government of the Roman people. With the death of this
personage, Mr Wegg terminated his first reading; long before which
consummation several total eclipses of Mrs Boffin’s candle behind
her black velvet disc, would have been very alarming, but for being
regularly accompanied by a potent smell of burnt pens when her feathers
took fire, which acted as a restorative and woke her. Mr Wegg, having
read on by rote and attached as few ideas as possible to the text, came
out of the encounter fresh; but, Mr Boffin, who had soon laid down his
unfinished pipe, and had ever since sat intently staring with his eyes
and mind at the confounding enormities of the Romans, was so severely
punished that he could hardly wish his literary friend Good-night, and
articulate ‘Tomorrow.’

‘Commodious,’ gasped Mr Boffin, staring at the moon, after letting
Wegg out at the gate and fastening it: ‘Commodious fights in that
wild-beast-show, seven hundred and thirty-five times, in one character
only! As if that wasn’t stunning enough, a hundred lions is turned into
the same wild-beast-show all at once! As if that wasn’t stunning enough,
Commodious, in another character, kills ‘em all off in a hundred goes!
As if that wasn’t stunning enough, Vittle-us (and well named too) eats
six millions’ worth, English money, in seven months! Wegg takes it easy,
but upon-my-soul to a old bird like myself these are scarers. And even
now that Commodious is strangled, I don’t see a way to our bettering
ourselves.’ Mr Boffin added as he turned his pensive steps towards the
Bower and shook his head, ‘I didn’t think this morning there was half so
many Scarers in Print. But I’m in for it now!’


