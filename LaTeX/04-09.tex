% REV00 Tue 04 May 2021 13:55:16 WIB
% START Tue 04 May 2021 13:55:16 WIB

\chapter{XXX}

\includegraphics[scale=2.3]{03-05-01}

Chapter 9

TWO PLACES VACATED


Set down by the omnibus at the corner of Saint Mary Axe, and trusting
to her feet and her crutch-stick within its precincts, the dolls’
dressmaker proceeded to the place of business of Pubsey and Co. All
there was sunny and quiet externally, and shady and quiet internally.
Hiding herself in the entry outside the glass door, she could see from
that post of observation the old man in his spectacles sitting writing
at his desk.

‘Boh!’ cried the dressmaker, popping in her head at the glass-door. ‘Mr
Wolf at home?’

The old man took his glasses off, and mildly laid them down beside him.
‘Ah Jenny, is it you? I thought you had given me up.’

‘And so I had given up the treacherous wolf of the forest,’ she replied;
‘but, godmother, it strikes me you have come back. I am not quite sure,
because the wolf and you change forms. I want to ask you a question or
two, to find out whether you are really godmother or really wolf. May
I?’

‘Yes, Jenny, yes.’ But Riah glanced towards the door, as if he thought
his principal might appear there, unseasonably.

‘If you’re afraid of the fox,’ said Miss Jenny, ‘you may dismiss all
present expectations of seeing that animal. HE won’t show himself
abroad, for many a day.’

‘What do you mean, my child?’

‘I mean, godmother,’ replied Miss Wren, sitting down beside the Jew,
‘that the fox has caught a famous flogging, and that if his skin and
bones are not tingling, aching, and smarting at this present instant, no
fox did ever tingle, ache, and smart.’ Therewith Miss Jenny related what
had come to pass in the Albany, omitting the few grains of pepper.

‘Now, godmother,’ she went on, ‘I particularly wish to ask you what has
taken place here, since I left the wolf here? Because I have an idea
about the size of a marble, rolling about in my little noddle. First and
foremost, are you Pubsey and Co., or are you either? Upon your solemn
word and honour.’

The old man shook his head.

‘Secondly, isn’t Fledgeby both Pubsey and Co.?’

The old man answered with a reluctant nod.

‘My idea,’ exclaimed Miss Wren, ‘is now about the size of an orange. But
before it gets any bigger, welcome back, dear godmother!’

The little creature folded her arms about the old man’s neck with great
earnestness, and kissed him. ‘I humbly beg your forgiveness, godmother.
I am truly sorry. I ought to have had more faith in you. But what could
I suppose when you said nothing for yourself, you know? I don’t mean to
offer that as a justification, but what could I suppose, when you were a
silent party to all he said? It did look bad; now didn’t it?’

‘It looked so bad, Jenny,’ responded the old man, with gravity, ‘that I
will straightway tell you what an impression it wrought upon me. I was
hateful in mine own eyes. I was hateful to myself, in being so hateful
to the debtor and to you. But more than that, and worse than that,
and to pass out far and broad beyond myself--I reflected that evening,
sitting alone in my garden on the housetop, that I was doing dishonour
to my ancient faith and race. I reflected--clearly reflected for the
first time--that in bending my neck to the yoke I was willing to wear,
I bent the unwilling necks of the whole Jewish people. For it is not, in
Christian countries, with the Jews as with other peoples. Men say, “This
is a bad Greek, but there are good Greeks. This is a bad Turk, but there
are good Turks.” Not so with the Jews. Men find the bad among us easily
enough--among what peoples are the bad not easily found?--but they take
the worst of us as samples of the best; they take the lowest of us as
presentations of the highest; and they say “All Jews are alike.” If,
doing what I was content to do here, because I was grateful for the past
and have small need of money now, I had been a Christian, I could have
done it, compromising no one but my individual self. But doing it as a
Jew, I could not choose but compromise the Jews of all conditions and
all countries. It is a little hard upon us, but it is the truth. I would
that all our people remembered it! Though I have little right to say so,
seeing that it came home so late to me.’

The dolls’ dressmaker sat holding the old man by the hand, and looking
thoughtfully in his face.

‘Thus I reflected, I say, sitting that evening in my garden on the
housetop. And passing the painful scene of that day in review before
me many times, I always saw that the poor gentleman believed the story
readily, because I was one of the Jews--that you believed the story
readily, my child, because I was one of the Jews--that the story itself
first came into the invention of the originator thereof, because I was
one of the Jews. This was the result of my having had you three before
me, face to face, and seeing the thing visibly presented as upon a
theatre. Wherefore I perceived that the obligation was upon me to leave
this service. But Jenny, my dear,’ said Riah, breaking off, ‘I promised
that you should pursue your questions, and I obstruct them.’

‘On the contrary, godmother; my idea is as large now as a pumpkin--and
YOU know what a pumpkin is, don’t you? So you gave notice that you
were going? Does that come next?’ asked Miss Jenny with a look of close
attention.

‘I indited a letter to my master. Yes. To that effect.’

‘And what said Tingling-Tossing-Aching-Screaming-Scratching-Smarter?’
asked Miss Wren with an unspeakable enjoyment in the utterance of those
honourable titles and in the recollection of the pepper.

‘He held me to certain months of servitude, which were his lawful term
of notice. They expire to-morrow. Upon their expiration--not before--I
had meant to set myself right with my Cinderella.’

‘My idea is getting so immense now,’ cried Miss Wren, clasping her
temples, ‘that my head won’t hold it! Listen, godmother; I am going to
expound. Little Eyes (that’s Screaming-Scratching-Smarter) owes you a
heavy grudge for going. Little Eyes casts about how best to pay you off.
Little Eyes thinks of Lizzie. Little Eyes says to himself, “I’ll find
out where he has placed that girl, and I’ll betray his secret because
it’s dear to him.” Perhaps Little Eyes thinks, “I’ll make love to her
myself too;” but that I can’t swear--all the rest I can. So, Little Eyes
comes to me, and I go to Little Eyes. That’s the way of it. And now the
murder’s all out, I’m sorry,’ added the dolls’ dressmaker, rigid from
head to foot with energy as she shook her little fist before her eyes,
‘that I didn’t give him Cayenne pepper and chopped pickled Capsicum!’

This expression of regret being but partially intelligible to Mr Riah,
the old man reverted to the injuries Fledgeby had received, and hinted
at the necessity of his at once going to tend that beaten cur.

‘Godmother, godmother, godmother!’ cried Miss Wren irritably, ‘I really
lose all patience with you. One would think you believed in the Good
Samaritan. How can you be so inconsistent?’

‘Jenny dear,’ began the old man gently, ‘it is the custom of our people
to help--’

‘Oh! Bother your people!’ interposed Miss Wren, with a toss of her head.
‘If your people don’t know better than to go and help Little Eyes, it’s
a pity they ever got out of Egypt. Over and above that,’ she added, ‘he
wouldn’t take your help if you offered it. Too much ashamed. Wants to
keep it close and quiet, and to keep you out of the way.’

They were still debating this point when a shadow darkened the entry,
and the glass door was opened by a messenger who brought a letter
unceremoniously addressed, ‘Riah.’ To which he said there was an answer
wanted.

The letter, which was scrawled in pencil uphill and downhill and round
crooked corners, ran thus:


‘OLD RIAH,

Your accounts being all squared, go. Shut up the place, turn out
directly, and send me the key by bearer. Go. You are an unthankful dog
of a Jew. Get out.

F.’


The dolls’ dressmaker found it delicious to trace the screaming and
smarting of Little Eyes in the distorted writing of this epistle. She
laughed over it and jeered at it in a convenient corner (to the great
astonishment of the messenger) while the old man got his few goods
together in a black bag. That done, the shutters of the upper windows
closed, and the office blind pulled down, they issued forth upon the
steps with the attendant messenger. There, while Miss Jenny held the
bag, the old man locked the house door, and handed over the key to him;
who at once retired with the same.

‘Well, godmother,’ said Miss Wren, as they remained upon the steps
together, looking at one another. ‘And so you’re thrown upon the world!’

‘It would appear so, Jenny, and somewhat suddenly.’

‘Where are you going to seek your fortune?’ asked Miss Wren.

The old man smiled, but looked about him with a look of having lost his
way in life, which did not escape the dolls’ dressmaker.

‘Verily, Jenny,’ said he, ‘the question is to the purpose, and more
easily asked than answered. But as I have experience of the ready
goodwill and good help of those who have given occupation to Lizzie, I
think I will seek them out for myself.’

‘On foot?’ asked Miss Wren, with a chop.

‘Ay!’ said the old man. ‘Have I not my staff?’

It was exactly because he had his staff, and presented so quaint an
aspect, that she mistrusted his making the journey.

‘The best thing you can do,’ said Jenny, ‘for the time being, at all
events, is to come home with me, godmother. Nobody’s there but my bad
child, and Lizzie’s lodging stands empty.’ The old man when satisfied
that no inconvenience could be entailed on any one by his compliance,
readily complied; and the singularly-assorted couple once more went
through the streets together.

Now, the bad child having been strictly charged by his parent to remain
at home in her absence, of course went out; and, being in the very last
stage of mental decrepitude, went out with two objects; firstly,
to establish a claim he conceived himself to have upon any licensed
victualler living, to be supplied with threepennyworth of rum for
nothing; and secondly, to bestow some maudlin remorse on Mr Eugene
Wrayburn, and see what profit came of it. Stumblingly pursuing these
two designs--they both meant rum, the only meaning of which he was
capable--the degraded creature staggered into Covent Garden Market and
there bivouacked, to have an attack of the trembles succeeded by an
attack of the horrors, in a doorway.

This market of Covent Garden was quite out of the creature’s line of
road, but it had the attraction for him which it has for the worst of
the solitary members of the drunken tribe. It may be the companionship
of the nightly stir, or it may be the companionship of the gin and
beer that slop about among carters and hucksters, or it may be the
companionship of the trodden vegetable refuse which is so like their own
dress that perhaps they take the Market for a great wardrobe; but be
it what it may, you shall see no such individual drunkards on doorsteps
anywhere, as there. Of dozing women-drunkards especially, you shall come
upon such specimens there, in the morning sunlight, as you might
seek out of doors in vain through London. Such stale vapid rejected
cabbage-leaf and cabbage-stalk dress, such damaged-orange countenance,
such squashed pulp of humanity, are open to the day nowhere else. So,
the attraction of the Market drew Mr Dolls to it, and he had out his two
fits of trembles and horrors in a doorway on which a woman had had out
her sodden nap a few hours before.

There is a swarm of young savages always flitting about this same place,
creeping off with fragments of orange-chests, and mouldy litter--Heaven
knows into what holes they can convey them, having no home!--whose bare
feet fall with a blunt dull softness on the pavement as the policeman
hunts them, and who are (perhaps for that reason) little heard by
the Powers that be, whereas in top-boots they would make a deafening
clatter. These, delighting in the trembles and the horrors of Mr Dolls,
as in a gratuitous drama, flocked about him in his doorway, butted
at him, leaped at him, and pelted him. Hence, when he came out of
his invalid retirement and shook off that ragged train, he was much
bespattered, and in worse case than ever. But, not yet at his worst;
for, going into a public-house, and being supplied in stress of business
with his rum, and seeking to vanish without payment, he was collared,
searched, found penniless, and admonished not to try that again,
by having a pail of dirty water cast over him. This application
superinduced another fit of the trembles; after which Mr Dolls, as
finding himself in good cue for making a call on a professional friend,
addressed himself to the Temple.

There was nobody at the chambers but Young Blight. That discreet youth,
sensible of a certain incongruity in the association of such a
client with the business that might be coming some day, with the best
intentions temporized with Dolls, and offered a shilling for coach-hire
home. Mr Dolls, accepting the shilling, promptly laid it out in
two threepennyworths of conspiracy against his life, and two
threepennyworths of raging repentance. Returning to the Chambers with
which burden, he was descried coming round into the court, by the wary
young Blight watching from the window: who instantly closed the outer
door, and left the miserable object to expend his fury on the panels.

The more the door resisted him, the more dangerous and imminent became
that bloody conspiracy against his life. Force of police arriving,
he recognized in them the conspirators, and laid about him hoarsely,
fiercely, staringly, convulsively, foamingly. A humble machine, familiar
to the conspirators and called by the expressive name of Stretcher,
being unavoidably sent for, he was rendered a harmless bundle of torn
rags by being strapped down upon it, with voice and consciousness gone
out of him, and life fast going. As this machine was borne out at the
Temple gate by four men, the poor little dolls’ dressmaker and her
Jewish friend were coming up the street.

‘Let us see what it is,’ cried the dressmaker. ‘Let us make haste and
look, godmother.’

The brisk little crutch-stick was but too brisk. ‘O gentlemen,
gentlemen, he belongs to me!’

‘Belongs to you?’ said the head of the party, stopping it.

‘O yes, dear gentlemen, he’s my child, out without leave. My poor bad,
bad boy! and he don’t know me, he don’t know me! O what shall I do,’
cried the little creature, wildly beating her hands together, ‘when my
own child don’t know me!’

The head of the party looked (as well he might) to the old man for
explanation. He whispered, as the dolls’ dressmaker bent over the
exhausted form and vainly tried to extract some sign of recognition from
it: ‘It’s her drunken father.’

As the load was put down in the street, Riah drew the head of the party
aside, and whispered that he thought the man was dying. ‘No, surely
not?’ returned the other. But he became less confident, on looking, and
directed the bearers to ‘bring him to the nearest doctor’s shop.’

Thither he was brought; the window becoming from within, a wall of
faces, deformed into all kinds of shapes through the agency of globular
red bottles, green bottles, blue bottles, and other coloured bottles. A
ghastly light shining upon him that he didn’t need, the beast so furious
but a few minutes gone, was quiet enough now, with a strange mysterious
writing on his face, reflected from one of the great bottles, as if
Death had marked him: ‘Mine.’

The medical testimony was more precise and more to the purpose than it
sometimes is in a Court of Justice. ‘You had better send for something
to cover it. All’s over.’

Therefore, the police sent for something to cover it, and it was covered
and borne through the streets, the people falling away. After it,
went the dolls’ dressmaker, hiding her face in the Jewish skirts, and
clinging to them with one hand, while with the other she plied her
stick. It was carried home, and, by reason that the staircase was very
narrow, it was put down in the parlour--the little working-bench being
set aside to make room for it--and there, in the midst of the dolls with
no speculation in their eyes, lay Mr Dolls with no speculation in his.

Many flaunting dolls had to be gaily dressed, before the money was in
the dressmaker’s pocket to get mourning for Mr Dolls. As the old man,
Riah, sat by, helping her in such small ways as he could, he found it
difficult to make out whether she really did realize that the deceased
had been her father.

‘If my poor boy,’ she would say, ‘had been brought up better, he might
have done better. Not that I reproach myself. I hope I have no cause for
that.’

‘None indeed, Jenny, I am very certain.’

‘Thank you, godmother. It cheers me to hear you say so. But you see it
is so hard to bring up a child well, when you work, work, work, all day.
When he was out of employment, I couldn’t always keep him near me. He
got fractious and nervous, and I was obliged to let him go into the
streets. And he never did well in the streets, he never did well out of
sight. How often it happens with children!’

‘Too often, even in this sad sense!’ thought the old man.

‘How can I say what I might have turned out myself, but for my back
having been so bad and my legs so queer, when I was young!’ the
dressmaker would go on. ‘I had nothing to do but work, and so I worked.
I couldn’t play. But my poor unfortunate child could play, and it turned
out the worse for him.’

‘And not for him alone, Jenny.’

‘Well! I don’t know, godmother. He suffered heavily, did my unfortunate
boy. He was very, very ill sometimes. And I called him a quantity of
names;’ shaking her head over her work, and dropping tears. ‘I don’t
know that his going wrong was much the worse for me. If it ever was, let
us forget it.’

‘You are a good girl, you are a patient girl.’

‘As for patience,’ she would reply with a shrug, ‘not much of that,
godmother. If I had been patient, I should never have called him names.
But I hope I did it for his good. And besides, I felt my responsibility
as a mother, so much. I tried reasoning, and reasoning failed. I tried
coaxing, and coaxing failed. I tried scolding and scolding failed. But I
was bound to try everything, you know, with such a charge upon my hands.
Where would have been my duty to my poor lost boy, if I had not tried
everything!’

With such talk, mostly in a cheerful tone on the part of the industrious
little creature, the day-work and the night-work were beguiled until
enough of smart dolls had gone forth to bring into the kitchen,
where the working-bench now stood, the sombre stuff that the occasion
required, and to bring into the house the other sombre preparations.
‘And now,’ said Miss Jenny, ‘having knocked off my rosy-cheeked young
friends, I’ll knock off my white-cheeked self.’ This referred to her
making her own dress, which at last was done. ‘The disadvantage of
making for yourself,’ said Miss Jenny, as she stood upon a chair to look
at the result in the glass, ‘is, that you can’t charge anybody else for
the job, and the advantage is, that you haven’t to go out to try on.
Humph! Very fair indeed! If He could see me now (whoever he is) I hope
he wouldn’t repent of his bargain!’

The simple arrangements were of her own making, and were stated to Riah
thus:

‘I mean to go alone, godmother, in my usual carriage, and you’ll be so
kind as keep house while I am gone. It’s not far off. And when I return,
we’ll have a cup of tea, and a chat over future arrangements. It’s a
very plain last house that I have been able to give my poor unfortunate
boy; but he’ll accept the will for the deed if he knows anything about
it; and if he doesn’t know anything about it,’ with a sob, and wiping
her eyes, ‘why, it won’t matter to him. I see the service in the
Prayer-book says, that we brought nothing into this world and it is
certain we can take nothing out. It comforts me for not being able to
hire a lot of stupid undertaker’s things for my poor child, and seeming
as if I was trying to smuggle ‘em out of this world with him, when of
course I must break down in the attempt, and bring ‘em all back again.
As it is, there’ll be nothing to bring back but me, and that’s quite
consistent, for I shan’t be brought back, some day!’

After that previous carrying of him in the streets, the wretched old
fellow seemed to be twice buried. He was taken on the shoulders of half
a dozen blossom-faced men, who shuffled with him to the churchyard,
and who were preceded by another blossom-faced man, affecting a
stately stalk, as if he were a Policeman of the D(eath) Division, and
ceremoniously pretending not to know his intimate acquaintances, as he
led the pageant. Yet, the spectacle of only one little mourner hobbling
after, caused many people to turn their heads with a look of interest.

At last the troublesome deceased was got into the ground, to be buried
no more, and the stately stalker stalked back before the solitary
dressmaker, as if she were bound in honour to have no notion of the way
home. Those Furies, the conventionalities, being thus appeased, he left
her.

‘I must have a very short cry, godmother, before I cheer up for good,’
said the little creature, coming in. ‘Because after all a child is a
child, you know.’

It was a longer cry than might have been expected. Howbeit, it wore
itself out in a shadowy corner, and then the dressmaker came forth, and
washed her face, and made the tea. ‘You wouldn’t mind my cutting out
something while we are at tea, would you?’ she asked her Jewish friend,
with a coaxing air.

‘Cinderella, dear child,’ the old man expostulated, ‘will you never
rest?’

‘Oh! It’s not work, cutting out a pattern isn’t,’ said Miss Jenny, with
her busy little scissors already snipping at some paper. ‘The truth is,
godmother, I want to fix it while I have it correct in my mind.’

‘Have you seen it to-day then?’ asked Riah.

‘Yes, godmother. Saw it just now. It’s a surplice, that’s what it
is. Thing our clergymen wear, you know,’ explained Miss Jenny, in
consideration of his professing another faith.

‘And what have you to do with that, Jenny?’

‘Why, godmother,’ replied the dressmaker, ‘you must know that we
Professors who live upon our taste and invention, are obliged to keep
our eyes always open. And you know already that I have many extra
expenses to meet just now. So, it came into my head while I was weeping
at my poor boy’s grave, that something in my way might be done with a
clergyman.’

‘What can be done?’ asked the old man.

‘Not a funeral, never fear!’ returned Miss Jenny, anticipating his
objection with a nod. ‘The public don’t like to be made melancholy, I
know very well. I am seldom called upon to put my young friends into
mourning; not into real mourning, that is; Court mourning they are
rather proud of. But a doll clergyman, my dear,--glossy black curls
and whiskers--uniting two of my young friends in matrimony,’ said Miss
Jenny, shaking her forefinger, ‘is quite another affair. If you don’t
see those three at the altar in Bond Street, in a jiffy, my name’s Jack
Robinson!’

With her expert little ways in sharp action, she had got a doll into
whitey-brown paper orders, before the meal was over, and was displaying
it for the edification of the Jewish mind, when a knock was heard at the
street-door. Riah went to open it, and presently came back, ushering in,
with the grave and courteous air that sat so well upon him, a gentleman.

The gentleman was a stranger to the dressmaker; but even in the moment
of his casting his eyes upon her, there was something in his manner
which brought to her remembrance Mr Eugene Wrayburn.

‘Pardon me,’ said the gentleman. ‘You are the dolls’ dressmaker?’

‘I am the dolls’ dressmaker, sir.’

‘Lizzie Hexam’s friend?’

‘Yes, sir,’ replied Miss Jenny, instantly on the defensive. ‘And Lizzie
Hexam’s friend.’

‘Here is a note from her, entreating you to accede to the request of
Mr Mortimer Lightwood, the bearer. Mr Riah chances to know that I am Mr
Mortimer Lightwood, and will tell you so.’

Riah bent his head in corroboration.

‘Will you read the note?’

‘It’s very short,’ said Jenny, with a look of wonder, when she had read
it.

‘There was no time to make it longer. Time was so very precious. My dear
friend Mr Eugene Wrayburn is dying.’

The dressmaker clasped her hands, and uttered a little piteous cry.

‘Is dying,’ repeated Lightwood, with emotion, ‘at some distance from
here. He is sinking under injuries received at the hands of a villain
who attacked him in the dark. I come straight from his bedside. He is
almost always insensible. In a short restless interval of sensibility,
or partial sensibility, I made out that he asked for you to be brought
to sit by him. Hardly relying on my own interpretation of the indistinct
sounds he made, I caused Lizzie to hear them. We were both sure that he
asked for you.’

The dressmaker, with her hands still clasped, looked affrightedly from
the one to the other of her two companions.

‘If you delay, he may die with his request ungratified, with his
last wish--intrusted to me--we have long been much more than
brothers--unfulfilled. I shall break down, if I try to say more.’

In a few moments the black bonnet and the crutch-stick were on duty, the
good Jew was left in possession of the house, and the dolls’ dressmaker,
side by side in a chaise with Mortimer Lightwood, was posting out of
town.



Chapter 10

THE DOLLS’ DRESSMAKER DISCOVERS A WORD


A darkened and hushed room; the river outside the windows flowing on
to the vast ocean; a figure on the bed, swathed and bandaged and bound,
lying helpless on its back, with its two useless arms in splints at its
sides. Only two days of usage so familiarized the little dressmaker
with this scene, that it held the place occupied two days ago by the
recollections of years.

He had scarcely moved since her arrival. Sometimes his eyes were open,
sometimes closed. When they were open, there was no meaning in their
unwinking stare at one spot straight before them, unless for a moment
the brow knitted into a faint expression of anger, or surprise. Then,
Mortimer Lightwood would speak to him, and on occasions he would be so
far roused as to make an attempt to pronounce his friend’s name. But, in
an instant consciousness was gone again, and no spirit of Eugene was in
Eugene’s crushed outer form.

They provided Jenny with materials for plying her work, and she had a
little table placed at the foot of his bed. Sitting there, with her rich
shower of hair falling over the chair-back, they hoped she might attract
his notice. With the same object, she would sing, just above her breath,
when he opened his eyes, or she saw his brow knit into that faint
expression, so evanescent that it was like a shape made in water. But
as yet he had not heeded. The ‘they’ here mentioned were the medical
attendant; Lizzie, who was there in all her intervals of rest; and
Lightwood, who never left him.

The two days became three, and the three days became four. At length,
quite unexpectedly, he said something in a whisper.

‘What was it, my dear Eugene?’

‘Will you, Mortimer--’

‘Will I--?

--‘Send for her?’

‘My dear fellow, she is here.’

Quite unconscious of the long blank, he supposed that they were still
speaking together.

The little dressmaker stood up at the foot of the bed, humming her song,
and nodded to him brightly. ‘I can’t shake hands, Jenny,’ said Eugene,
with something of his old look; ‘but I am very glad to see you.’

Mortimer repeated this to her, for it could only be made out by bending
over him and closely watching his attempts to say it. In a little while,
he added:

‘Ask her if she has seen the children.’

Mortimer could not understand this, neither could Jenny herself, until
he added:

‘Ask her if she has smelt the flowers.’

‘Oh! I know!’ cried Jenny. ‘I understand him now!’ Then, Lightwood
yielded his place to her quick approach, and she said, bending over the
bed, with that better look: ‘You mean my long bright slanting rows of
children, who used to bring me ease and rest? You mean the children who
used to take me up, and make me light?’

Eugene smiled, ‘Yes.’

‘I have not seen them since I saw you. I never see them now, but I am
hardly ever in pain now.’

‘It was a pretty fancy,’ said Eugene.

‘But I have heard my birds sing,’ cried the little creature, ‘and I have
smelt my flowers. Yes, indeed I have! And both were most beautiful and
most Divine!’

‘Stay and help to nurse me,’ said Eugene, quietly. ‘I should like you to
have the fancy here, before I die.’

She touched his lips with her hand, and shaded her eyes with that same
hand as she went back to her work and her little low song. He heard the
song with evident pleasure, until she allowed it gradually to sink away
into silence.

‘Mortimer.’

‘My dear Eugene.’

‘If you can give me anything to keep me here for only a few minutes--’

‘To keep you here, Eugene?’

‘To prevent my wandering away I don’t know where--for I begin to be
sensible that I have just come back, and that I shall lose myself
again--do so, dear boy!’

Mortimer gave him such stimulants as could be given him with safety
(they were always at hand, ready), and bending over him once more, was
about to caution him, when he said:

‘Don’t tell me not to speak, for I must speak. If you knew the
harassing anxiety that gnaws and wears me when I am wandering in those
places--where are those endless places, Mortimer? They must be at an
immense distance!’

He saw in his friend’s face that he was losing himself; for he added
after a moment: ‘Don’t be afraid--I am not gone yet. What was it?’

‘You wanted to tell me something, Eugene. My poor dear fellow, you
wanted to say something to your old friend--to the friend who has always
loved you, admired you, imitated you, founded himself upon you, been
nothing without you, and who, God knows, would be here in your place if
he could!’

‘Tut, tut!’ said Eugene with a tender glance as the other put his hand
before his face. ‘I am not worth it. I acknowledge that I like it,
dear boy, but I am not worth it. This attack, my dear Mortimer; this
murder--’

His friend leaned over him with renewed attention, saying: ‘You and I
suspect some one.’

‘More than suspect. But, Mortimer, while I lie here, and when I lie
here no longer, I trust to you that the perpetrator is never brought to
justice.’

‘Eugene?’

‘Her innocent reputation would be ruined, my friend. She would be
punished, not he. I have wronged her enough in fact; I have wronged her
still more in intention. You recollect what pavement is said to be made
of good intentions. It is made of bad intentions too. Mortimer, I am
lying on it, and I know!’

‘Be comforted, my dear Eugene.’

‘I will, when you have promised me. Dear Mortimer, the man must never be
pursued. If he should be accused, you must keep him silent and save
him. Don’t think of avenging me; think only of hushing the story
and protecting her. You can confuse the case, and turn aside the
circumstances. Listen to what I say to you. It was not the schoolmaster,
Bradley Headstone. Do you hear me? Twice; it was not the schoolmaster,
Bradley Headstone. Do you hear me? Three times; it was not the
schoolmaster, Bradley Headstone.’

He stopped, exhausted. His speech had been whispered, broken, and
indistinct; but by a great effort he had made it plain enough to be
unmistakeable.

‘Dear fellow, I am wandering away. Stay me for another moment, if you
can.’

Lightwood lifted his head at the neck, and put a wine-glass to his lips.
He rallied.

‘I don’t know how long ago it was done, whether weeks, days, or hours.
No matter. There is inquiry on foot, and pursuit. Say! Is there not?’

‘Yes.’

‘Check it; divert it! Don’t let her be brought in question. Shield
her. The guilty man, brought to justice, would poison her name. Let the
guilty man go unpunished. Lizzie and my reparation before all! Promise
me!’

‘Eugene, I do. I promise you!’

In the act of turning his eyes gratefully towards his friend, he
wandered away. His eyes stood still, and settled into that former intent
unmeaning stare.

Hours and hours, days and nights, he remained in this same condition.
There were times when he would calmly speak to his friend after a long
period of unconsciousness, and would say he was better, and would ask
for something. Before it could be given him, he would be gone again.

The dolls’ dressmaker, all softened compassion now, watched him with an
earnestness that never relaxed. She would regularly change the ice, or
the cooling spirit, on his head, and would keep her ear at the pillow
betweenwhiles, listening for any faint words that fell from him in his
wanderings. It was amazing through how many hours at a time she would
remain beside him, in a crouching attitude, attentive to his slightest
moan. As he could not move a hand, he could make no sign of distress;
but, through this close watching (if through no secret sympathy or
power) the little creature attained an understanding of him that
Lightwood did not possess. Mortimer would often turn to her, as if she
were an interpreter between this sentient world and the insensible man;
and she would change the dressing of a wound, or ease a ligature, or
turn his face, or alter the pressure of the bedclothes on him, with an
absolute certainty of doing right. The natural lightness and delicacy of
touch which had become very refined by practice in her miniature work,
no doubt was involved in this; but her perception was at least as fine.

The one word, Lizzie, he muttered millions of times. In a certain phase
of his distressful state, which was the worst to those who tended him,
he would roll his head upon the pillow, incessantly repeating the name
in a hurried and impatient manner, with the misery of a disturbed mind,
and the monotony of a machine. Equally, when he lay still and staring,
he would repeat it for hours without cessation, but then, always in a
tone of subdued warning and horror. Her presence and her touch upon his
breast or face would often stop this, and then they learned to expect
that he would for some time remain still, with his eyes closed, and that
he would be conscious on opening them. But, the heavy disappointment of
their hope--revived by the welcome silence of the room--was, that his
spirit would glide away again and be lost, in the moment of their joy
that it was there.

This frequent rising of a drowning man from the deep, to sink again, was
dreadful to the beholders. But, gradually the change stole upon him that
it became dreadful to himself. His desire to impart something that was
on his mind, his unspeakable yearning to have speech with his friend
and make a communication to him, so troubled him when he recovered
consciousness, that its term was thereby shortened. As the man rising
from the deep would disappear the sooner for fighting with the water, so
he in his desperate struggle went down again.

One afternoon when he had been lying still, and Lizzie, unrecognized,
had just stolen out of the room to pursue her occupation, he uttered
Lightwood’s name.

‘My dear Eugene, I am here.’

‘How long is this to last, Mortimer?’

Lightwood shook his head. ‘Still, Eugene, you are no worse than you
were.’

‘But I know there’s no hope. Yet I pray it may last long enough for you
to do me one last service, and for me to do one last action. Keep me
here a few moments, Mortimer. Try, try!’

His friend gave him what aid he could, and encouraged him to believe
that he was more composed, though even then his eyes were losing the
expression they so rarely recovered.

‘Hold me here, dear fellow, if you can. Stop my wandering away. I am
going!’

‘Not yet, not yet. Tell me, dear Eugene, what is it I shall do?’

‘Keep me here for only a single minute. I am going away again. Don’t let
me go. Hear me speak first. Stop me--stop me!’

‘My poor Eugene, try to be calm.’

‘I do try. I try so hard. If you only knew how hard! Don’t let me wander
till I have spoken. Give me a little more wine.’

Lightwood complied. Eugene, with a most pathetic struggle against the
unconsciousness that was coming over him, and with a look of appeal that
affected his friend profoundly, said:

‘You can leave me with Jenny, while you speak to her and tell her what I
beseech of her. You can leave me with Jenny, while you are gone. There’s
not much for you to do. You won’t be long away.’

‘No, no, no. But tell me what it is that I shall do, Eugene!’

‘I am going! You can’t hold me.’

‘Tell me in a word, Eugene!’

His eyes were fixed again, and the only word that came from his lips was
the word millions of times repeated. Lizzie, Lizzie, Lizzie.

But, the watchful little dressmaker had been vigilant as ever in her
watch, and she now came up and touched Lightwood’s arm as he looked down
at his friend, despairingly.

‘Hush!’ she said, with her finger on her lips. ‘His eyes are closing.
He’ll be conscious when he next opens them. Shall I give you a leading
word to say to him?’

‘O Jenny, if you could only give me the right word!’

‘I can. Stoop down.’

He stooped, and she whispered in his ear. She whispered in his ear one
short word of a single syllable. Lightwood started, and looked at her.

‘Try it,’ said the little creature, with an excited and exultant face.
She then bent over the unconscious man, and, for the first time, kissed
him on the cheek, and kissed the poor maimed hand that was nearest to
her. Then, she withdrew to the foot of the bed.

Some two hours afterwards, Mortimer Lightwood saw his consciousness come
back, and instantly, but very tranquilly, bent over him.

‘Don’t speak, Eugene. Do no more than look at me, and listen to me. You
follow what I say.’

He moved his head in assent.

‘I am going on from the point where we broke off. Is the word we should
soon have come to--is it--Wife?’

‘O God bless you, Mortimer!’

‘Hush! Don’t be agitated. Don’t speak. Hear me, dear Eugene. Your mind
will be more at peace, lying here, if you make Lizzie your wife. You
wish me to speak to her, and tell her so, and entreat her to be your
wife. You ask her to kneel at this bedside and be married to you, that
your reparation may be complete. Is that so?’

‘Yes. God bless you! Yes.’

‘It shall be done, Eugene. Trust it to me. I shall have to go away
for some few hours, to give effect to your wishes. You see this is
unavoidable?’

‘Dear friend, I said so.’

‘True. But I had not the clue then. How do you think I got it?’

Glancing wistfully around, Eugene saw Miss Jenny at the foot of the bed,
looking at him with her elbows on the bed, and her head upon her hands.
There was a trace of his whimsical air upon him, as he tried to smile at
her.

‘Yes indeed,’ said Lightwood, ‘the discovery was hers. Observe my dear
Eugene; while I am away you will know that I have discharged my trust
with Lizzie, by finding her here, in my present place at your bedside,
to leave you no more. A final word before I go. This is the right course
of a true man, Eugene. And I solemnly believe, with all my soul, that if
Providence should mercifully restore you to us, you will be blessed with
a noble wife in the preserver of your life, whom you will dearly love.’

‘Amen. I am sure of that. But I shall not come through it, Mortimer.’

‘You will not be the less hopeful or less strong, for this, Eugene.’

‘No. Touch my face with yours, in case I should not hold out till you
come back. I love you, Mortimer. Don’t be uneasy for me while you are
gone. If my dear brave girl will take me, I feel persuaded that I shall
live long enough to be married, dear fellow.’

Miss Jenny gave up altogether on this parting taking place between the
friends, and sitting with her back towards the bed in the bower made by
her bright hair, wept heartily, though noiselessly. Mortimer Lightwood
was soon gone. As the evening light lengthened the heavy reflections of
the trees in the river, another figure came with a soft step into the
sick room.

‘Is he conscious?’ asked the little dressmaker, as the figure took its
station by the pillow. For, Jenny had given place to it immediately, and
could not see the sufferer’s face, in the dark room, from her new and
removed position.

‘He is conscious, Jenny,’ murmured Eugene for himself. ‘He knows his
wife.’



Chapter 11

EFFECT IS GIVEN TO THE DOLLS’ DRESSMAKER’S DISCOVERY


Mrs John Rokesmith sat at needlework in her neat little room, beside a
basket of neat little articles of clothing, which presented so much of
the appearance of being in the dolls’ dressmaker’s way of business, that
one might have supposed she was going to set up in opposition to Miss
Wren. Whether the Complete British Family Housewife had imparted sage
counsel anent them, did not appear, but probably not, as that cloudy
oracle was nowhere visible. For certain, however, Mrs John Rokesmith
stitched at them with so dexterous a hand, that she must have taken
lessons of somebody. Love is in all things a most wonderful teacher,
and perhaps love (from a pictorial point of view, with nothing on but
a thimble), had been teaching this branch of needlework to Mrs John
Rokesmith.

It was near John’s time for coming home, but as Mrs John was desirous to
finish a special triumph of her skill before dinner, she did not go out
to meet him. Placidly, though rather consequentially smiling, she sat
stitching away with a regular sound, like a sort of dimpled little
charming Dresden-china clock by the very best maker.

A knock at the door, and a ring at the bell. Not John; or Bella would
have flown out to meet him. Then who, if not John? Bella was asking
herself the question, when that fluttering little fool of a servant
fluttered in, saying, ‘Mr Lightwood!’

Oh good gracious!

Bella had but time to throw a handkerchief over the basket, when Mr
Lightwood made his bow. There was something amiss with Mr Lightwood, for
he was strangely grave and looked ill.

With a brief reference to the happy time when it had been his privilege
to know Mrs Rokesmith as Miss Wilfer, Mr Lightwood explained what was
amiss with him and why he came. He came bearing Lizzie Hexam’s earnest
hope that Mrs John Rokesmith would see her married.

Bella was so fluttered by the request, and by the short narrative he had
feelingly given her, that there never was a more timely smelling-bottle
than John’s knock. ‘My husband,’ said Bella; ‘I’ll bring him in.’

But, that turned out to be more easily said than done; for, the instant
she mentioned Mr Lightwood’s name, John stopped, with his hand upon the
lock of the room door.

‘Come up stairs, my darling.’

Bella was amazed by the flush in his face, and by his sudden turning
away. ‘What can it mean?’ she thought, as she accompanied him up stairs.

‘Now, my life,’ said John, taking her on his knee, ‘tell me all about
it.’

All very well to say, ‘Tell me all about it;’ but John was very much
confused. His attention evidently trailed off, now and then, even while
Bella told him all about it. Yet she knew that he took a great interest
in Lizzie and her fortunes. What could it mean?

‘You will come to this marriage with me, John dear?’

‘N--no, my love; I can’t do that.’

‘You can’t do that, John?’

‘No, my dear, it’s quite out of the question. Not to be thought of.’

‘Am I to go alone, John?’

‘No, my dear, you will go with Mr Lightwood.’

‘Don’t you think it’s time we went down to Mr Lightwood, John dear?’
Bella insinuated.

‘My darling, it’s almost time you went, but I must ask you to excuse me
to him altogether.’

‘You never mean, John dear, that you are not going to see him? Why, he
knows you have come home. I told him so.’

‘That’s a little unfortunate, but it can’t be helped. Unfortunate or
fortunate, I positively cannot see him, my love.’

Bella cast about in her mind what could be his reason for this
unaccountable behaviour; as she sat on his knee looking at him in
astonishment and pouting a little. A weak reason presented itself.

‘John dear, you never can be jealous of Mr Lightwood?’

‘Why, my precious child,’ returned her husband, laughing outright: ‘how
could I be jealous of him? Why should I be jealous of him?’

‘Because, you know, John,’ pursued Bella, pouting a little more, ‘though
he did rather admire me once, it was not my fault.’

‘It was your fault that I admired you,’ returned her husband, with a
look of pride in her, ‘and why not your fault that he admired you? But,
I jealous on that account? Why, I must go distracted for life, if I
turned jealous of every one who used to find my wife beautiful and
winning!’

‘I am half angry with you, John dear,’ said Bella, laughing a little,
‘and half pleased with you; because you are such a stupid old fellow,
and yet you say nice things, as if you meant them. Don’t be mysterious,
sir. What harm do you know of Mr Lightwood?’

‘None, my love.’

‘What has he ever done to you, John?’

‘He has never done anything to me, my dear. I know no more against
him than I know against Mr Wrayburn; he has never done anything to me;
neither has Mr Wrayburn. And yet I have exactly the same objection to
both of them.’

‘Oh, John!’ retorted Bella, as if she were giving him up for a bad job,
as she used to give up herself. ‘You are nothing better than a sphinx!
And a married sphinx isn’t a--isn’t a nice confidential husband,’ said
Bella, in a tone of injury.

‘Bella, my life,’ said John Rokesmith, touching her cheek, with a grave
smile, as she cast down her eyes and pouted again; ‘look at me. I want
to speak to you.’

‘In earnest, Blue Beard of the secret chamber?’ asked Bella, clearing
her pretty face.

‘In earnest. And I confess to the secret chamber. Don’t you remember
that you asked me not to declare what I thought of your higher qualities
until you had been tried?’

‘Yes, John dear. And I fully meant it, and I fully mean it.’

‘The time will come, my darling--I am no prophet, but I say so,--when
you WILL be tried. The time will come, I think, when you will undergo
a trial through which you will never pass quite triumphantly for me,
unless you can put perfect faith in me.’

‘Then you may be sure of me, John dear, for I can put perfect faith in
you, and I do, and I always, always will. Don’t judge me by a little
thing like this, John. In little things, I am a little thing myself--I
always was. But in great things, I hope not; I don’t mean to boast, John
dear, but I hope not!’

He was even better convinced of the truth of what she said than she was,
as he felt her loving arms about him. If the Golden Dustman’s riches had
been his to stake, he would have staked them to the last farthing on the
fidelity through good and evil of her affectionate and trusting heart.

‘Now, I’ll go down to, and go away with, Mr Lightwood,’ said Bella,
springing up. ‘You are the most creasing and tumbling Clumsy-Boots of a
packer, John, that ever was; but if you’re quite good, and will promise
never to do so any more (though I don’t know what you have done!) you
may pack me a little bag for a night, while I get my bonnet on.’

He gaily complied, and she tied her dimpled chin up, and shook her head
into her bonnet, and pulled out the bows of her bonnet-strings, and
got her gloves on, finger by finger, and finally got them on her
little plump hands, and bade him good-bye and went down. Mr Lightwood’s
impatience was much relieved when he found her dressed for departure.

‘Mr Rokesmith goes with us?’ he said, hesitating, with a look towards
the door.

‘Oh, I forgot!’ replied Bella. ‘His best compliments. His face is
swollen to the size of two faces, and he is to go to bed directly, poor
fellow, to wait for the doctor, who is coming to lance him.’

‘It is curious,’ observed Lightwood, ‘that I have never yet seen Mr
Rokesmith, though we have been engaged in the same affairs.’

‘Really?’ said the unblushing Bella.

‘I begin to think,’ observed Lightwood, ‘that I never shall see him.’

‘These things happen so oddly sometimes,’ said Bella with a steady
countenance, ‘that there seems a kind of fatality in them. But I am
quite ready, Mr Lightwood.’

They started directly, in a little carriage that Lightwood had brought
with him from never-to-be-forgotten Greenwich; and from Greenwich they
started directly for London; and in London they waited at a railway
station until such time as the Reverend Frank Milvey, and Margaretta
his wife, with whom Mortimer Lightwood had been already in conference,
should come and join them.

That worthy couple were delayed by a portentous old parishioner of the
female gender, who was one of the plagues of their lives, and with whom
they bore with most exemplary sweetness and good-humour, notwithstanding
her having an infection of absurdity about her, that communicated itself
to everything with which, and everybody with whom, she came in contact.
She was a member of the Reverend Frank’s congregation, and made a point
of distinguishing herself in that body, by conspicuously weeping at
everything, however cheering, said by the Reverend Frank in his public
ministration; also by applying to herself the various lamentations of
David, and complaining in a personally injured manner (much in arrear of
the clerk and the rest of the respondents) that her enemies were digging
pit-falls about her, and breaking her with rods of iron. Indeed, this
old widow discharged herself of that portion of the Morning and Evening
Service as if she were lodging a complaint on oath and applying for
a warrant before a magistrate. But this was not her most inconvenient
characteristic, for that took the form of an impression, usually
recurring in inclement weather and at about daybreak, that she had
something on her mind and stood in immediate need of the Reverend Frank
to come and take it off. Many a time had that kind creature got up, and
gone out to Mrs Sprodgkin (such was the disciple’s name), suppressing
a strong sense of her comicality by his strong sense of duty, and
perfectly knowing that nothing but a cold would come of it. However,
beyond themselves, the Reverend Frank Milvey and Mrs Milvey seldom
hinted that Mrs Sprodgkin was hardly worth the trouble she gave; but
both made the best of her, as they did of all their troubles.

This very exacting member of the fold appeared to be endowed with a
sixth sense, in regard of knowing when the Reverend Frank Milvey least
desired her company, and with promptitude appearing in his little hall.
Consequently, when the Reverend Frank had willingly engaged that he and
his wife would accompany Lightwood back, he said, as a matter of course:
‘We must make haste to get out, Margaretta, my dear, or we shall be
descended on by Mrs Sprodgkin.’ To which Mrs Milvey replied, in her
pleasantly emphatic way, ‘Oh YES, for she IS such a marplot, Frank, and
DOES worry so!’ Words that were scarcely uttered when their theme
was announced as in faithful attendance below, desiring counsel on a
spiritual matter. The points on which Mrs Sprodgkin sought elucidation
being seldom of a pressing nature (as Who begat Whom, or some
information concerning the Amorites), Mrs Milvey on this special
occasion resorted to the device of buying her off with a present of tea
and sugar, and a loaf and butter. These gifts Mrs Sprodgkin accepted,
but still insisted on dutifully remaining in the hall, to curtsey to the
Reverend Frank as he came forth. Who, incautiously saying in his genial
manner, ‘Well, Sally, there you are!’ involved himself in a discursive
address from Mrs Sprodgkin, revolving around the result that she
regarded tea and sugar in the light of myrrh and frankincense, and
considered bread and butter identical with locusts and wild honey.
Having communicated this edifying piece of information, Mrs Sprodgkin
was left still unadjourned in the hall, and Mr and Mrs Milvey hurried in
a heated condition to the railway station. All of which is here recorded
to the honour of that good Christian pair, representatives of hundreds
of other good Christian pairs as conscientious and as useful, who merge
the smallness of their work in its greatness, and feel in no danger of
losing dignity when they adapt themselves to incomprehensible humbugs.

‘Detained at the last moment by one who had a claim upon me,’ was the
Reverend Frank’s apology to Lightwood, taking no thought of himself.
To which Mrs Milvey added, taking thought for him, like the championing
little wife she was; ‘Oh yes, detained at the last moment. But AS to
the claim, Frank, I MUST say that I DO think you are OVER-considerate
sometimes, and allow THAT to be a LITTLE abused.’

Bella felt conscious, in spite of her late pledge for herself, that her
husband’s absence would give disagreeable occasion for surprise to the
Milveys. Nor could she appear quite at her ease when Mrs Milvey asked:

‘HOW is Mr Rokesmith, and IS he gone before us, or DOES he follow us?’

It becoming necessary, upon this, to send him to bed again and hold him
in waiting to be lanced again, Bella did it. But not half as well on
the second occasion as on the first; for, a twice-told white one seems
almost to become a black one, when you are not used to it.

‘Oh DEAR!’ said Mrs Milvey, ‘I am SO sorry! Mr Rokesmith took SUCH an
interest in Lizzie Hexam, when we were there before. And if we had ONLY
known of his face, we COULD have given him something that would have
kept it down long enough for so SHORT a purpose.’

By way of making the white one whiter, Bella hastened to stipulate that
he was not in pain. Mrs Milvey was SO glad of it.

‘I don’t know HOW it is,’ said Mrs Milvey, ‘and I am SURE you don’t,
Frank, but the clergy and their wives seem to CAUSE swelled faces.
Whenever I take notice of a child in the school, it seems to me as if
its face swelled INSTANTLY. Frank NEVER makes acquaintance with a new
old woman, but she gets the face-ache. And another thing is, we DO make
the poor children sniff so. I don’t know HOW we do it, and I should
be so glad not to; but the MORE we take notice of them, the MORE they
sniff. Just as they do when the text is given out.--Frank, that’s a
schoolmaster. I have seen him somewhere.’

The reference was to a young man of reserved appearance, in a coat and
waistcoat of black, and pantaloons of pepper and salt. He had come
into the office of the station, from its interior, in an unsettled way,
immediately after Lightwood had gone out to the train; and he had been
hurriedly reading the printed bills and notices on the wall. He had had
a wandering interest in what was said among the people waiting there
and passing to and fro. He had drawn nearer, at about the time when
Mrs Milvey mentioned Lizzie Hexam, and had remained near, since: though
always glancing towards the door by which Lightwood had gone out. He
stood with his back towards them, and his gloved hands clasped behind
him. There was now so evident a faltering upon him, expressive of
indecision whether or no he should express his having heard himself
referred to, that Mr Milvey spoke to him.

‘I cannot recall your name,’ he said, ‘but I remember to have seen you
in your school.’

‘My name is Bradley Headstone, sir,’ he replied, backing into a more
retired place.

‘I ought to have remembered it,’ said Mr Milvey, giving him his hand. ‘I
hope you are well? A little overworked, I am afraid?’

‘Yes, I am overworked just at present, sir.’

‘Had no play in your last holiday time?’

‘No, sir.’

‘All work and no play, Mr Headstone, will not make dulness, in your
case, I dare say; but it will make dyspepsia, if you don’t take care.’

‘I will endeavour to take care, sir. Might I beg leave to speak to you,
outside, a moment?’

‘By all means.’

It was evening, and the office was well lighted. The schoolmaster, who
had never remitted his watch on Lightwood’s door, now moved by another
door to a corner without, where there was more shadow than light; and
said, plucking at his gloves:

‘One of your ladies, sir, mentioned within my hearing a name that I am
acquainted with; I may say, well acquainted with. The name of the sister
of an old pupil of mine. He was my pupil for a long time, and has got on
and gone upward rapidly. The name of Hexam. The name of Lizzie Hexam.’
He seemed to be a shy man, struggling against nervousness, and spoke in
a very constrained way. The break he set between his last two sentences
was quite embarrassing to his hearer.

‘Yes,’ replied Mr Milvey. ‘We are going down to see her.’

‘I gathered as much, sir. I hope there is nothing amiss with the sister
of my old pupil? I hope no bereavement has befallen her. I hope she is
in no affliction? Has lost no--relation?’

Mr Milvey thought this a man with a very odd manner, and a dark downward
look; but he answered in his usual open way.

‘I am glad to tell you, Mr Headstone, that the sister of your old pupil
has not sustained any such loss. You thought I might be going down to
bury some one?’

‘That may have been the connexion of ideas, sir, with your clerical
character, but I was not conscious of it.--Then you are not, sir?’

A man with a very odd manner indeed, and with a lurking look that was
quite oppressive.

‘No. In fact,’ said Mr Milvey, ‘since you are so interested in the
sister of your old pupil, I may as well tell you that I am going down to
marry her.’

The schoolmaster started back.

‘Not to marry her, myself,’ said Mr Milvey, with a smile, ‘because I
have a wife already. To perform the marriage service at her wedding.’

Bradley Headstone caught hold of a pillar behind him. If Mr Milvey knew
an ashy face when he saw it, he saw it then.

‘You are quite ill, Mr Headstone!’

‘It is not much, sir. It will pass over very soon. I am accustomed to be
seized with giddiness. Don’t let me detain you, sir; I stand in need
of no assistance, I thank you. Much obliged by your sparing me these
minutes of your time.’

As Mr Milvey, who had no more minutes to spare, made a suitable reply
and turned back into the office, he observed the schoolmaster to
lean against the pillar with his hat in his hand, and to pull at his
neckcloth as if he were trying to tear it off. The Reverend Frank
accordingly directed the notice of one of the attendants to him, by
saying: ‘There is a person outside who seems to be really ill, and to
require some help, though he says he does not.’

Lightwood had by this time secured their places, and the departure-bell
was about to be rung. They took their seats, and were beginning to
move out of the station, when the same attendant came running along the
platform, looking into all the carriages.

‘Oh! You are here, sir!’ he said, springing on the step, and holding
the window-frame by his elbow, as the carriage moved. ‘That person you
pointed out to me is in a fit.’

‘I infer from what he told me that he is subject to such attacks. He
will come to, in the air, in a little while.’

He was took very bad to be sure, and was biting and knocking about him
(the man said) furiously. Would the gentleman give him his card, as he
had seen him first? The gentleman did so, with the explanation that
he knew no more of the man attacked than that he was a man of a very
respectable occupation, who had said he was out of health, as his
appearance would of itself have indicated. The attendant received the
card, watched his opportunity for sliding down, slid down, and so it
ended.

Then, the train rattled among the house-tops, and among the ragged sides
of houses torn down to make way for it, and over the swarming streets,
and under the fruitful earth, until it shot across the river: bursting
over the quiet surface like a bomb-shell, and gone again as if it had
exploded in the rush of smoke and steam and glare. A little more, and
again it roared across the river, a great rocket: spurning the watery
turnings and doublings with ineffable contempt, and going straight to
its end, as Father Time goes to his. To whom it is no matter what living
waters run high or low, reflect the heavenly lights and darknesses,
produce their little growth of weeds and flowers, turn here, turn there,
are noisy or still, are troubled or at rest, for their course has one
sure termination, though their sources and devices are many.

Then, a carriage ride succeeded, near the solemn river, stealing away
by night, as all things steal away, by night and by day, so quietly
yielding to the attraction of the loadstone rock of Eternity; and the
nearer they drew to the chamber where Eugene lay, the more they feared
that they might find his wanderings done. At last they saw its dim light
shining out, and it gave them hope: though Lightwood faltered as he
thought: ‘If he were gone, she would still be sitting by him.’

But he lay quiet, half in stupor, half in sleep. Bella, entering with
a raised admonitory finger, kissed Lizzie softly, but said not a word.
Neither did any of them speak, but all sat down at the foot of the bed,
silently waiting. And now, in this night-watch, mingling with the flow
of the river and with the rush of the train, came the questions into
Bella’s mind again: What could be in the depths of that mystery of
John’s? Why was it that he had never been seen by Mr Lightwood, whom he
still avoided? When would that trial come, through which her faith
in, and her duty to, her dear husband, was to carry her, rendering him
triumphant? For, that had been his term. Her passing through the trial
was to make the man she loved with all her heart, triumphant. Term not
to sink out of sight in Bella’s breast.

Far on in the night, Eugene opened his eyes. He was sensible, and said
at once: ‘How does the time go? Has our Mortimer come back?’

Lightwood was there immediately, to answer for himself. ‘Yes, Eugene,
and all is ready.’

‘Dear boy!’ returned Eugene with a smile, ‘we both thank you heartily.
Lizzie, tell them how welcome they are, and that I would be eloquent if
I could.’

‘There is no need,’ said Mr Milvey. ‘We know it. Are you better, Mr
Wrayburn?’

‘I am much happier,’ said Eugene.

‘Much better too, I hope?’

Eugene turned his eyes towards Lizzie, as if to spare her, and answered
nothing.

Then, they all stood around the bed, and Mr Milvey, opening his book,
began the service; so rarely associated with the shadow of death; so
inseparable in the mind from a flush of life and gaiety and hope and
health and joy. Bella thought how different from her own sunny little
wedding, and wept. Mrs Milvey overflowed with pity, and wept too. The
dolls’ dressmaker, with her hands before her face, wept in her golden
bower. Reading in a low clear voice, and bending over Eugene, who kept
his eyes upon him, Mr Milvey did his office with suitable simplicity.
As the bridegroom could not move his hand, they touched his fingers with
the ring, and so put it on the bride. When the two plighted their troth,
she laid her hand on his and kept it there. When the ceremony was done,
and all the rest departed from the room, she drew her arm under his
head, and laid her own head down upon the pillow by his side.

‘Undraw the curtains, my dear girl,’ said Eugene, after a while, ‘and
let us see our wedding-day.’

The sun was rising, and his first rays struck into the room, as she came
back, and put her lips to his. ‘I bless the day!’ said Eugene. ‘I bless
the day!’ said Lizzie.

‘You have made a poor marriage of it, my sweet wife,’ said Eugene. ‘A
shattered graceless fellow, stretched at his length here, and next to
nothing for you when you are a young widow.’

‘I have made the marriage that I would have given all the world to dare
to hope for,’ she replied.

‘You have thrown yourself away,’ said Eugene, shaking his head. ‘But you
have followed the treasure of your heart. My justification is, that you
had thrown that away first, dear girl!’

‘No. I had given it to you.’

‘The same thing, my poor Lizzie!’

‘Hush! hush! A very different thing.’

There were tears in his eyes, and she besought him to close them. ‘No,’
said Eugene, again shaking his head; ‘let me look at you, Lizzie, while
I can. You brave devoted girl! You heroine!’

Her own eyes filled under his praises. And when he mustered strength to
move his wounded head a very little way, and lay it on her bosom, the
tears of both fell.

‘Lizzie,’ said Eugene, after a silence: ‘when you see me wandering away
from this refuge that I have so ill deserved, speak to me by my name,
and I think I shall come back.’

‘Yes, dear Eugene.’

‘There!’ he exclaimed, smiling. ‘I should have gone then, but for that!’

A little while afterwards, when he appeared to be sinking into
insensibility, she said, in a calm loving voice: ‘Eugene, my dear
husband!’ He immediately answered: ‘There again! You see how you can
recall me!’ And afterwards, when he could not speak, he still answered
by a slight movement of his head upon her bosom.

The sun was high in the sky, when she gently disengaged herself to give
him the stimulants and nourishment he required. The utter helplessness
of the wreck of him that lay cast ashore there, now alarmed her, but he
himself appeared a little more hopeful.

‘Ah, my beloved Lizzie!’ he said, faintly. ‘How shall I ever pay all I
owe you, if I recover!’

‘Don’t be ashamed of me,’ she replied, ‘and you will have more than paid
all.’

‘It would require a life, Lizzie, to pay all; more than a life.’

‘Live for that, then; live for me, Eugene; live to see how hard I will
try to improve myself, and never to discredit you.’

‘My darling girl,’ he replied, rallying more of his old manner than
he had ever yet got together. ‘On the contrary, I have been thinking
whether it is not the best thing I can do, to die.’

‘The best thing you can do, to leave me with a broken heart?’

‘I don’t mean that, my dear girl. I was not thinking of that. What I was
thinking of was this. Out of your compassion for me, in this maimed and
broken state, you make so much of me--you think so well of me--you love
me so dearly.’

‘Heaven knows I love you dearly!’

‘And Heaven knows I prize it! Well. If I live, you’ll find me out.’

‘I shall find out that my husband has a mine of purpose and energy, and
will turn it to the best account?’

‘I hope so, dearest Lizzie,’ said Eugene, wistfully, and yet somewhat
whimsically. ‘I hope so. But I can’t summon the vanity to think so. How
can I think so, looking back on such a trifling wasted youth as mine! I
humbly hope it; but I daren’t believe it. There is a sharp misgiving
in my conscience that if I were to live, I should disappoint your good
opinion and my own--and that I ought to die, my dear!’



Chapter 12

THE PASSING SHADOW


The winds and tides rose and fell a certain number of times, the earth
moved round the sun a certain number of times, the ship upon the ocean
made her voyage safely, and brought a baby-Bella home. Then who so blest
and happy as Mrs John Rokesmith, saving and excepting Mr John Rokesmith!

‘Would you not like to be rich NOW, my darling?’

‘How can you ask me such a question, John dear? Am I not rich?’

These were among the first words spoken near the baby Bella as she lay
asleep. She soon proved to be a baby of wonderful intelligence,
evincing the strongest objection to her grandmother’s society, and
being invariably seized with a painful acidity of the stomach when that
dignified lady honoured her with any attention.

It was charming to see Bella contemplating this baby, and finding out
her own dimples in that tiny reflection, as if she were looking in the
glass without personal vanity. Her cherubic father justly remarked
to her husband that the baby seemed to make her younger than before,
reminding him of the days when she had a pet doll and used to talk to it
as she carried it about. The world might have been challenged to produce
another baby who had such a store of pleasant nonsense said and sung
to it, as Bella said and sung to this baby; or who was dressed and
undressed as often in four-and-twenty hours as Bella dressed and
undressed this baby; or who was held behind doors and poked out to stop
its father’s way when he came home, as this baby was; or, in a word, who
did half the number of baby things, through the lively invention of a
gay and proud young mother, that this inexhaustible baby did.

The inexhaustible baby was two or three months old, when Bella began to
notice a cloud upon her husband’s brow. Watching it, she saw a gathering
and deepening anxiety there, which caused her great disquiet. More than
once, she awoke him muttering in his sleep; and, though he muttered
nothing worse than her own name, it was plain to her that his
restlessness originated in some load of care. Therefore, Bella at length
put in her claim to divide this load, and hear her half of it.

‘You know, John dear,’ she said, cheerily reverting to their former
conversation, ‘that I hope I may safely be trusted in great things. And
it surely cannot be a little thing that causes you so much uneasiness.
It’s very considerate of you to try to hide from me that you are
uncomfortable about something, but it’s quite impossible to be done,
John love.’

‘I admit that I am rather uneasy, my own.’

‘Then please to tell me what about, sir.’

But no, he evaded that. ‘Never mind!’ thought Bella, resolutely.
‘John requires me to put perfect faith in him, and he shall not be
disappointed.’

She went up to London one day, to meet him, in order that they might
make some purchases. She found him waiting for her at her journey’s
end, and they walked away together through the streets. He was in gay
spirits, though still harping on that notion of their being rich; and
he said, now let them make believe that yonder fine carriage was theirs,
and that it was waiting to take them home to a fine house they had; what
would Bella, in that case, best like to find in the house? Well! Bella
didn’t know: already having everything she wanted, she couldn’t say.
But, by degrees she was led on to confess that she would like to have
for the inexhaustible baby such a nursery as never was seen. It was
to be ‘a very rainbow for colours’, as she was quite sure baby noticed
colours; and the staircase was to be adorned with the most exquisite
flowers, as she was absolutely certain baby noticed flowers; and there
was to be an aviary somewhere, of the loveliest little birds, as there
was not the smallest doubt in the world that baby noticed birds.
Was there nothing else? No, John dear. The predilections of the
inexhaustible baby being provided for, Bella could think of nothing
else.

They were chatting on in this way, and John had suggested, ‘No jewels
for your own wear, for instance?’ and Bella had replied laughing. O! if
he came to that, yes, there might be a beautiful ivory case of jewels
on her dressing-table; when these pictures were in a moment darkened and
blotted out.

They turned a corner, and met Mr Lightwood.

He stopped as if he were petrified by the sight of Bella’s husband, who
in the same moment had changed colour.

‘Mr Lightwood and I have met before,’ he said.

‘Met before, John?’ Bella repeated in a tone of wonder. ‘Mr Lightwood
told me he had never seen you.’

‘I did not then know that I had,’ said Lightwood, discomposed on her
account. ‘I believed that I had only heard of--Mr Rokesmith.’ With an
emphasis on the name.

‘When Mr Lightwood saw me, my love,’ observed her husband, not avoiding
his eye, but looking at him, ‘my name was Julius Handford.’

Julius Handford! The name that Bella had so often seen in old
newspapers, when she was an inmate of Mr Boffin’s house! Julius
Handford, who had been publicly entreated to appear, and for
intelligence of whom a reward had been publicly offered!

‘I would have avoided mentioning it in your presence,’ said Lightwood to
Bella, delicately; ‘but since your husband mentions it himself, I must
confirm his strange admission. I saw him as Mr Julius Handford, and I
afterwards (unquestionably to his knowledge) took great pains to trace
him out.’

‘Quite true. But it was not my object or my interest,’ said Rokesmith,
quietly, ‘to be traced out.’

Bella looked from the one to the other, in amazement.

‘Mr Lightwood,’ pursued her husband, ‘as chance has brought us face to
face at last--which is not to be wondered at, for the wonder is, that,
in spite of all my pains to the contrary, chance has not confronted
us together sooner--I have only to remind you that you have been at my
house, and to add that I have not changed my residence.’

‘Sir’ returned Lightwood, with a meaning glance towards Bella, ‘my
position is a truly painful one. I hope that no complicity in a very
dark transaction may attach to you, but you cannot fail to know that
your own extraordinary conduct has laid you under suspicion.’

‘I know it has,’ was all the reply.

‘My professional duty,’ said Lightwood hesitating, with another glance
towards Bella, ‘is greatly at variance with my personal inclination; but
I doubt, Mr Handford, or Mr Rokesmith, whether I am justified in taking
leave of you here, with your whole course unexplained.’

Bella caught her husband by the hand.

‘Don’t be alarmed, my darling. Mr Lightwood will find that he is quite
justified in taking leave of me here. At all events,’ added Rokesmith,
‘he will find that I mean to take leave of him here.’

‘I think, sir,’ said Lightwood, ‘you can scarcely deny that when I came
to your house on the occasion to which you have referred, you avoided me
of a set purpose.’

‘Mr Lightwood, I assure you I have no disposition to deny it, or
intention to deny it. I should have continued to avoid you, in pursuance
of the same set purpose, for a short time longer, if we had not met now.
I am going straight home, and shall remain at home to-morrow until noon.
Hereafter, I hope we may be better acquainted. Good-day.’

Lightwood stood irresolute, but Bella’s husband passed him in the
steadiest manner, with Bella on his arm; and they went home without
encountering any further remonstrance or molestation from any one.

When they had dined and were alone, John Rokesmith said to his wife, who
had preserved her cheerfulness: ‘And you don’t ask me, my dear, why I
bore that name?’

‘No, John love. I should dearly like to know, of course;’ (which her
anxious face confirmed;) ‘but I wait until you can tell me of your own
free will. You asked me if I could have perfect faith in you, and I said
yes, and I meant it.’

It did not escape Bella’s notice that he began to look triumphant. She
wanted no strengthening in her firmness; but if she had had need of any,
she would have derived it from his kindling face.

‘You cannot have been prepared, my dearest, for such a discovery as that
this mysterious Mr Handford was identical with your husband?’

‘No, John dear, of course not. But you told me to prepare to be tried,
and I prepared myself.’

He drew her to nestle closer to him, and told her it would soon be over,
and the truth would soon appear. ‘And now,’ he went on, ‘lay stress,
my dear, on these words that I am going to add. I stand in no kind of
peril, and I can by possibility be hurt at no one’s hand.’

‘You are quite, quite sure of that, John dear?’

‘Not a hair of my head! Moreover, I have done no wrong, and have injured
no man. Shall I swear it?’

‘No, John!’ cried Bella, laying her hand upon his lips, with a proud
look. ‘Never to me!’

‘But circumstances,’ he went on ‘--I can, and I will, disperse them in
a moment--have surrounded me with one of the strangest suspicions ever
known. You heard Mr Lightwood speak of a dark transaction?’

‘Yes, John.’

‘You are prepared to hear explicitly what he meant?’

‘Yes, John.’

‘My life, he meant the murder of John Harmon, your allotted husband.’

With a fast palpitating heart, Bella grasped him by the arm. ‘You cannot
be suspected, John?’

‘Dear love, I can be--for I am!’

There was silence between them, as she sat looking in his face, with the
colour quite gone from her own face and lips. ‘How dare they!’ she cried
at length, in a burst of generous indignation. ‘My beloved husband, how
dare they!’

He caught her in his arms as she opened hers, and held her to his heart.
‘Even knowing this, you can trust me, Bella?’

‘I can trust you, John dear, with all my soul. If I could not trust you,
I should fall dead at your feet.’

The kindling triumph in his face was bright indeed, as he looked up and
rapturously exclaimed, what had he done to deserve the blessing of this
dear confiding creature’s heart! Again she put her hand upon his lips,
saying, ‘Hush!’ and then told him, in her own little natural pathetic
way, that if all the world were against him, she would be for him; that
if all the world repudiated him, she would believe him; that if he were
infamous in other eyes, he would be honoured in hers; and that, under
the worst unmerited suspicion, she could devote her life to consoling
him, and imparting her own faith in him to their little child.

A twilight calm of happiness then succeeding to their radiant noon, they
remained at peace, until a strange voice in the room startled them both.
The room being by that time dark, the voice said, ‘Don’t let the lady
be alarmed by my striking a light,’ and immediately a match rattled, and
glimmered in a hand. The hand and the match and the voice were then seen
by John Rokesmith to belong to Mr Inspector, once meditatively active in
this chronicle.

‘I take the liberty,’ said Mr Inspector, in a business-like manner, ‘to
bring myself to the recollection of Mr Julius Handford, who gave me his
name and address down at our place a considerable time ago. Would the
lady object to my lighting the pair of candles on the chimneypiece, to
throw a further light upon the subject? No? Thank you, ma’am. Now, we
look cheerful.’

Mr Inspector, in a dark-blue buttoned-up frock coat and pantaloons,
presented a serviceable, half-pay, Royal Arms kind of appearance, as he
applied his pocket handkerchief to his nose and bowed to the lady.

‘You favoured me, Mr Handford,’ said Mr Inspector, ‘by writing down your
name and address, and I produce the piece of paper on which you wrote
it. Comparing the same with the writing on the fly-leaf of this book on
the table--and a sweet pretty volume it is--I find the writing of the
entry, “Mrs John Rokesmith. From her husband on her birthday”--and very
gratifying to the feelings such memorials are--to correspond exactly.
Can I have a word with you?’

‘Certainly. Here, if you please,’ was the reply.

‘Why,’ retorted Mr Inspector, again using his pocket handkerchief,
‘though there’s nothing for the lady to be at all alarmed at, still,
ladies are apt to take alarm at matters of business--being of that
fragile sex that they’re not accustomed to them when not of a strictly
domestic character--and I do generally make it a rule to propose
retirement from the presence of ladies, before entering upon business
topics. Or perhaps,’ Mr Inspector hinted, ‘if the lady was to step
up-stairs, and take a look at baby now!’

‘Mrs Rokesmith,’--her husband was beginning; when Mr Inspector,
regarding the words as an introduction, said, ‘Happy I am sure, to have
the honour.’ And bowed, with gallantry.

‘Mrs Rokesmith,’ resumed her husband, ‘is satisfied that she can have no
reason for being alarmed, whatever the business is.’

‘Really? Is that so?’ said Mr Inspector. ‘But it’s a sex to live and
learn from, and there’s nothing a lady can’t accomplish when she once
fully gives her mind to it. It’s the case with my own wife. Well, ma’am,
this good gentleman of yours has given rise to a rather large amount
of trouble which might have been avoided if he had come forward and
explained himself. Well you see! He DIDN’T come forward and explain
himself. Consequently, now that we meet, him and me, you’ll say--and say
right--that there’s nothing to be alarmed at, in my proposing to him
TO come forward--or, putting the same meaning in another form, to come
along with me--and explain himself.’

When Mr Inspector put it in that other form, ‘to come along with me,’
there was a relishing roll in his voice, and his eye beamed with an
official lustre.

‘Do you propose to take me into custody?’ inquired John Rokesmith, very
coolly.

‘Why argue?’ returned Mr Inspector in a comfortable sort of
remonstrance; ‘ain’t it enough that I propose that you shall come along
with me?’

‘For what reason?’

‘Lord bless my soul and body!’ returned Mr Inspector, ‘I wonder at it in
a man of your education. Why argue?’

‘What do you charge against me?’

‘I wonder at you before a lady,’ said Mr Inspector, shaking his head
reproachfully: ‘I wonder, brought up as you have been, you haven’t a
more delicate mind! I charge you, then, with being some way concerned
in the Harmon Murder. I don’t say whether before, or in, or after, the
fact. I don’t say whether with having some knowledge of it that hasn’t
come out.’

‘You don’t surprise me. I foresaw your visit this afternoon.’

‘Don’t!’ said Mr Inspector. ‘Why, why argue? It’s my duty to inform you
that whatever you say, will be used against you.’

‘I don’t think it will.’

‘But I tell you it will,’ said Mr Inspector. ‘Now, having received the
caution, do you still say that you foresaw my visit this afternoon?’

‘Yes. And I will say something more, if you will step with me into the
next room.’

With a reassuring kiss on the lips of the frightened Bella, her husband
(to whom Mr Inspector obligingly offered his arm), took up a candle, and
withdrew with that gentleman. They were a full half-hour in conference.
When they returned, Mr Inspector looked considerably astonished.

‘I have invited this worthy officer, my dear,’ said John, ‘to make a
short excursion with me in which you shall be a sharer. He will take
something to eat and drink, I dare say, on your invitation, while you
are getting your bonnet on.’

Mr Inspector declined eating, but assented to the proposal of a glass of
brandy and water. Mixing this cold, and pensively consuming it, he broke
at intervals into such soliloquies as that he never did know such a
move, that he never had been so gravelled, and that what a game was
this to try the sort of stuff a man’s opinion of himself was made
of! Concurrently with these comments, he more than once burst out a
laughing, with the half-enjoying and half-piqued air of a man, who
had given up a good conundrum, after much guessing, and been told the
answer. Bella was so timid of him, that she noted these things in a
half-shrinking, half-perceptive way, and similarly noted that there was
a great change in his manner towards John. That coming-along-with-him
deportment was now lost in long musing looks at John and at herself and
sometimes in slow heavy rubs of his hand across his forehead, as if he
were ironing cut the creases which his deep pondering made there. He had
had some coughing and whistling satellites secretly gravitating towards
him about the premises, but they were now dismissed, and he eyed John as
if he had meant to do him a public service, but had unfortunately been
anticipated. Whether Bella might have noted anything more, if she
had been less afraid of him, she could not determine; but it was all
inexplicable to her, and not the faintest flash of the real state of the
case broke in upon her mind. Mr Inspector’s increased notice of herself
and knowing way of raising his eyebrows when their eyes by any chance
met, as if he put the question ‘Don’t you see?’ augmented her timidity,
and, consequently, her perplexity. For all these reasons, when he
and she and John, at towards nine o’clock of a winter evening went to
London, and began driving from London Bridge, among low-lying water-side
wharves and docks and strange places, Bella was in the state of a
dreamer; perfectly unable to account for her being there, perfectly
unable to forecast what would happen next, or whither she was going, or
why; certain of nothing in the immediate present, but that she confided
in John, and that John seemed somehow to be getting more triumphant. But
what a certainty was that!

They alighted at last at the corner of a court, where there was a
building with a bright lamp and wicket gate. Its orderly appearance was
very unlike that of the surrounding neighbourhood, and was explained by
the inscription POLICE STATION.

‘We are not going in here, John?’ said Bella, clinging to him.

‘Yes, my dear; but of our own accord. We shall come out again as easily,
never fear.’

The whitewashed room was pure white as of old, the methodical
book-keeping was in peaceful progress as of old, and some distant howler
was banging against a cell door as of old. The sanctuary was not a
permanent abiding-place, but a kind of criminal Pickford’s. The lower
passions and vices were regularly ticked off in the books, warehoused in
the cells, carted away as per accompanying invoice, and left little mark
upon it.

Mr Inspector placed two chairs for his visitors, before the fire, and
communed in a low voice with a brother of his order (also of a half-pay,
and Royal Arms aspect), who, judged only by his occupation at the
moment, might have been a writing-master, setting copies. Their
conference done, Mr Inspector returned to the fireplace, and, having
observed that he would step round to the Fellowships and see how matters
stood, went out. He soon came back again, saying, ‘Nothing could be
better, for they’re at supper with Miss Abbey in the bar;’ and then they
all three went out together.

Still, as in a dream, Bella found herself entering a snug old-fashioned
public-house, and found herself smuggled into a little three-cornered
room nearly opposite the bar of that establishment. Mr Inspector
achieved the smuggling of herself and John into this queer room, called
Cosy in an inscription on the door, by entering in the narrow passage
first in order, and suddenly turning round upon them with extended arms,
as if they had been two sheep. The room was lighted for their reception.

‘Now,’ said Mr Inspector to John, turning the gas lower; ‘I’ll mix with
‘em in a casual way, and when I say Identification, perhaps you’ll show
yourself.’

John nodded, and Mr Inspector went alone to the half-door of the bar.
From the dim doorway of Cosy, within which Bella and her husband stood,
they could see a comfortable little party of three persons sitting at
supper in the bar, and could hear everything that was said.

The three persons were Miss Abbey and two male guests. To whom
collectively, Mr Inspector remarked that the weather was getting sharp
for the time of year.

‘It need be sharp to suit your wits, sir,’ said Miss Abbey. ‘What have
you got in hand now?’

‘Thanking you for your compliment: not much, Miss Abbey,’ was Mr
Inspector’s rejoinder.

‘Who have you got in Cosy?’ asked Miss Abbey.

‘Only a gentleman and his wife, Miss.’

‘And who are they? If one may ask it without detriment to your deep
plans in the interests of the honest public?’ said Miss Abbey, proud of
Mr Inspector as an administrative genius.

‘They are strangers in this part of the town, Miss Abbey. They are
waiting till I shall want the gentleman to show himself somewhere, for
half a moment.’

‘While they’re waiting,’ said Miss Abbey, ‘couldn’t you join us?’

Mr Inspector immediately slipped into the bar, and sat down at the side
of the half-door, with his back towards the passage, and directly facing
the two guests. ‘I don’t take my supper till later in the night,’ said
he, ‘and therefore I won’t disturb the compactness of the table. But
I’ll take a glass of flip, if that’s flip in the jug in the fender.’

‘That’s flip,’ replied Miss Abbey, ‘and it’s my making, and if even you
can find out better, I shall be glad to know where.’ Filling him, with
hospitable hands, a steaming tumbler, Miss Abbey replaced the jug by
the fire; the company not having yet arrived at the flip-stage of their
supper, but being as yet skirmishing with strong ale.

‘Ah--h!’ cried Mr Inspector. ‘That’s the smack! There’s not a Detective
in the Force, Miss Abbey, that could find out better stuff than that.’

‘Glad to hear you say so,’ rejoined Miss Abbey. ‘You ought to know, if
anybody does.’

‘Mr Job Potterson,’ Mr Inspector continued, ‘I drink your health. Mr
Jacob Kibble, I drink yours. Hope you have made a prosperous voyage
home, gentlemen both.’

Mr Kibble, an unctuous broad man of few words and many mouthfuls, said,
more briefly than pointedly, raising his ale to his lips: ‘Same to you.’
Mr Job Potterson, a semi-seafaring man of obliging demeanour, said,
‘Thank you, sir.’

‘Lord bless my soul and body!’ cried Mr Inspector. ‘Talk of trades, Miss
Abbey, and the way they set their marks on men’ (a subject which nobody
had approached); ‘who wouldn’t know your brother to be a Steward!
There’s a bright and ready twinkle in his eye, there’s a neatness in his
action, there’s a smartness in his figure, there’s an air of reliability
about him in case you wanted a basin, which points out the steward! And
Mr Kibble; ain’t he Passenger, all over? While there’s that mercantile
cut upon him which would make you happy to give him credit for five
hundred pound, don’t you see the salt sea shining on him too?’

‘YOU do, I dare say,’ returned Miss Abbey, ‘but I don’t. And as for
stewarding, I think it’s time my brother gave that up, and took his
House in hand on his sister’s retiring. The House will go to pieces if
he don’t. I wouldn’t sell it for any money that could be told out, to a
person that I couldn’t depend upon to be a Law to the Porters, as I have
been.’

‘There you’re right, Miss,’ said Mr Inspector. ‘A better kept house is
not known to our men. What do I say? Half so well a kept house is not
known to our men. Show the Force the Six Jolly Fellowship Porters,
and the Force--to a constable--will show you a piece of perfection, Mr
Kibble.’

That gentleman, with a very serious shake of his head, subscribed the
article.

‘And talk of Time slipping by you, as if it was an animal at rustic
sports with its tail soaped,’ said Mr Inspector (again, a subject which
nobody had approached); ‘why, well you may. Well you may. How has it
slipped by us, since the time when Mr Job Potterson here present, Mr
Jacob Kibble here present, and an Officer of the Force here present,
first came together on a matter of Identification!’

Bella’s husband stepped softly to the half-door of the bar, and stood
there.

‘How has Time slipped by us,’ Mr Inspector went on slowly, with his eyes
narrowly observant of the two guests, ‘since we three very men, at an
Inquest in this very house--Mr Kibble? Taken ill, sir?’

Mr Kibble had staggered up, with his lower jaw dropped, catching
Potterson by the shoulder, and pointing to the half-door. He now cried
out: ‘Potterson! Look! Look there!’ Potterson started up, started back,
and exclaimed: ‘Heaven defend us, what’s that!’ Bella’s husband stepped
back to Bella, took her in his arms (for she was terrified by the
unintelligible terror of the two men), and shut the door of the little
room. A hurry of voices succeeded, in which Mr Inspector’s voice was
busiest; it gradually slackened and sank; and Mr Inspector reappeared.
‘Sharp’s the word, sir!’ he said, looking in with a knowing wink. ‘We’ll
get your lady out at once.’ Immediately, Bella and her husband were
under the stars, making their way back, alone, to the vehicle they had
kept in waiting.

All this was most extraordinary, and Bella could make nothing of it but
that John was in the right. How in the right, and how suspected of being
in the wrong, she could not divine. Some vague idea that he had never
really assumed the name of Handford, and that there was a remarkable
likeness between him and that mysterious person, was her nearest
approach to any definite explanation. But John was triumphant; that much
was made apparent; and she could wait for the rest.

When John came home to dinner next day, he said, sitting down on the
sofa by Bella and baby-Bella: ‘My dear, I have a piece of news to tell
you. I have left the China House.’

As he seemed to like having left it, Bella took it for granted that
there was no misfortune in the case.

‘In a word, my love,’ said John, ‘the China House is broken up and
abolished. There is no such thing any more.’

‘Then, are you already in another House, John?’

‘Yes, my darling. I am in another way of business. And I am rather
better off.’

The inexhaustible baby was instantly made to congratulate him, and
to say, with appropriate action on the part of a very limp arm and a
speckled fist: ‘Three cheers, ladies and gemplemorums. Hoo--ray!’

‘I am afraid, my life,’ said John, ‘that you have become very much
attached to this cottage?’

‘Afraid I have, John? Of course I have.’

‘The reason why I said afraid,’ returned John, ‘is, because we must
move.’

‘O John!’

‘Yes, my dear, we must move. We must have our head-quarters in London
now. In short, there’s a dwelling-house rent-free, attached to my new
position, and we must occupy it.’

‘That’s a gain, John.’

‘Yes, my dear, it is undoubtedly a gain.’

He gave her a very blithe look, and a very sly look. Which occasioned
the inexhaustible baby to square at him with the speckled fists, and
demand in a threatening manner what he meant?

‘My love, you said it was a gain, and I said it was a gain. A very
innocent remark, surely.’

‘I won’t,’ said the inexhaustible baby,
‘--allow--you--to--make--game--of--my--venerable--Ma.’ At each division
administering a soft facer with one of the speckled fists.

John having stooped down to receive these punishing visitations, Bella
asked him, would it be necessary to move soon? Why yes, indeed (said
John), he did propose that they should move very soon. Taking the
furniture with them, of course? (said Bella). Why, no (said John), the
fact was, that the house was--in a sort of a kind of a way--furnished
already.

The inexhaustible baby, hearing this, resumed the offensive, and said:
‘But there’s no nursery for me, sir. What do you mean, marble-hearted
parent?’ To which the marble-hearted parent rejoined that there was
a--sort of a kind of a--nursery, and it might be ‘made to do’. ‘Made to
do?’ returned the Inexhaustible, administering more punishment, ‘what do
you take me for?’ And was then turned over on its back in Bella’s lap,
and smothered with kisses.

‘But really, John dear,’ said Bella, flushed in quite a lovely manner
by these exercises, ‘will the new house, just as it stands, do for baby?
That’s the question.’

‘I felt that to be the question,’ he returned, ‘and therefore I arranged
that you should come with me and look at it, to-morrow morning.’
Appointment made, accordingly, for Bella to go up with him to-morrow
morning; John kissed; and Bella delighted.

When they reached London in pursuance of their little plan, they took
coach and drove westward. Not only drove westward, but drove into that
particular westward division, which Bella had seen last when she turned
her face from Mr Boffin’s door. Not only drove into that particular
division, but drove at last into that very street. Not only drove into
that very street, but stopped at last at that very house.

‘John dear!’ cried Bella, looking out of window in a flutter. ‘Do you
see where we are?’

‘Yes, my love. The coachman’s quite right.’

The house-door was opened without any knocking or ringing, and John
promptly helped her out. The servant who stood holding the door, asked
no question of John, neither did he go before them or follow them as
they went straight up-stairs. It was only her husband’s encircling arm,
urging her on, that prevented Bella from stopping at the foot of the
staircase. As they ascended, it was seen to be tastefully ornamented
with most beautiful flowers.

‘O John!’ said Bella, faintly. ‘What does this mean?’

‘Nothing, my darling, nothing. Let us go on.’

Going on a little higher, they came to a charming aviary, in which a
number of tropical birds, more gorgeous in colour than the flowers,
were flying about; and among those birds were gold and silver fish, and
mosses, and water-lilies, and a fountain, and all manner of wonders.

‘O my dear John!’ said Bella. ‘What does this mean?’

‘Nothing, my darling, nothing. Let us go on.’

They went on, until they came to a door. As John put out his hand to
open it, Bella caught his hand.

‘I don’t know what it means, but it’s too much for me. Hold me, John,
love.’

John caught her up in his arm, and lightly dashed into the room with
her.

Behold Mr and Mrs Boffin, beaming! Behold Mrs Boffin clapping her hands
in an ecstacy, running to Bella with tears of joy pouring down her
comely face, and folding her to her breast, with the words: ‘My deary
deary, deary girl, that Noddy and me saw married and couldn’t wish joy
to, or so much as speak to! My deary, deary, deary, wife of John and
mother of his little child! My loving loving, bright bright, Pretty
Pretty! Welcome to your house and home, my deary!’



Chapter 13

SHOWING HOW THE GOLDEN DUSTMAN HELPED TO SCATTER DUST


In all the first bewilderment of her wonder, the most bewilderingly
wonderful thing to Bella was the shining countenance of Mr Boffin. That
his wife should be joyous, open-hearted, and genial, or that her face
should express every quality that was large and trusting, and no quality
that was little or mean, was accordant with Bella’s experience. But,
that he, with a perfectly beneficent air and a plump rosy face, should
be standing there, looking at her and John, like some jovial good
spirit, was marvellous. For, how had he looked when she last saw him in
that very room (it was the room in which she had given him that piece of
her mind at parting), and what had become of all those crooked lines of
suspicion, avarice, and distrust, that twisted his visage then?

Mrs Boffin seated Bella on the large ottoman, and seated herself beside
her, and John her husband seated himself on the other side of her, and
Mr Boffin stood beaming at every one and everything he could see, with
surpassing jollity and enjoyment. Mrs Boffin was then taken with a
laughing fit of clapping her hands, and clapping her knees, and rocking
herself to and fro, and then with another laughing fit of embracing
Bella, and rocking her to and fro--both fits, of considerable duration.

‘Old lady, old lady,’ said Mr Boffin, at length; ‘if you don’t begin
somebody else must.’

‘I’m a going to begin, Noddy, my dear,’ returned Mrs Boffin. ‘Only it
isn’t easy for a person to know where to begin, when a person is in this
state of delight and happiness. Bella, my dear. Tell me, who’s this?’

‘Who is this?’ repeated Bella. ‘My husband.’

‘Ah! But tell me his name, deary!’ cried Mrs Boffin.

‘Rokesmith.’

‘No, it ain’t!’ cried Mrs Boffin, clapping her hands, and shaking her
head. ‘Not a bit of it.’

‘Handford then,’ suggested Bella.

‘No, it ain’t!’ cried Mrs Boffin, again clapping her hands and shaking
her head. ‘Not a bit of it.’

‘At least, his name is John, I suppose?’ said Bella.

‘Ah! I should think so, deary!’ cried Mrs Boffin. ‘I should hope so!
Many and many is the time I have called him by his name of John. But
what’s his other name, his true other name? Give a guess, my pretty!’

‘I can’t guess,’ said Bella, turning her pale face from one to another.

‘I could,’ cried Mrs Boffin, ‘and what’s more, I did! I found him out,
all in a flash as I may say, one night. Didn’t I, Noddy?’

‘Ay! That the old lady did!’ said Mr Boffin, with stout pride in the
circumstance.

‘Harkee to me, deary,’ pursued Mrs Boffin, taking Bella’s hands between
her own, and gently beating on them from time to time. ‘It was after a
particular night when John had been disappointed--as he thought--in
his affections. It was after a night when John had made an offer to a
certain young lady, and the certain young lady had refused it. It was
after a particular night, when he felt himself cast-away-like, and had
made up his mind to go seek his fortune. It was the very next night. My
Noddy wanted a paper out of his Secretary’s room, and I says to Noddy,
“I am going by the door, and I’ll ask him for it.” I tapped at his door,
and he didn’t hear me. I looked in, and saw him a sitting lonely by his
fire, brooding over it. He chanced to look up with a pleased kind of
smile in my company when he saw me, and then in a single moment every
grain of the gunpowder that had been lying sprinkled thick about him
ever since I first set eyes upon him as a man at the Bower, took fire!
Too many a time had I seen him sitting lonely, when he was a poor child,
to be pitied, heart and hand! Too many a time had I seen him in need of
being brightened up with a comforting word! Too many and too many a time
to be mistaken, when that glimpse of him come at last! No, no! I just
makes out to cry, “I know you now! You’re John!” And he catches me as
I drops.--So what,’ says Mrs Boffin, breaking off in the rush of her
speech to smile most radiantly, ‘might you think by this time that your
husband’s name was, dear?’

‘Not,’ returned Bella, with quivering lips; ‘not Harmon? That’s not
possible?’

‘Don’t tremble. Why not possible, deary, when so many things are
possible?’ demanded Mrs Boffin, in a soothing tone.

‘He was killed,’ gasped Bella.

‘Thought to be,’ said Mrs Boffin. ‘But if ever John Harmon drew the
breath of life on earth, that is certainly John Harmon’s arm round your
waist now, my pretty. If ever John Harmon had a wife on earth, that wife
is certainly you. If ever John Harmon and his wife had a child on earth,
that child is certainly this.’

By a master-stroke of secret arrangement, the inexhaustible baby here
appeared at the door, suspended in mid-air by invisible agency. Mrs
Boffin, plunging at it, brought it to Bella’s lap, where both Mrs and Mr
Boffin (as the saying is) ‘took it out of’ the Inexhaustible in a shower
of caresses. It was only this timely appearance that kept Bella from
swooning. This, and her husband’s earnestness in explaining further to
her how it had come to pass that he had been supposed to be slain, and
had even been suspected of his own murder; also, how he had put a pious
fraud upon her which had preyed upon his mind, as the time for its
disclosure approached, lest she might not make full allowance for
the object with which it had originated, and in which it had fully
developed.

‘But bless ye, my beauty!’ cried Mrs Boffin, taking him up short at this
point, with another hearty clap of her hands. ‘It wasn’t John only that
was in it. We was all of us in it.’

‘I don’t,’ said Bella, looking vacantly from one to another, ‘yet
understand--’

‘Of course you don’t, my deary,’ exclaimed Mrs Boffin. ‘How can you till
you’re told! So now I am a going to tell you. So you put your two hands
between my two hands again,’ cried the comfortable creature, embracing
her, ‘with that blessed little picter lying on your lap, and you shall
be told all the story. Now, I’m a going to tell the story. Once, twice,
three times, and the horses is off. Here they go! When I cries out that
night, “I know you now, you’re John!”--which was my exact words; wasn’t
they, John?’

‘Your exact words,’ said John, laying his hand on hers.

‘That’s a very good arrangement,’ cried Mrs Boffin. ‘Keep it there,
John. And as we was all of us in it, Noddy you come and lay yours a top
of his, and we won’t break the pile till the story’s done.’

Mr Boffin hitched up a chair, and added his broad brown right hand to
the heap.

‘That’s capital!’ said Mrs Boffin, giving it a kiss. ‘Seems quite a
family building; don’t it? But the horses is off. Well! When I cries
out that night, “I know you now! you’re John!” John catches of me, it
is true; but I ain’t a light weight, bless ye, and he’s forced to let me
down. Noddy, he hears a noise, and in he trots, and as soon as I anyways
comes to myself I calls to him, “Noddy, well I might say as I did say,
that night at the Bower, for the Lord be thankful this is John!” On
which he gives a heave, and down he goes likewise, with his head under
the writing-table. This brings me round comfortable, and that brings him
round comfortable, and then John and him and me we all fall a crying for
joy.’

‘Yes! They cry for joy, my darling,’ her husband struck in. ‘You
understand? These two, whom I come to life to disappoint and dispossess,
cry for joy!’

Bella looked at him confusedly, and looked again at Mrs Boffin’s radiant
face.

‘That’s right, my dear, don’t you mind him,’ said Mrs Boffin, ‘stick
to me. Well! Then we sits down, gradually gets cool, and holds a
confabulation. John, he tells us how he is despairing in his mind on
accounts of a certain fair young person, and how, if I hadn’t found him
out, he was going away to seek his fortune far and wide, and had fully
meant never to come to life, but to leave the property as our wrongful
inheritance for ever and a day. At which you never see a man so
frightened as my Noddy was. For to think that he should have come into
the property wrongful, however innocent, and--more than that--might have
gone on keeping it to his dying day, turned him whiter than chalk.’

‘And you too,’ said Mr Boffin.

‘Don’t you mind him, neither, my deary,’ resumed Mrs Boffin; ‘stick
to me. This brings up a confabulation regarding the certain fair young
person; when Noddy he gives it as his opinion that she is a deary
creetur. “She may be a leetle spoilt, and nat’rally spoilt,” he says,
“by circumstances, but that’s only the surface, and I lay my life,” he
says, “that she’s the true golden gold at heart.”’

‘So did you,’ said Mr Boffin.

‘Don’t you mind him a single morsel, my dear,’ proceeded Mrs Boffin,
‘but stick to me. Then says John, O, if he could but prove so! Then we
both of us ups and says, that minute, “Prove so!”’

With a start, Bella directed a hurried glance towards Mr Boffin. But,
he was sitting thoughtfully smiling at that broad brown hand of his, and
either didn’t see it, or would take no notice of it.

‘“Prove it, John!” we says,’ repeated Mrs Boffin. ‘“Prove it and
overcome your doubts with triumph, and be happy for the first time in
your life, and for the rest of your life.” This puts John in a state,
to be sure. Then we says, “What will content you? If she was to stand up
for you when you was slighted, if she was to show herself of a generous
mind when you was oppressed, if she was to be truest to you when you was
poorest and friendliest, and all this against her own seeming interest,
how would that do?” “Do?” says John, “it would raise me to the skies.”
 “Then,” says my Noddy, “make your preparations for the ascent, John, it
being my firm belief that up you go!”’

Bella caught Mr Boffin’s twinkling eye for half an instant; but he got
it away from her, and restored it to his broad brown hand.

‘From the first, you was always a special favourite of Noddy’s,’ said
Mrs Boffin, shaking her head. ‘O you were! And if I had been inclined
to be jealous, I don’t know what I mightn’t have done to you. But as I
wasn’t--why, my beauty,’ with a hearty laugh and an embrace, ‘I made you
a special favourite of my own too. But the horses is coming round the
corner. Well! Then says my Noddy, shaking his sides till he was fit to
make ‘em ache again: “Look out for being slighted and oppressed, John,
for if ever a man had a hard master, you shall find me from this present
time to be such to you.” And then he began!’ cried Mrs Boffin, in an
ecstacy of admiration. ‘Lord bless you, then he began! And how he DID
begin; didn’t he!’

Bella looked half frightened, and yet half laughed.

‘But, bless you,’ pursued Mrs Boffin, ‘if you could have seen him of a
night, at that time of it! The way he’d sit and chuckle over himself!
The way he’d say “I’ve been a regular brown bear to-day,” and take
himself in his arms and hug himself at the thoughts of the brute he had
pretended. But every night he says to me: “Better and better, old lady.
What did we say of her? She’ll come through it, the true golden gold.
This’ll be the happiest piece of work we ever done.” And then he’d say,
“I’ll be a grislier old growler to-morrow!” and laugh, he would, till
John and me was often forced to slap his back, and bring it out of his
windpipes with a little water.’

Mr Boffin, with his face bent over his heavy hand, made no sound,
but rolled his shoulders when thus referred to, as if he were vastly
enjoying himself.

‘And so, my good and pretty,’ pursued Mrs Boffin, ‘you was married, and
there was we hid up in the church-organ by this husband of yours; for
he wouldn’t let us out with it then, as was first meant. “No,” he says,
“she’s so unselfish and contented, that I can’t afford to be rich yet. I
must wait a little longer.” Then, when baby was expected, he says, “She
is such a cheerful, glorious housewife that I can’t afford to be rich
yet. I must wait a little longer.” Then when baby was born, he says,
“She is so much better than she ever was, that I can’t afford to be rich
yet. I must wait a little longer.” And so he goes on and on, till I says
outright, “Now, John, if you don’t fix a time for setting her up in her
own house and home, and letting us walk out of it, I’ll turn Informer.”
 Then he says he’ll only wait to triumph beyond what we ever thought
possible, and to show her to us better than even we ever supposed; and
he says, “She shall see me under suspicion of having murdered myself,
and YOU shall see how trusting and how true she’ll be.” Well! Noddy and
me agreed to that, and he was right, and here you are, and the horses is
in, and the story is done, and God bless you my Beauty, and God bless us
all!’

The pile of hands dispersed, and Bella and Mrs Boffin took a good long
hug of one another: to the apparent peril of the inexhaustible baby,
lying staring in Bella’s lap.

‘But IS the story done?’ said Bella, pondering. ‘Is there no more of
it?’

‘What more of it should there be, deary?’ returned Mrs Boffin, full of
glee.

‘Are you sure you have left nothing out of it?’ asked Bella.

‘I don’t think I have,’ said Mrs Boffin, archly.

‘John dear,’ said Bella, ‘you’re a good nurse; will you please hold
baby?’ Having deposited the Inexhaustible in his arms with those words,
Bella looked hard at Mr Boffin, who had moved to a table where he was
leaning his head upon his hand with his face turned away, and, quietly
settling herself on her knees at his side, and drawing one arm over his
shoulder, said: ‘Please I beg your pardon, and I made a small mistake of
a word when I took leave of you last. Please I think you are better (not
worse) than Hopkins, better (not worse) than Dancer, better (not worse)
than Blackberry Jones, better (not worse) than any of them! Please
something more!’ cried Bella, with an exultant ringing laugh as she
struggled with him and forced him to turn his delighted face to hers.
‘Please I have found out something not yet mentioned. Please I don’t
believe you are a hard-hearted miser at all, and please I don’t believe
you ever for one single minute were!’

At this, Mrs Boffin fairly screamed with rapture, and sat beating her
feet upon the floor, clapping her hands, and bobbing herself backwards
and forwards, like a demented member of some Mandarin’s family.

‘O, I understand you now, sir!’ cried Bella. ‘I want neither you nor any
one else to tell me the rest of the story. I can tell it to YOU, now, if
you would like to hear it.’

‘Can you, my dear?’ said Mr Boffin. ‘Tell it then.’

‘What?’ cried Bella, holding him prisoner by the coat with both hands.
‘When you saw what a greedy little wretch you were the patron of, you
determined to show her how much misused and misprized riches could
do, and often had done, to spoil people; did you? Not caring what she
thought of you (and Goodness knows THAT was of no consequence!) you
showed her, in yourself, the most detestable sides of wealth, saying in
your own mind, “This shallow creature would never work the truth out of
her own weak soul, if she had a hundred years to do it in; but a glaring
instance kept before her may open even her eyes and set her thinking.”
 That was what you said to yourself, was it, sir?’

‘I never said anything of the sort,’ Mr Boffin declared in a state of
the highest enjoyment.

‘Then you ought to have said it, sir,’ returned Bella, giving him two
pulls and one kiss, ‘for you must have thought and meant it. You saw
that good fortune was turning my stupid head and hardening my silly
heart--was making me grasping, calculating, insolent, insufferable--and
you took the pains to be the dearest and kindest fingerpost that ever
was set up anywhere, pointing out the road that I was taking and the end
it led to. Confess instantly!’

‘John,’ said Mr Boffin, one broad piece of sunshine from head to foot,
‘I wish you’d help me out of this.’

‘You can’t be heard by counsel, sir,’ returned Bella. ‘You must speak
for yourself. Confess instantly!’

‘Well, my dear,’ said Mr Boffin, ‘the truth is, that when we did go in
for the little scheme that my old lady has pinted out, I did put it to
John, what did he think of going in for some such general scheme as YOU
have pinted out? But I didn’t in any way so word it, because I didn’t in
any way so mean it. I only said to John, wouldn’t it be more consistent,
me going in for being a reg’lar brown bear respecting him, to go in as a
reg’lar brown bear all round?’

‘Confess this minute, sir,’ said Bella, ‘that you did it to correct and
amend me!’

‘Certainly, my dear child,’ said Mr Boffin, ‘I didn’t do it to harm you;
you may be sure of that. And I did hope it might just hint a caution.
Still, it ought to be mentioned that no sooner had my old lady found out
John, than John made known to her and me that he had had his eye upon a
thankless person by the name of Silas Wegg. Partly for the punishment of
which Wegg, by leading him on in a very unhandsome and underhanded
game that he was playing, them books that you and me bought so many
of together (and, by-the-by, my dear, he wasn’t Blackberry Jones, but
Blewberry) was read aloud to me by that person of the name of Silas Wegg
aforesaid.’

Bella, who was still on her knees at Mr Boffin’s feet, gradually sank
down into a sitting posture on the ground, as she meditated more and
more thoughtfully, with her eyes upon his beaming face.

‘Still,’ said Bella, after this meditative pause, ‘there remain two
things that I cannot understand. Mrs Boffin never supposed any part of
the change in Mr Boffin to be real; did she?--You never did; did you?’
asked Bella, turning to her.

‘No!’ returned Mrs Boffin, with a most rotund and glowing negative.

‘And yet you took it very much to heart,’ said Bella. ‘I remember its
making you very uneasy, indeed.’

‘Ecod, you see Mrs John has a sharp eye, John!’ cried Mr Boffin, shaking
his head with an admiring air. ‘You’re right, my dear. The old lady
nearly blowed us into shivers and smithers, many times.’

‘Why?’ asked Bella. ‘How did that happen, when she was in your secret?’

‘Why, it was a weakness in the old lady,’ said Mr Boffin; ‘and yet, to
tell you the whole truth and nothing but the truth, I’m rather proud of
it. My dear, the old lady thinks so high of me that she couldn’t abear
to see and hear me coming out as a reg’lar brown one. Couldn’t abear
to make-believe as I meant it! In consequence of which, we was
everlastingly in danger with her.’

Mrs Boffin laughed heartily at herself; but a certain glistening in her
honest eyes revealed that she was by no means cured of that dangerous
propensity.

‘I assure you, my dear,’ said Mr Boffin, ‘that on the celebrated
day when I made what has since been agreed upon to be my grandest
demonstration--I allude to Mew says the cat, Quack quack says the
duck, and Bow-wow-wow says the dog--I assure you, my dear, that on that
celebrated day, them flinty and unbelieving words hit my old lady so hard
on my account, that I had to hold her, to prevent her running out after
you, and defending me by saying I was playing a part.’

Mrs Boffin laughed heartily again, and her eyes glistened again, and
it then appeared, not only that in that burst of sarcastic eloquence
Mr Boffin was considered by his two fellow-conspirators to have outdone
himself, but that in his own opinion it was a remarkable achievement.
‘Never thought of it afore the moment, my dear!’ he observed to Bella.
‘When John said, if he had been so happy as to win your affections and
possess your heart, it come into my head to turn round upon him with
“Win her affections and possess her heart! Mew says the cat, Quack quack
says the duck, and Bow-wow-wow says the dog.” I couldn’t tell you how
it come into my head or where from, but it had so much the sound of a
rasper that I own to you it astonished myself. I was awful nigh bursting
out a laughing though, when it made John stare!’

‘You said, my pretty,’ Mrs Boffin reminded Bella, ‘that there was one
other thing you couldn’t understand.’

‘O yes!’ cried Bella, covering her face with her hands; ‘but that I
never shall be able to understand as long as I live. It is, how John
could love me so when I so little deserved it, and how you, Mr and Mrs
Boffin, could be so forgetful of yourselves, and take such pains and
trouble, to make me a little better, and after all to help him to so
unworthy a wife. But I am very very grateful.’

It was John Harmon’s turn then--John Harmon now for good, and John
Rokesmith for nevermore--to plead with her (quite unnecessarily) in
behalf of his deception, and to tell her, over and over again, that it
had been prolonged by her own winning graces in her supposed station of
life. This led on to many interchanges of endearment and enjoyment
on all sides, in the midst of which the Inexhaustible being observed
staring, in a most imbecile manner, on Mrs Boffin’s breast, was
pronounced to be supernaturally intelligent as to the whole transaction,
and was made to declare to the ladies and gemplemorums, with a wave of
the speckled fist (with difficulty detached from an exceedingly short
waist), ‘I have already informed my venerable Ma that I know all about
it!’

Then, said John Harmon, would Mrs John Harmon come and see her house?
And a dainty house it was, and a tastefully beautiful; and they went
through it in procession; the Inexhaustible on Mrs Boffin’s bosom (still
staring) occupying the middle station, and Mr Boffin bringing up the
rear. And on Bella’s exquisite toilette table was an ivory casket, and
in the casket were jewels the like of which she had never dreamed of,
and aloft on an upper floor was a nursery garnished as with rainbows;
‘though we were hard put to it,’ said John Harmon, ‘to get it done in so
short a time.’

The house inspected, emissaries removed the Inexhaustible, who was
shortly afterwards heard screaming among the rainbows; whereupon Bella
withdrew herself from the presence and knowledge of gemplemorums, and
the screaming ceased, and smiling Peace associated herself with that
young olive branch.

‘Come and look in, Noddy!’ said Mrs Boffin to Mr Boffin.

Mr Boffin, submitting to be led on tiptoe to the nursery door, looked in
with immense satisfaction, although there was nothing to see but Bella
in a musing state of happiness, seated in a little low chair upon the
hearth, with her child in her fair young arms, and her soft eyelashes
shading her eyes from the fire.

‘It looks as if the old man’s spirit had found rest at last; don’t it?’
said Mrs Boffin.

‘Yes, old lady.’

‘And as if his money had turned bright again, after a long long rust in
the dark, and was at last a beginning to sparkle in the sunlight?’

‘Yes, old lady.’

‘And it makes a pretty and a promising picter; don’t it?’

‘Yes, old lady.’

But, aware at the instant of a fine opening for a point, Mr Boffin
quenched that observation in this--delivered in the grisliest growling
of the regular brown bear. ‘A pretty and a hopeful picter? Mew,
Quack quack, Bow-wow!’ And then trotted silently downstairs, with his
shoulders in a state of the liveliest commotion.



Chapter 14

CHECKMATE TO THE FRIENDLY MOVE


Mr and Mrs John Harmon had so timed their taking possession of their
rightful name and their London house, that the event befel on the very
day when the last waggon-load of the last Mound was driven out at the
gates of Boffin’s Bower. As it jolted away, Mr Wegg felt that the
last load was correspondingly removed from his mind, and hailed the
auspicious season when that black sheep, Boffin, was to be closely
sheared.

Over the whole slow process of levelling the Mounds, Silas had kept
watch with rapacious eyes. But, eyes no less rapacious had watched the
growth of the Mounds in years bygone, and had vigilantly sifted the dust
of which they were composed. No valuables turned up. How should there
be any, seeing that the old hard jailer of Harmony Jail had coined every
waif and stray into money, long before?

Though disappointed by this bare result, Mr Wegg felt too sensibly
relieved by the close of the labour, to grumble to any great extent.
A foreman-representative of the dust contractors, purchasers of the
Mounds, had worn Mr Wegg down to skin and bone. This supervisor of the
proceedings, asserting his employers’ rights to cart off by daylight,
nightlight, torchlight, when they would, must have been the death of
Silas if the work had lasted much longer. Seeming never to need sleep
himself, he would reappear, with a tied-up broken head, in fantail hat
and velveteen smalls, like an accursed goblin, at the most unholy and
untimely hours. Tired out by keeping close ward over a long day’s work
in fog and rain, Silas would have just crawled to bed and be dozing,
when a horrid shake and rumble under his pillow would announce an
approaching train of carts, escorted by this Demon of Unrest, to fall to
work again. At another time, he would be rumbled up out of his soundest
sleep, in the dead of the night; at another, would be kept at his post
eight-and-forty hours on end. The more his persecutor besought him not
to trouble himself to turn out, the more suspicious was the crafty Wegg
that indications had been observed of something hidden somewhere, and
that attempts were on foot to circumvent him. So continually broken was
his rest through these means, that he led the life of having wagered
to keep ten thousand dog-watches in ten thousand hours, and looked
piteously upon himself as always getting up and yet never going to bed.
So gaunt and haggard had he grown at last, that his wooden leg showed
disproportionate, and presented a thriving appearance in contrast
with the rest of his plagued body, which might almost have been termed
chubby.

However, Wegg’s comfort was, that all his disagreeables were now over,
and that he was immediately coming into his property. Of late, the
grindstone did undoubtedly appear to have been whirling at his own nose
rather than Boffin’s, but Boffin’s nose was now to be sharpened fine.
Thus far, Mr Wegg had let his dusty friend off lightly, having been
baulked in that amiable design of frequently dining with him, by the
machinations of the sleepless dustman. He had been constrained to depute
Mr Venus to keep their dusty friend, Boffin, under inspection, while he
himself turned lank and lean at the Bower.

To Mr Venus’s museum Mr Wegg repaired when at length the Mounds
were down and gone. It being evening, he found that gentleman, as he
expected, seated over his fire; but did not find him, as he expected,
floating his powerful mind in tea.

‘Why, you smell rather comfortable here!’ said Wegg, seeming to take it
ill, and stopping and sniffing as he entered.

‘I AM rather comfortable, sir,’ said Venus.

‘You don’t use lemon in your business, do you?’ asked Wegg, sniffing
again.

‘No, Mr Wegg,’ said Venus. ‘When I use it at all, I mostly use it in
cobblers’ punch.’

‘What do you call cobblers’ punch?’ demanded Wegg, in a worse humour
than before.

‘It’s difficult to impart the receipt for it, sir,’ returned Venus,
‘because, however particular you may be in allotting your materials,
so much will still depend upon the individual gifts, and there being a
feeling thrown into it. But the groundwork is gin.’

‘In a Dutch bottle?’ said Wegg gloomily, as he sat himself down.

‘Very good, sir, very good!’ cried Venus. ‘Will you partake, sir?’

‘Will I partake?’ returned Wegg very surlily. ‘Why, of course I will!
WILL a man partake, as has been tormented out of his five senses by
an everlasting dustman with his head tied up! WILL he, too! As if he
wouldn’t!’

‘Don’t let it put you out, Mr Wegg. You don’t seem in your usual
spirits.’

‘If you come to that, you don’t seem in your usual spirits,’ growled
Wegg. ‘You seem to be setting up for lively.’

This circumstance appeared, in his then state of mind, to give Mr Wegg
uncommon offence.

‘And you’ve been having your hair cut!’ said Wegg, missing the usual
dusty shock.

‘Yes, Mr Wegg. But don’t let that put you out, either.’

‘And I am blest if you ain’t getting fat!’ said Wegg, with culminating
discontent. ‘What are you going to do next?’

‘Well, Mr Wegg,’ said Venus, smiling in a sprightly manner, ‘I suspect
you could hardly guess what I am going to do next.’

‘I don’t want to guess,’ retorted Wegg. ‘All I’ve got to say is, that
it’s well for you that the diwision of labour has been what it has been.
It’s well for you to have had so light a part in this business, when
mine has been so heavy. You haven’t had YOUR rest broke, I’ll be bound.’

‘Not at all, sir,’ said Venus. ‘Never rested so well in all my life, I
thank you.’

‘Ah!’ grumbled Wegg, ‘you should have been me. If you had been me, and
had been fretted out of your bed, and your sleep, and your meals, and
your mind, for a stretch of months together, you’d have been out of
condition and out of sorts.’

‘Certainly, it has trained you down, Mr Wegg,’ said Venus, contemplating
his figure with an artist’s eye. ‘Trained you down very low, it has! So
weazen and yellow is the kivering upon your bones, that one might almost
fancy you had come to give a look-in upon the French gentleman in the
corner, instead of me.’

Mr Wegg, glancing in great dudgeon towards the French gentleman’s
corner, seemed to notice something new there, which induced him to
glance at the opposite corner, and then to put on his glasses and stare
at all the nooks and corners of the dim shop in succession.

‘Why, you’ve been having the place cleaned up!’ he exclaimed.

‘Yes, Mr Wegg. By the hand of adorable woman.’

‘Then what you’re going to do next, I suppose, is to get married?’

‘That’s it, sir.’

Silas took off his glasses again--finding himself too intensely
disgusted by the sprightly appearance of his friend and partner to bear
a magnified view of him and made the inquiry:

‘To the old party?’

‘Mr Wegg!’ said Venus, with a sudden flush of wrath. ‘The lady in
question is not a old party.’

‘I meant,’ exclaimed Wegg, testily, ‘to the party as formerly objected?’

‘Mr Wegg,’ said Venus, ‘in a case of so much delicacy, I must trouble
you to say what you mean. There are strings that must not be played
upon. No sir! Not sounded, unless in the most respectful and tuneful
manner. Of such melodious strings is Miss Pleasant Riderhood formed.’

‘Then it IS the lady as formerly objected?’ said Wegg.

‘Sir,’ returned Venus with dignity, ‘I accept the altered phrase. It is
the lady as formerly objected.’

‘When is it to come off?’ asked Silas.

‘Mr Wegg,’ said Venus, with another flush. ‘I cannot permit it to be
put in the form of a Fight. I must temperately but firmly call upon you,
sir, to amend that question.’

‘When is the lady,’ Wegg reluctantly demanded, constraining his ill
temper in remembrance of the partnership and its stock in trade, ‘a
going to give her ‘and where she has already given her ‘art?’

‘Sir,’ returned Venus, ‘I again accept the altered phrase, and with
pleasure. The lady is a going to give her ‘and where she has already
given her ‘art, next Monday.’

‘Then the lady’s objection has been met?’ said Silas.

‘Mr Wegg,’ said Venus, ‘as I did name to you, I think, on a former
occasion, if not on former occasions--’

‘On former occasions,’ interrupted Wegg.

‘--What,’ pursued Venus, ‘what the nature of the lady’s objection was, I
may impart, without violating any of the tender confidences since sprung
up between the lady and myself, how it has been met, through the kind
interference of two good friends of mine: one, previously acquainted
with the lady: and one, not. The pint was thrown out, sir, by those two
friends when they did me the great service of waiting on the lady to
try if a union betwixt the lady and me could not be brought to bear--the
pint, I say, was thrown out by them, sir, whether if, after marriage,
I confined myself to the articulation of men, children, and the lower
animals, it might not relieve the lady’s mind of her feeling respecting
being as a lady--regarded in a bony light. It was a happy thought, sir,
and it took root.’

‘It would seem, Mr Venus,’ observed Wegg, with a touch of distrust,
‘that you are flush of friends?’

‘Pretty well, sir,’ that gentleman answered, in a tone of placid
mystery. ‘So-so, sir. Pretty well.’

‘However,’ said Wegg, after eyeing him with another touch of distrust,
‘I wish you joy. One man spends his fortune in one way, and another in
another. You are going to try matrimony. I mean to try travelling.’

‘Indeed, Mr Wegg?’

‘Change of air, sea-scenery, and my natural rest, I hope may bring me
round after the persecutions I have undergone from the dustman with his
head tied up, which I just now mentioned. The tough job being ended and
the Mounds laid low, the hour is come for Boffin to stump up. Would ten
to-morrow morning suit you, partner, for finally bringing Boffin’s nose
to the grindstone?’

Ten to-morrow morning would quite suit Mr Venus for that excellent
purpose.

‘You have had him well under inspection, I hope?’ said Silas.

Mr Venus had had him under inspection pretty well every day.

‘Suppose you was just to step round to-night then, and give him orders
from me--I say from me, because he knows I won’t be played with--to be
ready with his papers, his accounts, and his cash, at that time in the
morning?’ said Wegg. ‘And as a matter of form, which will be agreeable
to your own feelings, before we go out (for I’ll walk with you part of
the way, though my leg gives under me with weariness), let’s have a look
at the stock in trade.’

Mr Venus produced it, and it was perfectly correct; Mr Venus undertook
to produce it again in the morning, and to keep tryst with Mr Wegg on
Boffin’s doorstep as the clock struck ten. At a certain point of the
road between Clerkenwell and Boffin’s house (Mr Wegg expressly insisted
that there should be no prefix to the Golden Dustman’s name) the
partners separated for the night.

It was a very bad night; to which succeeded a very bad morning. The
streets were so unusually slushy, muddy, and miserable, in the morning,
that Wegg rode to the scene of action; arguing that a man who was, as
it were, going to the Bank to draw out a handsome property, could well
afford that trifling expense.

Venus was punctual, and Wegg undertook to knock at the door, and conduct
the conference. Door knocked at. Door opened.

‘Boffin at home?’

The servant replied that MR Boffin was at home.

‘He’ll do,’ said Wegg, ‘though it ain’t what I call him.’

The servant inquired if they had any appointment?

‘Now, I tell you what, young fellow,’ said Wegg, ‘I won’t have it. This
won’t do for me. I don’t want menials. I want Boffin.’

They were shown into a waiting-room, where the all-powerful Wegg wore
his hat, and whistled, and with his forefinger stirred up a clock that
stood upon the chimneypiece, until he made it strike. In a few minutes
they were shown upstairs into what used to be Boffin’s room; which,
besides the door of entrance, had folding-doors in it, to make it one
of a suite of rooms when occasion required. Here, Boffin was seated at a
library-table, and here Mr Wegg, having imperiously motioned the servant
to withdraw, drew up a chair and seated himself, in his hat, close
beside him. Here, also, Mr Wegg instantly underwent the remarkable
experience of having his hat twitched off his head and thrown out of a
window, which was opened and shut for the purpose.

‘Be careful what insolent liberties you take in that gentleman’s
presence,’ said the owner of the hand which had done this, ‘or I will
throw you after it.’

Wegg involuntarily clapped his hand to his bare head, and stared at the
Secretary. For, it was he addressed him with a severe countenance, and
who had come in quietly by the folding-doors.

‘Oh!’ said Wegg, as soon as he recovered his suspended power of speech.
‘Very good! I gave directions for YOU to be dismissed. And you ain’t
gone, ain’t you? Oh! We’ll look into this presently. Very good!’

‘No, nor I ain’t gone,’ said another voice.

Somebody else had come in quietly by the folding-doors. Turning his
head, Wegg beheld his persecutor, the ever-wakeful dustman, accoutred
with fantail hat and velveteen smalls complete. Who, untying his
tied-up broken head, revealed a head that was whole, and a face that was
Sloppy’s.

‘Ha, ha, ha, gentlemen!’ roared Sloppy in a peal of laughter, and with
immeasureable relish. ‘He never thought as I could sleep standing, and
often done it when I turned for Mrs Higden! He never thought as I used
to give Mrs Higden the Police-news in different voices! But I did lead
him a life all through it, gentlemen, I hope I really and truly DID!’
Here, Mr Sloppy opening his mouth to a quite alarming extent, and
throwing back his head to peal again, revealed incalculable buttons.

‘Oh!’ said Wegg, slightly discomfited, but not much as yet: ‘one and one
is two not dismissed, is it? Bof--fin! Just let me ask a question. Who
set this chap on, in this dress, when the carting began? Who employed
this fellow?’

‘I say!’ remonstrated Sloppy, jerking his head forward. ‘No fellows, or
I’ll throw you out of winder!’

Mr Boffin appeased him with a wave of his hand, and said: ‘I employed
him, Wegg.’

‘Oh! You employed him, Boffin? Very good. Mr Venus, we raise our terms,
and we can’t do better than proceed to business. Bof--fin! I want the
room cleared of these two scum.’

‘That’s not going to be done, Wegg,’ replied Mr Boffin, sitting
composedly on the library-table, at one end, while the Secretary sat
composedly on it at the other.

‘Bof--fin! Not going to be done?’ repeated Wegg. ‘Not at your peril?’

‘No, Wegg,’ said Mr Boffin, shaking his head good-humouredly. ‘Not at my
peril, and not on any other terms.’

Wegg reflected a moment, and then said: ‘Mr Venus, will you be so good
as hand me over that same dockyment?’

‘Certainly, sir,’ replied Venus, handing it to him with much politeness.
‘There it is. Having now, sir, parted with it, I wish to make a small
observation: not so much because it is anyways necessary, or expresses
any new doctrine or discovery, as because it is a comfort to my mind.
Silas Wegg, you are a precious old rascal.’

Mr Wegg, who, as if anticipating a compliment, had been beating
time with the paper to the other’s politeness until this unexpected
conclusion came upon him, stopped rather abruptly.

‘Silas Wegg,’ said Venus, ‘know that I took the liberty of taking Mr
Boffin into our concern as a sleeping partner, at a very early period of
our firm’s existence.’

‘Quite true,’ added Mr Boffin; ‘and I tested Venus by making him a
pretended proposal or two; and I found him on the whole a very honest
man, Wegg.’

‘So Mr Boffin, in his indulgence, is pleased to say,’ Venus remarked:
‘though in the beginning of this dirt, my hands were not, for a few
hours, quite as clean as I could wish. But I hope I made early and full
amends.’

‘Venus, you did,’ said Mr Boffin. ‘Certainly, certainly, certainly.’

Venus inclined his head with respect and gratitude. ‘Thank you, sir.
I am much obliged to you, sir, for all. For your good opinion now, for
your way of receiving and encouraging me when I first put myself in
communication with you, and for the influence since so kindly brought
to bear upon a certain lady, both by yourself and by Mr John Harmon.’ To
whom, when thus making mention of him, he also bowed.

Wegg followed the name with sharp ears, and the action with sharp eyes,
and a certain cringing air was infusing itself into his bullying air,
when his attention was re-claimed by Venus.

‘Everything else between you and me, Mr Wegg,’ said Venus, ‘now explains
itself, and you can now make out, sir, without further words from me.
But totally to prevent any unpleasantness or mistake that might arise on
what I consider an important point, to be made quite clear at the close
of our acquaintance, I beg the leave of Mr Boffin and Mr John Harmon to
repeat an observation which I have already had the pleasure of bringing
under your notice. You are a precious old rascal!’

‘You are a fool,’ said Wegg, with a snap of his fingers, ‘and I’d have
got rid of you before now, if I could have struck out any way of doing
it. I have thought it over, I can tell you. You may go, and welcome. You
leave the more for me. Because, you know,’ said Wegg, dividing his next
observation between Mr Boffin and Mr Harmon, ‘I am worth my price, and
I mean to have it. This getting off is all very well in its way, and it
tells with such an anatomical Pump as this one,’ pointing out Mr Venus,
‘but it won’t do with a Man. I am here to be bought off, and I have
named my figure. Now, buy me, or leave me.’

‘I’ll leave you, Wegg,’ said Mr Boffin, laughing, ‘as far as I am
concerned.’

‘Bof--fin!’ replied Wegg, turning upon him with a severe air, ‘I
understand YOUR new-born boldness. I see the brass underneath YOUR
silver plating. YOU have got YOUR nose out of joint. Knowing that you’ve
nothing at stake, you can afford to come the independent game. Why,
you’re just so much smeary glass to see through, you know! But Mr Harmon
is in another sitiwation. What Mr Harmon risks, is quite another pair
of shoes. Now, I’ve heerd something lately about this being Mr
Harmon--I make out now, some hints that I’ve met on that subject in
the newspaper--and I drop you, Bof--fin, as beneath my notice. I ask Mr
Harmon whether he has any idea of the contents of this present paper?’

‘It is a will of my late father’s, of more recent date than the will
proved by Mr Boffin (address whom again, as you have addressed him
already, and I’ll knock you down), leaving the whole of his property
to the Crown,’ said John Harmon, with as much indifference as was
compatible with extreme sternness.

‘Bight you are!’ cried Wegg. ‘Then,’ screwing the weight of his body
upon his wooden leg, and screwing his wooden head very much on one side,
and screwing up one eye: ‘then, I put the question to you, what’s this
paper worth?’

‘Nothing,’ said John Harmon.

Wegg had repeated the word with a sneer, and was entering on some
sarcastic retort, when, to his boundless amazement, he found himself
gripped by the cravat; shaken until his teeth chattered; shoved back,
staggering, into a corner of the room; and pinned there.

‘You scoundrel!’ said John Harmon, whose seafaring hold was like that of
a vice.

‘You’re knocking my head against the wall,’ urged Silas faintly.

‘I mean to knock your head against the wall,’ returned John Harmon,
suiting his action to his words, with the heartiest good will; ‘and I’d
give a thousand pounds for leave to knock your brains out. Listen, you
scoundrel, and look at that Dutch bottle.’

Sloppy held it up, for his edification.

‘That Dutch bottle, scoundrel, contained the latest will of the many
wills made by my unhappy self-tormenting father. That will gives
everything absolutely to my noble benefactor and yours, Mr Boffin,
excluding and reviling me, and my sister (then already dead of a broken
heart), by name. That Dutch bottle was found by my noble benefactor and
yours, after he entered on possession of the estate. That Dutch bottle
distressed him beyond measure, because, though I and my sister were
both no more, it cast a slur upon our memory which he knew we had
done nothing in our miserable youth, to deserve. That Dutch bottle,
therefore, he buried in the Mound belonging to him, and there it lay
while you, you thankless wretch, were prodding and poking--often very
near it, I dare say. His intention was, that it should never see the
light; but he was afraid to destroy it, lest to destroy such a document,
even with his great generous motive, might be an offence at law. After
the discovery was made here who I was, Mr Boffin, still restless on the
subject, told me, upon certain conditions impossible for such a hound as
you to appreciate, the secret of that Dutch bottle. I urged upon him the
necessity of its being dug up, and the paper being legally produced and
established. The first thing you saw him do, and the second thing has
been done without your knowledge. Consequently, the paper now rattling
in your hand as I shake you--and I should like to shake the life out
of you--is worth less than the rotten cork of the Dutch bottle, do you
understand?’

Judging from the fallen countenance of Silas as his head wagged
backwards and forwards in a most uncomfortable manner, he did
understand.

‘Now, scoundrel,’ said John Harmon, taking another sailor-like turn on
his cravat and holding him in his corner at arms’ length, ‘I shall make
two more short speeches to you, because I hope they will torment you.
Your discovery was a genuine discovery (such as it was), for nobody had
thought of looking into that place. Neither did we know you had made it,
until Venus spoke to Mr Boffin, though I kept you under good observation
from my first appearance here, and though Sloppy has long made it
the chief occupation and delight of his life, to attend you like your
shadow. I tell you this, that you may know we knew enough of you to
persuade Mr Boffin to let us lead you on, deluded, to the last possible
moment, in order that your disappointment might be the heaviest possible
disappointment. That’s the first short speech, do you understand?’

Here, John Harmon assisted his comprehension with another shake.

‘Now, scoundrel,’ he pursued, ‘I am going to finish. You supposed me
just now, to be the possessor of my father’s property.--So I am. But
through any act of my father’s, or by any right I have? No. Through the
munificence of Mr Boffin. The conditions that he made with me, before
parting with the secret of the Dutch bottle, were, that I should take
the fortune, and that he should take his Mound and no more. I owe
everything I possess, solely to the disinterestedness, uprightness,
tenderness, goodness (there are no words to satisfy me) of Mr and Mrs
Boffin. And when, knowing what I knew, I saw such a mud-worm as you
presume to rise in this house against this noble soul, the wonder is,’
added John Harmon through his clenched teeth, and with a very ugly turn
indeed on Wegg’s cravat, ‘that I didn’t try to twist your head off,
and fling THAT out of window! So. That’s the last short speech, do you
understand?’

Silas, released, put his hand to his throat, cleared it, and looked as
if he had a rather large fishbone in that region. Simultaneously with
this action on his part in his corner, a singular, and on the surface
an incomprehensible, movement was made by Mr Sloppy: who began backing
towards Mr Wegg along the wall, in the manner of a porter or heaver who
is about to lift a sack of flour or coals.

‘I am sorry, Wegg,’ said Mr Boffin, in his clemency, ‘that my old lady
and I can’t have a better opinion of you than the bad one we are forced
to entertain. But I shouldn’t like to leave you, after all said and
done, worse off in life than I found you. Therefore say in a word,
before we part, what it’ll cost to set you up in another stall.’

‘And in another place,’ John Harmon struck in. ‘You don’t come outside
these windows.’

‘Mr Boffin,’ returned Wegg in avaricious humiliation: ‘when I first had
the honour of making your acquaintance, I had got together a collection
of ballads which was, I may say, above price.’

‘Then they can’t be paid for,’ said John Harmon, ‘and you had better not
try, my dear sir.’

‘Pardon me, Mr Boffin,’ resumed Wegg, with a malignant glance in the
last speaker’s direction, ‘I was putting the case to you, who, if my
senses did not deceive me, put the case to me. I had a very choice
collection of ballads, and there was a new stock of gingerbread in the
tin box. I say no more, but would rather leave it to you.’

‘But it’s difficult to name what’s right,’ said Mr Boffin uneasily, with
his hand in his pocket, ‘and I don’t want to go beyond what’s right,
because you really have turned out such a very bad fellow. So artful,
and so ungrateful you have been, Wegg; for when did I ever injure you?’

‘There was also,’ Mr Wegg went on, in a meditative manner, ‘a errand
connection, in which I was much respected. But I would not wish to be
deemed covetous, and I would rather leave it to you, Mr Boffin.’

‘Upon my word, I don’t know what to put it at,’ the Golden Dustman
muttered.

‘There was likewise,’ resumed Wegg, ‘a pair of trestles, for which alone
a Irish person, who was deemed a judge of trestles, offered five and
six--a sum I would not hear of, for I should have lost by it--and there
was a stool, a umbrella, a clothes-horse, and a tray. But I leave it to
you, Mr Boffin.’

The Golden Dustman seeming to be engaged in some abstruse calculation,
Mr Wegg assisted him with the following additional items.

‘There was, further, Miss Elizabeth, Master George, Aunt Jane, and Uncle
Parker. Ah! When a man thinks of the loss of such patronage as that;
when a man finds so fair a garden rooted up by pigs; he finds it hard
indeed, without going high, to work it into money. But I leave it wholly
to you, sir.’

Mr Sloppy still continued his singular, and on the surface his
incomprehensible, movement.

‘Leading on has been mentioned,’ said Wegg with a melancholy air, ‘and
it’s not easy to say how far the tone of my mind may have been lowered
by unwholesome reading on the subject of Misers, when you was leading me
and others on to think you one yourself, sir. All I can say is, that
I felt my tone of mind a lowering at the time. And how can a man put a
price upon his mind! There was likewise a hat just now. But I leave the
ole to you, Mr Boffin.’

‘Come!’ said Mr Boffin. ‘Here’s a couple of pound.’

‘In justice to myself, I couldn’t take it, sir.’

The words were but out of his mouth when John Harmon lifted his finger,
and Sloppy, who was now close to Wegg, backed to Wegg’s back, stooped,
grasped his coat collar behind with both hands, and deftly swung him
up like the sack of flour or coals before mentioned. A countenance of
special discontent and amazement Mr Wegg exhibited in this position,
with his buttons almost as prominently on view as Sloppy’s own, and
with his wooden leg in a highly unaccommodating state. But, not for many
seconds was his countenance visible in the room; for, Sloppy lightly
trotted out with him and trotted down the staircase, Mr Venus attending
to open the street door. Mr Sloppy’s instructions had been to deposit
his burden in the road; but, a scavenger’s cart happening to stand
unattended at the corner, with its little ladder planted against the
wheel, Mr S. found it impossible to resist the temptation of shooting Mr
Silas Wegg into the cart’s contents. A somewhat difficult feat, achieved
with great dexterity, and with a prodigious splash.



Chapter 15

WHAT WAS CAUGHT IN THE TRAPS THAT WERE SET


How Bradley Headstone had been racked and riven in his mind since the
quiet evening when by the river-side he had risen, as it were, out of
the ashes of the Bargeman, none but he could have told. Not even he
could have told, for such misery can only be felt.

First, he had to bear the combined weight of the knowledge of what he
had done, of that haunting reproach that he might have done it so much
better, and of the dread of discovery. This was load enough to crush
him, and he laboured under it day and night. It was as heavy on him in
his scanty sleep, as in his red-eyed waking hours. It bore him down with
a dread unchanging monotony, in which there was not a moment’s variety.
The overweighted beast of burden, or the overweighted slave, can for
certain instants shift the physical load, and find some slight respite
even in enforcing additional pain upon such a set of muscles or such
a limb. Not even that poor mockery of relief could the wretched man
obtain, under the steady pressure of the infernal atmosphere into which
he had entered.

Time went by, and no visible suspicion dogged him; time went by, and
in such public accounts of the attack as were renewed at intervals,
he began to see Mr Lightwood (who acted as lawyer for the injured man)
straying further from the fact, going wider of the issue, and evidently
slackening in his zeal. By degrees, a glimmering of the cause of this
began to break on Bradley’s sight. Then came the chance meeting with Mr
Milvey at the railway station (where he often lingered in his leisure
hours, as a place where any fresh news of his deed would be circulated,
or any placard referring to it would be posted), and then he saw in the
light what he had brought about.

For, then he saw that through his desperate attempt to separate those
two for ever, he had been made the means of uniting them. That he had
dipped his hands in blood, to mark himself a miserable fool and tool.
That Eugene Wrayburn, for his wife’s sake, set him aside and left him to
crawl along his blasted course. He thought of Fate, or Providence, or
be the directing Power what it might, as having put a fraud upon
him--overreached him--and in his impotent mad rage bit, and tore, and
had his fit.

New assurance of the truth came upon him in the next few following days,
when it was put forth how the wounded man had been married on his bed,
and to whom, and how, though always in a dangerous condition, he was a
shade better. Bradley would far rather have been seized for his murder,
than he would have read that passage, knowing himself spared, and
knowing why.

But, not to be still further defrauded and overreached--which he would
be, if implicated by Riderhood, and punished by the law for his abject
failure, as though it had been a success--he kept close in his school
during the day, ventured out warily at night, and went no more to the
railway station. He examined the advertisements in the newspapers for
any sign that Riderhood acted on his hinted threat of so summoning him
to renew their acquaintance, but found none. Having paid him handsomely
for the support and accommodation he had had at the Lock House, and
knowing him to be a very ignorant man who could not write, he began to
doubt whether he was to be feared at all, or whether they need ever meet
again.

All this time, his mind was never off the rack, and his raging sense of
having been made to fling himself across the chasm which divided those
two, and bridge it over for their coming together, never cooled down.
This horrible condition brought on other fits. He could not have said
how many, or when; but he saw in the faces of his pupils that they had
seen him in that state, and that they were possessed by a dread of his
relapsing.

One winter day when a slight fall of snow was feathering the sills and
frames of the schoolroom windows, he stood at his black board, crayon in
hand, about to commence with a class; when, reading in the countenances
of those boys that there was something wrong, and that they seemed in
alarm for him, he turned his eyes to the door towards which they faced.
He then saw a slouching man of forbidding appearance standing in the
midst of the school, with a bundle under his arm; and saw that it was
Riderhood.

He sat down on a stool which one of his boys put for him, and he had a
passing knowledge that he was in danger of falling, and that his face
was becoming distorted. But, the fit went off for that time, and he
wiped his mouth, and stood up again.

‘Beg your pardon, governor! By your leave!’ said Riderhood, knuckling
his forehead, with a chuckle and a leer. ‘What place may this be?’

‘This is a school.’

‘Where young folks learns wot’s right?’ said Riderhood, gravely nodding.
‘Beg your pardon, governor! By your leave! But who teaches this school?’

‘I do.’

‘You’re the master, are you, learned governor?’

‘Yes. I am the master.’

‘And a lovely thing it must be,’ said Riderhood, ‘fur to learn young
folks wot’s right, and fur to know wot THEY know wot you do it. Beg your
pardon, learned governor! By your leave!--That there black board; wot’s
it for?’

‘It is for drawing on, or writing on.’

‘Is it though!’ said Riderhood. ‘Who’d have thought it, from the
looks on it! WOULD you be so kind as write your name upon it, learned
governor?’ (In a wheedling tone.)

Bradley hesitated for a moment; but placed his usual signature,
enlarged, upon the board.

‘I ain’t a learned character myself,’ said Riderhood, surveying the
class, ‘but I do admire learning in others. I should dearly like to hear
these here young folks read that there name off, from the writing.’

The arms of the class went up. At the miserable master’s nod, the shrill
chorus arose: ‘Bradley Headstone!’

‘No?’ cried Riderhood. ‘You don’t mean it? Headstone! Why, that’s in a
churchyard. Hooroar for another turn!’

Another tossing of arms, another nod, and another shrill chorus:

‘Bradley Headstone!’

‘I’ve got it now!’ said Riderhood, after attentively listening, and
internally repeating: ‘Bradley. I see. Chris’en name, Bradley sim’lar to
Roger which is my own. Eh? Fam’ly name, Headstone, sim’lar to Riderhood
which is my own. Eh?’

Shrill chorus. ‘Yes!’

‘Might you be acquainted, learned governor,’ said Riderhood, ‘with a
person of about your own heighth and breadth, and wot ‘ud pull down in
a scale about your own weight, answering to a name sounding summat like
Totherest?’

With a desperation in him that made him perfectly quiet, though his jaw
was heavily squared; with his eyes upon Riderhood; and with traces of
quickened breathing in his nostrils; the schoolmaster replied, in a
suppressed voice, after a pause: ‘I think I know the man you mean.’

‘I thought you knowed the man I mean, learned governor. I want the man.’

With a half glance around him at his pupils, Bradley returned:

‘Do you suppose he is here?’

‘Begging your pardon, learned governor, and by your leave,’ said
Riderhood, with a laugh, ‘how could I suppose he’s here, when there’s
nobody here but you, and me, and these young lambs wot you’re a learning
on? But he is most excellent company, that man, and I want him to come
and see me at my Lock, up the river.’

‘I’ll tell him so.’

‘D’ye think he’ll come?’ asked Riderhood.

‘I am sure he will.’

‘Having got your word for him,’ said Riderhood, ‘I shall count upon him.
P’raps you’d so fur obleege me, learned governor, as tell him that if he
don’t come precious soon, I’ll look him up.’

‘He shall know it.’

‘Thankee. As I says a while ago,’ pursued Riderhood, changing his hoarse
tone and leering round upon the class again, ‘though not a learned
character my own self, I do admire learning in others, to be sure! Being
here and having met with your kind attention, Master, might I, afore I
go, ask a question of these here young lambs of yourn?’

‘If it is in the way of school,’ said Bradley, always sustaining his
dark look at the other, and speaking in his suppressed voice, ‘you may.’

‘Oh! It’s in the way of school!’ cried Riderhood. ‘I’ll pound it,
Master, to be in the way of school. Wot’s the diwisions of water, my
lambs? Wot sorts of water is there on the land?’

Shrill chorus: ‘Seas, rivers, lakes, and ponds.’

‘Seas, rivers, lakes, and ponds,’ said Riderhood. ‘They’ve got all the
lot, Master! Blowed if I shouldn’t have left out lakes, never having
clapped eyes upon one, to my knowledge. Seas, rivers, lakes, and ponds.
Wot is it, lambs, as they ketches in seas, rivers, lakes, and ponds?’

Shrill chorus (with some contempt for the ease of the question):

‘Fish!’

‘Good a-gin!’ said Riderhood. ‘But wot else is it, my lambs, as they
sometimes ketches in rivers?’

Chorus at a loss. One shrill voice: ‘Weed!’

‘Good agin!’ cried Riderhood. ‘But it ain’t weed neither. You’ll never
guess, my dears. Wot is it, besides fish, as they sometimes ketches in
rivers? Well! I’ll tell you. It’s suits o’ clothes.’

Bradley’s face changed.

‘Leastways, lambs,’ said Riderhood, observing him out of the corners
of his eyes, ‘that’s wot I my own self sometimes ketches in rivers. For
strike me blind, my lambs, if I didn’t ketch in a river the wery bundle
under my arm!’

The class looked at the master, as if appealing from the irregular
entrapment of this mode of examination. The master looked at the
examiner, as if he would have torn him to pieces.

‘I ask your pardon, learned governor,’ said Riderhood, smearing his
sleeve across his mouth as he laughed with a relish, ‘tain’t fair to the
lambs, I know. It wos a bit of fun of mine. But upon my soul I drawed
this here bundle out of a river! It’s a Bargeman’s suit of clothes. You
see, it had been sunk there by the man as wore it, and I got it up.’

‘How do you know it was sunk by the man who wore it?’ asked Bradley.

‘Cause I see him do it,’ said Riderhood.

They looked at each other. Bradley, slowly withdrawing his eyes, turned
his face to the black board and slowly wiped his name out.

‘A heap of thanks, Master,’ said Riderhood, ‘for bestowing so much of
your time, and of the lambses’ time, upon a man as hasn’t got no other
recommendation to you than being a honest man. Wishing to see at my Lock
up the river, the person as we’ve spoke of, and as you’ve answered for,
I takes my leave of the lambs and of their learned governor both.’

With those words, he slouched out of the school, leaving the master
to get through his weary work as he might, and leaving the whispering
pupils to observe the master’s face until he fell into the fit which had
been long impending.

The next day but one was Saturday, and a holiday. Bradley rose early,
and set out on foot for Plashwater Weir Mill Lock. He rose so early that
it was not yet light when he began his journey. Before extinguishing the
candle by which he had dressed himself, he made a little parcel of his
decent silver watch and its decent guard, and wrote inside the paper:
‘Kindly take care of these for me.’ He then addressed the parcel to Miss
Peecher, and left it on the most protected corner of the little seat in
her little porch.

It was a cold hard easterly morning when he latched the garden gate
and turned away. The light snowfall which had feathered his schoolroom
windows on the Thursday, still lingered in the air, and was falling
white, while the wind blew black. The tardy day did not appear until he
had been on foot two hours, and had traversed a greater part of London
from east to west. Such breakfast as he had, he took at the comfortless
public-house where he had parted from Riderhood on the occasion of
their night-walk. He took it, standing at the littered bar, and looked
loweringly at a man who stood where Riderhood had stood that early
morning.

He outwalked the short day, and was on the towing-path by the river,
somewhat footsore, when the night closed in. Still two or three miles
short of the Lock, he slackened his pace then, but went steadily on. The
ground was now covered with snow, though thinly, and there were floating
lumps of ice in the more exposed parts of the river, and broken sheets
of ice under the shelter of the banks. He took heed of nothing but the
ice, the snow, and the distance, until he saw a light ahead, which he
knew gleamed from the Lock House window. It arrested his steps, and he
looked all around. The ice, and the snow, and he, and the one light, had
absolute possession of the dreary scene. In the distance before him, lay
the place where he had struck the worse than useless blows that mocked
him with Lizzie’s presence there as Eugene’s wife. In the distance
behind him, lay the place where the children with pointing arms had
seemed to devote him to the demons in crying out his name. Within there,
where the light was, was the man who as to both distances could give him
up to ruin. To these limits had his world shrunk.

He mended his pace, keeping his eyes upon the light with a strange
intensity, as if he were taking aim at it. When he approached it so
nearly as that it parted into rays, they seemed to fasten themselves
to him and draw him on. When he struck the door with his hand, his foot
followed so quickly on his hand, that he was in the room before he was
bidden to enter.

The light was the joint product of a fire and a candle. Between the two,
with his feet on the iron fender, sat Riderhood, pipe in mouth.

He looked up with a surly nod when his visitor came in. His visitor
looked down with a surly nod. His outer clothing removed, the visitor
then took a seat on the opposite side of the fire.

‘Not a smoker, I think?’ said Riderhood, pushing a bottle to him across
the table.

‘No.’

They both lapsed into silence, with their eyes upon the fire.

‘You don’t need to be told I am here,’ said Bradley at length. ‘Who is
to begin?’

‘I’ll begin,’ said Riderhood, ‘when I’ve smoked this here pipe out.’

He finished it with great deliberation, knocked out the ashes on the
hob, and put it by.

‘I’ll begin,’ he then repeated, ‘Bradley Headstone, Master, if you wish
it.’

‘Wish it? I wish to know what you want with me.’

‘And so you shall.’ Riderhood had looked hard at his hands and his
pockets, apparently as a precautionary measure lest he should have any
weapon about him. But, he now leaned forward, turning the collar of
his waistcoat with an inquisitive finger, and asked, ‘Why, where’s your
watch?’

‘I have left it behind.’

‘I want it. But it can be fetched. I’ve took a fancy to it.’

Bradley answered with a contemptuous laugh.

‘I want it,’ repeated Riderhood, in a louder voice, ‘and I mean to have
it.’

‘That is what you want of me, is it?’

‘No,’ said Riderhood, still louder; ‘it’s on’y part of what I want of
you. I want money of you.’

‘Anything else?’

‘Everythink else!’ roared Riderhood, in a very loud and furious way.
‘Answer me like that, and I won’t talk to you at all.’

Bradley looked at him.

‘Don’t so much as look at me like that, or I won’t talk to you at all,’
vociferated Riderhood. ‘But, instead of talking, I’ll bring my hand
down upon you with all its weight,’ heavily smiting the table with great
force, ‘and smash you!’

‘Go on,’ said Bradley, after moistening his lips.

‘O! I’m a going on. Don’t you fear but I’ll go on full-fast enough for
you, and fur enough for you, without your telling. Look here, Bradley
Headstone, Master. You might have split the T’other governor to chips
and wedges, without my caring, except that I might have come upon you
for a glass or so now and then. Else why have to do with you at all? But
when you copied my clothes, and when you copied my neckhankercher, and
when you shook blood upon me after you had done the trick, you did wot
I’ll be paid for and paid heavy for. If it come to be throw’d upon you,
you was to be ready to throw it upon me, was you? Where else but
in Plashwater Weir Mill Lock was there a man dressed according as
described? Where else but in Plashwater Weir Mill Lock was there a
man as had had words with him coming through in his boat? Look at the
Lock-keeper in Plashwater Weir Mill Lock, in them same answering clothes
and with that same answering red neckhankercher, and see whether his
clothes happens to be bloody or not. Yes, they do happen to be bloody.
Ah, you sly devil!’

Bradley, very white, sat looking at him in silence.

‘But two could play at your game,’ said Riderhood, snapping his fingers
at him half a dozen times, ‘and I played it long ago; long afore you
tried your clumsy hand at it; in days when you hadn’t begun croaking
your lecters or what not in your school. I know to a figure how you
done it. Where you stole away, I could steal away arter you, and do it
knowinger than you. I know how you come away from London in your own
clothes, and where you changed your clothes, and hid your clothes. I see
you with my own eyes take your own clothes from their hiding-place
among them felled trees, and take a dip in the river to account for
your dressing yourself, to any one as might come by. I see you rise up
Bradley Headstone, Master, where you sat down Bargeman. I see you pitch
your Bargeman’s bundle into the river. I hooked your Bargeman’s bundle
out of the river. I’ve got your Bargeman’s clothes, tore this way and
that way with the scuffle, stained green with the grass, and spattered
all over with what bust from the blows. I’ve got them, and I’ve got you.
I don’t care a curse for the T’other governor, alive or dead, but I care
a many curses for my own self. And as you laid your plots agin me and
was a sly devil agin me, I’ll be paid for it--I’ll be paid for it--I’ll
be paid for it--till I’ve drained you dry!’

Bradley looked at the fire, with a working face, and was silent for a
while. At last he said, with what seemed an inconsistent composure of
voice and feature:

‘You can’t get blood out of a stone, Riderhood.’

‘I can get money out of a schoolmaster though.’

‘You can’t get out of me what is not in me. You can’t wrest from me what
I have not got. Mine is but a poor calling. You have had more than two
guineas from me, already. Do you know how long it has taken me (allowing
for a long and arduous training) to earn such a sum?’

‘I don’t know, nor I don’t care. Yours is a ‘spectable calling. To
save your ‘spectability, it’s worth your while to pawn every article of
clothes you’ve got, sell every stick in your house, and beg and borrow
every penny you can get trusted with. When you’ve done that and handed
over, I’ll leave you. Not afore.’

‘How do you mean, you’ll leave me?’

‘I mean as I’ll keep you company, wherever you go, when you go away from
here. Let the Lock take care of itself. I’ll take care of you, once I’ve
got you.’

Bradley again looked at the fire. Eyeing him aside, Riderhood took up
his pipe, refilled it, lighted it, and sat smoking. Bradley leaned his
elbows on his knees, and his head upon his hands, and looked at the fire
with a most intent abstraction.

‘Riderhood,’ he said, raising himself in his chair, after a long
silence, and drawing out his purse and putting it on the table. ‘Say
I part with this, which is all the money I have; say I let you have
my watch; say that every quarter, when I draw my salary, I pay you a
certain portion of it.’

‘Say nothink of the sort,’ retorted Riderhood, shaking his head as he
smoked. ‘You’ve got away once, and I won’t run the chance agin. I’ve had
trouble enough to find you, and shouldn’t have found you, if I hadn’t
seen you slipping along the street overnight, and watched you till you
was safe housed. I’ll have one settlement with you for good and all.’

‘Riderhood, I am a man who has lived a retired life. I have no resources
beyond myself. I have absolutely no friends.’

‘That’s a lie,’ said Riderhood. ‘You’ve got one friend as I knows of;
one as is good for a Savings-Bank book, or I’m a blue monkey!’

Bradley’s face darkened, and his hand slowly closed on the purse and
drew it back, as he sat listening for what the other should go on to
say.

‘I went into the wrong shop, fust, last Thursday,’ said Riderhood.
‘Found myself among the young ladies, by George! Over the young ladies,
I see a Missis. That Missis is sweet enough upon you, Master, to sell
herself up, slap, to get you out of trouble. Make her do it then.’

Bradley stared at him so very suddenly that Riderhood, not quite knowing
how to take it, affected to be occupied with the encircling smoke from
his pipe; fanning it away with his hand, and blowing it off.

‘You spoke to the mistress, did you?’ inquired Bradley, with that
former composure of voice and feature that seemed inconsistent, and with
averted eyes.

‘Poof! Yes,’ said Riderhood, withdrawing his attention from the smoke.
‘I spoke to her. I didn’t say much to her. She was put in a fluster by
my dropping in among the young ladies (I never did set up for a lady’s
man), and she took me into her parlour to hope as there was nothink
wrong. I tells her, “O no, nothink wrong. The master’s my wery good
friend.” But I see how the land laid, and that she was comfortable off.’

Bradley put the purse in his pocket, grasped his left wrist with his
right hand, and sat rigidly contemplating the fire.

‘She couldn’t live more handy to you than she does,’ said Riderhood,
‘and when I goes home with you (as of course I am a going), I recommend
you to clean her out without loss of time. You can marry her, arter you
and me have come to a settlement. She’s nice-looking, and I know
you can’t be keeping company with no one else, having been so lately
disapinted in another quarter.’

Not one other word did Bradley utter all that night. Not once did he
change his attitude, or loosen his hold upon his wrist. Rigid before the
fire, as if it were a charmed flame that was turning him old, he sat,
with the dark lines deepening in his face, its stare becoming more and
more haggard, its surface turning whiter and whiter as if it were being
overspread with ashes, and the very texture and colour of his hair
degenerating.

Not until the late daylight made the window transparent, did this
decaying statue move. Then it slowly arose, and sat in the window
looking out.

Riderhood had kept his chair all night. In the earlier part of the night
he had muttered twice or thrice that it was bitter cold; or that the
fire burnt fast, when he got up to mend it; but, as he could elicit from
his companion neither sound nor movement, he had afterwards held his
peace. He was making some disorderly preparations for coffee, when
Bradley came from the window and put on his outer coat and hat.

‘Hadn’t us better have a bit o’ breakfast afore we start?’ said
Riderhood. ‘It ain’t good to freeze a empty stomach, Master.’

Without a sign to show that he heard, Bradley walked out of the Lock
House. Catching up from the table a piece of bread, and taking his
Bargeman’s bundle under his arm, Riderhood immediately followed him.
Bradley turned towards London. Riderhood caught him up, and walked at
his side.

The two men trudged on, side by side, in silence, full three miles.
Suddenly, Bradley turned to retrace his course. Instantly, Riderhood
turned likewise, and they went back side by side.

Bradley re-entered the Lock House. So did Riderhood. Bradley sat down in
the window. Riderhood warmed himself at the fire. After an hour or more,
Bradley abruptly got up again, and again went out, but this time turned
the other way. Riderhood was close after him, caught him up in a few
paces, and walked at his side.

This time, as before, when he found his attendant not to be shaken off,
Bradley suddenly turned back. This time, as before, Riderhood turned
back along with him. But, not this time, as before, did they go into the
Lock House, for Bradley came to a stand on the snow-covered turf by the
Lock, looking up the river and down the river. Navigation was impeded by
the frost, and the scene was a mere white and yellow desert.

‘Come, come, Master,’ urged Riderhood, at his side. ‘This is a dry game.
And where’s the good of it? You can’t get rid of me, except by coming to
a settlement. I am a going along with you wherever you go.’

Without a word of reply, Bradley passed quickly from him over the wooden
bridge on the lock gates. ‘Why, there’s even less sense in this move
than t’other,’ said Riderhood, following. ‘The Weir’s there, and you’ll
have to come back, you know.’

Without taking the least notice, Bradley leaned his body against a post,
in a resting attitude, and there rested with his eyes cast down. ‘Being
brought here,’ said Riderhood, gruffly, ‘I’ll turn it to some use by
changing my gates.’ With a rattle and a rush of water, he then swung-to
the lock gates that were standing open, before opening the others. So,
both sets of gates were, for the moment, closed.

‘You’d better by far be reasonable, Bradley Headstone, Master,’ said
Riderhood, passing him, ‘or I’ll drain you all the dryer for it, when we
do settle.--Ah! Would you!’

Bradley had caught him round the body. He seemed to be girdled with an
iron ring. They were on the brink of the Lock, about midway between the
two sets of gates.

‘Let go!’ said Riderhood, ‘or I’ll get my knife out and slash you
wherever I can cut you. Let go!’

Bradley was drawing to the Lock-edge. Riderhood was drawing away from
it. It was a strong grapple, and a fierce struggle, arm and leg. Bradley
got him round, with his back to the Lock, and still worked him backward.

‘Let go!’ said Riderhood. ‘Stop! What are you trying at? You can’t drown
Me. Ain’t I told you that the man as has come through drowning can never
be drowned? I can’t be drowned.’

‘I can be!’ returned Bradley, in a desperate, clenched voice. ‘I am
resolved to be. I’ll hold you living, and I’ll hold you dead. Come
down!’

Riderhood went over into the smooth pit, backward, and Bradley Headstone
upon him. When the two were found, lying under the ooze and scum behind
one of the rotting gates, Riderhood’s hold had relaxed, probably in
falling, and his eyes were staring upward. But, he was girdled still
with Bradley’s iron ring, and the rivets of the iron ring held tight.



Chapter 16

PERSONS AND THINGS IN GENERAL


Mr and Mrs John Harmon’s first delightful occupation was, to set all
matters right that had strayed in any way wrong, or that might, could,
would, or should, have strayed in any way wrong, while their name was in
abeyance. In tracing out affairs for which John’s fictitious death was
to be considered in any way responsible, they used a very broad and free
construction; regarding, for instance, the dolls’ dressmaker as having
a claim on their protection, because of her association with Mrs Eugene
Wrayburn, and because of Mrs Eugene’s old association, in her turn, with
the dark side of the story. It followed that the old man, Riah, as a
good and serviceable friend to both, was not to be disclaimed. Nor even
Mr Inspector, as having been trepanned into an industrious hunt on a
false scent. It may be remarked, in connexion with that worthy officer,
that a rumour shortly afterwards pervaded the Force, to the effect that
he had confided to Miss Abbey Potterson, over a jug of mellow flip in
the bar of the Six Jolly Fellowship Porters, that he ‘didn’t stand to
lose a farthing’ through Mr Harmon’s coming to life, but was quite as
well satisfied as if that gentleman had been barbarously murdered, and
he (Mr Inspector) had pocketed the government reward.

In all their arrangements of such nature, Mr and Mrs John Harmon derived
much assistance from their eminent solicitor, Mr Mortimer Lightwood; who
laid about him professionally with such unwonted despatch and intention,
that a piece of work was vigorously pursued as soon as cut out; whereby
Young Blight was acted on as by that transatlantic dram which is
poetically named An Eye-Opener, and found himself staring at real
clients instead of out of window. The accessibility of Riah proving
very useful as to a few hints towards the disentanglement of Eugene’s
affairs, Lightwood applied himself with infinite zest to attacking and
harassing Mr Fledgeby: who, discovering himself in danger of being blown
into the air by certain explosive transactions in which he had been
engaged, and having been sufficiently flayed under his beating, came
to a parley and asked for quarter. The harmless Twemlow profited by
the conditions entered into, though he little thought it. Mr Riah
unaccountably melted; waited in person on him over the stable yard in
Duke Street, St James’s, no longer ravening but mild, to inform him
that payment of interest as heretofore, but henceforth at Mr Lightwood’s
offices, would appease his Jewish rancour; and departed with the secret
that Mr John Harmon had advanced the money and become the creditor.
Thus, was the sublime Snigsworth’s wrath averted, and thus did he snort
no larger amount of moral grandeur at the Corinthian column in the
print over the fireplace, than was normally in his (and the British)
constitution.


Mrs Wilfer’s first visit to the Mendicant’s bride at the new abode of
Mendicancy, was a grand event. Pa had been sent for into the City,
on the very day of taking possession, and had been stunned with
astonishment, and brought-to, and led about the house by one ear, to
behold its various treasures, and had been enraptured and enchanted. Pa
had also been appointed Secretary, and had been enjoined to give instant
notice of resignation to Chicksey, Veneering, and Stobbles, for ever and
ever. But Ma came later, and came, as was her due, in state.

The carriage was sent for Ma, who entered it with a bearing worthy of
the occasion, accompanied, rather than supported, by Miss Lavinia, who
altogether declined to recognize the maternal majesty. Mr George Sampson
meekly followed. He was received in the vehicle, by Mrs Wilfer, as if
admitted to the honour of assisting at a funeral in the family, and she
then issued the order, ‘Onward!’ to the Mendicant’s menial.

‘I wish to goodness, Ma,’ said Lavvy, throwing herself back among the
cushions, with her arms crossed, ‘that you’d loll a little.’

‘How!’ repeated Mrs Wilfer. ‘Loll!’

‘Yes, Ma.’

‘I hope,’ said the impressive lady, ‘I am incapable of it.’

‘I am sure you look so, Ma. But why one should go out to dine with one’s
own daughter or sister, as if one’s under-petticoat was a backboard, I
do NOT understand.’

‘Neither do I understand,’ retorted Mrs Wilfer, with deep scorn, ‘how
a young lady can mention the garment in the name of which you have
indulged. I blush for you.’

‘Thank you, Ma,’ said Lavvy, yawning, ‘but I can do it for myself, I am
obliged to you, when there’s any occasion.’

Here, Mr Sampson, with the view of establishing harmony, which he never
under any circumstances succeeded in doing, said with an agreeable
smile: ‘After all, you know, ma’am, we know it’s there.’ And immediately
felt that he had committed himself.

‘We know it’s there!’ said Mrs Wilfer, glaring.

‘Really, George,’ remonstrated Miss Lavinia, ‘I must say that I don’t
understand your allusions, and that I think you might be more delicate
and less personal.’

‘Go it!’ cried Mr Sampson, becoming, on the shortest notice, a prey to
despair. ‘Oh yes! Go it, Miss Lavinia Wilfer!’

‘What you may mean, George Sampson, by your omnibus-driving expressions,
I cannot pretend to imagine. Neither,’ said Miss Lavinia, ‘Mr George
Sampson, do I wish to imagine. It is enough for me to know in my own
heart that I am not going to--’ having imprudently got into a sentence
without providing a way out of it, Miss Lavinia was constrained to
close with ‘going to it’. A weak conclusion which, however, derived some
appearance of strength from disdain.

‘Oh yes!’ cried Mr Sampson, with bitterness. ‘Thus it ever is. I
never--’

‘If you mean to say,’ Miss Lavvy cut him short, that you never brought
up a young gazelle, you may save yourself the trouble, because nobody
in this carriage supposes that you ever did. We know you better.’ (As if
this were a home-thrust.)

‘Lavinia,’ returned Mr Sampson, in a dismal vein, ‘I did not mean to
say so. What I did mean to say, was, that I never expected to retain my
favoured place in this family, after Fortune shed her beams upon it. Why
do you take me,’ said Mr Sampson, ‘to the glittering halls with which
I can never compete, and then taunt me with my moderate salary? Is it
generous? Is it kind?’

The stately lady, Mrs Wilfer, perceiving her opportunity of delivering a
few remarks from the throne, here took up the altercation.

‘Mr Sampson,’ she began, ‘I cannot permit you to misrepresent the
intentions of a child of mine.’

‘Let him alone, Ma,’ Miss Lavvy interposed with haughtiness. ‘It is
indifferent to me what he says or does.’

‘Nay, Lavinia,’ quoth Mrs Wilfer, ‘this touches the blood of the family.
If Mr George Sampson attributes, even to my youngest daughter--’

[‘I don’t see why you should use the word “even”, Ma,’ Miss Lavvy
interposed, ‘because I am quite as important as any of the others.’)

‘Peace!’ said Mrs Wilfer, solemnly. ‘I repeat, if Mr George Sampson
attributes, to my youngest daughter, grovelling motives, he attributes
them equally to the mother of my youngest daughter. That mother
repudiates them, and demands of Mr George Sampson, as a youth of honour,
what he WOULD have? I may be mistaken--nothing is more likely--but Mr
George Sampson,’ proceeded Mrs Wilfer, majestically waving her gloves,
‘appears to me to be seated in a first-class equipage. Mr George Sampson
appears to me to be on his way, by his own admission, to a residence
that may be termed Palatial. Mr George Sampson appears to me to be
invited to participate in the--shall I say the--Elevation which has
descended on the family with which he is ambitious, shall I say to
Mingle? Whence, then, this tone on Mr Sampson’s part?’

‘It is only, ma’am,’ Mr Sampson explained, in exceedingly low spirits,
‘because, in a pecuniary sense, I am painfully conscious of my
unworthiness. Lavinia is now highly connected. Can I hope that she will
still remain the same Lavinia as of old? And is it not pardonable if
I feel sensitive, when I see a disposition on her part to take me up
short?’

‘If you are not satisfied with your position, sir,’ observed Miss
Lavinia, with much politeness, ‘we can set you down at any turning you
may please to indicate to my sister’s coachman.’

‘Dearest Lavinia,’ urged Mr Sampson, pathetically, ‘I adore you.’

‘Then if you can’t do it in a more agreeable manner,’ returned the young
lady, ‘I wish you wouldn’t.’

‘I also,’ pursued Mr Sampson, ‘respect you, ma’am, to an extent which
must ever be below your merits, I am well aware, but still up to an
uncommon mark. Bear with a wretch, Lavinia, bear with a wretch, ma’am,
who feels the noble sacrifices you make for him, but is goaded almost to
madness,’ Mr Sampson slapped his forehead, ‘when he thinks of competing
with the rich and influential.’

‘When you have to compete with the rich and influential, it will
probably be mentioned to you,’ said Miss Lavvy, ‘in good time. At least,
it will if the case is MY case.’

Mr Sampson immediately expressed his fervent Opinion that this was ‘more
than human’, and was brought upon his knees at Miss Lavinia’s feet.

It was the crowning addition indispensable to the full enjoyment of both
mother and daughter, to bear Mr Sampson, a grateful captive, into the
glittering halls he had mentioned, and to parade him through the same,
at once a living witness of their glory, and a bright instance of their
condescension. Ascending the staircase, Miss Lavinia permitted him to
walk at her side, with the air of saying: ‘Notwithstanding all these
surroundings, I am yours as yet, George. How long it may last is another
question, but I am yours as yet.’ She also benignantly intimated to him,
aloud, the nature of the objects upon which he looked, and to which he
was unaccustomed: as, ‘Exotics, George,’ ‘An aviary, George,’ ‘An
ormolu clock, George,’ and the like. While, through the whole of the
decorations, Mrs Wilfer led the way with the bearing of a Savage Chief,
who would feel himself compromised by manifesting the slightest token of
surprise or admiration.

Indeed, the bearing of this impressive woman, throughout the day, was a
pattern to all impressive women under similar circumstances. She renewed
the acquaintance of Mr and Mrs Boffin, as if Mr and Mrs Boffin had said
of her what she had said of them, and as if Time alone could quite wear
her injury out. She regarded every servant who approached her, as her
sworn enemy, expressly intending to offer her affronts with the dishes,
and to pour forth outrages on her moral feelings from the decanters.
She sat erect at table, on the right hand of her son-in-law, as half
suspecting poison in the viands, and as bearing up with native force of
character against other deadly ambushes. Her carriage towards Bella was
as a carriage towards a young lady of good position, whom she had met in
society a few years ago. Even when, slightly thawing under the influence
of sparkling champagne, she related to her son-in-law some passages of
domestic interest concerning her papa, she infused into the narrative
such Arctic suggestions of her having been an unappreciated blessing to
mankind, since her papa’s days, and also of that gentleman’s having
been a frosty impersonation of a frosty race, as struck cold to the
very soles of the feet of the hearers. The Inexhaustible being produced,
staring, and evidently intending a weak and washy smile shortly, no
sooner beheld her, than it was stricken spasmodic and inconsolable. When
she took her leave at last, it would have been hard to say whether it
was with the air of going to the scaffold herself, or of leaving the
inmates of the house for immediate execution. Yet, John Harmon enjoyed
it all merrily, and told his wife, when he and she were alone, that her
natural ways had never seemed so dearly natural as beside this foil,
and that although he did not dispute her being her father’s daughter,
he should ever remain stedfast in the faith that she could not be her
mother’s.


This visit was, as has been said, a grand event. Another event, not
grand but deemed in the house a special one, occurred at about the same
period; and this was, the first interview between Mr Sloppy and Miss
Wren.

The dolls’ dressmaker, being at work for the Inexhaustible upon a
full-dressed doll some two sizes larger than that young person, Mr
Sloppy undertook to call for it, and did so.

‘Come in, sir,’ said Miss Wren, who was working at her bench. ‘And who
may you be?’

Mr Sloppy introduced himself by name and buttons.

‘Oh indeed!’ cried Jenny. ‘Ah! I have been looking forward to knowing
you. I heard of your distinguishing yourself.’

‘Did you, Miss?’ grinned Sloppy. ‘I am sure I am glad to hear it, but I
don’t know how.’

‘Pitching somebody into a mud-cart,’ said Miss Wren.

‘Oh! That way!’ cried Sloppy. ‘Yes, Miss.’ And threw back his head and
laughed.

‘Bless us!’ exclaimed Miss Wren, with a start. ‘Don’t open your mouth
as wide as that, young man, or it’ll catch so, and not shut again some
day.’

Mr Sloppy opened it, if possible, wider, and kept it open until his
laugh was out.

‘Why, you’re like the giant,’ said Miss Wren, ‘when he came home in the
land of Beanstalk, and wanted Jack for supper.’

‘Was he good-looking, Miss?’ asked Sloppy.

‘No,’ said Miss Wren. ‘Ugly.’

Her visitor glanced round the room--which had many comforts in it now,
that had not been in it before--and said: ‘This is a pretty place,
Miss.’

‘Glad you think so, sir,’ returned Miss Wren. ‘And what do you think of
Me?’

The honesty of Mr Sloppy being severely taxed by the question, he
twisted a button, grinned, and faltered.

‘Out with it!’ said Miss Wren, with an arch look. ‘Don’t you think me
a queer little comicality?’ In shaking her head at him after asking the
question, she shook her hair down.

‘Oh!’ cried Sloppy, in a burst of admiration. ‘What a lot, and what a
colour!’

Miss Wren, with her usual expressive hitch, went on with her work. But,
left her hair as it was; not displeased by the effect it had made.

‘You don’t live here alone; do you, Miss?’ asked Sloppy.

‘No,’ said Miss Wren, with a chop. ‘Live here with my fairy godmother.’

‘With;’ Mr Sloppy couldn’t make it out; ‘with who did you say, Miss?’

‘Well!’ replied Miss Wren, more seriously. ‘With my second father. Or
with my first, for that matter.’ And she shook her head, and drew a
sigh. ‘If you had known a poor child I used to have here,’ she added,
‘you’d have understood me. But you didn’t, and you can’t. All the
better!’

‘You must have been taught a long time,’ said Sloppy, glancing at the
array of dolls in hand, ‘before you came to work so neatly, Miss, and
with such a pretty taste.’

‘Never was taught a stitch, young man!’ returned the dress-maker,
tossing her head. ‘Just gobbled and gobbled, till I found out how to do
it. Badly enough at first, but better now.’

‘And here have I,’ said Sloppy, in something of a self-reproachful tone,
‘been a learning and a learning, and here has Mr Boffin been a paying
and a paying, ever so long!’

‘I have heard what your trade is,’ observed Miss Wren; ‘it’s
cabinet-making.’

Mr Sloppy nodded. ‘Now that the Mounds is done with, it is. I’ll tell
you what, Miss. I should like to make you something.’

‘Much obliged. But what?’

‘I could make you,’ said Sloppy, surveying the room, ‘I could make you
a handy set of nests to lay the dolls in. Or I could make you a handy
little set of drawers, to keep your silks and threads and scraps in. Or
I could turn you a rare handle for that crutch-stick, if it belongs to
him you call your father.’

‘It belongs to me,’ returned the little creature, with a quick flush of
her face and neck. ‘I am lame.’

Poor Sloppy flushed too, for there was an instinctive delicacy behind
his buttons, and his own hand had struck it. He said, perhaps, the best
thing in the way of amends that could be said. ‘I am very glad it’s
yours, because I’d rather ornament it for you than for any one else.
Please may I look at it?’

Miss Wren was in the act of handing it to him over her bench, when she
paused. ‘But you had better see me use it,’ she said, sharply. ‘This is
the way. Hoppetty, Kicketty, Pep-peg-peg. Not pretty; is it?’

‘It seems to me that you hardly want it at all,’ said Sloppy.

The little dressmaker sat down again, and gave it into his hand, saying,
with that better look upon her, and with a smile: ‘Thank you!’

‘And as concerning the nests and the drawers,’ said Sloppy, after
measuring the handle on his sleeve, and softly standing the stick aside
against the wall, ‘why, it would be a real pleasure to me. I’ve heerd
tell that you can sing most beautiful; and I should be better paid with
a song than with any money, for I always loved the likes of that, and
often giv’ Mrs Higden and Johnny a comic song myself, with “Spoken” in
it. Though that’s not your sort, I’ll wager.’

‘You are a very kind young man,’ returned the dressmaker; ‘a really kind
young man. I accept your offer.--I suppose He won’t mind,’ she added as
an afterthought, shrugging her shoulders; ‘and if he does, he may!’

‘Meaning him that you call your father, Miss,’ asked Sloppy.

‘No, no,’ replied Miss Wren. ‘Him, Him, Him!’

‘Him, him, him?’ repeated Sloppy; staring about, as if for Him.

‘Him who is coming to court and marry me,’ returned Miss Wren. ‘Dear me,
how slow you are!’

‘Oh! HIM!’ said Sloppy. And seemed to turn thoughtful and a little
troubled. ‘I never thought of him. When is he coming, Miss?’

‘What a question!’ cried Miss Wren. ‘How should I know!’

‘Where is he coming from, Miss?’

‘Why, good gracious, how can I tell! He is coming from somewhere or
other, I suppose, and he is coming some day or other, I suppose. I don’t
know any more about him, at present.’

This tickled Mr Sloppy as an extraordinarily good joke, and he threw
back his head and laughed with measureless enjoyment. At the sight of
him laughing in that absurd way, the dolls’ dressmaker laughed very
heartily indeed. So they both laughed, till they were tired.

‘There, there, there!’ said Miss Wren. ‘For goodness’ sake, stop, Giant,
or I shall be swallowed up alive, before I know it. And to this minute
you haven’t said what you’ve come for.’

‘I have come for little Miss Harmonses doll,’ said Sloppy.

‘I thought as much,’ remarked Miss Wren, ‘and here is little Miss
Harmonses doll waiting for you. She’s folded up in silver paper, you
see, as if she was wrapped from head to foot in new Bank notes. Take
care of her, and there’s my hand, and thank you again.’

‘I’ll take more care of her than if she was a gold image,’ said Sloppy,
‘and there’s both MY hands, Miss, and I’ll soon come back again.’


But, the greatest event of all, in the new life of Mr and Mrs John
Harmon, was a visit from Mr and Mrs Eugene Wrayburn. Sadly wan and worn
was the once gallant Eugene, and walked resting on his wife’s arm, and
leaning heavily upon a stick. But, he was daily growing stronger and
better, and it was declared by the medical attendants that he might not
be much disfigured by-and-by. It was a grand event, indeed, when Mr
and Mrs Eugene Wrayburn came to stay at Mr and Mrs John Harmon’s house:
where, by the way, Mr and Mrs Boffin (exquisitely happy, and daily
cruising about, to look at shops,) were likewise staying indefinitely.

To Mr Eugene Wrayburn, in confidence, did Mrs John Harmon impart what
she had known of the state of his wife’s affections, in his reckless
time. And to Mrs John Harmon, in confidence, did Mr Eugene Wrayburn
impart that, please God, she should see how his wife had changed him!

‘I make no protestations,’ said Eugene; ‘--who does, who means them!--I
have made a resolution.’

‘But would you believe, Bella,’ interposed his wife, coming to resume
her nurse’s place at his side, for he never got on well without her:
‘that on our wedding day he told me he almost thought the best thing he
could do, was to die?’

‘As I didn’t do it, Lizzie,’ said Eugene, ‘I’ll do that better thing you
suggested--for your sake.’

That same afternoon, Eugene lying on his couch in his own room upstairs,
Lightwood came to chat with him, while Bella took his wife out for a
ride. ‘Nothing short of force will make her go,’ Eugene had said; so,
Bella had playfully forced her.

‘Dear old fellow,’ Eugene began with Lightwood, reaching up his hand,
‘you couldn’t have come at a better time, for my mind is full, and I
want to empty it. First, of my present, before I touch upon my future.
M. R. F., who is a much younger cavalier than I, and a professed admirer
of beauty, was so affable as to remark the other day (he paid us a visit
of two days up the river there, and much objected to the accommodation
of the hotel), that Lizzie ought to have her portrait painted. Which,
coming from M. R. F., may be considered equivalent to a melodramatic
blessing.’

‘You are getting well,’ said Mortimer, with a smile.

‘Really,’ said Eugene, ‘I mean it. When M. R. F. said that, and followed
it up by rolling the claret (for which he called, and I paid), in his
mouth, and saying, “My dear son, why do you drink this trash?” it was
tantamount in him--to a paternal benediction on our union, accompanied
with a gush of tears. The coolness of M. R. F. is not to be measured by
ordinary standards.’

‘True enough,’ said Lightwood.

‘That’s all,’ pursued Eugene, ‘that I shall ever hear from M. R. F. on
the subject, and he will continue to saunter through the world with
his hat on one side. My marriage being thus solemnly recognized at the
family altar, I have no further trouble on that score. Next, you really
have done wonders for me, Mortimer, in easing my money-perplexities, and
with such a guardian and steward beside me, as the preserver of my life
(I am hardly strong yet, you see, for I am not man enough to refer
to her without a trembling voice--she is so inexpressibly dear to me,
Mortimer!), the little that I can call my own will be more than it ever
has been. It need be more, for you know what it always has been in my
hands. Nothing.’

‘Worse than nothing, I fancy, Eugene. My own small income (I devoutly
wish that my grandfather had left it to the Ocean rather than to me!)
has been an effective Something, in the way of preventing me from
turning to at Anything. And I think yours has been much the same.’

‘There spake the voice of wisdom,’ said Eugene. ‘We are shepherds both.
In turning to at last, we turn to in earnest. Let us say no more of
that, for a few years to come. Now, I have had an idea, Mortimer, of
taking myself and my wife to one of the colonies, and working at my
vocation there.’

‘I should be lost without you, Eugene; but you may be right.’

‘No,’ said Eugene, emphatically. ‘Not right. Wrong!’

He said it with such a lively--almost angry--flash, that Mortimer showed
himself greatly surprised.

‘You think this thumped head of mine is excited?’ Eugene went on, with a
high look; ‘not so, believe me. I can say to you of the healthful music
of my pulse what Hamlet said of his. My blood is up, but wholesomely up,
when I think of it. Tell me! Shall I turn coward to Lizzie, and sneak
away with her, as if I were ashamed of her! Where would your friend’s
part in this world be, Mortimer, if she had turned coward to him, and on
immeasurably better occasion?’

‘Honourable and stanch,’ said Lightwood. ‘And yet, Eugene--’

‘And yet what, Mortimer?’

‘And yet, are you sure that you might not feel (for her sake, I say for
her sake) any slight coldness towards her on the part of--Society?’

‘O! You and I may well stumble at the word,’ returned Eugene, laughing.
‘Do we mean our Tippins?’

‘Perhaps we do,’ said Mortimer, laughing also.

‘Faith, we DO!’ returned Eugene, with great animation. ‘We may hide
behind the bush and beat about it, but we DO! Now, my wife is something
nearer to my heart, Mortimer, than Tippins is, and I owe her a little
more than I owe to Tippins, and I am rather prouder of her than I ever
was of Tippins. Therefore, I will fight it out to the last gasp, with
her and for her, here, in the open field. When I hide her, or strike
for her, faint-heartedly, in a hole or a corner, do you whom I love next
best upon earth, tell me what I shall most righteously deserve to be
told:--that she would have done well to turn me over with her foot that
night when I lay bleeding to death, and spat in my dastard face.’

The glow that shone upon him as he spoke the words, so irradiated his
features that he looked, for the time, as though he had never been
mutilated. His friend responded as Eugene would have had him respond,
and they discoursed of the future until Lizzie came back. After resuming
her place at his side, and tenderly touching his hands and his head, she
said:

‘Eugene, dear, you made me go out, but I ought to have stayed with you.
You are more flushed than you have been for many days. What have you
been doing?’

‘Nothing,’ replied Eugene, ‘but looking forward to your coming back.’

‘And talking to Mr Lightwood,’ said Lizzie, turning to him with a smile.
‘But it cannot have been Society that disturbed you.’

‘Faith, my dear love!’ retorted Eugene, in his old airy manner, as he
laughed and kissed her, ‘I rather think it WAS Society though!’

The word ran so much in Mortimer Lightwood’s thoughts as he went home to
the Temple that night, that he resolved to take a look at Society, which
he had not seen for a considerable period.



Chapter 17

THE VOICE OF SOCIETY


Behoves Mortimer Lightwood, therefore, to answer a dinner card from Mr
and Mrs Veneering requesting the honour, and to signify that Mr Mortimer
Lightwood will be happy to have the other honour. The Veneerings have
been, as usual, indefatigably dealing dinner cards to Society, and
whoever desires to take a hand had best be quick about it, for it is
written in the Books of the Insolvent Fates that Veneering shall make a
resounding smash next week. Yes. Having found out the clue to that great
mystery how people can contrive to live beyond their means, and having
over-jobbed his jobberies as legislator deputed to the Universe by the
pure electors of Pocket-Breaches, it shall come to pass next week that
Veneering will accept the Chiltern Hundreds, that the legal gentleman in
Britannia’s confidence will again accept the Pocket-Breaches Thousands,
and that the Veneerings will retire to Calais, there to live on Mrs
Veneering’s diamonds (in which Mr Veneering, as a good husband, has from
time to time invested considerable sums), and to relate to Neptune and
others, how that, before Veneering retired from Parliament, the House
of Commons was composed of himself and the six hundred and fifty-seven
dearest and oldest friends he had in the world. It shall likewise come
to pass, at as nearly as possible the same period, that Society will
discover that it always did despise Veneering, and distrust Veneering,
and that when it went to Veneering’s to dinner it always had
misgivings--though very secretly at the time, it would seem, and in a
perfectly private and confidential manner.

The next week’s books of the Insolvent Fates, however, being not yet
opened, there is the usual rush to the Veneerings, of the people who go
to their house to dine with one another and not with them. There is Lady
Tippins. There are Podsnap the Great, and Mrs Podsnap. There is Twemlow.
There are Buffer, Boots, and Brewer. There is the Contractor, who
is Providence to five hundred thousand men. There is the Chairman,
travelling three thousand miles per week. There is the brilliant genius
who turned the shares into that remarkably exact sum of three hundred
and seventy five thousand pounds, no shillings, and nopence.

To whom, add Mortimer Lightwood, coming in among them with a
reassumption of his old languid air, founded on Eugene, and belonging to
the days when he told the story of the man from Somewhere.

That fresh fairy, Tippins, all but screams at sight of her false
swain. She summons the deserter to her with her fan; but the deserter,
predetermined not to come, talks Britain with Podsnap. Podsnap always
talks Britain, and talks as if he were a sort of Private Watchman
employed, in the British interests, against the rest of the world. ‘We
know what Russia means, sir,’ says Podsnap; ‘we know what France wants;
we see what America is up to; but we know what England is. That’s enough
for us.’

However, when dinner is served, and Lightwood drops into his old place
over against Lady Tippins, she can be fended off no longer. ‘Long
banished Robinson Crusoe,’ says the charmer, exchanging salutations,
‘how did you leave the Island?’

‘Thank you,’ says Lightwood. ‘It made no complaint of being in pain
anywhere.’

‘Say, how did you leave the savages?’ asks Lady Tippins.

‘They were becoming civilized when I left Juan Fernandez,’ says
Lightwood. ‘At least they were eating one another, which looked like
it.’

‘Tormentor!’ returns the dear young creature. ‘You know what I mean, and
you trifle with my impatience. Tell me something, immediately, about the
married pair. You were at the wedding.’

‘Was I, by-the-by?’ Mortimer pretends, at great leisure, to consider.
‘So I was!’

‘How was the bride dressed? In rowing costume?’

Mortimer looks gloomy, and declines to answer.

‘I hope she steered herself, skiffed herself, paddled herself,
larboarded and starboarded herself, or whatever the technical term may
be, to the ceremony?’ proceeds the playful Tippins.

‘However she got to it, she graced it,’ says Mortimer.

Lady Tippins with a skittish little scream, attracts the general
attention. ‘Graced it! Take care of me if I faint, Veneering. He means
to tell us, that a horrid female waterman is graceful!’

‘Pardon me. I mean to tell you nothing, Lady Tippins,’ replies
Lightwood. And keeps his word by eating his dinner with a show of the
utmost indifference.

‘You shall not escape me in this way, you morose backwoodsman,’ retorts
Lady Tippins. ‘You shall not evade the question, to screen your friend
Eugene, who has made this exhibition of himself. The knowledge shall be
brought home to you that such a ridiculous affair is condemned by the
voice of Society. My dear Mrs Veneering, do let us resolve ourselves
into a Committee of the whole House on the subject.’

Mrs Veneering, always charmed by this rattling sylph, cries. ‘Oh yes!
Do let us resolve ourselves into a Committee of the whole House!
So delicious!’ Veneering says, ‘As many as are of that opinion, say
Aye,--contrary, No--the Ayes have it.’ But nobody takes the slightest
notice of his joke.

‘Now, I am Chairwoman of Committees!’ cries Lady Tippins.

[‘What spirits she has!’ exclaims Mrs Veneering; to whom likewise nobody
attends.)

‘And this,’ pursues the sprightly one, ‘is a Committee of the whole
House to what-you-may-call-it--elicit, I suppose--the voice of Society.
The question before the Committee is, whether a young man of very fair
family, good appearance, and some talent, makes a fool or a wise man of
himself in marrying a female waterman, turned factory girl.’

‘Hardly so, I think,’ the stubborn Mortimer strikes in. ‘I take the
question to be, whether such a man as you describe, Lady Tippins, does
right or wrong in marrying a brave woman (I say nothing of her beauty),
who has saved his life, with a wonderful energy and address; whom he
knows to be virtuous, and possessed of remarkable qualities; whom he has
long admired, and who is deeply attached to him.’

‘But, excuse me,’ says Podsnap, with his temper and his shirt-collar
about equally rumpled; ‘was this young woman ever a female waterman?’

‘Never. But she sometimes rowed in a boat with her father, I believe.’

General sensation against the young woman. Brewer shakes his head. Boots
shakes his head. Buffer shakes his head.

‘And now, Mr Lightwood, was she ever,’ pursues Podsnap, with his
indignation rising high into those hair-brushes of his, ‘a factory
girl?’

‘Never. But she had some employment in a paper mill, I believe.’

General sensation repeated. Brewer says, ‘Oh dear!’ Boots says, ‘Oh
dear!’ Buffer says, ‘Oh dear!’ All, in a rumbling tone of protest.

‘Then all I have to say is,’ returns Podsnap, putting the thing away
with his right arm, ‘that my gorge rises against such a marriage--that
it offends and disgusts me--that it makes me sick--and that I desire to
know no more about it.’

[‘Now I wonder,’ thinks Mortimer, amused, ‘whether YOU are the Voice of
Society!’)

‘Hear, hear, hear!’ cries Lady Tippins. ‘Your opinion of this
MESALLIANCE, honourable colleagues of the honourable member who has just
sat down?’

Mrs Podsnap is of opinion that in these matters there should be an
equality of station and fortune, and that a man accustomed to Society
should look out for a woman accustomed to Society and capable of bearing
her part in it with--an ease and elegance of carriage--that.’ Mrs
Podsnap stops there, delicately intimating that every such man should
look out for a fine woman as nearly resembling herself as he may hope to
discover.

[‘Now I wonder,’ thinks Mortimer, ‘whether you are the Voice!’)

Lady Tippins next canvasses the Contractor, of five hundred thousand
power. It appears to this potentate, that what the man in question
should have done, would have been, to buy the young woman a boat and a
small annuity, and set her up for herself. These things are a question
of beefsteaks and porter. You buy the young woman a boat. Very good. You
buy her, at the same time, a small annuity. You speak of that annuity in
pounds sterling, but it is in reality so many pounds of beefsteaks and
so many pints of porter. On the one hand, the young woman has the boat.
On the other hand, she consumes so many pounds of beefsteaks and so many
pints of porter. Those beefsteaks and that porter are the fuel to that
young woman’s engine. She derives therefrom a certain amount of power to
row the boat; that power will produce so much money; you add that to the
small annuity; and thus you get at the young woman’s income. That (it
seems to the Contractor) is the way of looking at it.

The fair enslaver having fallen into one of her gentle sleeps during the
last exposition, nobody likes to wake her. Fortunately, she comes
awake of herself, and puts the question to the Wandering Chairman. The
Wanderer can only speak of the case as if it were his own. If such a
young woman as the young woman described, had saved his own life, he
would have been very much obliged to her, wouldn’t have married her, and
would have got her a berth in an Electric Telegraph Office, where young
women answer very well.

What does the Genius of the three hundred and seventy-five thousand
pounds, no shillings, and nopence, think? He can’t say what he thinks,
without asking: Had the young woman any money?

‘No,’ says Lightwood, in an uncompromising voice; ‘no money.’

‘Madness and moonshine,’ is then the compressed verdict of the Genius.
‘A man may do anything lawful, for money. But for no money!--Bosh!’

What does Boots say?

Boots says he wouldn’t have done it under twenty thousand pound.

What does Brewer say?

Brewer says what Boots says.

What does Buffer say?

Buffer says he knows a man who married a bathing-woman, and bolted.

Lady Tippins fancies she has collected the suffrages of the whole
Committee (nobody dreaming of asking the Veneerings for their opinion),
when, looking round the table through her eyeglass, she perceives Mr
Twemlow with his hand to his forehead.

Good gracious! My Twemlow forgotten! My dearest! My own! What is his
vote?

Twemlow has the air of being ill at ease, as he takes his hand from his
forehead and replies.

‘I am disposed to think,’ says he, ‘that this is a question of the
feelings of a gentleman.’

‘A gentleman can have no feelings who contracts such a marriage,’
flushes Podsnap.

‘Pardon me, sir,’ says Twemlow, rather less mildly than usual, ‘I don’t
agree with you. If this gentleman’s feelings of gratitude, of respect,
of admiration, and affection, induced him (as I presume they did) to
marry this lady--’

‘This lady!’ echoes Podsnap.

‘Sir,’ returns Twemlow, with his wristbands bristling a little, ‘YOU
repeat the word; I repeat the word. This lady. What else would you call
her, if the gentleman were present?’

This being something in the nature of a poser for Podsnap, he merely
waves it away with a speechless wave.

‘I say,’ resumes Twemlow, ‘if such feelings on the part of this
gentleman, induced this gentleman to marry this lady, I think he is the
greater gentleman for the action, and makes her the greater lady. I beg
to say, that when I use the word, gentleman, I use it in the sense in
which the degree may be attained by any man. The feelings of a gentleman
I hold sacred, and I confess I am not comfortable when they are made the
subject of sport or general discussion.’

‘I should like to know,’ sneers Podsnap, ‘whether your noble relation
would be of your opinion.’

‘Mr Podsnap,’ retorts Twemlow, ‘permit me. He might be, or he might not
be. I cannot say. But, I could not allow even him to dictate to me on a
point of great delicacy, on which I feel very strongly.’

Somehow, a canopy of wet blanket seems to descend upon the company, and
Lady Tippins was never known to turn so very greedy or so very cross.
Mortimer Lightwood alone brightens. He has been asking himself, as to
every other member of the Committee in turn, ‘I wonder whether you are
the Voice!’ But he does not ask himself the question after Twemlow has
spoken, and he glances in Twemlow’s direction as if he were grateful.
When the company disperse--by which time Mr and Mrs Veneering have had
quite as much as they want of the honour, and the guests have had quite
as much as THEY want of the other honour--Mortimer sees Twemlow home,
shakes hands with him cordially at parting, and fares to the Temple,
gaily.



POSTSCRIPT

IN LIEU OF PREFACE


When I devised this story, I foresaw the likelihood that a class of
readers and commentators would suppose that I was at great pains to
conceal exactly what I was at great pains to suggest: namely, that Mr
John Harmon was not slain, and that Mr John Rokesmith was he. Pleasing
myself with the idea that the supposition might in part arise out
of some ingenuity in the story, and thinking it worth while, in the
interests of art, to hint to an audience that an artist (of whatever
denomination) may perhaps be trusted to know what he is about in his
vocation, if they will concede him a little patience, I was not alarmed
by the anticipation.

To keep for a long time unsuspected, yet always working itself out,
another purpose originating in that leading incident, and turning it to
a pleasant and useful account at last, was at once the most interesting
and the most difficult part of my design. Its difficulty was much
enhanced by the mode of publication; for, it would be very unreasonable
to expect that many readers, pursuing a story in portions from month
to month through nineteen months, will, until they have it before them
complete, perceive the relations of its finer threads to the whole
pattern which is always before the eyes of the story-weaver at his loom.
Yet, that I hold the advantages of the mode of publication to outweigh
its disadvantages, may be easily believed of one who revived it in the
Pickwick Papers after long disuse, and has pursued it ever since.

There is sometimes an odd disposition in this country to dispute as
improbable in fiction, what are the commonest experiences in fact.
Therefore, I note here, though it may not be at all necessary, that
there are hundreds of Will Cases (as they are called), far more
remarkable than that fancied in this book; and that the stores of the
Prerogative Office teem with instances of testators who have made,
changed, contradicted, hidden, forgotten, left cancelled, and left
uncancelled, each many more wills than were ever made by the elder Mr
Harmon of Harmony Jail.

In my social experiences since Mrs Betty Higden came upon the scene and
left it, I have found Circumlocutional champions disposed to be
warm with me on the subject of my view of the Poor Law. Mr friend Mr
Bounderby could never see any difference between leaving the Coketown
‘hands’ exactly as they were, and requiring them to be fed with turtle
soup and venison out of gold spoons. Idiotic propositions of a parallel
nature have been freely offered for my acceptance, and I have been
called upon to admit that I would give Poor Law relief to anybody,
anywhere, anyhow. Putting this nonsense aside, I have observed a
suspicious tendency in the champions to divide into two parties; the
one, contending that there are no deserving Poor who prefer death by
slow starvation and bitter weather, to the mercies of some Relieving
Officers and some Union Houses; the other, admitting that there are such
Poor, but denying that they have any cause or reason for what they do.
The records in our newspapers, the late exposure by THE LANCET, and the
common sense and senses of common people, furnish too abundant evidence
against both defences. But, that my view of the Poor Law may not be
mistaken or misrepresented, I will state it. I believe there has been
in England, since the days of the STUARTS, no law so often infamously
administered, no law so often openly violated, no law habitually so
ill-supervised. In the majority of the shameful cases of disease and
death from destitution, that shock the Public and disgrace the country,
the illegality is quite equal to the inhumanity--and known language
could say no more of their lawlessness.

On Friday the Ninth of June in the present year, Mr and Mrs Boffin (in
their manuscript dress of receiving Mr and Mrs Lammle at breakfast)
were on the South Eastern Railway with me, in a terribly destructive
accident. When I had done what I could to help others, I climbed back
into my carriage--nearly turned over a viaduct, and caught aslant upon
the turn--to extricate the worthy couple. They were much soiled, but
otherwise unhurt. The same happy result attended Miss Bella Wilfer on
her wedding day, and Mr Riderhood inspecting Bradley Headstone’s red
neckerchief as he lay asleep. I remember with devout thankfulness that I
can never be much nearer parting company with my readers for ever, than
I was then, until there shall be written against my life, the two words
with which I have this day closed this book:--THE END.

September 2nd, 1865.





End of the Project Gutenberg EBook of Our Mutual Friend, by Charles Dickens

*** END OF THIS PROJECT GUTENBERG EBOOK OUR MUTUAL FRIEND ***

***** This file should be named 883-0.txt or 883-0.zip *****
This and all associated files of various formats will be found in:
        http://www.gutenberg.org/8/8/883/

Produced by Donald Lainson; David Widger

Updated editions will replace the previous one--the old editions
will be renamed.

Creating the works from public domain print editions means that no
one owns a United States copyright in these works, so the Foundation
(and you!) can copy and distribute it in the United States without
permission and without paying copyright royalties.  Special rules,
set forth in the General Terms of Use part of this license, apply to
copying and distributing Project Gutenberg-tm electronic works to
protect the PROJECT GUTENBERG-tm concept and trademark.  Project
Gutenberg is a registered trademark, and may not be used if you
charge for the eBooks, unless you receive specific permission.  If you
do not charge anything for copies of this eBook, complying with the
rules is very easy.  You may use this eBook for nearly any purpose
such as creation of derivative works, reports, performances and
research.  They may be modified and printed and given away--you may do
practically ANYTHING with public domain eBooks.  Redistribution is
subject to the trademark license, especially commercial
redistribution.



*** START: FULL LICENSE ***

THE FULL PROJECT GUTENBERG LICENSE
PLEASE READ THIS BEFORE YOU DISTRIBUTE OR USE THIS WORK

To protect the Project Gutenberg-tm mission of promoting the free
distribution of electronic works, by using or distributing this work
(or any other work associated in any way with the phrase “Project
Gutenberg”), you agree to comply with all the terms of the Full Project
Gutenberg-tm License (available with this file or online at
http://gutenberg.org/license).


Section 1.  General Terms of Use and Redistributing Project Gutenberg-tm
electronic works

1.A.  By reading or using any part of this Project Gutenberg-tm
electronic work, you indicate that you have read, understand, agree to
and accept all the terms of this license and intellectual property
(trademark/copyright) agreement.  If you do not agree to abide by all
the terms of this agreement, you must cease using and return or destroy
all copies of Project Gutenberg-tm electronic works in your possession.
If you paid a fee for obtaining a copy of or access to a Project
Gutenberg-tm electronic work and you do not agree to be bound by the
terms of this agreement, you may obtain a refund from the person or
entity to whom you paid the fee as set forth in paragraph 1.E.8.

1.B.  “Project Gutenberg” is a registered trademark.  It may only be
used on or associated in any way with an electronic work by people who
agree to be bound by the terms of this agreement.  There are a few
things that you can do with most Project Gutenberg-tm electronic works
even without complying with the full terms of this agreement.  See
paragraph 1.C below.  There are a lot of things you can do with Project
Gutenberg-tm electronic works if you follow the terms of this agreement
and help preserve free future access to Project Gutenberg-tm electronic
works.  See paragraph 1.E below.

1.C.  The Project Gutenberg Literary Archive Foundation (“the Foundation”
 or PGLAF), owns a compilation copyright in the collection of Project
Gutenberg-tm electronic works.  Nearly all the individual works in the
collection are in the public domain in the United States.  If an
individual work is in the public domain in the United States and you are
located in the United States, we do not claim a right to prevent you from
copying, distributing, performing, displaying or creating derivative
works based on the work as long as all references to Project Gutenberg
are removed.  Of course, we hope that you will support the Project
Gutenberg-tm mission of promoting free access to electronic works by
freely sharing Project Gutenberg-tm works in compliance with the terms of
this agreement for keeping the Project Gutenberg-tm name associated with
the work.  You can easily comply with the terms of this agreement by
keeping this work in the same format with its attached full Project
Gutenberg-tm License when you share it without charge with others.

1.D.  The copyright laws of the place where you are located also govern
what you can do with this work.  Copyright laws in most countries are in
a constant state of change.  If you are outside the United States, check
the laws of your country in addition to the terms of this agreement
before downloading, copying, displaying, performing, distributing or
creating derivative works based on this work or any other Project
Gutenberg-tm work.  The Foundation makes no representations concerning
the copyright status of any work in any country outside the United
States.

1.E.  Unless you have removed all references to Project Gutenberg:

1.E.1.  The following sentence, with active links to, or other immediate
access to, the full Project Gutenberg-tm License must appear prominently
whenever any copy of a Project Gutenberg-tm work (any work on which the
phrase “Project Gutenberg” appears, or with which the phrase “Project
Gutenberg” is associated) is accessed, displayed, performed, viewed,
copied or distributed:

This eBook is for the use of anyone anywhere at no cost and with
almost no restrictions whatsoever.  You may copy it, give it away or
re-use it under the terms of the Project Gutenberg License included
with this eBook or online at www.gutenberg.org

1.E.2.  If an individual Project Gutenberg-tm electronic work is derived
from the public domain (does not contain a notice indicating that it is
posted with permission of the copyright holder), the work can be copied
and distributed to anyone in the United States without paying any fees
or charges.  If you are redistributing or providing access to a work
with the phrase “Project Gutenberg” associated with or appearing on the
work, you must comply either with the requirements of paragraphs 1.E.1
through 1.E.7 or obtain permission for the use of the work and the
Project Gutenberg-tm trademark as set forth in paragraphs 1.E.8 or
1.E.9.

1.E.3.  If an individual Project Gutenberg-tm electronic work is posted
with the permission of the copyright holder, your use and distribution
must comply with both paragraphs 1.E.1 through 1.E.7 and any additional
terms imposed by the copyright holder.  Additional terms will be linked
to the Project Gutenberg-tm License for all works posted with the
permission of the copyright holder found at the beginning of this work.

1.E.4.  Do not unlink or detach or remove the full Project Gutenberg-tm
License terms from this work, or any files containing a part of this
work or any other work associated with Project Gutenberg-tm.

1.E.5.  Do not copy, display, perform, distribute or redistribute this
electronic work, or any part of this electronic work, without
prominently displaying the sentence set forth in paragraph 1.E.1 with
active links or immediate access to the full terms of the Project
Gutenberg-tm License.

1.E.6.  You may convert to and distribute this work in any binary,
compressed, marked up, nonproprietary or proprietary form, including any
word processing or hypertext form.  However, if you provide access to or
distribute copies of a Project Gutenberg-tm work in a format other than
“Plain Vanilla ASCII” or other format used in the official version
posted on the official Project Gutenberg-tm web site (www.gutenberg.org),
you must, at no additional cost, fee or expense to the user, provide a
copy, a means of exporting a copy, or a means of obtaining a copy upon
request, of the work in its original “Plain Vanilla ASCII” or other
form.  Any alternate format must include the full Project Gutenberg-tm
License as specified in paragraph 1.E.1.

1.E.7.  Do not charge a fee for access to, viewing, displaying,
performing, copying or distributing any Project Gutenberg-tm works
unless you comply with paragraph 1.E.8 or 1.E.9.

1.E.8.  You may charge a reasonable fee for copies of or providing
access to or distributing Project Gutenberg-tm electronic works provided
that

- You pay a royalty fee of 20% of the gross profits you derive from
     the use of Project Gutenberg-tm works calculated using the method
     you already use to calculate your applicable taxes.  The fee is
     owed to the owner of the Project Gutenberg-tm trademark, but he
     has agreed to donate royalties under this paragraph to the
     Project Gutenberg Literary Archive Foundation.  Royalty payments
     must be paid within 60 days following each date on which you
     prepare (or are legally required to prepare) your periodic tax
     returns.  Royalty payments should be clearly marked as such and
     sent to the Project Gutenberg Literary Archive Foundation at the
     address specified in Section 4, “Information about donations to
     the Project Gutenberg Literary Archive Foundation.”

- You provide a full refund of any money paid by a user who notifies
     you in writing (or by e-mail) within 30 days of receipt that s/he
     does not agree to the terms of the full Project Gutenberg-tm
     License.  You must require such a user to return or
     destroy all copies of the works possessed in a physical medium
     and discontinue all use of and all access to other copies of
     Project Gutenberg-tm works.

- You provide, in accordance with paragraph 1.F.3, a full refund of any
     money paid for a work or a replacement copy, if a defect in the
     electronic work is discovered and reported to you within 90 days
     of receipt of the work.

- You comply with all other terms of this agreement for free
     distribution of Project Gutenberg-tm works.

1.E.9.  If you wish to charge a fee or distribute a Project Gutenberg-tm
electronic work or group of works on different terms than are set
forth in this agreement, you must obtain permission in writing from
both the Project Gutenberg Literary Archive Foundation and Michael
Hart, the owner of the Project Gutenberg-tm trademark.  Contact the
Foundation as set forth in Section 3 below.

1.F.

1.F.1.  Project Gutenberg volunteers and employees expend considerable
effort to identify, do copyright research on, transcribe and proofread
public domain works in creating the Project Gutenberg-tm
collection.  Despite these efforts, Project Gutenberg-tm electronic
works, and the medium on which they may be stored, may contain
“Defects,” such as, but not limited to, incomplete, inaccurate or
corrupt data, transcription errors, a copyright or other intellectual
property infringement, a defective or damaged disk or other medium, a
computer virus, or computer codes that damage or cannot be read by
your equipment.

1.F.2.  LIMITED WARRANTY, DISCLAIMER OF DAMAGES - Except for the “Right
of Replacement or Refund” described in paragraph 1.F.3, the Project
Gutenberg Literary Archive Foundation, the owner of the Project
Gutenberg-tm trademark, and any other party distributing a Project
Gutenberg-tm electronic work under this agreement, disclaim all
liability to you for damages, costs and expenses, including legal
fees.  YOU AGREE THAT YOU HAVE NO REMEDIES FOR NEGLIGENCE, STRICT
LIABILITY, BREACH OF WARRANTY OR BREACH OF CONTRACT EXCEPT THOSE
PROVIDED IN PARAGRAPH F3.  YOU AGREE THAT THE FOUNDATION, THE
TRADEMARK OWNER, AND ANY DISTRIBUTOR UNDER THIS AGREEMENT WILL NOT BE
LIABLE TO YOU FOR ACTUAL, DIRECT, INDIRECT, CONSEQUENTIAL, PUNITIVE OR
INCIDENTAL DAMAGES EVEN IF YOU GIVE NOTICE OF THE POSSIBILITY OF SUCH
DAMAGE.

1.F.3.  LIMITED RIGHT OF REPLACEMENT OR REFUND - If you discover a
defect in this electronic work within 90 days of receiving it, you can
receive a refund of the money (if any) you paid for it by sending a
written explanation to the person you received the work from.  If you
received the work on a physical medium, you must return the medium with
your written explanation.  The person or entity that provided you with
the defective work may elect to provide a replacement copy in lieu of a
refund.  If you received the work electronically, the person or entity
providing it to you may choose to give you a second opportunity to
receive the work electronically in lieu of a refund.  If the second copy
is also defective, you may demand a refund in writing without further
opportunities to fix the problem.

1.F.4.  Except for the limited right of replacement or refund set forth
in paragraph 1.F.3, this work is provided to you ‘AS-IS’ WITH NO OTHER
WARRANTIES OF ANY KIND, EXPRESS OR IMPLIED, INCLUDING BUT NOT LIMITED TO
WARRANTIES OF MERCHANTIBILITY OR FITNESS FOR ANY PURPOSE.

1.F.5.  Some states do not allow disclaimers of certain implied
warranties or the exclusion or limitation of certain types of damages.
If any disclaimer or limitation set forth in this agreement violates the
law of the state applicable to this agreement, the agreement shall be
interpreted to make the maximum disclaimer or limitation permitted by
the applicable state law.  The invalidity or unenforceability of any
provision of this agreement shall not void the remaining provisions.

1.F.6.  INDEMNITY - You agree to indemnify and hold the Foundation, the
trademark owner, any agent or employee of the Foundation, anyone
providing copies of Project Gutenberg-tm electronic works in accordance
with this agreement, and any volunteers associated with the production,
promotion and distribution of Project Gutenberg-tm electronic works,
harmless from all liability, costs and expenses, including legal fees,
that arise directly or indirectly from any of the following which you do
or cause to occur: (a) distribution of this or any Project Gutenberg-tm
work, (b) alteration, modification, or additions or deletions to any
Project Gutenberg-tm work, and (c) any Defect you cause.


Section  2.  Information about the Mission of Project Gutenberg-tm

Project Gutenberg-tm is synonymous with the free distribution of
electronic works in formats readable by the widest variety of computers
including obsolete, old, middle-aged and new computers.  It exists
because of the efforts of hundreds of volunteers and donations from
people in all walks of life.

Volunteers and financial support to provide volunteers with the
assistance they need, is critical to reaching Project Gutenberg-tm’s
goals and ensuring that the Project Gutenberg-tm collection will
remain freely available for generations to come.  In 2001, the Project
Gutenberg Literary Archive Foundation was created to provide a secure
and permanent future for Project Gutenberg-tm and future generations.
To learn more about the Project Gutenberg Literary Archive Foundation
and how your efforts and donations can help, see Sections 3 and 4
and the Foundation web page at http://www.pglaf.org.


Section 3.  Information about the Project Gutenberg Literary Archive
Foundation

The Project Gutenberg Literary Archive Foundation is a non profit
501(c)(3) educational corporation organized under the laws of the
state of Mississippi and granted tax exempt status by the Internal
Revenue Service.  The Foundation’s EIN or federal tax identification
number is 64-6221541.  Its 501(c)(3) letter is posted at
http://pglaf.org/fundraising.  Contributions to the Project Gutenberg
Literary Archive Foundation are tax deductible to the full extent
permitted by U.S. federal laws and your state’s laws.

The Foundation’s principal office is located at 4557 Melan Dr. S.
Fairbanks, AK, 99712., but its volunteers and employees are scattered
throughout numerous locations.  Its business office is located at
809 North 1500 West, Salt Lake City, UT 84116, (801) 596-1887, email
business@pglaf.org.  Email contact links and up to date contact
information can be found at the Foundation’s web site and official
page at http://pglaf.org

For additional contact information:
     Dr. Gregory B. Newby
     Chief Executive and Director
     gbnewby@pglaf.org


Section 4.  Information about Donations to the Project Gutenberg
Literary Archive Foundation

Project Gutenberg-tm depends upon and cannot survive without wide
spread public support and donations to carry out its mission of
increasing the number of public domain and licensed works that can be
freely distributed in machine readable form accessible by the widest
array of equipment including outdated equipment.  Many small donations
($1 to $5,000) are particularly important to maintaining tax exempt
status with the IRS.

The Foundation is committed to complying with the laws regulating
charities and charitable donations in all 50 states of the United
States.  Compliance requirements are not uniform and it takes a
considerable effort, much paperwork and many fees to meet and keep up
with these requirements.  We do not solicit donations in locations
where we have not received written confirmation of compliance.  To
SEND DONATIONS or determine the status of compliance for any
particular state visit http://pglaf.org

While we cannot and do not solicit contributions from states where we
have not met the solicitation requirements, we know of no prohibition
against accepting unsolicited donations from donors in such states who
approach us with offers to donate.

International donations are gratefully accepted, but we cannot make
any statements concerning tax treatment of donations received from
outside the United States.  U.S. laws alone swamp our small staff.

Please check the Project Gutenberg Web pages for current donation
methods and addresses.  Donations are accepted in a number of other
ways including checks, online payments and credit card donations.
To donate, please visit: http://pglaf.org/donate


Section 5.  General Information About Project Gutenberg-tm electronic
works.

Professor Michael S. Hart is the originator of the Project Gutenberg-tm
concept of a library of electronic works that could be freely shared
with anyone.  For thirty years, he produced and distributed Project
Gutenberg-tm eBooks with only a loose network of volunteer support.


Project Gutenberg-tm eBooks are often created from several printed
editions, all of which are confirmed as Public Domain in the U.S.
unless a copyright notice is included.  Thus, we do not necessarily
keep eBooks in compliance with any particular paper edition.


Most people start at our Web site which has the main PG search facility:

     http://www.gutenberg.org

This Web site includes information about Project Gutenberg-tm,
including how to make donations to the Project Gutenberg Literary
Archive Foundation, how to help produce our new eBooks, and how to
subscribe to our email newsletter to hear about new eBooks.
